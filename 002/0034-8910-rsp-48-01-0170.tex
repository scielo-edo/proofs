% Generated by jats2tex@0.11.1.0
\documentclass{article}
\usepackage[T1]{fontenc}
\usepackage[utf8]{inputenc} %% *
\usepackage[portuges,spanish,english,german,italian,russian]{babel} %% *
\usepackage{amstext}
\usepackage{authblk}
\usepackage{unicode-math}
\usepackage{multirow}
\usepackage{graphicx}
\usepackage{etoolbox}
\usepackage{xtab}
\usepackage{enumerate}
\usepackage{hyperref}
\usepackage{penalidades}
\usepackage[footnotesize,bf,hang]{caption}
\usepackage[nodayofweek,level]{datetime}
\usepackage[top=0.85in,left=2.75in,footskip=0.75in]{geometry}
\newlength\savedwidth
\newcommand\thickcline[1]{\noalign{\global
\savedwidth
\arrayrulewidth
\global\arrayrulewidth 2pt}
\cline{#1}
\noalign{\vskip\arrayrulewidth}
\noalign{\global\arrayrulewidth\savedwidth}}
\newcommand\thickhline{\noalign{\global
\savedwidth\arrayrulewidth
\global\arrayrulewidth 2pt}
\hline
\noalign{\global\arrayrulewidth\savedwidth}}
\usepackage{lastpage,fancyhdr}
\usepackage{epstopdf}
\pagestyle{myheadings}
\pagestyle{fancy}
\fancyhf{}
\setlength{\headheight}{27.023pt}
\lhead{\includegraphics[width=10mm]{logo.png}}
\rhead{\ifdef{\journaltitle}{\journaltitle}{}
\ifdef{\volume}{vol.\,\volume}{}
\ifdef{\issue}{(\issue)}{}
\ifdef{\fpage}{\fpage--\lpage\,pp.}}
\rfoot{\thepage/\pageref{LastPage}}
\renewcommand{\footrule}{\hrule height 2pt \vspace{2mm}}
\fancyheadoffset[L]{2.25in}
\fancyfootoffset[L]{2.25in}
\lfoot{\sf \ifdef{\articledoi}{\articledoi}{}}
\setmainfont{Linux Libertine O}
\renewcommand*{\thefootnote}{\alph{footnote}}
\makeatletter
\newcommand{\fn}{\afterassignment\fn@aux\count0=}
\newcommand{\fn@aux}{\csname fn\the\count0\endcsname}
\makeatother

\newcommand{\journalid}{Rev Saude Publica}
\newcommand{\journaltitle}{Revista de Saúde Pública}
\newcommand{\abbrevjournaltitle}{Rev. Saúde Pública}
\newcommand{\issnppub}{0034-8910}
\newcommand{\issnepub}{1518-8787}
\newcommand{\publishername}{Faculdade de Saúde Pública da Universidade de São
Paulo}
\newcommand\articledoi{\textsc{doi} 10.1590/S0034-8910.2014048004498}
\def\subject{Revisões}\newcommand{\subtitlestyle}[1]{-- \emph{#1}\medskip}
\newcommand{\transtitlestyle}[1]{\par\medskip\Large #1}
\newcommand{\transsubtitlestyle}[1]{-- \Large\emph{ #1}}

\newcommand{\titlegroup}{
\ifdef{\subtitle}{\subtitlestyle{\subtitle}}{}
\ifdef{\transtitle}{\transtitlestyle{\transtitle}}{}
\ifdef{\transsubtitle}{\transsubtitlestyle{\transsubtitle}}{}}

\title{Educação permanente em saúde: metassíntese\titlegroup{}}
\newcommand{\transtitle}{Permanent education in health: a review}
\newcommand{\transtitle}{Educación permanente en salud: metasíntesis}
\author[{I}]{Miccas, Fernanda Luppino}
\author[{II}]{Batista, Sylvia Helena Souza da Silva}
\affil[i]{Universidade Federal de São Paulo}
\affil[ii]{Universidade Federal de São Paulo}
\def\authornotes{Correspondência | Correspondence : Fernanda Luppino Miccas. Rua
Expedicionário Poitena, 58 Centro. 11740-000 Itanhaém, SP, Brasil. E-mail:
femiccas@hotmail.com
Os autores declaram não haver conflito de interesses.}
\date{ 02 2014}
\def\volume{48}
\def\issue{1}
\def\fpage{170}
\def\lpage{185}
\def\permissions{This is an Open Access article distributed under the terms of
the Creative Commons Attribution Non-Commercial License, which permits
unrestricted non-commercial use, distribution, and reproduction in any medium,
provided the original work is properly cited.}
\newcommand{\kwdgroup}{Educação Continuada, Educação Profissional em Saúde
Pública, Educação Profissional, Saúde Pública, Revisão}
\newcommand{\kwdgroupen}{Education, Continuing, Education, Public Health
Professional, Education, Professional, Public Health, Review}
\newcommand{\kwdgroupes}{Educación Continua, Educación en Salud Pública
Profesional, Educación Profesional, Salud Pública, Revisión}
%%% Nota fn1 %%%%%%%%%%%%%%%%%%%%%%%%%%%%%%%%%%%%%%%%%%%%%%%%%%%%%%%%
\expandafter\newcommand\csname fn1\endcsname{
Organizacion Mundial de la Salud. Organización Panamericana de La Salud.
Ministerio de Salud de Canadá. Ministerio de Salud y Cuidados Prolongados de la
Provincia de Ontario. Llamado a la acción de Toronto 2006-2015: hacia una década
de recursos humanos en salud para las Américas. Reunión regional de los
observatorios de recursos humanos en salud. Toronto; 2005 [citado 2012 jan 08].
Disponível em: http://www.observatoriorh.org/sites/default/files/webfiles/fullte
xt/\textsc{ops}\_{}desafios\_{}toronto\_{}2005.pdf}
%%% Nota fn2 %%%%%%%%%%%%%%%%%%%%%%%%%%%%%%%%%%%%%%%%%%%%%%%%%%%%%%%%
\expandafter\newcommand\csname fn2\endcsname{
Organizacion Mundial de la Salud. Organización Panamericana de La Salud. 27ª
Conferencia Sanitaria Panamericana. 59ª Sesion del Comité Regional. Metas
regionales en materia de recursos humanos para la salud 2007-2015. Punto 4.6 del
orden del día \textsc{csp}27/10. Washington (DC); 2007 [citado 2012 jan 12]. Disponível
em: http://www.paho.org/spanish/gov/csp/csp27-10-s.pdf}
%%% Nota fn3 %%%%%%%%%%%%%%%%%%%%%%%%%%%%%%%%%%%%%%%%%%%%%%%%%%%%%%%%
\expandafter\newcommand\csname fn3\endcsname{
Estorniolo Filho J, organizador. End Note Web: Guia de uso. São Paulo: Faculdade
de Saúde Pública da \textsc{usp}; 2011 [citado 2011 jun 23]. Disponível em: http://citrus.uspnet.usp.br/sibi/tutoriais/Manual\_{}EndNoteWeb\_{}publicacao.pdf}
%%% Nota fn4 %%%%%%%%%%%%%%%%%%%%%%%%%%%%%%%%%%%%%%%%%%%%%%%%%%%%%%%%
\expandafter\newcommand\csname fn4\endcsname{
Ferraz F. Contextos e processos de desenvolvimento das comissões permanentes de
integração de ensino-serviço: perspectiva dos sujeitos sociais pautada na
concepção dialógica de Paulo Freire [tese de doutorado]. Florianópolis:
Universidade Federal de Santa Catarina; 2011.}
%%% Nota fn5 %%%%%%%%%%%%%%%%%%%%%%%%%%%%%%%%%%%%%%%%%%%%%%%%%%%%%%%%
\expandafter\newcommand\csname fn5\endcsname{
Ministério da Saúde. Pólos de educação permanente em saúde: política de educação
e desenvolvimento para o \textsc{sus}, caminhos para a educação permanente em saúde.
Brasília (DF); 2004.}
%%% Nota fn6 %%%%%%%%%%%%%%%%%%%%%%%%%%%%%%%%%%%%%%%%%%%%%%%%%%%%%%%%
\expandafter\newcommand\csname fn6\endcsname{
Pinheiro R, Ceccim RB. Experimentação, formação, cuidado e conhecimento em
saúde: articulando concepções, percepções e sensações para efetivar o ensino da
integralidade. In: Pineiro R, Ceccim RB, Mattos RA, editores. Ensinar Saúde: a
integralidade do \textsc{sus} nos cursos de graduação na área da saúde. Rio de Janeiro:
\textsc{uerj}, \textsc{cepesc}, \textsc{abrasco}; 2006. p.13-35.}
%%% Nota fn7 %%%%%%%%%%%%%%%%%%%%%%%%%%%%%%%%%%%%%%%%%%%%%%%%%%%%%%%%
\expandafter\newcommand\csname fn7\endcsname{
Ministério da Saúde. Portaria GM/MS n\textsuperscript{o}
1.996/07, de 20 de agosto de 2007. Dispõe sobre as diretrizes para a
implementação da Política nacional de educação permanente em saúde.
\textit{Diario Oficial Uniao.}
22 ago 2007 [citado 2012 jan 24]. Disponível em: http://www.saude.gov.br/sgtes}
%%% Nota fn8 %%%%%%%%%%%%%%%%%%%%%%%%%%%%%%%%%%%%%%%%%%%%%%%%%%%%%%%%
\expandafter\newcommand\csname fn8\endcsname{
Ministério da Saúde. Secretaria de Gestão do Trabalho e da Educação na Saúde.
Política nacional de educação permanente em saúde. Brasília (DF); 2009.}
%%% Nota fn9 %%%%%%%%%%%%%%%%%%%%%%%%%%%%%%%%%%%%%%%%%%%%%%%%%%%%%%%%
\expandafter\newcommand\csname fn9\endcsname{
Ministério da Saúde. Secretaria de Gestão Programa Nacional de Reorientação da
Formação Profissional em Saúde. Pró-saúde: objetivos, impletação e
desenvolvimento potencial. Brasília (DF); 2007.}

\begin{document}
\selectlanguage{portuges}
\section*{Metadados não aplicados}
\begin{itemize}
\item[\textbf{língua do artigo}]{Português}
\ifdef{\journalid}{\item[\textbf{journalid}] \journalid}{}
\ifdef{\journaltitle}{\item[\textbf{journaltitle}] \journaltitle}{}

\ifdef{\journalsubtitle}{\item[\textbf{journalsubtitle}] \journaltitle}{}
\ifdef{\transjournaltitle}{\item[\textbf{journaltitle}] \journaltitle}{}
\ifdef{\transjournalsubtitle}{\item[\textbf{journalsubtitle}] \journaltitle}{}

\ifdef{\abbrevjournaltitle}{\item[\textbf{abbrevjournaltitle}]
\abbrevjournaltitle}{}
\ifdef{\issnppub}{\item[\textbf{issnppub}] \issnppub}{}
\ifdef{\issnepub}{\item[\textbf{issnepub}] \issnepub}{}
\ifdef{\publishername}{\item[\textbf{publishername}] \publishername}{}
\ifdef{\publisherid}{\item[\textbf{publisherid}] \publisherid}{}
\ifdef{\subject}{\item[\textbf{subject}] \subject}{}
\ifdef{\transtitle}{\item[\textbf{transtitle}] \transtitle}{}
\ifdef{\authornotes}{\item[\textbf{authornotes}] \authornotes}{}
\ifdef{\articleid}{\item[\textbf{articleid}] \articleid}{}
\ifdef{\articledoi}{\item[\textbf{articledoi}] \articledoi}{}
\ifdef{\volume}{\item[\textbf{volume}] \volume}{}
\ifdef{\issue}{\item[\textbf{issue}] \issue}{}
\ifdef{\fpage}{\item[\textbf{fpage}] \fpage}{}
\ifdef{\lpage}{\item[\textbf{lpage}] \lpage}{}
\ifdef{\permissions}{\item[\textbf{permissions}] \permissions}{}
\end{itemize}
\maketitle

\begingroup

\begin{abstract}
\section{\textsc{objetivo}}

: Realizar metassíntese da literatura sobre os principais conceitos e práticas
relacionados à educação permanente em saúde.

\section{\textsc{métodos}}

: Foi realizada busca bibliográfica de artigos originais nas bases de dados
PubMed, Web of Science, Lilacs, \textsc{ibecs} e Sci\textsc{elo}, utilizando os seguintes
descritores: “ \textit{public health professional education}
”, “ \textit{permanent education”, “continuing education}
”, “ \textit{permanent education health}
”. De um total de 590 artigos identificados, após os critérios de inclusão e
exclusão, foram selecionados 48 para análise, os quais foram submetidos à
análise individual, análise comparativa, análise com critérios de agrupamentos
de elementos-chave e submetidos à metassíntese.

\section{\textsc{resultados}}

: Os 48 artigos originais foram classificados como elementos-chave em quatro
unidades temáticas: 1) Concepções; 2) Estratégias e dificuldades; 3) Políticas
públicas; e 4) Instituições formadoras. Foram encontradas três concepções
principais de educação permanente em saúde: problematizadora e focada no
trabalho em equipe, diretamente relacionada à educação continuada e educação que
se dá ao longo da vida. As principais estratégias para efetivação da educação
permanente foram a problematização, manutenção de espaços para a educação
permanente e polos de educação permanente. O maior fator limitante foi
relacionado à gerência direta ou indireta. Foram indicadas a necessidade de
implementação e manutenção de políticas públicas, além de disponibilidade de
recursos financeiros e de recursos humanos. As instituições formadoras teriam
necessidade de articular ensino e serviço para a formação de egressos
críticos-reflexivos.

\section{\textsc{conclusões}}

: A articulação educação e saúde encontra-se pautada tanto nas ações dos
serviços de saúde, quanto de gestão e de instituições formadoras. Assim,
torna-se um desafio implementar processos de ensino-aprendizagem que sejam
respaldados por ações crítico-reflexivas. É necessário realizar propostas de
educação permanente em saúde com a participação de profissionais dos serviços,
professores e profissionais das instituições de ensino.

\iflanguage{portuges}{\medskip\noindent\textbf{Palavras-chave:} \kwdgroup}{}
\iflanguage{english}{\medskip\noindent\textbf{Keywords:} \kwdgroupen}{}
\iflanguage{spanish}{\medskip\noindent\textbf{Palavras claves:} \kwdgroupes}{}
\iflanguage{french}{\medskip\noindent\textbf{Mots clés:} \kwdgroupfr}{}
\end{abstract}
\endgroup

\begingroup
\renewcommand{\section}[1]{\subsection*{#1}}
\begin{otherlanguage}{english}

\begin{abstract}
\section{\textsc{objective}}

: To undertake a meta-synthesis of the literature on the main concepts and
practices related to permanent education in health.

\section{\textsc{methods}}

: A bibliographical search was conducted for original articles in the PubMed,
Web of Science, \textsc{lilacs}, \textsc{ibecs} and Sci\textsc{elo} databases, using the following search
terms: “public health professional education”, “permanent education”,
“continuing education”, “permanent education health”. Of the 590 articles
identified, after applying inclusion and exclusion criteria, 48 were selected
for further analysis, grouped according to the criteria of key elements, and
then underwent meta-synthesis.

\section{\textsc{results}}

: The 48 original publications were classified according to four thematic units
of key elements: 1) concepts, 2) strategies and difficulties, 3) public policies
and 4) educational institutions. Three main conceptions of permanent education
in health were found: problem-focused and team work, directly related to
continuing education and education that takes place throughout life. The main
strategies for executing permanent education in health are discussion,
maintaining an open space for permanent education, and permanent education
clusters. The most limiting factor is mainly related to directly or indirect
management. Another highlight is the requirement for implementation and
maintenance of public policies, and the availability of financial and human
resources. The educational institutions need to combine education and service
aiming to form critical-reflexive graduates.

\section{\textsc{conclusions}}

: The coordination between health and education is based as much on the actions
of health services as on management and educational institutions. Thus, it
becomes a challenge to implement the teaching-learning processes that are
supported by critical-reflexive actions. It is necessary to carry out proposals
for permanent education in health involving the participation of health
professionals, teachers and educational institutions.

\ifdef{\kwdgroupen}{\medskip\noindent\textbf{Keywords:} \kwdgroupen}{}
\end{abstract}
\end{otherlanguage}
\endgroup

\begingroup
\renewcommand{\section}[1]{\subsection*{#1}}
\begin{otherlanguage}{spanish}

\begin{abstract}
\section{\textsc{objetivo}}

: Realizar metasíntesis de la literatura sobre los principales conceptos y
prácticas relacionados con la educación permanente en salud.

\section{\textsc{métodos}}

: Se realizó búsqueda bibliográfica de artículos originales en las bases de
datos PubMed, Web of Science, Lilacs, \textsc{ibecs} y Sci\textsc{elo}, utilizando los siguientes
descriptores: “ \textit{public health professional education”, “permanent
education”, “continuing education”, “permanent education health}
”. De un total de 590 artículos identificados, posterior a los criterios de
inclusión y exclusión, fueron seleccionados 48 para análisis, los cuales fueron
sometidos al análisis individual, análisis comparativo, análisis con criterios
de agrupamiento de elementos-clave y sometidos a metasíntesis.

\section{\textsc{resultados}}

: Los 48 artículos originales fueron clasificados como elementos-clave en cuatro
unidades temáticas: 1) Concepciones; 2) Estrategias y dificultades; 3) Políticas
públicas e 4) Instituciones formadoras. Se encontraron tres concepciones
principales de educación permanente en salud: ubicación del problema y enfocarlo
en el trabajo en equipo, directamente relacionado con la educación continua y
educación que se da a lo largo de la vida. Las principales estrategias para
efectivar la educación permanente fueron la ubicación del problema,
mantenimiento de espacios para la educación permanente y polos de educación
permanente. El mayor factor limitante estuvo relacionado con la gerencia directa
o indirecta. Fueron mencionadas la necesidad de implementación y mantenimiento
de políticas públicas, así como la disponibilidad de recursos financieros y
humanos. Las instituciones formadoras tendrían la necesidad de articular
educación y servicio para la formación de egresados críticos-reflexivos.

\section{\textsc{conclusiones}}

: La articulación educación y salud se encuentra pautada tanto en las acciones
de los servicios de salud, cuanto en la gestión y de instituciones formadoras.
Así, se torna un desafío implementar procesos de educación-aprendizaje que sean
respaldados por acciones crítico-reflexivas. Es necesario realizar propuestas de
educación permanente en salud con la participación de profesionales de los
servicios, profesores y profesionales de las instituciones de educación.

\ifdef{\kwdgroupes}{\medskip\noindent\textbf{Palavras claves:} \kwdgroupes}{}
\end{abstract}
\end{otherlanguage}
\endgroup
\section{\textsc{introdução}}

A criação de uma equipe de profissionais da área da saúde se apresenta como um
processo desafiador que perpassa as definições políticas e as práticas
institucionais para atender às expectativas na qualidade do atendimento à
comunidade.\textsuperscript{[}\textsuperscript{57}\textsuperscript{]}

O Sistema Único de Saúde (\textsc{sus}), pela dimensão e amplitude que tem, aparece na
arena dos processos educacionais de saúde como um lugar privilegiado para o
ensino e a aprendizagem, especialmente os lugares de assistência à saúde. Educar
“no” e “para o” trabalho é o pressuposto da proposta de educação permanente em
saúde (\textsc{eps}). Os lugares de produção de cuidado, visando integralidade,
corresponsabilidade e resolutividade são, ao mesmo tempo, cenários de produção
pedagógica, pois concentram, o encontro criativo entre trabalhadores e usuários.\textsuperscript{[}\textsuperscript{23}\textsuperscript{]}
A proposta da \textsc{eps} surgiu na década de 1980, por iniciativa da Organização
Pan-Americana da Saúde e da Organização Mundial da Saúde (\textsc{opas}/\textsc{oms}) para o
desenvolvimento dos Recursos Humanos na Saúde. No Brasil, foi lançada como
política nacional em 2003, constituindo papel importante na concepção de um \textsc{sus}
democrático, equitativo e eficiente.\textsuperscript{[}\textsuperscript{40}\textsuperscript{]}\textsuperscript{,}\textsuperscript{[}\textsuperscript{43}\textsuperscript{]}

A \textsc{eps} construída como instrumento para transformar o profissional de saúde em um
profundo conhecedor da sua realidade local. Por isso, foi necessário situar a
formação dentro de um marco de regionalização, com programas adaptados para os
profissionais e equipes de saúde em cada nível local do sistema de saúde.\textsuperscript{[}\textsuperscript{33}\textsuperscript{]}

O primeiro passo para provocar mudanças nos processos de formação é entender que
as propostas não podem mais ser construídas isoladamente e nem de cima para
baixo, hierarquizadas. Elas devem fazer parte de uma grande estratégia, estar
articuladas entre si e ser criadas a partir da problematização das realidades
locais, envolvendo os diversos segmentos.\textsuperscript{[}\textsuperscript{66}\textsuperscript{]}

Para o desenvolvimento exitoso dos recursos humanos em saúde, o planejamento e a
formulação de políticas devem resultar de um esforço multissetorial entre saúde,
educação, trabalho e finanças, articulando-se atores governamentais e não
governamentais.\footnote{\fn1}\textsuperscript{,}\footnote{\fn2}

Sob este prisma, verifica-se a necessidade de impulsionar o ensino profissional
de modo que o perfil do trabalhador seja voltado para a integralidade do cuidado
e permanente reestruturação de seus conhecimentos a partir da problematização e
demandas internas sob a lógica da \textsc{eps}.

O objetivo do presente estudo foi realizar metassíntese da literatura sobre os
principais conceitos e práticas relacionados à educação permanente em saúde.

\section{\textsc{métodos}}

Utilizamos a metassíntese como modalidade de revisão de literatura,\textsuperscript{[}\textsuperscript{60}\textsuperscript{]}
por ser uma ferramenta metodológica importante para integrar as informações de
conjuntos de estudos realizados separadamente sobre determinada intervenção ou
área.\textsuperscript{[}\textsuperscript{32}\textsuperscript{]}
Abrange tanto o processo analítico, quanto a interpretação dos resultados,
possibilitando sintetizar e obter ampla compreensão conceitual.\textsuperscript{[}\textsuperscript{59}\textsuperscript{]}
Essas integrações vão além da soma das partes, uma vez que oferecem nova
interpretação, que não pode ser encontrada em nenhum relatório primário, pois
todos os artigos tornaram-se uma única amostra.\textsuperscript{[}\textsuperscript{37}\textsuperscript{]}

Desta forma, as etapas do estudo elencadas a seguir partiram da pergunta de
pesquisa: “Como têm sido construídos os critérios de pesquisa e prática
destacados na literatura sobre a educação permanente em saúde?”.

Foi desenvolvida busca ampla e exaustiva em diferentes bases de dados e áreas do
conhecimento. Optou-se por bases multidisciplinares norte-americanas (PubMed e
Web of Science), europeia (\textsc{ibecs}) e latino-americanas (Lilacs e Sci\textsc{elo}).

Como critérios de inclusão foram considerados os artigos originais publicados em
periódicos indexados entre os anos de 2000 e 2011. Este critério foi
estabelecido pelo fato de o embasamento das definições, políticas e práticas de
\textsc{eps} ter sido mais claro e disseminado nesse período. No entanto, buscou-se por
referências dos anos anteriores a 2000 em todas as bases para verificar se não
haveria publicações originais importantes a esta pesquisa, não sendo encontrados
artigos relevantes ao estudo.

Também foram incluídos estudos com amostras de pessoas de ambos os sexos,
trabalhadores de serviços de saúde ou instituições de ensino, que estabeleciam
como prática ou pesquisa a \textsc{eps} e pesquisas produzidas e discutidas de maneira
qualitativa, ou qualitativa e quantitativa concomitantemente, como forma de
gerar elementos para a análise dos dados qualitativos.

Foram excluídos os estudos que não continham como pesquisa principal educação
continuada ou permanente.

Para a seleção dos artigos foram definidas estratégias de buscas diferentes para
cada base de dados, sendo contemplados os descritores, palavras-chaves e
assuntos mais adequados utilizados nas buscas.

A primeira busca foi realizada na base PubMed. Inicialmente, foram utilizados os
descritores nos termos \textit{MeSH (Medical Subject Heading)}
que se aproximassem ao tema de pesquisa.

1. \textit{public health professional education[MeSH] \textsc{and} continuing
education[MeSH].}
Desta forma, foram encontrados 33 resultados.

2. “ \textit{permanent education}
” \textit{\textsc{and} health.}
Com resultado de 46 artigos.

3. \textit{professional education}
[ \textit{MeSH}] \textit{\textsc{and} interprofessional relations}
[ \textit{MeSH}] \textit{\textsc{and} “permanent education”.}
Com 15 resultados.

4. \textit{continuing education}
[ \textit{MeSH}] \textit{\textsc{and} “permanent education”.}
Com 27 resultados.

5. \textit{health policy}
[ \textit{MeSH}] \textit{and continuing education}
[ \textit{MeSH}]. Com 133 resultados.

O total dos resultados alcançou 254 artigos.

Na base de dados \textit{Web of Science}
foram usadas palavras “ \textit{permanent}
” \textsc{and} “ \textit{education}
” para a busca e foram encontrados 39 resultados.

Na base \textsc{ibecs} foram selecionados sete artigos provenientes da busca pelo assunto
“ \textit{permanent education}
”. Para abranger um número maior de resultados, foram utilizadas na busca as
palavras \textit{permanent}
\textsc{and} \textit{educat\$}
\textsc{and} \textit{health}, o que resultou em mais cinco artigos localizados.

Na base Lilacs foram recuperados 35 artigos pelo assunto “ \textit{permanent
education health}
”. Também foram considerados os 110 resultados provenientes da busca pelo
assunto “ \textit{continuing education health}
” para posterior seleção pelos critérios estabelecidos.

Por fim, foram pesquisados artigos no Sci\textsc{elo} pelo assunto \textit{permanent
education,}
tendo sido encontrados 88 artigos. Outra estratégia de busca foi agregar os
assuntos \textit{permanent education}
\textsc{and} \textit{professional education}, a qual recuperou 51 documentos.

Com o intuito de uniformizar e padronizar as seleções, foi usado como conceito
base de \textsc{eps} proposta construída nas noções de aprendizagem significativa e de
problematização, constituindo-se, assim, em processos educativos que buscam
promover a transformação das práticas de saúde e de educação.\textsuperscript{[}\textsuperscript{12}\textsuperscript{]}\textsuperscript{,}\textsuperscript{[}\textsuperscript{51}\textsuperscript{]}

Todos os artigos foram salvos e encaminhados a uma conta no gerenciador de
referências \textit{EndNote Web}.\footnote{\fn3}
Em seguida, os artigos foram comparados e selecionados seguindo critérios
pré-estabelecidos, primeiro pelo título, em seguida pelo resumo e, por fim,
procedeu-se a leitura do texto na íntegra e seleção da bibliografia e citação
dos autores, nas quais não foram gerados novos resultados de artigos
complementares.

Os artigos selecionados de acordo com os critérios de inclusão foram submetidos
à análise individual, realizada pela pesquisadora principal, sob supervisão de
sua orientadora, minimizando as possíveis interferências de se ter apenas um
avaliador. Em seguida, os artigos foram analisados por meio da análise
comparativa, na qual foram estabelecidos critérios para agrupamentos de
elementos-chave e submetidos à metassíntese.

Durante todas as fases de desenvolvimento do estudo foram respeitados os
preceitos da ética em pesquisa, tendo sido submetido ainda à aprovação do Comitê
de Ética em Pesquisa da Universidade Federal de São Paulo, sob o parecer
0726/11, obedecendo a norma institucional da Universidade.

\section{\textsc{discussão} \textsc{dos} \textsc{conceitos} E \textsc{práticas} DA \textsc{educação} \textsc{permanente} EM \textsc{saúde}}

Foram recuperadas 590 publicações, das quais 87 eram duplicações. Foram
excluídos: 254 artigos na análise do título; 128 artigos após análise das
palavras chave; e 73 artigos, após a leitura na íntegra, por não enquadrarem nos
objetivos da pesquisa. Desta forma, foram selecionadas e analisadas 48
publicações originais. A Tabela 1 apresenta o processo de inclusão e exclusão
dos estudos.

Tabela 1Resultado das estratégias de busca realizadas nas bases de dados
selecionadas, segundo critérios de inclusão e exclusão, entre 2000-2011.
\begin{table}
\begin{xtabular}{ l | l | l | l l | l | l | l | l }
\hline
Base de dados & Artigos encontrados & Excluídos & Incluídos\\ \hline
Duplicidade &\textsuperscript{a}
& Título & Resumo & Íntegra\\ \hline
PubMed
& 254
& 25
& 111
& 85
& 26
& 7
\\ \hline

Web of Science
& 39
& 9
& 21
& 4
& 5
& 0
\\ \hline

\textsc{ibecs}
& 12
& 6
& 5
& 1
& 0
& 0
\\ \hline

Lilacs
& 145
& 18
& 85
& 17
& 16
& 9
\\ \hline

Sci\textsc{elo}
& 140
& 29
& 32
& 21
& 26
& 32
\\ \hline

Total
& 590
& 87
& 254
& 128
& 73
& 48
\\ \hline

\end{xtabular}
\end{table}

Para a metassíntese usou-se a proposta da análise focada e constante.\textsuperscript{[}\textsuperscript{61}\textsuperscript{]}
A partir deste conceito foram estabelecidos como elementos-chave centrais de
comparação quatro unidades temáticas: 1) Concepções: principais conceitos
extraídos dos artigos acerca da \textsc{eps}; 2) Estratégias, facilidades e
dificuldades:relatos que integrem as dificuldades e possibilidades da
implantação e manutenção da \textsc{eps}; 3) Políticas públicas: identificação do quanto
a implementação das políticas públicas se aproximou da \textsc{eps} e 4) Instituições
formadoras: incorporação da \textsc{eps} às instituições, principalmente as de ensino
superior e pós-graduação e a articulação entre o ensino e o serviço.

A Tabela 2 apresenta os artigos analisados, seus principais objetivos e
elementos-chave sintetizados.

Tabela 2a Intra e Inter bases de dados.. Resultados dos artigos analisados segundo autores, ano de publicação,
objetivos e elementos-chave.

\begin{table}
\begin{xtabular}{ l | l | l }
\hline
Autoria e ano de publicação & Objetivo & Elementos-chave para metassíntese\\
\hline
Abdalla IG, Stella \textsc{rcs}, Perim GL, Aguilar-da-Silva RH, Lampert JB, Costa \textsc{nmsc}
(2009)\textsuperscript{1}

& Apresentar e analisar o eixo pedagógico do curso de graduação em medicina.
& Instituições formadoras: Experiências de instituições formadoras e articulação
entre ensino e serviço
\\ \hline

Arruda MP, Araújo MP, Locks GA, Pagliosa FL (2008)\textsuperscript{3}

& Descrever a experiência em \textsc{eps} do curso e do corpo docente de medicina da
Universidade do Planalto Catarinense
& Concepções: \textsc{eps} como prática que desenvolve profissional e pessoal; trabalho
em equipe Instituições formadoras: Experiências de instituições formadoras,
educador facilita a aprendizagem
\\ \hline

Barreto \textsc{ichc}, Andrade \textsc{lom}, Loiola F, Paula JB, Miranda AS, Goya N (2006)\textsuperscript{4}

& Descrever a implementação da \textsc{eps} em instituições de saúde-escola
& Concepções: \textsc{eps} como prática problematizadora e transformadora a partir da
equipe e da realidade do trabalho; Estratégias, facilidades e dificuldades: uso
de portfólio, conhecimento contextualizado, parcerias com ensino, comunidades,
gestores e profissionais
\\ \hline

Bogus CM, Martins CM, Dimitrov P, Fortes \textsc{pac}, Capucci PF, Nemes Filho A et al
(2003)\textsuperscript{7}

& Descrever o processo de educação permanente de conselheiros de saúde
& Políticas públicas: formulação e implantação
\\ \hline

Camps E, Calliat MC, Spalvieri M, Dante V (2003)\textsuperscript{9}

& Elaborar um programa de formação de recursos humanos na área de bioquímica que
tenha continuidade
& Instituições formadoras: experiências com metodologias ativas, melhora da
qualidade e atenção profissional
\\ \hline

Carotta F, Kawamura D, Salazar J (2009)\textsuperscript{10}

& Descrição e análise de facilitadores de educação permanente e da criação de um
polo de educação permanente em saúde
& Estratégias, facilidades e dificuldades: polo de educação permanente em saúde,
facilitadores, supervisão de equipe, reflexão crítica do processo de trabalho,
rodas de conversas, soluções coletivas
\\ \hline

Ciconet RM, Marques GQ, Lima \textsc{mads} (2008)\textsuperscript{13}

& Relatar a experiência de educação permanente com trabalhadores de um serviço
de atendimento pré-hospitalar de urgência
& Concepções: \textsc{eps} como prática problematizadora, que articula situações
vivenciadas e experiências cotidianas Estratégias, facilidades e dificuldades:
Rotina e demanda do serviço como entrave, revisão de protocolos de atendimento,
apoio a gestão do serviço
\\ \hline

Costa \textsc{ccc}, Bezerra Filho JG, Machado \textsc{mmt}, Machado \textsc{mfas}, Jorge AC, Furtado \textsc{aaa} et
al (2008)\textsuperscript{14}

& Analisar o corpo técnico do curso técnico de enfermagem na visão dos
supervisores
& Concepções: \textsc{eps} como articulação entre prática e teoria reflexivas e
problematizadoras Estratégias, facilidades e dificuldades: fortalecimento de
parcerias entre diferentes esferas Políticas Públicas: necessidade de
investimento para melhora da formação profissional Instituições formadoras:
técnicas ativas de ensino-aprendizagem
\\ \hline

Costa e Silva V, Rivera \textsc{fju}, Hortale VE (2007)\textsuperscript{15}

& Descrever experiência de integração entre serviços de saúde e analisar sua
contribuição para o desenvolvimento de práticas de cuidado integral
& Concepções: \textsc{eps} como prática reflexiva contínua pautada na realidade do
serviço e na integralidade do cuidado
\\ \hline

De Marco MA (2006)\textsuperscript{16}

& Demonstrar a implantação de modelo de atenção biopsicossocial na formação de
alunos de graduação em medicina
& Instituições formadoras: experiência em que o aluno é alvo de ações e também
agente transformador
\\ \hline

Demers AL, Marary E, Ebin VJ (2011)\textsuperscript{19}

& Descrever o desenvolvimento de colaboração para proporcionar oportunidades de
educação continuada
& Instituições formadoras: avanços tecnológicos como forma de inovar o ensino,
problematização como estratégia de ensino
\\ \hline

Dreisinger M, Leet TL, Baker EA, Gillespie KN, Haas B, Brownson RC (2008)\textsuperscript{20}

& Utilizar a saúde pública baseada em evidências para análise de práticas
profissionais
& Políticas públicas: formulação e implantação de políticas para melhorar o
raciocínio clínico e a qualidade do atendimento
\\ \hline

Feliciano \textsc{kvo}, Kovacs MH, Costa \textsc{ier}, Oliveira MG, Araújo \textsc{ams} (2008)\textsuperscript{21}

& Realizar avaliação continuada da educação permanente no âmbito da atenção à
saúde da criança
& Estratégias, facilidades e dificuldades: reflexões coletivas sobre a prática,
interdependência da autonomia profissional, número insuficiente de profissionais
dificulta trabalho em profundidade Políticas públicas: estratégias em
consonância com a prioridade conferida pelo Ministério da Saúde
\\ \hline

Fortuna CM, Franceschini, \textsc{trc}, Mishima SM, Matumoto S, Pereira \textsc{mjb} (2011)\textsuperscript{22}

& Cartografar os movimentos e efeitos desencadeados na região do \textsc{drs}-\textsc{iii} a
partir do curso de facilitadores de educação permanente
& Concepções: \textsc{eps} como rede de conversação em rodas, assumir posições
democráticas Estratégias, facilidades e dificuldades: construção da
corresponsabilidade, inclusão de facilitadores, estímulo à autonomia
\\ \hline

Horta NC, Sena RR, Silva \textsc{meo}, Oliveira SR, Rezende VA (2009)\textsuperscript{27}

& Identificar as ações de promoção da saúde predominantes entre os membros da
\textsc{esf}, identificar a existência de ações intersetoriais
& Concepções: Relacionada à educação continuada por meio de cursos e repasse de
informes Estratégias, facilidades e dificuldades: treinamentos ocorrem de forma
pontual, sobrecarga de trabalho, importância da reflexão sobre o cotidiano
\\ \hline

Jones M, Tyrer A, Kalekzi T, Lancashire S (2008)\textsuperscript{28}

& Analisar o impacto do conhecimento da prática e bem-estar de equipes de saúde
mental para prática baseada em evidências
& Estratégias, facilidades e dificuldades: trabalho em equipe, envolvimento e
colaboração terapêutica
\\ \hline

Kleba ME, Comerlatto D, Colliselli L (2007)\textsuperscript{29}

& Relatar a experiência de um curso de capacitação envolvendo gestores, trazendo
para o debate os desafios inerentes à participação social
& Concepções: metodologia ativa e problematizadora, por meio da apreensão da
realidade Instituições formadoras: experiência da instituição formadora em
aplicar conceitos da \textsc{eps} e contribuir com o empoderamento para o trabalho
\\ \hline

Lazarine CA, Francischetti I (2010)\textsuperscript{30}

& Descrever a percepção dos docentes da Unidade de Prática Profissional e dos
tutores em relação ao programa de educação permanente
& Concepções: prática problematizadora e transformadora contínua, troca de
experiência Instituições formadoras: inclusão de facilitadores para o processo
de \textsc{eps} dos docentes
\\ \hline

Lima SG, Macedo LA, Vidal ML, Sá \textsc{mpbo} (2009)\textsuperscript{31}

& Avaliar o impacto de um programa de educação permanente e treinamento em \textsc{sbv} e
\textsc{sav} e o conhecimento do profissional de enfermagem
& Estratégias, facilidades e dificuldades: profissionais relatam a importância
da implantação da \textsc{eps}, mas não há incentivo, há excessiva demanda e sobrecarga
de trabalho
\\ \hline

Maciel \textsc{eln}, Figueiredo PF, Prado TN, Galavote HS, Ramos MC, Araújo MD et al
(2010)\textsuperscript{34}

& Avaliar a contribuição do curso de pós-graduação em saúde da família e as
mudanças da prática a partir dele
& Concepções: aprimoramento pessoal, social e profissional contínuo Instituições
formadoras: experiência de formação conceitual para a mudança das práticas
clínicas que são produzidas em ambientes complexos e mutáveis de trabalho
\\ \hline

Marães \textsc{vrfs}, Martins EF, Junior GC, Acezedo AC, Pinho \textsc{dlm} (2010)\textsuperscript{35}

& Apresentar uma proposta de projeto pedagógico inovadora para um curso de
graduação
& Instituições formadoras: experiência de implantação de curso voltada para
prática humanista, crítica e reflexiva, além da aprendizagem significativa e
atuação em equipe
\\ \hline

Marques ES, Cotta MM, Franceschini \textsc{scc}, Botelho \textsc{miv}, Araújo \textsc{rma}, Junqueira TS
(2009)\textsuperscript{36}

& Identificar o significado do aleitamento materno para os profissionais que
atuam no Programa de Saúde da Família
& Estratégias, facilidades e dificuldades: falta de investimento em capacitação
e sensibilização contínua Instituições formadoras: necessidade de formar
profissionais capazes de suprir as demandas sociais
\\ \hline

Matos E, Pires \textsc{dep} (2009)\textsuperscript{38}

& Conhecer práticas assistenciais que potencializem os cuidados em saúde na
perspectiva da integralidade e do agir interdisciplinar
& Concepções: prática potencializadora da interdisciplinaridade,
problematizadora, pautada na realidade vivenciada Estratégias, facilidades e
dificuldades: reuniões de equipe, visitas multiprofissionais aos leitos, estudos
de caso, conferências com famílias, reuniões na sala de espera, tomadas de
decisões conjuntas
\\ \hline

Matumoto S, Fortuna CM, Kawata LS, Mishima SM, Pereira \textsc{mjb} (2011)\textsuperscript{39}

& Apresentar o movimento de ressignificação dos sentidos da prática clínica, na
perspectiva da clínica ampliada e educação permanente
& Concepções: análise e reflexão da prática a partir do trabalho desempenhado no
cotidiano Estratégias, facilidades e dificuldades: inexistência de apoio
técnico, necessidade da clínica do cuidado, problematização, disputas pessoais
entre trabalhadores usuários e gestores
\\ \hline

Medeiros AC, Pereira \textsc{qlc}, Siqueira \textsc{hch}, Cecagno D, Moraes CM (2010)\textsuperscript{40}

& Conhecer as estratégias de gestão construídas pelas enfermeiras, com base na
\textsc{eps}
& Concepções: transformações das práticas do cuidado por meio das experiências e
trocas vivenciadas no dia-a-dia Estratégias, facilidades e dificuldades: espaços
para troca de saberes, autonomia, construção em equipe, estratégias de gestão
\\ \hline

Montanha D, Peduzzi M (2010)\textsuperscript{44}

& Analisar o levantamento de necessidades para a implantação de atividades
educativas de trabalhadores de enfermagem e os resultados esperados
& Concepções: prática problematizadora e transformadora a partir da realidade do
trabalho; relacionada diretamente à educação continuada Estratégias, facilidades
e dificuldades: decisões coletivas e autônomas, reflexão crítica, ações
educativas voltadas para soluções de problemas pontuais
\\ \hline

Monteiro MI, Chilida \textsc{msp}, Bargas EB (2004)\textsuperscript{45}

& Contextualizar e analisar as atividades de educação continuada desenvolvidas
com trabalhadores que atuam no setor de limpeza de um hospital universitário
& Estratégias, facilidades e dificuldades: dificuldade em articular \textsc{eps} pela
ideologia do processo de trabalho, baixa escolaridade como fator que dificulta a
aprendizagem
\\ \hline

Murofuse NT, Rizzotto \textsc{mlf}, Muzzolon \textsc{abf}, Nicola AL (2009)\textsuperscript{46}

& Identificar as atividades de formação frequentadas por profissionais da rede
de saúde
& Concepções: Estratégias de problematização, democratização dos espaços de
trabalho Estratégias, facilidades e dificuldades: atividades com foco no
relacionamento pessoal, qualidade do serviço, motivação e humanização. Oferta de
cursos como forma de melhorar as condições de trabalho
\\ \hline

Nicoletto \textsc{scs}, Mendonça FF, Bueno \textsc{vlr}dC, Brevilheri \textsc{ecl}, Almeida \textsc{dcs}, Rezende
LR, Carvalho GS, González AD (2009)\textsuperscript{47}

& Analisar o processo de implantação da política de \textsc{eps} no Paraná
& Concepções: Construção coletiva a partir das demandas locais, relacionada a
mudança da prática, problematização e associada também à Educação Continuada
Estratégias, facilidades e dificuldades: rodas de conversação nos polos de \textsc{eps},
trabalho em equipe
\\ \hline

Nunes MF, Leles CR, Pereira MF, Alves, RT (2008)\textsuperscript{48}

& Acompanhar a formação de dentistas como facilitadores de \textsc{eps} nos polos de
educação permanente
& Estratégias, facilidades e dificuldades: articulação ensino-serviço,
surgimento de aquisições e competências, transformação ativa, problematização
\\ \hline

Olson D, Hoeppner M, Larson S, Ehrenberg A, Leitheiser AT (2008)\textsuperscript{49}

& Descrever um modelo de aprendizado ao longo da vida para a prática em educação
em saúde pública
& Concepções: percepção de educação ao longo da vida, aprimoramento contínuo
\\ \hline

Paschoal AS, Mantovani MF, Méier MJ (2007)\textsuperscript{50}

& Discutir a educação permanente, continuada e em serviço, junto aos enfermeiros
de um hospital de ensino
& Concepções: educação do início ao fim da vida, engloba a educação continuada e
educação em serviço
\\ \hline

Peduzzi M, Del Guerra DA, Braga CP, Lucena FS, Silva \textsc{jam} (2009)\textsuperscript{51}

& Estudar as atividades educativas de trabalhadores da rede básica na
perspectiva microssocial, com objetivo de analisar a prática de atividades
educativas de trabalhadores da saúde segundo \textsc{eps} e educação continuada
& Concepções: \textsc{eps} e educação continuada como concepções complementares
Estratégias, facilidades e dificuldades: reuniões de equipe, estratégias de
ensino participativas, integralidade no cuidado, problematização Políticas
públicas: necessidade de ampliação do debate sobre \textsc{eps}
\\ \hline

Pessanha RV, Cunha \textsc{fts} (2009)\textsuperscript{52}

& Analisar o processo de trabalho dos profissionais de enfermagem, medicina e
odontologia das equipes multiprofissionais de um módulo do Programa de Saúde da
Família
& Concepções: \textsc{eps} diretamente relacionada à Educação Continuada, mas prática
problematizadora Estratégias, facilidades e dificuldades: corresponsabilidade,
aprendizagem-trabalho, trabalho em equipe, autonomia
\\ \hline

Ricaldoni \textsc{cac}, Sena RR (2006)\textsuperscript{53}

& Analisar os efeitos das ações de educação permanente na qualidade de
assistência de enfermagem
& Concepções: pedagogia da problematização como transformadora da realidade
Estratégias, facilidades e dificuldades: estímulo à reflexão da prática, no
entanto há desarticulação na compreensão do porque se faz, o que prejudica a
qualidade do cuidado
\\ \hline

Robinson S, Murrells T, Smith EM (2005)\textsuperscript{54}

& Descrever e analisar profissionais com curso superior baseado em qualificação
em saúde mental
& Instituições formadoras: experiências na carreira relatadas como ligadas a
qualidade dos estágios e da preceptoria
\\ \hline

Rodrigues \textsc{acs}, Vieira \textsc{glc}, Torres HC (2010)\textsuperscript{55}

& Relatar a experiência do processo de \textsc{eps} por meio de oficinas educativas em
diabetes
& Estratégias, facilidades e dificuldades: oficinas educativas, estudo de caso,
jogos educativos, problematizações sobre o despreparo da equipe, integralidade.
Alta demanda impele a práticas prescritivas
\\ \hline

Rodrigues \textsc{rrj} (2001)\textsuperscript{56}

& Criar um espaço para discutir coletivamente com funcionários temas que
pudessem contribuir para o desenvolvimento interpessoal
& Concepções: aprimoramento pessoal, profissional e pessoal contínuo, a fim de
incrementar o desenvolvimento interpessoal Estratégias, facilidades e
dificuldades: estabelecimento de vínculo, autoconhecimento, relacionamento
interpessoal
\\ \hline

Rossetto M, Silva \textsc{laa} (2010)\textsuperscript{58}

& Conhecer as ações de \textsc{eps} com os agentes comunitários de saúde
& Estratégias, facilidades e dificuldades: cursos realizados a partir das
demandas internas, atualmente atividades descontínuas, tecnicistas e
assistemáticas Políticas Públicas: necessidade de formulação e implantação de
políticas de \textsc{eps} para ampliação do debate
\\ \hline

Smeke \textsc{elm}, Oliveira \textsc{nls} (2009)\textsuperscript{63}

& Avaliar as práticas educativas em saúde desenvolvidas em centros do \textsc{sus},
durante implantação do Paidéia Saúde da Família
& Concepções: Construção compartilhada de conceitos, reflexão ético-política dos
significados Estratégias, facilidades e dificuldades: responsabilidade
cogestora, ressignificação das ações, pressão da demanda dificulta implantação
de grupos educativos
\\ \hline

Sousa MF, Merchán-Hamann E (2009)\textsuperscript{64}

& Análise da implantação do programa de saúde da família sob as dimensões
política, técnico-financeira e simbólica
& Estratégias, facilidades e dificuldades: polos de educação permanente, oferta
de cursos de atualização, ampliação de parcerias inter setoriais Políticas
Públicas: ampliação do debate sobre polos de \textsc{eps} para retorno de sua articulação
\\ \hline

Souza \textsc{rcr}, Soares E, Souza IA, Oliveira JC, Salles RS, Cordeiro \textsc{cem} (2010)\textsuperscript{65}

& Investigar as demandas dos usuários à ouvidoria relacionadas à assistência em
enfermagem e discutir sua contribuição para a educação permanente
& Estratégias, facilidades e dificuldades: importância da ouvidoria para o
aprimoramento professional, processo avaliativo, problematização, conhecimento
baseado na cotidiano local
\\ \hline

Sudan \textsc{lcp}, Corrêa AC (2008)\textsuperscript{67}

& Apreender os significados atribuídos pelos egressos do curso de enfermagem, as
experiências vivenciadas na realização de atividades educativas junto aos
trabalhadores, nos serviços de saúde e estágio supervisionado
& Instituições formadoras: atividades educativas articuladas com o serviço em
saúde
\\ \hline

Tanji S, Silva \textsc{cmslmd} (2010)\textsuperscript{68}

& Identificar os conflitos encontrados no cotidiano do trabalho no decorrer de
um curso de especialização
& Concepções: Integração entre ensino-trabalho, sincronizando profissionais,
alunos e docentes, problematização e relacionada à educação continuada
Estratégias, facilidades e dificuldades: metodologias ativas, práticas em
consonância com o nível individual e coletivo Instituições formadoras: Integrar
ensino-trabalho-cidadania na perspectiva da problematização
\\ \hline

Tavares \textsc{cmm} (2006)\textsuperscript{69}

& Analisar a necessidade de educação permanente da equipe de enfermagem para o
cuidado nos serviços de saúde mental
& Concepções: \textsc{eps} relacionada à Educação Continuada Estratégias, facilidades e
dificuldades: a rede não dispõe de \textsc{eps} problematizadora, são as buscas
individuais que movem escolhas e treinamentos
\\ \hline

Tronchin \textsc{dmr}, Mira VL, Peduzzi M, Ciampone \textsc{mht}, Melleiro MM, Silva \textsc{jam}, et al
(2009)\textsuperscript{70}

& Identificar, caracterizar e analisar as práticas educativas desenvolvidas com
profissionais de saúde em hospitais
& Estratégias, facilidades e dificuldades: escasso debate em torno da atenção
integral, estratégias de ensino tradicionais atividades com origem nas demandas
internas, problematização
\\ \hline

Ximenes Neto \textsc{frg}, Sampaio \textsc{jjc} (2007)\textsuperscript{72}

& Elaborar o perfil sociodemográfico e educacional dos gerentes de território da
\textsc{esf}, identificar os tipos de qualificação e de educação permanente
& Concepções: Ressignificação do processo de trabalho pela prática no território
e em serviço Estratégias, facilidades e dificuldades: articulação
ensino-serviço, pouco investimento em aprimoramento profissional
\\ \hline

Yaping D, Stanton P (2002)\textsuperscript{73}

& Aprimorar e descrever programa de treinamento de gerenciamento em saúde na
China
& Instituições formadoras: articulação ensino-serviço com abordagem integrada,
necessidade de melhora na participação dos alunos
\\ \hline

\end{xtabular}
\end{table}

\textsc{drs}-\textsc{iii}: Departamento Regional de Saúde \textsc{iii}; \textsc{eps}: Educação permanente em saúde;
\textsc{sbv}: Suporte Básico de Vida; \textsc{sav}: Suporte Avançado de Vida. Resultados dos artigos analisados segundo autores, ano de publicação,
objetivos e elementos-chave.

\textsc{drs}-\textsc{iii}: Departamento Regional de Saúde \textsc{iii}; \textsc{eps}: Educação permanente em saúde;
\textsc{sbv}: Suporte Básico de Vida; \textsc{sav}: Suporte Avançado de Vida

Tais concepções abrangeram a prática transformadora e problematizadora, bem como
a relação com a educação continuada e educação ao longo da vida. Tais conceitos
são baseados nas premissas das políticas públicas e das mudanças históricas da
maneira de lidar e reconhecer a educação profissional de adultos em serviço.\footnote{\fn4}
Os artigos que fundamentam a pesquisa na compreensão da \textsc{eps} como prática
transformadora e problematizadora, pautada na realidade dos serviços e que
promove integração entre o universo do ensino e do trabalho, apresentaram como
principal população do estudo os profissionais dos serviços de saúde e
identificaram que a “participação em atividades de formação constitui-se numa
forma de democratização nas relações institucionais e pode ser estratégica para
a recomposição das relações entre a população, os trabalhadores e os gestores”.\textsuperscript{[}\textsuperscript{51}\textsuperscript{]}

Aspecto recorrente foi a íntima relação entre \textsc{eps} e trabalho em equipe multi ou
interdisciplinar,\textsuperscript{[}\textsuperscript{22}\textsuperscript{]}\textsuperscript{,}\textsuperscript{[}\textsuperscript{38}\textsuperscript{]}\textsuperscript{,}\textsuperscript{[}\textsuperscript{39}\textsuperscript{]}\textsuperscript{,}\textsuperscript{[}\textsuperscript{44}\textsuperscript{]}\textsuperscript{,}\textsuperscript{[}\textsuperscript{46}\textsuperscript{]}\textsuperscript{,}\textsuperscript{[}\textsuperscript{47}\textsuperscript{]}\textsuperscript{,}\footnote{\fn4}
articulando os processos de trabalho para corresponder às necessidades de saúde
da população.

Desta forma, compreende-se e discute-se a proposta política da \textsc{eps} como
construção compartilhada de conceitos\textsuperscript{[}\textsuperscript{15}\textsuperscript{]}\textsuperscript{,}\textsuperscript{[}\textsuperscript{63}\textsuperscript{]}
que supera a cultura organizacional baseada na centralidade de decisões.\textsuperscript{[}\textsuperscript{46}\textsuperscript{]}
Adicionalmente, pressupõe uma organização com rede de relações tecida por todos
os participantes por meio das ideias, necessidades e sentimentos presentes nas
interações sociais, o que se reflete nas percepções e vivências da realidade.\textsuperscript{[}\textsuperscript{56}\textsuperscript{]}

Os espaços coletivos construídos para trocas de saberes, reflexões e avaliações
foram descritos como caminhos para o delineamento de novos modos de produção do
cuidado\textsuperscript{[}\textsuperscript{40}\textsuperscript{]}
que exige a apreensão da realidade, não para a adaptação a ela, mas para nela
intervir.\textsuperscript{[}\textsuperscript{29}\textsuperscript{]}

A problematização da prática é a compreensão que a aprendizagem se realiza na
ação-reflexão-ação\textsuperscript{[}\textsuperscript{4}\textsuperscript{]}\textsuperscript{,}\textsuperscript{[}\textsuperscript{51}\textsuperscript{]}\textsuperscript{,}\textsuperscript{[}\textsuperscript{72}\textsuperscript{]}
caracterizada pelo compromisso e auto-implicação\textsuperscript{[}\textsuperscript{30}\textsuperscript{]}
fundamentada a partir do conhecimento dos participantes e da aprendizagem
significativa, por sua prática ser desenvolvida em serviço e por haver a
apropriação efetiva do território.\textsuperscript{[}\textsuperscript{29}\textsuperscript{]}

Outro elemento-chave presente em sete publicações\textsuperscript{[}\textsuperscript{27}\textsuperscript{]}\textsuperscript{,}\textsuperscript{[}\textsuperscript{44}\textsuperscript{]}\textsuperscript{,}\textsuperscript{[}\textsuperscript{47}\textsuperscript{]}\textsuperscript{,}\textsuperscript{[}\textsuperscript{50}\textsuperscript{]}\textsuperscript{,}\textsuperscript{[}\textsuperscript{51}\textsuperscript{]}\textsuperscript{,}\textsuperscript{[}\textsuperscript{6}\textsuperscript{]}\textsuperscript{[}\textsuperscript{8}\textsuperscript{]}\textsuperscript{,}\textsuperscript{[}\textsuperscript{69}\textsuperscript{]}
foi a \textsc{eps} relacionada diretamente à educação continuada, baseada em ações de
caráter pontual, fragmentadas, com metodologias tradicionais de ensino. Nesses
artigos os conceitos foram usados na prática como sinônimos.\textsuperscript{[}\textsuperscript{70}\textsuperscript{]}

Observou-se por vezes que as populações estudadas de profissionais de serviços
de saúde narraram a realização de cursos, reuniões de equipe para repasse de
informações administrativas e capacitações específicas como sendo conceitos e
práticas de \textsc{eps}.\textsuperscript{[}\textsuperscript{13}\textsuperscript{]}\textsuperscript{,}\textsuperscript{[}\textsuperscript{26}\textsuperscript{]}\textsuperscript{,}\textsuperscript{[}\textsuperscript{47}\textsuperscript{]}
O que se preconizava em alguns serviços era a aquisição de competências
profissionais, importava não só a posse dos saberes disciplinares ou
técnico-profissionais, mas a capacidade de mobilizá-los para enfrentar os
imprevistos na situação de trabalho. A aquisição de competências parecia
remeter, prioritariamente, às características individuais dos trabalhadores\textsuperscript{[}\textsuperscript{18}\textsuperscript{]}
não privilegiando o trabalho em equipe e a problematização coletiva como foco da
aprendizagem.

Outros estudos corroboram que a \textsc{eps} engloba a educação continuada. Nesse caso os
estudos revelam conceitos entendidos separadamente, no entanto, o que ocorre é o
predomínio de ações ligadas à educação continuada, porém com possibilidades de
ações com novos formatos, conteúdos e sentidos.\textsuperscript{[}\textsuperscript{44}\textsuperscript{]}\textsuperscript{,}\textsuperscript{[}\textsuperscript{47}\textsuperscript{]}\textsuperscript{,}\textsuperscript{[}\textsuperscript{50}\textsuperscript{]}\textsuperscript{,}\textsuperscript{[}\textsuperscript{58}\textsuperscript{]}
Ou seja, há a intencionalidade no discurso de se produzir \textsc{eps}, mas a prática
permanece restrita à educação continuada.

O terceiro elemento central identificado foi a concepção também vigente de \textsc{eps}
como educação ao longo da vida, por meio de ressignificação do desenvolvimento
pessoal e interpessoal contínuo, concebendo o aprendizado no trabalho vai além
da dimensão técnica.\textsuperscript{[}\textsuperscript{56}\textsuperscript{]}

Esta ideia começou a ser discutida nos anos 1920, mas foi em 1966 na Conferência
Geral da \textsc{unesco}, que se definiu como objetivo prioritário para alavancar a
educação contínua ou por toda a vida.\textsuperscript{[}\textsuperscript{24}\textsuperscript{]}

Desta maneira, os três artigos que trazem essa referência de educação ao longo
da vida discutem principalmente o autoaprimoramento contínuo na busca de
competência pessoal, profissional e pessoal\textsuperscript{[}\textsuperscript{34}\textsuperscript{]}\textsuperscript{,}\textsuperscript{[}\textsuperscript{50}\textsuperscript{]}
e pouco consideram as situações de problematização do trabalho para
transformação da realidade.

Esse elemento central pode ser apreendido em relação às práticas do cotidiano
dos serviços e instituições sobre o tema.

Na perspectiva de 25 estudos analisados,\textsuperscript{[}\textsuperscript{4}\textsuperscript{]}\textsuperscript{,}\textsuperscript{[}\textsuperscript{10}\textsuperscript{]}\textsuperscript{,}\textsuperscript{[}\textsuperscript{13}\textsuperscript{]}\textsuperscript{,}\textsuperscript{[}\textsuperscript{14}\textsuperscript{]}\textsuperscript{,}\textsuperscript{[}\textsuperscript{21}\textsuperscript{]}\textsuperscript{,}\textsuperscript{[}\textsuperscript{2}\textsuperscript{]}\textsuperscript{[}\textsuperscript{2}\textsuperscript{]}\textsuperscript{,}\textsuperscript{[}\textsuperscript{38}\textsuperscript{]}\textsuperscript{,}\textsuperscript{[}\textsuperscript{39}\textsuperscript{]}\textsuperscript{,}\textsuperscript{[}\textsuperscript{40}\textsuperscript{]}\textsuperscript{,}\textsuperscript{[}\textsuperscript{44}\textsuperscript{]}\textsuperscript{,}\textsuperscript{[}\textsuperscript{46}\textsuperscript{]}\textsuperscript{-}\textsuperscript{[}\textsuperscript{48}\textsuperscript{]}\textsuperscript{,}\textsuperscript{[}\textsuperscript{51}\textsuperscript{]}\textsuperscript{-}\textsuperscript{[}\textsuperscript{53}\textsuperscript{]}\textsuperscript{,}\textsuperscript{[}\textsuperscript{55}\textsuperscript{]}\textsuperscript{,}\textsuperscript{[}\textsuperscript{58}\textsuperscript{]}\textsuperscript{,}\textsuperscript{[}\textsuperscript{63}\textsuperscript{]}\textsuperscript{-}\textsuperscript{[}\textsuperscript{65}\textsuperscript{]}\textsuperscript{,}\textsuperscript{[}\textsuperscript{68}\textsuperscript{]}\textsuperscript{-}\textsuperscript{[}\textsuperscript{70}\textsuperscript{]}\textsuperscript{,}\textsuperscript{[}\textsuperscript{72}\textsuperscript{]}
as \textsc{eps} foram as ações e decisões coletivas fundamentadas em práticas
problematizadoras que têm por base a aprendizagem deslocada para o ambiente de
serviço, concebendo a ação-reflexão-ação como foco norteador.\textsuperscript{[}\textsuperscript{55}\textsuperscript{]}
A problematização encontra nas formulações de Paulo Freire um sentido de
inserção crítica na realidade para dela retirar os elementos que atribuirão
significado às aprendizagens e levar em conta as implicações pessoais e as
interações entre os diferentes sujeitos que aprendem e ensinam.\textsuperscript{[}\textsuperscript{5}\textsuperscript{]}
A construção do conhecimento e aprendizagem significativa enquadraram-se como
traço definidor da apropriação de informações e explicação da realidade.\textsuperscript{[}\textsuperscript{6}\textsuperscript{]}

A partir disso, são apreendidas e discutidas informações dos profissionais de
situações reais do trabalho como processo de formação permanente, no qual
conhecimentos teóricos, práticos e contextualizados são abordados em toda sua
complexidade.\textsuperscript{[}\textsuperscript{17}\textsuperscript{]}\textsuperscript{,}\textsuperscript{[}\textsuperscript{51}\textsuperscript{]}

Nos serviços de saúde estudados, por intermédio dos artigos, os participantes
indicaram a problematização como fundamental no aprendizado e nas relações do
trabalho, pois fornece respaldo na prática de forma contínua e articulada.\textsuperscript{[}\textsuperscript{11}\textsuperscript{]}

Esse processo só foi possível quando esteve interligado à problematização e à
articulação em equipe,\textsuperscript{[}\textsuperscript{47}\textsuperscript{]}\textsuperscript{,}\textsuperscript{[}\textsuperscript{51}\textsuperscript{]}\textsuperscript{,}\textsuperscript{[}\textsuperscript{55}\textsuperscript{]}\textsuperscript{,}\textsuperscript{[}\textsuperscript{68}\textsuperscript{]}
necessárias para a desconstrução do modelo assistencial vigente
hospitalocêntrico.\textsuperscript{[}\textsuperscript{69}\textsuperscript{]}
Assim, a tomada de decisão coletiva era vista como forma de superar
dificuldades, as responsabilidades eram divididas e co-gestões assumidas como
forma de facilitar o processo. Adicionalmente, havia assunção de que
interdependência da autonomia profissional manteria melhor relacionamento
grupal.\textsuperscript{[}\textsuperscript{21}\textsuperscript{]}\textsuperscript{,}\textsuperscript{[}\textsuperscript{27}\textsuperscript{]}\textsuperscript{,}\textsuperscript{[}\textsuperscript{38}\textsuperscript{]}\textsuperscript{,}\textsuperscript{[}\textsuperscript{63}\textsuperscript{]}

Outra prática estabelecida nos setores de saúde e educação foi a capacitação.
Doze estudos indicaram a importância da criação e manutenção dos espaços de \textsc{eps}\textsuperscript{[}\textsuperscript{39}\textsuperscript{]}
por meio de planejamento coletivo e desenvolvimento de treinamentos baseados em
discussões problematizadoras\textsuperscript{[}\textsuperscript{36}\textsuperscript{]}\textsuperscript{,}\textsuperscript{[}\textsuperscript{40}\textsuperscript{]}
acerca das demandas do território e dos profissionais e população ali inseridos.
Esse procedimento facilitou a atualização técnica-científica,\textsuperscript{[}\textsuperscript{48}\textsuperscript{]}
a construção do trabalho em equipe e a comunicação.\textsuperscript{[}\textsuperscript{40}\textsuperscript{]}
Também foi enfatizado que as ofertas de cursos tradicionais que não consideram a
aprendizagem-trabalho, nem o contexto do local, não surtem efeito no cotidiano
dos serviços.\textsuperscript{[}\textsuperscript{11}\textsuperscript{]}\textsuperscript{,}\textsuperscript{[}\textsuperscript{15}\textsuperscript{]}\textsuperscript{,}\textsuperscript{[}\textsuperscript{47}\textsuperscript{]}\textsuperscript{,}\textsuperscript{[}\textsuperscript{64}\textsuperscript{]}

Dentre as práticas que facilitaram a implementação e gerenciamento da \textsc{eps} estão
os polos de \textsc{eps} instituídos como política pública, como espaços de diálogo e de
negociação entre os atores das ações e serviços do \textsc{sus} e instituições
formadoras.\textsuperscript{[}\textsuperscript{10}\textsuperscript{]}\textsuperscript{,}\textsuperscript{[}\textsuperscript{25}\textsuperscript{]}\textsuperscript{,}\textsuperscript{[}\textsuperscript{47}\textsuperscript{]}\textsuperscript{,}\textsuperscript{[}\textsuperscript{48}\textsuperscript{]}\textsuperscript{,}\textsuperscript{[}\textsuperscript{64}\textsuperscript{]}\textsuperscript{,}\footnote{\fn5}
Os polos constituíram espaços para a identificação de necessidades e construção
de estratégias e de políticas no campo da formação e desenvolvimento, na
perspectiva de ampliação da qualidade da gestão, da qualidade e do
aperfeiçoamento da atenção integral à saúde, do domínio popularizado do conceito
ampliado de saúde e do fortalecimento do controle social no \textsc{sus}.\footnote{\fn6}

O principal benefício dessas práticas nos serviços está ligado à existência de
diálogo em rodas de conversa, constituídas por grupos de discussão formados por
profissionais de instuições de saúde e facilitadores dos polos, com afirmações
positivas relacionadas ao comprometimento com o trabalho, fortalecimento da
integração ensino-serviço, preparando o profissional por meio do desenvolvimento
da capacidade crítica, criativa e postura pró-ativa.\textsuperscript{[}\textsuperscript{47}\textsuperscript{]}\textsuperscript{,}\textsuperscript{[}\textsuperscript{48}\textsuperscript{]}\textsuperscript{,}\textsuperscript{[}\textsuperscript{55}\textsuperscript{]}

Entre os anos de 2005 e 2006 foram instalados no Brasil 96 polos de \textsc{eps},
contando com a participação de mais de 1.400 atores institucionais,
profissionais de saúde, gestores e educadores.\textsuperscript{[}\textsuperscript{43}\textsuperscript{]}
No entanto, a falta de continuidade nos investimentos, o pouco comprometimento
de gestores, a dificuldade de construir uma dinâmica de trabalho ágil e disputas
por poder resultaram em redução e constituição mais lenta dos espaços.\textsuperscript{[}\textsuperscript{22}\textsuperscript{]}\textsuperscript{,}\textsuperscript{[}\textsuperscript{41}\textsuperscript{]}\textsuperscript{,}\textsuperscript{[}\textsuperscript{48}\textsuperscript{]}
Esse fato pode ser observado pela constatação de que apenas sete\textsuperscript{[}\textsuperscript{10}\textsuperscript{]}\textsuperscript{,}\textsuperscript{[}\textsuperscript{22}\textsuperscript{]}\textsuperscript{,}\textsuperscript{[}\textsuperscript{25}\textsuperscript{]}\textsuperscript{,}\textsuperscript{[}\textsuperscript{47}\textsuperscript{]}\textsuperscript{,}\textsuperscript{[}\textsuperscript{48}\textsuperscript{]}\textsuperscript{,}\textsuperscript{[}\textsuperscript{64}\textsuperscript{]}\textsuperscript{,}\footnote{\fn6}
dos 48 artigos discutiram polos de educação permantente.

Os polos estiveram relacionados à integração entre
universidade-serviço-comunidade\textsuperscript{[}\textsuperscript{4}\textsuperscript{]}\textsuperscript{,}\textsuperscript{[}\textsuperscript{10}\textsuperscript{]}\textsuperscript{,}\textsuperscript{[}\textsuperscript{22}\textsuperscript{]}\textsuperscript{,}\textsuperscript{[}\textsuperscript{55}\textsuperscript{]}\textsuperscript{,}\textsuperscript{[}\textsuperscript{64}\textsuperscript{]}
como forma de apropriação da realidade local, além de fornecer subsídios para
uma prática crítica-reflexiva no cotidiano mutável dos serviços de saúde.\textsuperscript{[}\textsuperscript{32}\textsuperscript{]}
Para que essa integração fosse possível, discutiu-se enfaticamente a necessidade
de investimento em novas tecnologias como ensino à distância, informatização dos
sistemas e inovação pedagógica.\textsuperscript{[}\textsuperscript{4}\textsuperscript{]}\textsuperscript{,}\textsuperscript{[}\textsuperscript{19}\textsuperscript{]}

A discussão sobre as dificuldades de implantação e gerenciamento da \textsc{eps} nos
serviços indicou como desafio na articulação ensino-trabalho-cidadania: baixa
disponibilidade de profissionais ou sua alta rotatividade nos setores,
distribuição irregular com grande concentração em centros urbanos e regiões mais
desenvolvidas, crescente especialização e dependência de tecnologias mais
sofisticadas, predomínio da formação hospitalar, conceitos imprecisos de
integralidade e promoção da saúde e cisão nas equipes em relação a treinamentos,
capacitações e reuniões, apresentam-se.\textsuperscript{[}\textsuperscript{4}\textsuperscript{]}\textsuperscript{,}\textsuperscript{[}\textsuperscript{19}\textsuperscript{]}\textsuperscript{,}\textsuperscript{[}\textsuperscript{25}\textsuperscript{]}\textsuperscript{,}\textsuperscript{[}\textsuperscript{38}\textsuperscript{]}\textsuperscript{,}\textsuperscript{[}\textsuperscript{47}\textsuperscript{]}\textsuperscript{,}\textsuperscript{[}\textsuperscript{56}\textsuperscript{]}\textsuperscript{,}\textsuperscript{[}\textsuperscript{68}\textsuperscript{]}\textsuperscript{,}\textsuperscript{[}\textsuperscript{72}\textsuperscript{]}

Tradicionalmente, fala-se da formação como se os trabalhadores pudessem ser
administrados como um dos componentes de um espectro de recursos materiais,
financeiros, infraestruturais, entre outros, como se fosse possível apenas
“prescrever” habilidades, comportamentos e perfis aos trabalhadores do setor
para que as ações e os serviços sejam implementados com a qualidade desejada. As
prescrições de trabalho, entretanto, não podem ser consideradas sinônimo de
trabalho realizado, pois a prática dos serviços muitas vezes pode ser diferente
do que está previsto e prescrito na teoria.\footnote{\fn7}
O fator limitante mais expressivo para implementação da \textsc{eps}, na análise de 15
artigos,\textsuperscript{[}\textsuperscript{22}\textsuperscript{]}\textsuperscript{,}\textsuperscript{[}\textsuperscript{27}\textsuperscript{]}\textsuperscript{,}\textsuperscript{[}\textsuperscript{31}\textsuperscript{]}\textsuperscript{,}\textsuperscript{[}\textsuperscript{39}\textsuperscript{]}\textsuperscript{,}\textsuperscript{[}\textsuperscript{40}\textsuperscript{]}\textsuperscript{,}\textsuperscript{[}\textsuperscript{45}\textsuperscript{]}\textsuperscript{-}\textsuperscript{[}\textsuperscript{47}\textsuperscript{]}\textsuperscript{,}\textsuperscript{[}\textsuperscript{51}\textsuperscript{]}\textsuperscript{,}\textsuperscript{[}\textsuperscript{56}\textsuperscript{]}\textsuperscript{,}\textsuperscript{[}\textsuperscript{58}\textsuperscript{]}\textsuperscript{,}\textsuperscript{[}\textsuperscript{64}\textsuperscript{]}\textsuperscript{,}\textsuperscript{[}\textsuperscript{69}\textsuperscript{]}\textsuperscript{,}\textsuperscript{[}\textsuperscript{70}\textsuperscript{]}\textsuperscript{,}\textsuperscript{[}\textsuperscript{72}\textsuperscript{]}
foi relacionado às gerências e gestões pelo escasso debate em torno da atenção
integral, pressionados pela demanda dos serviços, limitações pedagógicas e de
recursos.\textsuperscript{[}\textsuperscript{34}\textsuperscript{]}\textsuperscript{,}\textsuperscript{[}\textsuperscript{5}\textsuperscript{]}\textsuperscript{[}\textsuperscript{3}\textsuperscript{]}\textsuperscript{,}\textsuperscript{[}\textsuperscript{69}\textsuperscript{]}\textsuperscript{,}\textsuperscript{[}\textsuperscript{72}\textsuperscript{]}
A pouca articulação das diversas gerências responsáveis pelo mesmo programa, em
sua compartimentalização por categorias profissionais foi atribuída ao fato de
os profissionais, nunca ou quase nunca atualizados, participarem do seu
planejamento.\textsuperscript{[}\textsuperscript{71}\textsuperscript{]}

Além disso, a falta de articulação entre ensino-serviço-comunidade mostrou ser
um ponto importante para a não concretização do processo de \textsc{eps}, pois não
efetiva o planejamento de ações e define as necessidades de ações de modo
aleatório.\textsuperscript{[}\textsuperscript{25}\textsuperscript{]}

Assim, as estratégias, facilidades e dificuldades relacionadas à \textsc{eps} se
entrelaçam a partir das categorias apresentadas e possibilitam a reflexão
coletiva sobre o trabalho no \textsc{sus}.\textsuperscript{[}\textsuperscript{43}\textsuperscript{]}
E aí está o cerne de um grande desafio: produzir questionamentos no agir do
cuidado e colocar-se ético-politicamente em discussão, no plano individual e
coletivo, do trabalho.\textsuperscript{[}\textsuperscript{42}\textsuperscript{]}

Processos conflituosos fazem parte do cotidiano e aprender a enfrentá-los é uma
forma de ampliar a capacidade de análise sobre si, os outros e o contexto,
aumentando, por consequência, a possibilidade de agir sobre estas situações.
Assim, os conflitos “trazem consigo a possibilidade de inclusão e produção da
mudança, movendo as pessoas do lugar da conservação para o lugar da
transformação”.\textsuperscript{[}\textsuperscript{42}\textsuperscript{]}

Essa transformação não envolve apenas a pedagogia e processos de ensino e
aprendizagem, mas também uma profunda incorporação crítica de tecnologias
materiais, como a eficácia da clínica produzida, os padrões de escuta, as
relações estabelecidas com os usuários e entre os profissionais.\textsuperscript{[}\textsuperscript{12}\textsuperscript{]}

As principais categorias deste elemento-chave são a formulação e implantação das
políticas públicas e necessidade de ampliação dos debates sobre o tema.

A articulação de educação e trabalho deve orientar a formação e a gestão,
comprometidas não apenas com a qualidade da técnica, mas conjugadas às
necessidades da população.\textsuperscript{[}\textsuperscript{71}\textsuperscript{]}
O debate principal está na necessidade de formulação ou reformulação e
implantação de políticas. Assim, o primeiro passo é a construção de um
diagnóstico nacional do problema,\textsuperscript{[}\textsuperscript{48}\textsuperscript{]}
identificando as falhas de gestão e gerência e a não consonância com as
indicações do Ministério da Saúde.\textsuperscript{[}\textsuperscript{21}\textsuperscript{]}

Além disso, novas políticas precisam ser construídas coletivamente, com foco nos
sujeitos envolvidos, professores, estudantes, usuários, profissionais, gestores
e comunidade.\textsuperscript{[}\textsuperscript{6}\textsuperscript{]}

Dado imprescindível, na perspectiva de oito estudos\textsuperscript{[}\textsuperscript{7}\textsuperscript{]}\textsuperscript{,}\textsuperscript{[}\textsuperscript{20}\textsuperscript{]}\textsuperscript{,}\textsuperscript{[}\textsuperscript{21}\textsuperscript{]}\textsuperscript{,}\textsuperscript{[}\textsuperscript{4}\textsuperscript{]}\textsuperscript{[}\textsuperscript{6}\textsuperscript{]}\textsuperscript{,}\textsuperscript{[}\textsuperscript{48}\textsuperscript{]}\textsuperscript{,}\textsuperscript{[}\textsuperscript{49}\textsuperscript{]}\textsuperscript{,}\textsuperscript{[}\textsuperscript{51}\textsuperscript{]}\textsuperscript{,}\textsuperscript{[}\textsuperscript{58}\textsuperscript{]}
analisados, foi a disponibilização de recursos financeiros e humanos para a
operacionalização do trabalho, uma relação custo/benefício muito alta resultaria
na exaustão desses recursos. Por outra via, a dificuldade de se praticar a
integralidade no cuidado é um componente que favorece o corporativismo e o
privilégio aos hospitais em detrimento da atenção básica nas políticas públicas.\textsuperscript{[}\textsuperscript{70}\textsuperscript{]}

Dessa combinação resultam, muitas vezes, programas fragilizados em sua
estrutura, aplicados por profissionais não integrados, pouco conscientes do
objetivo geral do projeto e geralmente interessados em aspectos técnicos
específicos.\textsuperscript{[}\textsuperscript{71}\textsuperscript{]}

Iniciativas importantes foram tomadas a partir da criação pelo Ministério da
Saúde, Secretaria de Gestão do Trabalho e da Educação na Saúde (\textsc{sgtes}), com os
Departamentos de Gestão da Educação e de Regulação do Trabalho, em 2003. Dessa
forma, a instituição assumiu com mais clareza seu papel, atuando na formulação e
execução das políticas de Recursos Humanos para o Sistema Único de Saúde.\footnote{\fn8}

A partir daí foram desencadeadas a Política Nacional de Educação Permanente
(2003); a Mesa Nacional de Negociação Permanente do \textsc{sus} (2003); o Programa
Nacional de Reorientação Profissional (Pró-saúde, 2005); as Diretrizes Nacionais
para a elaboração dos Planos de Carreira, Cargos e Salários dos trabalhadores do
\textsc{sus} (2004); a Rede Nacional de Saúde do Trabalhador (2005); a elaboração da
Norma Operacional Básica de Recursos Humanos (\textsc{nob}/RH-\textsc{sus}, 2005).\textsuperscript{[}\textsuperscript{70}\textsuperscript{]}\textsuperscript{,}\footnote{\fn8}

No entanto, observamos a necessidade de articular as experiências e
transformações da \textsc{eps} nos serviços com as mudanças estruturais e pedagógicas das
instituições de ensino e formação.

As instituições de ensino, da maneira como foram organizadas, em sua maioria
desarticuladas da rede de atenção, privavam a capacidade educativa de outros
cenários, sobretudo o serviço. Considerando que todas as instituições têm um
efeito educativo secundário que se agrega à formação inicial do profissional, é
imprescindível a articulação ensino-serviço, pois o saber posterior à formação
escolar do trabalhador da saúde se aprende privilegiadamente pelo trabalho.\textsuperscript{[}\textsuperscript{66}\textsuperscript{]}

Esse elemento-chave compreende, na análise de 12 artigos,\textsuperscript{[}\textsuperscript{1}\textsuperscript{]}\textsuperscript{,}\textsuperscript{[}\textsuperscript{10}\textsuperscript{]}\textsuperscript{,}\textsuperscript{[}\textsuperscript{14}\textsuperscript{]}\textsuperscript{,}\textsuperscript{[}\textsuperscript{16}\textsuperscript{]}\textsuperscript{,}\textsuperscript{[}\textsuperscript{19}\textsuperscript{]}\textsuperscript{,}\textsuperscript{[}\textsuperscript{29}\textsuperscript{]}\textsuperscript{,}\textsuperscript{[}\textsuperscript{30}\textsuperscript{]}\textsuperscript{,}\textsuperscript{[}\textsuperscript{34}\textsuperscript{]}\textsuperscript{,}\textsuperscript{[}\textsuperscript{35}\textsuperscript{]}\textsuperscript{,}\textsuperscript{[}\textsuperscript{67}\textsuperscript{]}\textsuperscript{,}\textsuperscript{[}\textsuperscript{68}\textsuperscript{]}\textsuperscript{,}\textsuperscript{[}\textsuperscript{73}\textsuperscript{]}
que as experiências das instituições de ensino partem da problematização e da
construção voltada para a \textsc{eps}, entendendo o espaço de encontro para formação de
profissionais interdisciplinares e críticos-reflexivos.\textsuperscript{[}\textsuperscript{2}\textsuperscript{]}

A principal noção observada foi a reflexão das práticas nos contextos reais dos
serviços, em especial nas instituições de ensino em nível de graduação e
pós-graduação.\textsuperscript{[}\textsuperscript{12}\textsuperscript{]}
A pedagogia problematizadora no processo de ensino foi indicada como instrumento
que permite o respaldo na prática e orienta os alunos sobre o universo do
trabalho.\textsuperscript{[}\textsuperscript{3}\textsuperscript{]}\textsuperscript{,}\textsuperscript{[}\textsuperscript{8}\textsuperscript{]}\textsuperscript{,}\textsuperscript{[}\textsuperscript{9}\textsuperscript{]}\textsuperscript{,}\textsuperscript{[}\textsuperscript{14}\textsuperscript{]}\textsuperscript{,}\textsuperscript{[}\textsuperscript{16}\textsuperscript{]}\textsuperscript{,}\textsuperscript{[}\textsuperscript{35}\textsuperscript{]}\textsuperscript{,}\textsuperscript{[}\textsuperscript{67}\textsuperscript{]}\textsuperscript{,}\textsuperscript{[}\textsuperscript{73}\textsuperscript{]}

A articulação ensino-serviço prioriza as necessidades de educação relacionadas
com os exercícios reais, proporciona novo olhar sobre o ensino, em que alunos
tornam-se sujeitos da aprendizagem e responsáveis.\textsuperscript{[}\textsuperscript{3}\textsuperscript{]}

Uma crítica indicada por quatro\textsuperscript{[}\textsuperscript{1}\textsuperscript{]}\textsuperscript{,}\textsuperscript{[}\textsuperscript{14}\textsuperscript{]}\textsuperscript{,}\textsuperscript{[}\textsuperscript{33}\textsuperscript{]}\textsuperscript{,}\textsuperscript{[}\textsuperscript{66}\textsuperscript{]}
artigos referiu-se à organização curricular por disciplinas, adotadas por muitos
cursos da saúde, caracterizando-se pela fragmentação do conteúdo, desvinculado
do processo de trabalho, o que dificulta a formação do egresso
crítico-reflexivo.\textsuperscript{[}\textsuperscript{67}\textsuperscript{]}

Sob esse prisma enfatiza-se que as instituições têm responsabilidades e
potenciais para fortalecer o processo de empoderamento de atores em seus
diferentes espaços de inserção.\textsuperscript{[}\textsuperscript{29}\textsuperscript{]}
Logo, as parcerias institucionais são condição \textit{sine qua non}
para a efetividade e implantação da \textsc{eps} e a melhora dos cuidados em saúde.\textsuperscript{[}\textsuperscript{62}\textsuperscript{]}

Como forma de alavancar esse processo, em 2005, o Governo Federal, por meio da
Secretaria de Gestão do Trabalho e da Educação na Saúde, do Ministério da Saúde,
em parceria com o Ministério da Educação e a \textsc{opas}/\textsc{oms}, lançou o programa
Pró-Saúde (Programa Nacional de Reorientação da Formação Profissional).
Inicialmente destinado para os cursos de Medicina, Enfermagem e Odontologia, foi
ampliado em 2007 para as demais áreas da saúde e tem como principal objetivo a
integração ensino-serviço, promovendo transformações nos processos de geração de
conhecimento, de ensino e aprendizagem e de prestação de serviço à população.
Além disso, o Pró-Saúde tem por finalidade estabelecer mecanismos de cooperação
entre os gestores do \textsc{sus} e as escolas, visando à melhoria da qualidade e à
resolubilidade da atenção prestada ao cidadão, à integração da rede pública de
serviços de saúde e à formação dos profissionais de saúde na graduação e na
educação permanente.\footnote{\fn9}

\section{\textsc{conclusões}}

Retomando a pergunta de pesquisa sobre como a \textsc{eps} tem sido compreendida quanto
aos seus pressupostos teóricos, metodológicos, resultados e conteúdos práticos
dos serviços, concluímos que ambas as composições – saúde e educação e trabalho
e educação – são envolvidas por processos políticos, sociais, econômicos,
desejos e demandas pessoais, pensamentos ideológicos, diferenças disciplinares
profissionais e instituições formadoras. São também permeadas por dificuldades
de infraestrutura material, de gestão e de recursos humanos para desenvolver ou
continuar multiplicando e aplicando a educação permanente.

Desta maneira, torna-se desafio ainda maior implementar processos de ensino
aprendizagem que sejam respaldados por ações crítico-reflexivas e participativas
de que a promover mudanças nas diferentes realidades de cada serviço.

Em conclusão, é possível inferir que a articulação educação e saúde encontra-se
pautada tanto nas ações dos serviços de saúde, quanto de gestão e de
instituições formadoras. Para atingir as metas propostas pelos documentos da
\textsc{opas}/\textsc{oms} e Ministério da Saúde, é necessário realizar propostas de \textsc{eps} com
profissionais dos serviços, professores e profissionais das instituições de
ensino a fim de que sejam incorporadas novas mudanças na estrutura do trabalho e
do ensino.

\section*{\textsc{referências}}
\begin{itemize}

\item[1] Abdalla IG, Stella \textsc{rcs}, Perim GL, Aguilar-da-Silva RH, Lampert JB,
Costa \textsc{nmsc}. Projeto pedagógico e as mudanças na educação médica. \textit{Rev
Bras Educ Med.}
2009;33(Supl.1):44-52. \textsc{doi}:10.1590/S0100-55022009000500005

\item[2] Arias \textsc{ehl}, Vitalino HA, Machado, MH, Aguiar Filho W, Cruz \textsc{lam}. Gestão
do trabalho no \textsc{sus}. \textit{Cad RH Saude.}
2006;3(1):119-24.

\item[3] Arruda MP, Araújo AP, Locks GA, Pagliosa FL. Educação permanente: uma
estratégia metodológica para os professores da saúde. \textit{Rev Bras Edu Med.}
2008;32(4):518-24. \textsc{doi}:10.1590/S0100-55022008000400015

\item[4] Barreto \textsc{ichc}, Andrade \textsc{lom}, Loiola F, Paula JB, Miranda AS, Goya N. A
educação permanente e a construção de Sistemas Municipais de Saúde-Escola: o
caso de Fortaleza (CE). \textit{Divulg Saude Debate.}
2006;34:31-46.

\item[5] Batista N, Batista SH, Goldenberg P, Seiffert O, Sonzogno MC. O enfoque
problematizador na formação de profissionais da saúde. \textit{Rev Saude
Publica.}
2005;39(2):231-7. \textsc{doi}:10.1590/S0034-89102005000200014

\item[6] Berbel \textsc{nan}. A problematização e a aprendizagem baseada em problemas:
diferentes termos ou diferentes caminhos? \textit{Interface Comun Saude Educ.}
1998;2(2):139-54. \textsc{doi}:10.1590/S1414-32831998000100008

\item[7] Bogus CM, Martin, Dimitrov P, Fortes \textsc{pac}, Capucci PF, Nemes Filho A et
al \textit{.}
Programa de capacitação permanente de conselheiros populares de saúde na cidade
de São Paulo. \textit{Saúde Soc.}
2003;12(2):56-67. \textsc{doi}:10.1590/S0104-12902003000200006

\item[8] Campos FE, Pierantoni CR, Viana \textsc{ala}, Faria \textsc{rmb}, Haddad AE. Os desafios
atuais para a educação permanente no \textsc{sus}. \textit{Cad RH Saude}. 2006;3(1):41-51.

\item[9] Camps E, Calliat MC, Spalvieri M, Dante V. La Educación Contínua como
una herramienta de intervención estrategica en la formación de recursos humanos.
\textit{Acta Bioquím Clín Latinoam}. 2003;37(3):289-306.

\item[10] Carotta F, Kawamura D, Salazar J. Educação permanente em saúde: uma
estratégia de gestão para pensar, refletir e construir práticas educativas e
processos de trabalhos. \textit{Saude Soc.}
2009;18(Supl 1):48-51. \textsc{doi}:10.1590/S0104-12902009000500008

\item[11] Ceccim RB, Feuerwerker \textsc{lcm}. O quadrilátero da formação para a área da
saúde: ensino, gestão, atenção e controle social. \textit{Physis}. 2004;14(1):41-65. \textsc{doi}:10.1590/S0103-73312004000100004

\item[12] Ceccim RB. Educação Permanente em Saúde: desafio ambicioso e
necessário. \textit{Interface Comun Saude Educ.}
2005;9(16):161-8. \textsc{doi}:10.1590/S1414-32832005000100013

\item[13] Ciconet RM, Marques GQ, Lima \textsc{mads}. Educação em serviço para
profissionais de saúde do Serviço de Atendimento Móvel de Urgência (\textsc{samu}):
relato da experiência de Porto Alegre-RS. \textit{Interface Comun Saude Educ.}
2008;12(26):659-66. \textsc{doi}:10.1590/S1414-32832008000300016

\item[14] Costa \textsc{ccc}, Bezerra Filho JG, Machado \textsc{mmt}, Machado \textsc{mfas}, Jorge AC,
Furtado \textsc{aaa}, et al. Curso técnico de enfermagem do \textsc{profae}-Ceará: a voz dos
supervisores. \textit{Texto Contexto - Enferm.}
2008;17(4):705-13. \textsc{doi}:10.1590/S0104-07072008000400011

\item[15] Costa-e-Silva V, Rivera \textsc{fju}, Hortale VA. Projeto Integrar: avaliação
da implantação de serviços integrados de saúde no Município de Vitória, Espírito
Santo, Brasil. \textit{Cad Saude Publica.}
2007;23(6):1405-14. \textsc{doi}:10.1590/S0102-311X2007000600015

\item[16] De Marco MA. Do modelo biomédico ao modelo biopsicossocial: um projeto
de educação permanente. \textit{Rev Bras Edu Med.}
2006;30(1):60-72. \textsc{doi}:10.1590/S0100-55022006000100010

\item[17] De Sordi \textsc{mrl}, Bagnato \textsc{mhs}. Subsídios para uma formação profissional
crítico-reflexiva na área da saúde: o desafio da virada do século. \textit{Rev
Latino-Am Enfermagem.}
1998;6(2):83-8. \textsc{doi}:10.1590/S0104-11691998000200012

\item[18] Deluiz N. O modelo das competências profissionais no mundo do trabalho
e na educação: implicações para o currículo. \textit{Bol Tec Senac.}
2001;27(3).

\item[19] Demers AL, Mamary E, Ebin VJ. Creating opportunities for training
California’s public health workforce. \textit{J Contin Educ Health Prof.}
2011;31(1):64-9. \textsc{doi}:10.1002/chp.20102

\item[20] Dreisinger M, Leet TL, Baker EA, Gillespie KN, Haas B, Brownson RC.
Improving the public health workforce: evaluation of training course to enhance
evidence-based decision making. \textit{J Public Health Manag Pract}. 2008;14(2):138-43. \textsc{doi}: 10.1097/01.\textsc{phh}.0000311891.73078.50.

\item[21] Feliciano \textsc{kvo}, Kovacs MH, Costa \textsc{ier}, Oliveira MG, Araújo \textsc{ams}.
Avaliação continuada da educação permanente na atenção à criança na estratégia
saúde da família. \textit{Rev Bras Saude Mater Infant.}
2008;8(1):45-53. \textsc{doi}:10.1590/S1519-38292008000100006

\item[22] Fortuna CM, Franceschini \textsc{trd}, Mishima SM, Matumoto S, Pereira \textsc{mjb}.
Movements of permanent health education triggered by the training of
facilitators. \textit{Rev Latino-Am Enfermagem.}
2011;19(2):411-20. \textsc{doi}:10.1590/S0104-11692011000200025

\item[23] Franco TB. Produção do cuidado e produção pedagógica: integração de
cenários do sistema de saúde no Brasil. \textit{Interface Comun Saude Educ.}
2007;11(23):427-38. \textsc{doi}:10.1590/S1414-32832007000300003

\item[24] Girade MG, Cruz \textsc{emnt}, Stefanelli MC. Educação continuada em enfermagem
psiquiátrica: reflexão sobre conceitos. \textit{Rev Esc Enferm \textsc{usp}.}
2006;40(1):105-10. \textsc{doi}:10.1590/S0080-62342006000100015

\item[25] González AD, Almeida MJ. Movimentos de mudança na formação em saúde:
da medicina comunitária às diretrizes curriculares. \textit{Physis}. 2010;20(2):551-70. \textsc{doi}:10.1590/S0103-73312010000200012

\item[26] Heidmann \textsc{itsb}, Almeida \textsc{mcp}, Eggert AB, Wosny AM, Monticelli M.
Promoção à saúde: trajetória histórica de suas concepções. \textit{Texto
Contexto - Enferm.}
2006;15(2):352-8. \textsc{doi}:10.1590/S0104-07072006000200021

\item[27] Horta NC, Sena R, Silva \textsc{meo}, Oliveira \textsc{srr}, Rezende VA. A prática das
equipes de saúde da família: desafios para a promoção de saúde. \textit{Rev Bras
Enferm.}
2009;62(4):524-9. \textsc{doi}:10.1590/S0034-71672009000400005

\item[28] Jones M, Tyrer A, Kalekzi T, Lancashire S. Research Summary: the
effect of whole team training in evidence-based interventions on the knowledge,
well-being and morale of inpatient mental health workers. \textit{J Psychiatr
Ment Health Nurs.}
2008;15(9):784-6. \textsc{doi}:10.1111/j.1365-2850.2008.01301.x.

\item[29] Kleba ME, Comerlatto D, Colliselli L. Promoção do empoderamento com
conselhos gestores de um pólo de educação permanente em saúde. \textit{Texto
Contexto - Enferm.}
2007;16(2):335-42. \textsc{doi}:10.1590/S0104-07072007000200018

\item[30] Lazarini CA, Francischetti I. Educação permanente: uma ferramenta para
o desenvolvimento docente na graduação. \textit{Rev Bras Edu Med.}
2010;34(4):481-6. \textsc{doi}:10.1590/S0100-55022010000400002

\item[31] Lima SG, Macedo LA, Vidal ML, Sá \textsc{mpbo}. Educação permanente em \textsc{sbv} e
\textsc{savc}: impacto no conhecimento dos profissionais de enfermagem. \textit{Arq.
Bras. Cardiol}. [Internet] 2009;93(6):582-588. \textsc{doi}:10.1590/S0066-782X2009001200012.

\item[32] Linde K, Willich SN. How objective are systematic reviews? Differences
between reviews on complementary medicine. \textit{J R Soc Med.}
2003;96(1):17-22. \textsc{doi}:10.1258/jrsm.96.1.17

\item[33] Lopes \textsc{srs}, Piovesan \textsc{eta}, Melo LO, Pereira MF. Potencialidades da
educação permanente para a transformação das práticas de saúde. \textit{Comun
Cienc Saude.}
2007;18(2):147-55.

\item[34] Maciel \textsc{eln}, Figueiredo PF, Prado TN, Galavote HS, Ramos MC, Araújo MD,
et al. Avaliação dos egressos do curso de especialização em Saúde da Família no
Espírito Santo, Brasil. \textit{Cienc Saude Coletiva.}
2010;15(4):2021-8. \textsc{doi}:10.1590/S1413-81232010000400016

\item[35] Marães \textsc{vrfs}, Martins EF, Cipriano Jr G, Acevedo AC, Pinho \textsc{dlm}. Projeto
pedagógico do curso de Fisioterapia da Universidade de Brasília.
\textit{Fisioter Mov.}
2010;23(2):311-21. \textsc{doi}:10.1590/S0103-51502010000200014

\item[36] Marques ES, Cotta \textsc{rmm}, Franceschini \textsc{scc}, Botelho \textsc{miv}, Araújo \textsc{rma},
Junqueira TS. Práticas e percepções acerca do aleitamento materno: consensos e
dissensos no cotidiano de cuidado numa Unidade de Saúde da Família.
\textit{Physis}. 2009;19(2):439-55. \textsc{doi}:10.1590/S0103-73312009000200011

\item[37] Matheus \textsc{mcc}. Metassíntese qualitativa: desenvolvimento e contribuições
para a prática baseada em evidências. \textit{Acta Paul Enferm.}
2009;22(Spe1):543-5. \textsc{doi}:10.1590/S0103-21002009000800019

\item[38] Matos E, Pires \textsc{dep}. Práticas de cuidado na perspectiva
interdisciplinar: um caminho promissor. \textit{Texto Contexto - Enferm.}
2009;18(2):338-46. \textsc{doi}:10.1590/S0104-07072009000200018

\item[39] Matumoto S, Fortuna CM, Kawata LS, Mishima SM, Pereira \textsc{mjb}. A prática
clínica do enfermeiro na atenção básica: um processo em construção. \textit{Rev
Latino-Am Enfermagem.}
2011;19(1):123-30. \textsc{doi}:10.1590/S0104-11692011000100017

\item[40] Medeiros AC, Pereira \textsc{qlc}, Siqueira \textsc{hch}, Cecagno D, Moraes CL. Gestão
participativa na educação permanente em saúde: olhar das enfermeiras.
\textit{Rev Bras Enferm.}
2010;63(1):38-42. \textsc{doi}:10.1590/S0034-71672010000100007

\item[41] Mendonça \textsc{mhm}, Giovanella L. Formação em política pública de saúde e
domínio da informação para o desenvolvimento profissional. \textit{Cienc Saude
Coletiva.}
2007;12(3):601-10. \textsc{doi}:10.1590/S1413-81232007000300010

\item[42] Merhy EE. O desafio que a educação permanente tem em si: a pedagogia
da implicação. \textit{Interface Comun Saude Educ.}
2005;9(16):172-4. \textsc{doi}:10.1590/S1414-32832005000100015

\item[43] Merhy EE, Feuerwerker \textsc{lcm}, Ceccim RB. Educación permanente en salud:
una estrategia para intervenir en la micropolítica del trabajo en salud.
\textit{Salud Colectiva.}
2006;2(2):147-60.

\item[44] Montanha D, Peduzzi M. Educação permanente em enfermagem: levantamento
de necessidades e resultados esperados segundo a concepção dos trabalhadores.
\textit{Rev Esc Enferm \textsc{usp}}. 2010;44(3):597-604. \textsc{doi}:10.1590/S0080-62342010000300007

\item[45] Monteiro MI, Chilida \textsc{msp}, Bargas EB. Educação continuada em um serviço
terceirizado de limpeza de um hospital universitário. \textit{Rev Latino-Am
Enfermagem.}
2004;12(3):541-8. \textsc{doi}: 10.1590/S0104-11692004000300013

\item[46] Murofuse NT, Rizzoto \textsc{mlf}, Muzzolon \textsc{abf}, Nicola AL. Diagnóstico da
situação dos trabalhadores em saúde e o processo de formação no polo regional de
educação permanente em saúde. \textit{Rev Latino-Am Enfermagem.}
2009;17(3):314-20. \textsc{doi}:10.1590/S0104-11692009000300006

\item[47] Nicoletto \textsc{scs}, Mendonça FF, Brevilheri \textsc{ecl}, Rezende LR, Carvalho GS,
Durán González A, et al. Polos de educação permanente em saúde: uma análise da
vivência dos atores sociais no norte do Paraná. \textit{Interface Comun Saude
Educ.}
2009;13(30):209-19. \textsc{doi}:10.1590/S1414-32832009000300017

\item[48] Nunes MF, Leles CR, Pereira MF, Alves RT. The proposal of permanent
education in the formation of dentists in std/hiv/aids. \textit{Interface Comun
Saude Educ.}
2008;12(25):413-20. \textsc{doi}:10.1590/S1414-32832008000200015

\item[49] Olson D, Hoeppner M, Larson S, Ehrenberg A, Leitheiser AT. Lifelong
learnig for public health practice education: a model curriculum for
bioterrorism and emergency readiness. \textit{Public Health Rep.}
2008;123(Suppl 2):53-64.

\item[50] Paschoal AS, Mantovani MF, Méier MJ. Percepção da educação permanente,
continuada e em serviço para enfermeiros de um hospital de ensino. \textit{Rev
Esc Enferm \textsc{usp}.}
2007;41(3):478-84. \textsc{doi}:10.1590/S0080-62342007000300019

\item[51] Peduzzi M, Del Guerra DA, Braga CP, Lucena FS, Silva \textsc{jam}. Atividades
educativas de trabalhadores na atenção primária: concepções de educação
permanente e de educação continuada em saúde presentes no cotidiano de Unidades
Básicas de Saúde em São Paulo. \textit{Interface Comun Saude Educ.}
2009;13(30):121-34. \textsc{doi}:10.1590/S1414-32832009000300011

\item[52] Pessanha RV, Cunha, \textsc{fts}. A aprendizagem-trabalho e as tecnologias de
saúde na estratégia saúde da família. \textit{Texto Contexto - Enferm}. 2009;18(2):233-40. \textsc{doi}:10.1590/S0104-07072009000200005

\item[53] Ricaldoni \textsc{cac}, Sena RRd. Educação permanente: uma ferramenta para
pensar e agir no trabalho de enfermagem. \textit{Rev Latino-Am Enferm.}
2006;14(6):837-42. \textsc{doi}:10.1590/S0104-11692006000600002

\item[54] Robinson S, Retainig the mental health nursing work-force: early
indicators of retention and attrition. \textit{Int J Ment Health Nurs}. 2005;14(4):230-42. \textsc{doi}:10.1111/j.1440-0979.2005.00387.x

\item[55] Rodrigues \textsc{acs}, Vieira \textsc{glc}, Torres HC. A proposta da educação
permanente em saúde na atualização da equipe de saúde em diabetes mellitus.
\textit{Rev Esc Enferm \textsc{usp}.}
2010;44(2):531-7. \textsc{doi}:10.1590/S0080-62342010000200041

\item[56] Rodrigues \textsc{rrj}, Imai RY, Ferreira WF. Um espaço para o desenvolvimento
interpessoal no trabalho. \textit{Psicol Estud.}
2001;6(2):123-7. \textsc{doi}:10.1590/S1413-73722001000200017

\item[57] Rodriguez C, Pozzebon M. The implementation evaluation of primary care
groups of practice: a focus and organization identity. \textit{\textsc{bmc} Fam Pract.}
2010;11:15.

\item[58] Rossetto M, Silva \textsc{laa}. Ações de educação permanente desenvolvidas para
os agentes comunitários de saúde. \textit{Cogitare Enferm.}
2010;15(4):723-9.

\item[59] Sandelowski M, Barroso J. Toward a metasynthesis of qualitative
findings on motherhood in \textsc{hiv}-positive women. \textit{Res Nurs Health.}
2003;26(2):153-70. \textsc{doi}:10.1002/nur.10072

\item[60] Sandelowski M, Barroso J. Writing the proposal for a qualitative
research methodology project. \textit{Qual Health Res.}
2003;13(6):781-820. \textsc{doi}:10.1177/1049732303013006003

\item[61] Sandelowski MB, Barroso J. Handbook for synthesizing qualitative
reserach. New York: Springer Publishing Company; 2007.

\item[62] Silva AM, Peduzzi M. Caracterização das atividades educativas de
trabalhadores de enfermagem na ótica da educação permanente. \textit{Rev Eletr
Enf.}
2009;11(3):518-26.

\item[63] Smeke \textsc{elm}, Oliveira \textsc{nls}. Avaliação participante de práticas educativas
em serviços de saúde. \textit{Cad \textsc{cedes}.}
2009;29(79):347-60. \textsc{doi}:10.1590/S0101-32622009000300005

\item[64] Sousa MF, Merchán-Hamann E. Saúde da Família no Brasil: estratégia de
superação da desigualdade na saúde? \textit{Physis.}
2009;19(3):711-29. \textsc{doi}:10.1590/S0103-73312009000300009

\item[65] Souza \textsc{rcr}, Soares E, Souza \textsc{iag}, Oliveira JC, Salles RS, Cordeiro \textsc{cem}.
Educação permanente em enfermagem e a interface com a ouvidoria hospitalar.
\textit{Rev \textsc{rene}.}
11(4):85-94.

\item[66] Souza RR. O sistema público de saúde brasileiro. In: Negri B, Viana
\textsc{ala}, editores. O Sistema Único de Saúde em dez anos de desafio: o passo a passo
de uma reforma que alarga o desenvolvimento e estreita a desigualdade social.
São Paulo: Centro de Estudos Augusto Leopoldo Ayrosa Galvão; 2002.

\item[67] Sudan \textsc{lcp}, Corrêa AK. Práticas educativas de trabalhadores de saúde:
vivência de graduandos de enfermagem. \textit{Rev Bras Enferm.}
2008;61(5):576-82. \textsc{doi}:10.1590/S0034-71672008000500008

\item[68] Tanji S, Silva \textsc{cmslmd}, Albuquerque VS, Viana LO, Santos \textsc{nmp}.
Integração ensino-trabalho-cidadania na formação de enfermeiros. \textit{Rev
Gaucha Enferm.}
2010;31(3):483-90. \textsc{doi}:10.1590/S1983-14472010000300011

\item[69] Tavares \textsc{cmm}. A educação permanente da equipe de enfermagem para o
cuidado nos serviços de saúde mental. \textit{Texto Contexto - Enferm.}
2006;15(2):287-95. \textsc{doi}:10.1590/S0104-07072006000200013

\item[70] Tronchin \textsc{dmr}, Mira VL, Peduzzi M, Ciampone \textsc{mht}, Melleiro MM, Silva
\textsc{jam}, et al. Educação permanente de profissionais de saúde em instituições
públicas hospitalares. \textit{Rev Esc Enferm \textsc{usp}.}
2009;43(Spe 2):1210-5. \textsc{doi}:10.1590/S0080-62342009000600011

\item[71] Vincent SP. Educação permanente: componente estratégico para a
implementação da política nacional de atenção oncológica. \textit{Rev Bras
Cancerol.}
2007;53(1):79-85.

\item[72] Ximenes Neto \textsc{frg}, Sampaio \textsc{jjc}. Gerentes do território na estratégia
Saúde da Família: análise e perfil de necessidades de qualificação. \textit{Rev
Bras Enferm.}
2007;60(6):687-95. \textsc{doi}:10.1590/S0034-71672007000600013

\item[73] Yaping D, Stanton P. Evaluation of the health services management
training course of Jiangsu, China. \textit{Aust Health Rev.}
2002;25(3):161-70. \textsc{doi}:10.1071/AH020161

\end{itemize}

\end{document}
