% Generated by jats2tex@0.11.1.0
\documentclass{article}
\usepackage[T1]{fontenc}
\usepackage[utf8]{inputenc} %% *
\usepackage[portuges,spanish,english,german,italian,russian]{babel} %% *
\usepackage{amstext}
\usepackage{authblk}
\usepackage{unicode-math}
\usepackage{multirow}
\usepackage{graphicx}
\usepackage{etoolbox}
\usepackage{xtab}
\usepackage{enumerate}
\usepackage{hyperref}
\usepackage{penalidades}
\usepackage[footnotesize,bf,hang]{caption}
\usepackage[nodayofweek,level]{datetime}
\usepackage[top=0.85in,left=2.75in,footskip=0.75in]{geometry}
\newlength\savedwidth
\newcommand\thickcline[1]{\noalign{\global
\savedwidth
\arrayrulewidth
\global\arrayrulewidth 2pt}
\cline{#1}
\noalign{\vskip\arrayrulewidth}
\noalign{\global\arrayrulewidth\savedwidth}}
\newcommand\thickhline{\noalign{\global
\savedwidth\arrayrulewidth
\global\arrayrulewidth 2pt}
\hline
\noalign{\global\arrayrulewidth\savedwidth}}
\usepackage{lastpage,fancyhdr}
\usepackage{epstopdf}
\pagestyle{myheadings}
\pagestyle{fancy}
\fancyhf{}
\setlength{\headheight}{27.023pt}
\lhead{\includegraphics[width=10mm]{logo.png}}
\rhead{\ifdef{\journaltitle}{\journaltitle}{}
\ifdef{\volume}{vol.\,\volume}{}
\ifdef{\issue}{(\issue)}{}
\ifdef{\fpage}{\fpage--\lpage\,pp.}}
\rfoot{\thepage/\pageref{LastPage}}
\renewcommand{\footrule}{\hrule height 2pt \vspace{2mm}}
\fancyheadoffset[L]{2.25in}
\fancyfootoffset[L]{2.25in}
\lfoot{\sf \ifdef{\articledoi}{\articledoi}{}}
\setmainfont{Linux Libertine O}
\renewcommand*{\thefootnote}{\alph{footnote}}
\makeatletter
\newcommand{\fn}{\afterassignment\fn@aux\count0=}
\newcommand{\fn@aux}{\csname fn\the\count0\endcsname}
\makeatother

\newcommand{\journalid}{Cad Saude Publica}
\newcommand{\journaltitle}{Cadernos de Saúde Pública}
\newcommand{\abbrevjournaltitle}{Cad. Saúde
Pública}
\newcommand{\issnppub}{0102-311X}
\newcommand{\issnepub}{1678-4464}
\newcommand{\publishername}{Fundação Oswaldo Cruz}
\newcommand\articledoi{\textsc{doi} 10.1590/0102-311\textsc{xco}07S114}
\def\subject{Os Autores Respondem}\newcommand{\subtitlestyle}[1]{--
\emph{#1}\medskip}
\newcommand{\transtitlestyle}[1]{\par\medskip\Large #1}
\newcommand{\transsubtitlestyle}[1]{-- \Large\emph{ #1}}

\newcommand{\titlegroup}{
\ifdef{\subtitle}{\subtitlestyle{\subtitle}}{}
\ifdef{\transtitle}{\transtitlestyle{\transtitle}}{}
\ifdef{\transsubtitle}{\transsubtitlestyle{\transsubtitle}}{}}

\title{Ampliando o debate\titlegroup{}}
\author{Leal, Maria do Carmo}
\author{Pereira, Ana Paula Esteves}
\author{Domingues, Rosa Maria Soares Madeira}
\author{Filha, Mariza Miranda Theme}
\author{Dias, Marcos Augusto Bastos}
\author{Nakamura-Pereira, Marcos}
\author{Bastos, Maria Helena}
\author{Gama, Silvana Granado Nogueira da}
\date{ 08 2014}
\def\volume{30}
\def\issue{Suppl 1}
\def\fpage{S43}
\def\lpage{S47}
\def\permissions{This is an Open Access article distributed under the terms of
the
Creative Commons Attribution Non-Commercial License, which permits
unrestricted non-commercial use, distribution, and reproduction in any
medium, provided the original work is properly cited.}

\begin{document}
\selectlanguage{portuges}
\section*{Metadados não aplicados}
\begin{itemize}
\item[\textbf{língua do artigo}]{Português}
\ifdef{\journalid}{\item[\textbf{journalid}] \journalid}{}
\ifdef{\journaltitle}{\item[\textbf{journaltitle}] \journaltitle}{}

\ifdef{\journalsubtitle}{\item[\textbf{journalsubtitle}] \journaltitle}{}
\ifdef{\transjournaltitle}{\item[\textbf{journaltitle}] \journaltitle}{}
\ifdef{\transjournalsubtitle}{\item[\textbf{journalsubtitle}] \journaltitle}{}

\ifdef{\abbrevjournaltitle}{\item[\textbf{abbrevjournaltitle}]
\abbrevjournaltitle}{}
\ifdef{\issnppub}{\item[\textbf{issnppub}] \issnppub}{}
\ifdef{\issnepub}{\item[\textbf{issnepub}] \issnepub}{}
\ifdef{\publishername}{\item[\textbf{publishername}] \publishername}{}
\ifdef{\publisherid}{\item[\textbf{publisherid}] \publisherid}{}
\ifdef{\subject}{\item[\textbf{subject}] \subject}{}
\ifdef{\transtitle}{\item[\textbf{transtitle}] \transtitle}{}
\ifdef{\authornotes}{\item[\textbf{authornotes}] \authornotes}{}
\ifdef{\articleid}{\item[\textbf{articleid}] \articleid}{}
\ifdef{\articledoi}{\item[\textbf{articledoi}] \articledoi}{}
\ifdef{\volume}{\item[\textbf{volume}] \volume}{}
\ifdef{\issue}{\item[\textbf{issue}] \issue}{}
\ifdef{\fpage}{\item[\textbf{fpage}] \fpage}{}
\ifdef{\lpage}{\item[\textbf{lpage}] \lpage}{}
\ifdef{\permissions}{\item[\textbf{permissions}] \permissions}{}
\end{itemize}
\maketitle

Os autores agradecem aos colegas por aceitaram participar como debatedores deste
artigo
sobre as intervenções obstétricas durante o trabalho de parto e parto em
mulheres
brasileiras classificadas como de risco obstétrico habitual. A contribuição que
cada um
trouxe enriqueceu a discussão no campo científico da saúde e ampliou o debate,
colocando-o no contexto da cultura, da ética e das relações sociais como um
todo.

É com grande entusiasmo e motivação que respondemos, selecionando apenas alguns
aspectos
dos comentários de cada um, dado o espaço disponível nesta seção.

Guilherme Cecatti fez importantes considerações sobre as questões metodológicas
do
estudo, tendo chamado a atenção para a ausência de informação no artigo sobre o
uso de
fórceps no Brasil. O tema havia sido suprimido dada a quantidade de desfechos
abordados
no trabalho, entretanto, sendo um dos objetivos centrais desse texto dar um
panorama
sobre a atenção ao parto no Brasil em mulheres classificadas como de risco
obstétrico
habitual, agradecemos ao Cecatti a oportunidade de poder comentar sobre esse
tópico. A
frequência de uso de fórceps foi muito baixa, de 1,4\% para todas as mulheres e
de 1,9\%
para as de risco obstétrico habitual, sendo maior a prevalência na Região
Sudeste, nas
capitais e usuárias do \textsc{sus}, bem como nas adolescentes, brancas e primíparas, sem
distinção com o grupo de risco obstétrico. Quanto aos aspectos obstétricos,
naquelas em
que foi utilizado esse instrumental foi muito mais frequente o uso de todas as
demais
intervenções, notadamente a analgesia peridural, uso de ocitocina, a manobra de
Kristeller e a episiotomia, que alcançaram cifras elevadas, de 60\%, 56\% e
86\%,
respectivamente. Alguns estudos mostram que é baixa a frequência de uso de
fórceps no
país, e as principais razões para o abandono quase que completo dessa conduta
obstétrica
são a deficiência na formação do médico para qualificá-lo na atenção ao parto
vaginal
operatório, bem como a preocupação com processos judiciais\textsuperscript{[}\textsuperscript{1}\textsuperscript{]}\textsuperscript{,}\textsuperscript{[}\textsuperscript{2}\textsuperscript{]}. Na nossa pesquisa, não temos como avaliar a adequação da
utilização do fórceps e, portanto, ficamos impossibilitados de afirmar se a
baixa taxa
de utilização é um fator positivo, ainda mais quando tal prática se associa à
elevada
prevalência da manobra de Kristeller e de maiores taxas de morbidade grave e
\textit{near miss}
materno\textsuperscript{[}\textsuperscript{3}\textsuperscript{]}.

Outro ponto destacado por Cecatti foi a decisão dos autores de incluir mulheres
com
cesariana prévia no grupo de mulheres classificadas como de risco obstétrico
habitual. É
correto pensar que essas mulheres podem estar submetidas a maiores riscos
durante o
trabalho de parto e parto do que aquelas sem experiência de cesariana anterior,
contudo,
não considerar uma cesariana prévia como critério para excluí-las do grupo de
mulheres
classificadas como de risco obstétrico habitual nos pareceu acertado porque: (a)
nesse
grupo, a proporção de cesariana anterior foi a mesma do grupo de risco
obstétrico, 20\%,
ou seja, com relação a esse aspecto, os dois grupos não diferiam; (b) as
evidências
científicas mostram experiências exitosas na realização de parto vaginal após
cesariana\textsuperscript{[}\textsuperscript{4}\textsuperscript{]}. Os principais protocolos
internacionais, dentre eles o do Colégio Americano de Ginecologistas e Obstetras\textsuperscript{[}\textsuperscript{5}\textsuperscript{]}, apontam para a importância de oferecer
uma prova de trabalho de parto para mulheres com cesariana anterior como uma
estratégia
para reduzir as taxas de repetição dessa cirurgia. No Brasil, onde as taxas de
cesariana
são as mais elevadas do mundo, é fundamental que essa estratégia esteja presente
na
rotina das maternidades. Em nosso estudo, mais de 80\% das mulheres
classificadas como de
risco obstétrico habitual com cesariana prévia foram submetidas a outra
cesariana e, 88\%
delas foram realizadas sem que a gestante entrasse em trabalho de parto. A
excessiva
prática de cesariana nesse grupo projeta um cenário preocupante no país, dada a
tendência crescente dessa cirurgia em primíparas.

Quanto à utilização da analgesia epidural, concordamos com Cecatti que sua
utilização não
deve ser classificada como desnecessária, mas o que impressiona no país é a
iniquidade
na oferta de métodos farmacológicos e não farmacológicos de alívio da dor. O
modelo de
atenção ao parto e nascimento que não priorize a parturição fisiológica,
medicaliza em
excesso, aumenta a dor e o medo do parto e, paradoxalmente, usa como alternativa
para
aliviar o sofrimento da mulher mais medicação.

Em relação à pressão do fundo uterino, não há evidência sobre seu uso ser
benéfico. Os
riscos potenciais do uso da manobra de Kristeller incluem a ruptura uterina,
lesão do
esfíncter anal, fraturas em recém-nascidos ou dano cerebral, dentre outros\textsuperscript{[}\textsuperscript{6}\textsuperscript{]}. Infelizmente, não temos como avaliar
em que condições e de que forma a manobra foi feita nas mulheres que
participaram desse
estudo, no entanto acreditamos que não há suporte para sua utilização em mais de
um
terço das mulheres classificadas como de risco obstétrico habitual, conforme
identificado. Num cenário de tantas intervenções e violências institucionais, a
manobra
de Kristeller deve ser apontada como uma prática desnecessária, até que
evidências
robustas de efetividade e segurança justifiquem sua utilização.

Maria Luiza G. Riesco destaca a importância do estudo, a originalidade do
trabalho e
chama a atenção para uma possível redução na ocorrência de episiotomia,
considerando os
dados publicados pela \textit{Pesquisa Nacional de Demografia e Saúde}
(\textsc{pnds}) de
2006. Pode ser que tenha ocorrido alguma redução na incidência da episiotomia no
Brasil,
aqui, porém, cabe comentar que é necessário ter cuidado ao se compararem os
dados da
\textsc{pnds} com os oriundos da pesquisa \textit{Nascer no Brasil}
porque: (1) a \textsc{pnds}
incluiu todas as mulheres que fizeram parto vaginal e a pesquisa \textit{Nascer
no
Brasil}, neste artigo, considerou apenas as mulheres classificadas como de
risco obstétrico habitual; (2) a \textsc{pnds} perguntou sobre a ocorrência de
episiotomia para
mães de nascidos vivos nos últimos 5 anos, podendo envolver importantes vieses
de
memória. Na pesquisa \textit{Nascer no Brasil}, a pergunta foi feita na
maternidade, no pós- parto imediato; (3) a pergunta da \textsc{pnds} era se foi feito um
corte na
vagina (episiotomia)\textsuperscript{[}\textsuperscript{7}\textsuperscript{]}, na pesquisa
\textit{Nascer no Brasil}, as mulheres eram indagadas sobre como ficou o
períneo após o parto e as alternativas de respostas eram: Não rompeu, não cortou
e não
deu pontos; Rompeu um pouco, mas não precisou dar pontos; Não levou pontos, mas
não sabe
se rompeu; Rompeu e deram pontos; Cortaram e deram pontos; Não soube informar.
Levou-se
em conta que foi feita a episiotomia apenas para as mulheres que informaram que
cortaram
e deram pontos, ou seja, naquelas em que a mulher informou com certeza que no
períneo
havia sido feita uma incisão pelo profissional de saúde. Se considerarmos as
mulheres
que referiram ter sido suturadas, a proporção corresponderia a 71\% das que
fizeram parto
vaginal, valor muito próximo ao descrito pela \textsc{pnds}. Fica, portanto, a dúvida se
avançamos nesse quesito ou se a diferença entre os resultados das duas pesquisas
é
consequência do uso de metodologias distintas.

Suzanne Serruya foca seus comentários na ausência de autonomia da mulher na
condução do
seu parto no Brasil, destacando que as rotinas hospitalares que definem o que
pode e não
pode ser feito são em si uma padronização de condutas que ignora “a experiência
única e
sempre singular de parir”. E o hospital, como espaço tradicional para tratar
enfermos,
impõe suas regras de jejuar, ficar restrita ao leito com veia puncionada,
aproximando a
atenção ao parto às práticas curativo-terapêuticas. Traz também o profissional
de saúde
para o cenário do parto, destacando que a incorporação de práticas com
fundamentos
científicos e que respeitem o pertencimento do parto à mulher podem ser muito
recompensadoras para eles. Contribuiria para diminuir a assimetria de poder e
abrir
possibilidades para conhecer o novo, ressignificar sua atuação na assistência ao
trabalho de parto e parto, gerando um ambiente de criatividade e bem-estar.
Precisa ser
frisado que essa mesma assimetria de poder que existe entre os profissionais de
saúde e
usuárias dos serviços está também presente entre os próprios profissionais –
médicos x
enfermeiros, enfermeiros x técnicos de enfermagem, etc. A dificuldade de
trabalhar em
equipe gera um ambiente de isolamento e não compartilhamento das experiências e,
particularmente, no Brasil, a quase exclusão da enfermeira
obstétrica/obstetrizes do
atendimento ao parto, na contramão do que acontece em países desenvolvidos.
Neste
estudo, apenas 15\% dos partos foram acompanhados por enfermeira
obstétrica/obstetrizes,
sendo mais frequente em regiões com carência de médicos. Ainda assim, pôde-se
ver que
nos partos vaginais assistidos por tais profissionais o uso de boas práticas
foram mais
frequentes e as intervenções obstétricas mais escassas (dados não mostrados), em
concordância com o que vem sendo descrito na literatura internacional\textsuperscript{[}\textsuperscript{8}\textsuperscript{]}.

Serruya chama ainda a atenção para a importância de mudanças na formação
profissional,
com cuja ideia compartilhamos completamente, dado que novas coortes de
profissionais
estão sendo formadas dentro das mesmas e velhas práticas. Por isso, os hospitais
de
ensino necessitam ser priorizados na implementação da Rede Cegonha, uma
estratégia
inovadora do Ministério da Saúde, que pretende implantar, até 2015, 280 Centros
de Parto
Normal (\textsc{cpn}) intra e peri-hospitalar, ao mesmo tempo que articula um sistema de
referência da atenção pré-natal para os hospitais, transporte para a
maternidade,
implementação de boas práticas em trabalho de parto e nascimento, incluindo o
direito à
livre escolha das mulheres de acompanhante durante todo o período da
hospitalização.
Outra característica desse programa é favorecer a priorização de enfermeiras
obstétricas/obstetrizes no acompanhamento do trabalho de parto e parto de risco
habitual
em trabalho colaborativo com a equipe médica, bem como promover a adequação do
espaço e
ambiência apropriada para a promoção do trabalho de parto, parto e nascimento
fisiológicos\textsuperscript{[}\textsuperscript{9}\textsuperscript{]}.

A iniciativa da Rede Cegonha está dirigida ao sistema público de saúde que
atende a mais
de 75\% dos partos da população brasileira e, se devidamente implementada no
\textsc{sus}, pode
modificar o atual panorama da assistência ao parto no Brasil e impactar nos
indicadores
obstétricos e perinatais do país.

Soo Downe enfatiza os custos financeiros com intervenções desnecessárias que
poderiam
estar sendo canalizados para atender a outras necessidades de saúde do país, com
o que
concordamos plenamente. O Global Health Report da \textsc{oms} de 2010, sobre os custos
globais
da cesariana, já enfocava que o excesso de uso dessa tecnologia se constituía em
uma
barreira para a cobertura universal em saúde\textsuperscript{[}\textsuperscript{10}\textsuperscript{]}. Nesse documento, o Brasil e a China são apresentados como
responsáveis por quase 50\% das cesarianas desnecessárias do mundo. No Brasil,
se
somarmos esses custos aos das demais intervenções desnecessárias sobre o parto
vaginal,
alcançaríamos cifras surpreendentes de desperdício no país.

Outro ponto muito importante levantado pela Soo Down foram os princípios éticos
na
relação profissional de saúde/paciente, destacando que o primeiro compromisso
ético
“\textit{Primum non nocere: Primeiro não cause Dano}
” ou princípio da não
maleficência, que é um dos princípios hipocráticos da bioética ensinados a todos
os
estudantes da área de saúde em todo o mundo. Esse princípio lembra aos
profissionais de
saúde que eles devem considerar o possível prejuízo que qualquer intervenção
possa
trazer. O princípio da beneficência, por outro lado, considera julgar se uma
determinada
ação ou intervenção seria uma prática suficiente de beneficência e o quão ela é
aceitável e adequada. Apesar das diferenças de opinião, existem conceitos sobre
os quais
há um amplo acordo, um deles é o conceito ético de respeito às mulheres na
assistência
ao parto. Quando não há evidências científicas suficientemente estabelecidas,
como por
exemplo, em relação às boas práticas obstétricas de
deambular\textsuperscript{[}\textsuperscript{11}\textsuperscript{]}, alimentar-se\textsuperscript{[}\textsuperscript{12}\textsuperscript{]}, liberdade de posição para o parto\textsuperscript{[}\textsuperscript{13}\textsuperscript{]}
etc., a recomendação é de que o procedimento seja de
livre escolha da parturiente.

As origens do termo “humanização do parto” e a avaliação da atenção obstétrica
como
evento que ignora os aspectos afetivos e sociais do nascimento têm induzido
diversas
políticas públicas no Brasil. Muitos pesquisadores têm explorado o entendimento
do termo
“humanização”, descrevendo os diferentes significados (muitas vezes
contraditórios),
suas possibilidades de mudar a cultura médica, a compreensão da anatomia e
fisiologia
das mulheres e as relações de gênero\textsuperscript{[}\textsuperscript{14}\textsuperscript{]}.

Com relação à cesariana, argumenta-se que as brasileiras tendem a perceber esse
procedimento como mais seguro que o parto normal, de maior qualidade de
atendimento e,
muitas vezes, como reafirmação das diferenças de classe so-cial\textsuperscript{[}\textsuperscript{15}\textsuperscript{]}\textsuperscript{,}\textsuperscript{[}\textsuperscript{16}\textsuperscript{]}. Para algumas mulheres, o esforço para medicalizar o
processo de nascimento representa uma solução prática para superar os problemas
encontrados dentro do próprio sistema de saúde.

No que diz respeito às condições sociobiológicas que produziram esse rápido
aumento no
uso da tecnologia para iniciar, acelerar, regular e monitorar o processo de
nascimento,
argumentam que o parto vaginal é percebido como se fosse de alto risco para
saúde da
mulher e para sua vida sexual, criando o paradigma do corte acima e corte
abaixo, ou
seja, cesarianas e episiotomies\textsuperscript{[}\textsuperscript{17}\textsuperscript{]}.

A ausência de consenso sobre o parto normal à luz de um parto sem intervenções é
definitivamente uma questão fundamental para esse debate. Para reduzir
procedimentos
desnecessários, a estratégia Rede Cegonha publicou em portaria ministerial a
definição
de parto normal como: de início espontâneo, sem indução, sem aceleração, sem o
uso de
intervenções como fórceps ou cesariana, sem o uso de anestesia geral, raquidiana
ou
epidural durante o parto\textsuperscript{[}\textsuperscript{9}\textsuperscript{]}\textsuperscript{,}\textsuperscript{[}\textsuperscript{18}\textsuperscript{]}.

Soo Downe acrescenta que publicações recentes vêm sinalizando para prejuízos
decorrentes
do uso excessivo de tecnologias obstétricas como ocitocina, antibióticos e
outros para a
saúde a longo prazo, tais como a ocorrência de diabetes mellitus tipo 1, asma,
esclerose
múltipla, alergias, obesidade, etc. Embora as evidências científicas sobre essas
relações sejam ainda frágeis por dependerem de um período longo de
acompanhamento
acurado de coortes de nascimentos, essas hipóteses parecem plausíveis. O
desenvolvimento
da teoria epigenética, que reforça o conceito de multicausalidade e de
influências
recíprocas entre os diferentes níveis de organização (molecular, celular,
orgânico,
comportamental e social) amplificam muito as possibilidades de entendimento do
adoecimento humano\textsuperscript{[}\textsuperscript{19}\textsuperscript{]}.

Eugene Declercq se surpreende com a insistência do Brasil em aumentar
sistematicamente as
taxas de cesariana, em contraste com alguns países desenvolvidos que as têm
mantido
estáveis, em níveis baixos, ou com outros de níveis não tão baixos, mas que
recentemente
vêm fazendo esforços e tendo êxito em reluzi-las. Ele também questiona se as
mulheres
classificadas como de risco obstétrico habitual atendidas no setor privado são
diferentes daquelas do setor público para justificar a maior quantidade de
intervenções
nas primeiras e sobre como os esforços do Ministério da Saúde do Brasil podem
ter
influenciado a maior frequência de boas práticas obstétricas no setor público.

Nós também compartilhamos dessa pressa em ter melhores dias para a atenção ao
parto e
nascimento no Brasil, um país de dimensão continental, com marcantes
desigualdades
sociais e regionais, o que torna complexa a análise dos indicadores de saúde e
seus
determinantes. As questões colocadas por Eugene Declercq são profundas e
contundentes,
certamente não será possível responder a elas em detalhes aqui, mas faremos um
esforço
para dar uma visão geral do que vem acontecendo no país.

No âmbito da assistência ao parto, os últimos vinte anos foram marcados por
intensos
debates e conflitos. Por um lado, o Ministério da Saúde vem adotando iniciativas
governamentais visando a modificações no modelo de atenção ao parto, tais como
portarias
regulamentando a atuação da enfermagem obstétrica/obstetrizes e o direito ao
acompanhante durante a internação para o parto, a implantação de centros de
parto
normal, a delimitação de percentuais máximos permitidos de cesarianas por
unidade
hospitalar, incentivos financeiros para a adequação dos ambientes hospitalares,
capacitação de profissionais para adesão às boas práticas obstétricas, dentre
outros.
Ciente de que a mera divulgação de protocolos não é suficiente para a
modificação das
práticas assistenciais, o Ministério da Saúde adotou a estratégia de apoiadores,
que
atuam diretamente com as equipes locais na modificação das rotinas assistenciais
e na
implantação de boas práticas, inicialmente em maternidades das regiões Norte e
Nordeste
e, mais recentemente, em todo país por meio da estratégia Rede Cegonha.

Entretanto, essas estratégias não foram consensuais, e assistimos a resistências
a essas
mudanças, desencadeadas por alguns órgãos representantes de classe, muitas vezes
com
medidas arbitrárias que extrapolavam suas competências.

As iniciativas governamentais, as experiências exitosas de alguns serviços, a
realização
de eventos científicos com exposição de modelos de outros países, foram
acompanhadas por
um movimento social de mulheres, pleiteando mudanças na forma como os partos são
assistidos no país. Como reflexo, vem se observando um número crescente de
grupos de
discussão nas redes sociais, manifestações em passeatas e atos públicos,
presença do
parto como tema na mídia e o crescimento da demanda das mulheres por parto
domiciliar.

Nesses 20 anos, assistimos também a um aumento da preferência da mulher pela
cesariana
desde o início da gestação. Tal preferência se distingue conforme sua parição,
antecedente de cesariana e financiamento privado do parto. As mulheres atendidas
no
setor privado têm melhor condição socioeconômica, mais acesso à assistência
pré-natal
adequada e são acompanhadas pelo mesmo profissional médico na gestação.
Destaca-se que,
na assistência privada, a cesariana atinge quase 90\% dos partos, a maioria
deles sem
indicação clínica\textsuperscript{[}\textsuperscript{20}\textsuperscript{]}\textsuperscript{,}\textsuperscript{[}\textsuperscript{21}\textsuperscript{]}. A opção pela cesariana não é apenas
por influência do médico, mas de todo um contexto cultural que se criou em
relação ao
risco do parto normal, como a ausência de garantias de que a mulher terá
controle sobre
seu processo de parturição, a vaga na maternidade, o medo da dor, ter o médico
conhecido
na hora do parto, dentre outros argumentos.

Ressalta-se que maior uso de intervenções obstétricas e de cesarianas em
mulheres
atendidas no setor privado tem sido observado em diversos países do mundo, e não
apenas
no Bra-sil\textsuperscript{[}\textsuperscript{22}\textsuperscript{]}\textsuperscript{,}\textsuperscript{[}\textsuperscript{23}\textsuperscript{]}\textsuperscript{,}\textsuperscript{[}\textsuperscript{24}\textsuperscript{]}.

Finalmente, para concluir, retomamos as colocações da Maria A. Gomes que
salienta os
movimentos em curso no Brasil para uma mudança no modelo de atenção obstétrica
no país.
Os bons resultados encontrados na região Sudeste, onde as cidades de Belo
Horizonte e do
Rio de Janeiro desenvolvem continuada implantação de boas práticas na atenção
perinatal,
sobressaem. No Rio de Janeiro e em São Paulo, recentemente, mulheres detentoras
de
planos de saúde privados têm cada vez mais procurado maternidades públicas para
parir,
uma vez que apresentam a opção do parto normal, cada vez mais escassa no serviço
privado.

Aspecto relevante com o qual compartilhamos trazido por ela é o investimento
nacional na
qualificação de enfermeiras obstétricas/obstetrizes e a recente mudança de
atitude da
\textsc{febrasgo}, que passou a defender práticas reconhecidas por aumentar a satisfação
da
parturiente com o parto normal, apoio aos \textsc{cnp} e à inserção da enfermagem
obstétrica/obstetrizes na atenção ao parto também na rede privada. Esses fatos
sinalizam
para a valorização da obstetrícia baseada em evidências científicas, projetando
um
cenário de esperança para a mudança do modelo.

Os resultados mostrados neste estudo, de uma menor proporção de intervenções
obstétricas
na Região Sudeste, podem estar indicando uma tendência que se consolidará com o
aumento
da participação das mulheres na definição das políticas de atenção ao parto e
com a
melhoria da qualidade dessa assistência no Sistema Público de Saúde.

Não foi por acaso que a pesquisa \textit{Nascer no Brasil}
ocorreu nesse momento,
com financiamento público, para mostrar, pela primeira vez, em nível nacional,
um
panorama da atenção ao parto e nascimento no país. Para mudar, todos os atores,
instituições, organizações não governamentais, profissionais de saúde,
movimentos
sociais, mães e famílias precisam, primeiramente, conhecer essa realidade e se
incomodar
com ela. Esperamos que a pesquisa \textit{Nascer no Brasil}
contribua no
cumprimento dessa etapa.

\section*{}
\begin{itemize}

\item[1] Faundes A, Cecatti JG. A operação Cesárea no Brasil. Incidência,
tendências, causas, consequências e propostas de ação. Cad Saúde Pública 1991;
7:150-73.

\item[2] Dias \textsc{mab}, Deslandes SF. Cesarianas: percepção de risco e sua
indicação pelo obstetra em uma maternidade pública no Município do Rio de
Janeiro. Cad Saúde Pública 2004; 20:109-16.

\item[3] Dias \textsc{mab}, Domingues \textsc{rmsm}, Schilithz \textsc{aoc}, Nakamura-Pereira M, Diniz
\textsc{csg}, Brum IR, et al. Incidência do near miss materno no parto e pós- parto
hospitalar: dados da pesquisa Nascer no Brasil. Cad Saúde Pública 2014; 30
Suppl:169-81.

\item[4] Rossi AC, D’Addario V. Maternal morbidity following a trial of
labor after cesarean section vs elective repeat cesarean delivery: a systematic
review with metaanalysis. Am J Obstet Gynecol 2008; 199:
224-31.

\item[5] American College of Obstetricians and Gynecologists. \textsc{acog} Practice
bulletin no. 115: Vaginal birth after previous cesarean delivery. Obstet Gynecol
2010; 116(2 Pt 1):450-63.

\item[6] Verheijen EC, Raven JH, Hofmeyr GJ. Fundal pressure during the
second stage of labour. Cochrane Database Syst Rev 2009;
(4):CD006067.

\item[7] Ministério da Saúde. Pesquisa Nacional de Demografia e Saúde da
Criança e da Mulher – \textsc{pnds} 2006: dimensões do processo reprodutivo e da saúde da
criança. Brasília: Ministério da Saúde; 2009. (Série G Estatística e Informação
em Saúde).

\item[8] Sandall J, Soltani H, Gates S, Shennan A, Devane D. Midwife-led
continuity models versus other models of care for childbearing women. Cochrane
Database Syst Rev 2013; 8:CD004667.

\item[9] Ministério da Saúde. Portaria GM
n\textsuperscript{o}
904, de 29 de maio de 2013. Diário
Oficial da União 2013; 31 mai.

\item[10] Gibbons L, Belizan JM, Lauer JS, Betrán AP, Merialdi M, Althabe F.
The Global Numbers and Costs of Additionally Needed and Unnecessary Caesarean
Sections Performed per Year: Overuse as a Barrier to Universal Coverage. Geneva:
World Health Organization; 2010. (World Health Report).

\item[11] Lawrence A, Lewis L, Hofmeyr GJ, Styles C. Maternal positions and
mobility during first stage labour. Cochrane Database Syst Rev 2013;
8:CD003934.

\item[12] Singata M, Tranmer J, Gyte GM. Restricting oral fluid and food
intake during labour. Cochrane Database Syst Rev 2013;
8:CD003930.

\item[13] Kemp E, Kingswood CJ, Kibuka M, Thornton JG. Position in the
second stage of labour for women with epidural anaesthesia. Cochrane Database
Syst Rev 2013; 1:CD008070.

\item[14] Diniz \textsc{csg}. Humanização da assistência ao parto no Brasil: os
muitos sentidos de um movimento. Ciênc Saúde Coletiva 2005;
10:627-37.

\item[15] Behague DP, Victora CG, Barros FC. Consumer demand for caesarean
sections in Brazil: informed decision making, patient choice, or social
inequality? A population based birth cohort study linking ethnographic and
epidemiological methods. \textsc{bmj} 2002; 324:942-5.

\item[16] Barros AJ, Santos IS, Matijasevich A, Domingues MR, Silveira M,
Barros FC, et al. Patterns of deliveries in a Brazilian birth cohort: almost
universal cesarean sections for the better-off. Rev Saúde Pública 2011;
45:635-43.

\item[17] Diniz SG, Chacham AS. “The cut above” and “the cut below”: the
abuse of caesareans and episiotomy in Sao Paulo, Brazil. Reprod Health Matters
2004; 12:100-10.

\item[18] Brasil. Portaria n\textsuperscript{o}
1.459/GM/MS,
de 24 de junho de 2011, que instituiu, no âmbito do \textsc{sus}, a Rede Cegonha. Diário
Oficial da União 2011; 27 jun.

\item[19] Dahlen HG, Kennedy HP, Anderson CM, Bell AF, Clark A, Foureur M,
et al. The \textsc{epiic} hypothesis: intrapartum effects on the neonatal epigenome and
consequent health outcomes. Med Hypotheses 2013; 80:656-62.

\item[20] Leal MC, Pereira \textsc{ape}, Domingues \textsc{rmsm}, Theme Filha MM, Dias \textsc{mab},
Nakamura-Pereira M, et al. Intervenções obstétricas durante o trabalho de parto
e parto em mulheres brasileiras de risco habitual. Cad Saúde Pública 2014; 30
Suppl: 17-47.

\item[21] Domingues \textsc{rmsm}, Dias \textsc{mab}, Nakamura-Pereira M, Torres JA, d’Orsi E,
Pereira \textsc{ape}, et al. Processo de decisão pelo tipo de parto no Brasil: da
preferência inicial das mulheres à via de parto final. Cad Saúde Pública 2014;
30 Suppl:101-16.

\item[22] Dahlen HG, Tracy S, Tracy M, Bisits A, Brown C, Thornton C. Rates
of obstetric intervention among low-risk women giving birth in private and
public hospitals in \textsc{nsw}: a population-based descriptive study. \textsc{bmj} Open 2012;
2:e001723.

\item[23] Coulm B, Le Ray C, Lelong N, Drewniak N, Zeitlin J, Blondel B.
Obstetric interventions for low-risk pregnant women in France: do maternity unit
characteristics make a difference? Birth 2012; 39:183-91.

\item[24] Lutomski JE, Murphy M, Devane D, Meaney S, Greene RA. Private
health care coverage and increased risk of obstetric intervention. \textsc{bmc} Pregnancy
Childbirth 2014; 14:13.

\end{itemize}

\end{document}
