% Generated by jats2tex@0.11.1.0
\documentclass{article}
\usepackage{scielo}

\newcommand{\journalid}{Cad Saude Publica}
\newcommand{\journaltitle}{Cadernos de Saúde Pública}
\newcommand{\abbrevjournaltitle}{Cad. Saúde
Pública}
\newcommand{\issnppub}{0102-311X}
\newcommand{\issnepub}{1678-4464}
\newcommand{\publishername}{Fundação Oswaldo Cruz}
\newcommand\articledoi{\textsc{doi} 10.1590/0102-311X00151513}
\def\subject{\textsc{debate}}
\title{Intervenções obstétricas durante o trabalho de parto e
parto em mulheres brasileiras de risco habitual\titlegroup{}}
\author[{1}]{Leal, Maria do Carmo}
\author[{1}]{Pereira, Ana Paula Esteves}
\author[{2}]{Domingues, Rosa Maria Soares Madeira}
\author[{1}]{Filha, Mariza Miranda Theme}
\author[{3}]{Dias, Marcos Augusto Bastos}
\author[{1}]{Nakamura-Pereira, Marcos}
\author[{1}]{Bastos, Maria Helena}
\author[{1}]{Gama, Silvana Granado Nogueira da}
\affil[1]{Fundação Oswaldo Cruz}
\affil[2]{Fundação Oswaldo Cruz}
\affil[3]{Fundação Oswaldo Cruz}
\def\authornotes{Correspondência M. C. Leal. Escola Nacional de Saúde Pública
Sergio
Arouca, Fundação Oswaldo Cruz. Rua Leopoldo Bulhões 1480, Rio de Janeiro, RJ
21041-210, Brasil. duca@fiocruz.br
Colaboradores

M. C. Leal foi responsável pela concepção do estudo, desenho e coordenação,
análise dos dados, elaboração e revisão do manuscrito. A. P. E. Pereira
contribuiu na análise dos dados, elaboração e revisão do texto. R. M. S. M.
Domingues, M. M. Theme Filha, M. A. B. Dias, M. Nakamura-Pereira, M. H.
Bastos e S. G. N. Gama contribuíram na elaboração, revisão do artigo e
aprovação da versão final.}
\date{ 08 2014}
\def\volume{30}
\def\issue{Suppl 1}
\def\fpage{S17}
\def\lpage{S32}
\newcommand{\cclicense}{\ccbync}
\newcommand{\kwdgroup}{Práticas de Saúde Pública, Saúde Materno-Infantil,
Trabalho de Parto, Parto}
\newcommand{\kwdgroupes}{Práctica de Salud Pública, Salud Materno-Infantil,
Trabajo de Parto, Parto}
%%% Nota %%%%%%%%%%%%%%%%%%%%%%%%%%%%%%%%%%%%%%%%%%%%%%%%%%%%%%%%
\expandafter\newcommand\csname \endcsname{
Financiamento}
%%% Nota %%%%%%%%%%%%%%%%%%%%%%%%%%%%%%%%%%%%%%%%%%%%%%%%%%%%%%%%
\expandafter\newcommand\csname \endcsname{
Conselho Nacional de Desenvolvimento Científico e Tecnológico (\textsc{cnp}q);
Departamento de Ciência e Tecnologia, Secretaria de Ciência, Tecnologia de
Insumos Estratégicos, Ministério da Saúde; Escola Nacional de Saúde Pública
Sergio Arouca, Fundação Oswaldo Cruz (Projeto \textsc{inova}); e Fundação de Amparo à
Pesquisa do Estado do Rio de Janeiro (Faperj).}

\begin{document}
\selectlanguage{portuges}
\newcommand{\lingua}{Português}
\maketitle
\tableofcontents

Intervenciones obstétricas durante el trabajo de parto y parto en
mujeres brasileñas de bajo riesgo
\begingroup
\renewcommand{\section}[1]{\subsection*{#1}}

\begin{abstract}

Este artigo avaliou o uso das boas práticas (alimentação, deambulação, uso de
métodos não farmacológicos para alívio da dor e de partograma) e de intervenções
obstétricas na assistência ao trabalho de parto e parto de mulheres de risco
obstétrico habitual. Foram utilizados dados da pesquisa Nascer no Brasil, estudo
de base hospitalar realizada em 2011/2012, com entrevistas de 23.894 mulheres.
As boas práticas durante o trabalho de parto ocorreram em menos de 50\% das
mulheres, sendo menos frequentes nas regiões Norte, Nordeste e Centro-oeste. O
uso de ocitocina e amniotomia foi de 40\%, sendo maior no setor público e nas
mulheres com menor escolaridade. A manobra de Kristeller, episiotomia e
litotomia foram utilizada, em 37\%, 56\% e 92\% das mulheres, respectivamente. A
cesariana foi menos frequente nas usuárias do setor público, não brancas, com
menor escolaridade e multíparas. Para melhorar a saúde de mães e crianças e
promover a qualidade de vida, o Sistema Único de Saúde (\textsc{sus}) e, sobretudo o
setor privado, necessitam mudar o modelo de atenção obstétrica promovendo um
cuidado baseado em evidências científicas.

\ifdef{\kwdgroup}{\iflanguage{portuges}{\medskip\noindent\textbf{Palavras-chave:} \kwdgroup}{}}{}
\ifdef{\kwdgroupen}{\iflanguage{english}{\medskip\noindent\textbf{Keywords:}
\kwdgroupen}{}}{}
\ifdef{\kwdgroupes}{\iflanguage{spanish}{\medskip\noindent\textbf{Palavras
claves:} \kwdgroupes}{}}{}
\ifdef{\kwdgroupfr}{\iflanguage{french}{\medskip\noindent\textbf{Mots clés:}
\kwdgroupfr}{}}{}
\end{abstract}
\endgroup

\begingroup
\renewcommand{\section}[1]{\subsection*{#1}}
\begin{otherlanguage}{spanish}

\begin{abstract}

Se evaluó el uso de buenas prácticas (alimentación, métodos no farmacológicos
para el alivio del dolor, caminar y el uso del partograma), además de las
intervenciones obstétricas durante el parto, en mujeres con un riesgo obstétrico
habitual. Los datos provienen del estudio Nacer en Brasil, una cohorte de base
hospitalaria realizada en 2011-2012, con entrevistas a 23.894 mujeres. Las
buenas prácticas durante el parto se produjeron en menos de un 50\% y fueron
menos frecuentes en el Norte, Nordeste y Centro-oeste. El uso de oxitocina y
amniotomía fue del 40\%, principalmente, en el sector público y en las mujeres
de
menor nivel educativo. La presión fúndica uterina, episiotomía y litotomía
fueron utilizados en: un 37\%, 56\% y 92\% respectivamente. La cesárea fue menos
frecuente en mujeres que son usuarias del sector público, no blancas, con menor
nivel educativo y multíparas. Para mejorar la salud de las madres y los niños, y
con el fin de promover la calidad de vida, el Sistema Único de Salud (\textsc{sus}), y
sobre todo el sector privado, necesitará cambiar el modelo de atención
obstétrica mediante la adopción de evidencias científicas.

\ifdef{\kwdgroupes}{\medskip\noindent\textbf{Palavras claves:} \kwdgroupes}{}
\end{abstract}
\end{otherlanguage}
\endgroup
\section{Introdução}

Há evidências científicas de que várias práticas na assistência à gestação e ao
parto
são promotoras de melhores resultados obstétricos e são efetivas para a redução
de
desfechos perinatais negativos. Fatores da saúde materna que atuam durante o
período
gestacional influenciam os resultados da gravidez, e a assistência pré-natal de
qualidade contribui para a redução de danos à gestante e ao recém-nascido. Da
mesma
forma, uma parcela importante das complicações que podem ocorrer ao longo do
trabalho de parto e no momento do parto pode ser reduzida por cuidado obstétrico
apropriado, realizado com o uso adequado de tecnologia. Por outro lado, o uso
inadequado de tecnologias ou a realização de intervenções desnecessárias pode
trazer
prejuízos para a mãe e seu concepto\textsuperscript{[}\textsuperscript{1}\textsuperscript{]}.

Embora o Brasil tenha atingido uma elevada cobertura na assistência pré-natal e
a
taxa de parto hospitalar tenha sido maior que 98\% em 2010, ainda persistem
elevadas
a razão da mortalidade materna (68,2/100 mil nascidos vivos) e a de mortalidade
perinatal, sugerindo problemas na qualidade da atenção materna e perinatal
(Departamento de Informática do \textsc{sus} – \textsc{datasus}.
http://tabnet.datasus.gov.br/cgi/idb2011/C03b.htm).

Várias pesquisas nacionais foram realizadas na última década para avaliar os
cuidados
pré-natais, época do início, número de consultas, procedimentos realizados,
dentre
outros\textsuperscript{[}\textsuperscript{2}\textsuperscript{]}, mas, até hoje, nenhum
estudo foi realizado, em nível nacional, para descrever os processos e
procedimentos
empregados na atenção ao parto e nascimento no Brasil. Desconhece-se a
frequência de
utilização das boas práticas de atenção e de intervenções obstétricas, bem como
a
sua distribuição por área geográfica, condição social das mulheres, paridade e
por
tipo de serviço de saúde: público e privado.

Iniciativas do Ministério da Saúde como a elaboração de manuais técnicos\textsuperscript{[}\textsuperscript{3}\textsuperscript{]}
e outros materiais educativos para
os profissionais que atendem à gestação e ao parto, embora relevantes, têm se
mostrado insuficientes para reverter o modelo de atenção obstétrica do Brasil
que é
reconhecido como extremamente intervencionista, tendo como expressão maior disso
as
taxas de cesárea mais elevadas do mundo\textsuperscript{[}\textsuperscript{4}\textsuperscript{]}.

Questiona-se se a persistência desses maus indicadores maternos e perinatais do
Brasil são reflexos da baixa qualidade da atenção obstétrica, dada a elevada
cobertura da assistência ofertada pelo sistema de saúde.

O objetivo deste estudo é descrever as boas práticas de atenção ao parto
(alimentação
e movimentação durante o trabalho de parto e parto, uso de métodos não
farmacológicos para alívio da dor e monitoramento do trabalho de parto pelo
partograma) e as intervenções obstétricas (uso de cateter venoso, ocitocina para
acelerar o trabalho de parto, amniotomia, analgesia peridural, manobra de
Kristeller, episiotomia e cesariana) realizadas em mulheres de risco obstétrico
habitual que pariram em uma amostra representativa de hospitais brasileiros com
500
ou mais partos/ano.

\section{Métodos}

\textit{Nascer no Brasil}
foi um estudo nacional de base hospitalar composto
por puérperas e seus recém-nascidos, realizado no período de fevereiro de 2011 a
outubro de 2012.

A amostra foi selecionada em três estágios. O primeiro foi composto por
hospitais
com 500 ou mais partos/ano em 2007, estratificado pelas cinco macrorregiões do
país, localização (capital ou não capital), e por tipo de hospital (privado,
público e misto). Em cada estrato os hospitais foram selecionados com
probabilidade proporcional ao número de partos/ano. No segundo estágio, um
método de amostragem inversa foi utilizado para selecionar o número de dias
(mínimo de sete) necessários para alcançar 90 puérperas em cada hospital. O
terceiro foi composto pelas puérperas elegíveis – por exemplo, as que tiveram
parto hospitalar de um nascido vivo, independentemente da idade gestacional e
peso, ou um nascido morto com mais de 500g ou idade gestacional maior ou igual a
22 semanas.

Os pesos amostrais foram baseados na probabilidade inversa de inclusão na
amostra. Para assegurar que as estimativas dos totais fossem equivalentes ao
número de nascimentos em hospitais com 500 ou mais partos/ano em 2011, um
processo de calibração foi usado em cada estrato selecionado. Os resultados
apresentados são estimativas para a população de estudo (2.227.476) baseadas na
amostra de 23.940 puérperas.

A amostra em cada estrato foi calculada considerando o desfecho cesariana,
estimado em 46,6\% (dados de 2007), com 5\% de significância estatística para
detectar diferenças de pelo menos 14\% entre os tipos de hospitais (público,
misto, privado), com um poder de 95\% e efeito de desenho de 1,3, resultando em
uma amostra mínima de 450 por estrato e um total de 266 hospitais espalhados em
191 municípios.

Na primeira fase do estudo foram realizadas entrevistas face a face com as
puérperas durante a internação hospitalar, e extraídos dados do prontuário da
mulher e do recém-nato, utilizando-se formulários eletrônicos. As entrevistas
foram conduzidas das primeiras 24 horas após o parto. Dados do prontuário
hospitalar foram coletados após a alta (ou óbito). No caso de estadia hospitalar
prolongada, os dados do prontuário da mulher foram coletados após 42 dias de
hospitalização (após o parto) e para os recém-natos após o
28\textsuperscript{o}
dia (período neonatal). No caso de
transferência hospitalar da mulher e/ou do recém-nato, os dados foram coletados
do hospital, mesmo quando este hospital não fazia parte da amostra dos
estabelecimentos de saúde sorteados. Recusas ou altas precoces foram repostas
por uma nova seleção de puérpera no mesmo hospital. Cartões de pré-natal, quando
disponíveis, foram fotografados e posteriormente os dados relevantes foram
extraídos e digitados em formulário eletrônicos. Entrevistas telefônicas de
\textit{follow-up}
foram realizadas antes dos seis meses e aos 12
meses após o parto para obter informações sobre desfechos maternos e neonatais.
Todo o trabalho de campo foi conduzido por profissionais ou estudantes
universitários da área da saúde, com a supervisão do grupo de pesquisa.
Informação detalhada sobre o processo de amostragem e coleta de está em outros
artigos\textsuperscript{[}\textsuperscript{5}\textsuperscript{]}\textsuperscript{,}\textsuperscript{[}\textsuperscript{6}\textsuperscript{]}.

Neste artigo analisamos somente os dados de mulheres de risco obstétrico
habitual. Elas foram definidas como mulheres sem história de diabetes ou
hipertensão arterial gestacional ou pré-gestacional, não obesas (\textsc{imc} < 30),
\textsc{hiv} negativas, com idade gestacional entre 37-41 semanas ao nascer, gravidez
única, com feto em apresentação cefálica, com peso ao nascer entre 2.500g e
4.499g e entre o 5\textsuperscript{o}
e
95\textsuperscript{o}
centil de peso ao nascer por idade
gestacional. Em analogia a Dahlen et al.\textsuperscript{[}\textsuperscript{7}\textsuperscript{]}, esses fatores neonatais foram considerados como
uma \textit{proxy}
de risco obstétrico habitual e acredita-se que eles
tenham sido capazes de excluir gestantes com outras patologias não incluídas no
critério.

No total, a amostra do \textit{Nascer no Brasil}
foi de 23.940, das quais
56,8\% foram classificadas como risco obstétrico habitual. Na análise do tipo de
parto todas com risco obstétrico habitual foram incluídas. Para as intervenções
durante o trabalho de parto, as mulheres que não entraram neste procedimento
foram excluídas (28,1\%), permanecendo na análise 40,8\% de todas as mulheres
entrevistadas. Finalmente, para intervenções durante o parto vaginal, todas as
cesáreas foram excluídas, 45,5\%, permanecendo na análise 30,9\% de todas as
mulheres entrevistadas.

Uma vez que o número de puérperas selecionadas para essa análise foi menor que a
amostra total do estudo, um cálculo amostral \textit{post-hoc}
foi
realizado. Considerando uma prevalência de 50\% de puérperas recebendo uma
intervenção durante o parto e o nível de significância de 5\%, o menor tamanho
de
amostra utilizada neste artigo (de 6.740 para analisar intervenções durante o
parto vaginal) teve um poder de 90\% para detectar diferenças de pelo menos
3,5\%.

As variáveis de exposição estudadas foram: Região (Norte, Nordeste, Sudeste,
Sul,
Centro- oeste); fonte de pagamento (pública; privada); idade (< 20; 20-34; 35
ou mais); anos de estudos (7 ou menos; 8-10; 11-14; 15 ou mais); cor da
pele/raça autorrelatada (segundo o Instituto Brasileiro de Geografia e
Estatística: branca; preta; parda; amarela; indígena) e número de partos
anteriores (0; 1-2; 3 ou mais).

Mulheres com parto em unidades públicas e mulheres com parto em unidades mistas
que não foram pagos por plano de saúde foram classificadas como tendo “fonte de
pagamento pública”. Mulheres com parto pago por plano de saúde, tendo o parto
ocorrido em unidades mistas ou privadas, e mulheres com parto em unidades
privadas, independente do parto ter sido pago ou não por plano de saúde, foram
classificadas como tendo “fonte de pagamento privada”.

Os desfechos avaliados foram boas práticas e intervenções obstétricas durante o
trabalho de parto e parto. Nós consideramos como boas práticas: ingesta de
líquidos ou alimentos durante o trabalho de parto, uso de métodos não
farmacológicos para alívio da dor, mobilidade durante o primeiro estágio do
trabalho de parto e monitoramento do progresso do trabalho de parto pelo
partograma. Outra boa prática é a presença de acompanhante durante todo o
período de hospitalização, a qual será objeto de estudo em outro artigo deste
mesmo número temático\textsuperscript{[}\textsuperscript{8}\textsuperscript{]}. Nós
consideramos como intervenções durante o trabalho de parto: uso de cateter
venoso, ocitocina para aceleração do trabalho de parto, amniotomia (para
mulheres com bolsa íntegra na admissão) analgesia raque/epidural; e finalmente
como intervenções durante o parto: litotomia, manobra de Kristeller e
episiotomia para parto vaginal, além de cesariana. Informações sobre alimentação
durante o trabalho de parto, uso de métodos não farmacológicos para alívio da
dor, mobilidade durante o primeiro estágio do trabalho de parto, uso de cateter
venoso, litotomia e manobra de Kristeller foram reportadas pelas puérperas na
entrevista. As informações sobre uso de partograma, ocitocina para aceleração do
trabalho de parto, aminiotomia, analgesia raqui/epidural e tipo de parto foram
coletadas dos prontuários médicos.

Todas as proporções, testes, intervalos de confiança e modelos foram estimados
considerando a característica complexa da amostra: estratos, conglomerados e
pesos. Modelos de regressão logística múltipla foram desenvolvidos para
identificar as características sociodemográficas associadas aos desfechos. Todas
as variáveis de exposição foram incluídas além da localização dos hospitais
(capital, outras cidades). Essas variáveis foram cuidadosamente escolhidas por
representarem diferentes dimensões sociodemográficas com potencial para
interferir na incidência dessas práticas. Nós adotamos o nível de significância
de 5\%. O programa estatístico utilizado foi \textsc{ibm} \textsc{spss}, versão 19.0 (\textsc{ibm} Corp.,
Armonk, Estados Unidos).

Esta pesquisa foi orientada pela \textit{Resolução
n}\textsuperscript{\textit{o}}
\textit{
196/1996}
do Conselho Nacional de Saúde, que define as recomendações
e procedimentos padrão para pesquisas em seres humanos, tendo sido protocolada
no Comitê de Ética em Pesquisa \textsc{ensp} n\textsuperscript{o}
92/10.
Todos os diretores de instituições e todas as puérperas assinaram o
consentimento informado.

\section{Resultados}

Menos de um terço do grupo de risco obstétrico habitual se alimentou durante o
trabalho de parto e utilizou procedimentos não farmacológicos para alívio da
dor.
Aproximadamente 45\% delas referiram ter se movimentado durante o trabalho de
parto e
tiveram o progresso monitorado pelo partograma. Em mais de 70\% das mulheres foi
realizada a punção venosa periférica, enquanto o uso de ocitocina e a
aminiotomia
ocorreu em cerca de 40\% delas, e a analgesia raqui/epidural em cerca de 30\%.
Durante
o parto, a incidência da posição de litotomia, manobra de Kristeller e
episiotomia
foram de 92\%, 37\% e 56\%, respectivamente. Com exceção de alimentar-se durante
o
trabalho de parto, todas as demais boas práticas obstétricas, bem como as
intervenções durante o trabalho de parto e parto, tiveram frequências mais
elevadas
no grupo de risco obstétrico habitual: ocitocina, amniotomia, manobra de
Kristeller
e episiotomia. Do total de partos, 48,1\% foram vaginais, 5\% vaginais sem
nenhuma
intervenção durante o trabalho de parto e parto (parto normal sem intervenção) e
51,9\% cesariana. Considerando-se somente as de risco obstétrico habitual, a
taxa de
cesárea decresceu para 45,5\% e o parto normal sem intervenção aumentou para
5,6\%
(Tabela 1).

Tabela 1Incidência de boas práticas e intervenções durante o trabalho de
parto e parto. Brasil, 2011.\begin{table}
%\begin{adjustbox}{width=1.1\textwidth}
\small\centering
\begin{tabulary}{\linewidth}{ C C C C C }
\hline... & Risco obstétrico habitual (\%) & Não risco obstétrico habitual (\%) &
Todas as mulheres (\%) & Valor de p *\\ \hline
Para mulheres que entraram em trabalho de parto
& ...
& ...
& ...
& ...
\\ \hline

Boas práticas durante o trabalho de parto
& ...
& ...
& ...
& ...
\\ \hline

Alimentação
& 25,6
& 24,5
& 25,2
& 0,408
\\ \hline

Movimentação
& 46,3
& 41,1
& 44,3
& < 0,001
\\ \hline

Procedimentos não farmacológios para alívio da dor
& 28,0
& 24,7
& 26,7
& 0,012
\\ \hline

Uso de partograma
& 44,2
& 36,9
& 41,4
& < 0,001
\\ \hline

Intervenções durante o trabalho de parto
& ...
& ...
& ...
& ...
\\ \hline

Cateter venoso periférico
& 73,8
& 76,7
& 74,9
& 0,043
\\ \hline

Ocitocina
& 38,2
& 33,3
& 36,4
& 0,001
\\ \hline

Analgesia epidural
& 31,5
& 37,8
& 33,9
& < 0,001
\\ \hline

Amniotomia **
& 40,7
& 36,4
& 39,1
& < 0,001
\\ \hline

Para mulheres com parto vaginal
& ...
& ...
& ...
& ...
\\ \hline

Intervenções durante o parto
& ...
& ...
& ...
& ...
\\ \hline

Litotomia
& 91,7
& 91,8
& 91,7
& 0,946
\\ \hline

Manobra de Kristeler
& 37,3
& 33,9
& 36,1
& 0,017
\\ \hline

Episiotomia
& 56,1
& 48,6
& 53,5
& < 0,001
\\ \hline

Para todas as mulheres
& ...
& ...
& ...
& ...
\\ \hline

Cesariana
& 45,5
& 60,3
& 51,9
& < 0,001
\\ \hline

Parto natural ***
& 5,6
& 4,2
& 5,0
& 0,845
\\ \hline

\end{tabulary}
%\end{adjustbox}
\caption*{\footnotesize }
\caption{}
\end{table}

* Valor de p de teste qui-quadrado na comparação entre risco
obstétrico habitual e não risco obstétrico habitual;

** Também foram excluídas as mulheres com ruptura espontânea de
membranas anterior à hospitalização;

*** Parto vaginal sem qualquer intervenção durante o trabalho de
parto e parto.

* Valor de p de teste qui-quadrado na comparação entre risco
obstétrico habitual e não risco obstétrico habitual;

** Também foram excluídas as mulheres com ruptura espontânea de
membranas anterior à hospitalização;

*** Parto vaginal sem qualquer intervenção durante o trabalho de
parto e parto.

Na Região Sudeste foi mais frequente a prática de alimentação durante o trabalho
de
parto, enquanto o uso de procedimentos não farmacológicos para alívio da dor e
uso
de partograma foram menos frequentes nas regiões Norte, Nordeste e Centro-oeste.
Em
contraste, mulheres da Região Sul tiveram maior chance de se movimentarem
durante o
trabalho de parto quando comparadas às da região Sudeste. Se por um lado,
mulheres
sem plano de saúde – que realizaram seu parto em hospitais públicos – tiveram
maior
chance para todas as boas práticas, por outro, a maior escolaridade e a cor da
pele/raça não se associaram com nenhuma destas boas práticas. Mulheres mais
velhas
usaram menos os métodos não farmacológicos para o alívio da dor e trabalho de
parto
monitorado por partograma, enquanto que todas as boas práticas foram mais
frequentes
em primíparas, exceto o uso de partograma (Tabelas
2 e 3).

Tabela 2Incidência de boas práticas durante o trabalho de parto em mulheres
de risco obstétrico habitual, de acordo com características
sociodemográficas. Brasil, 2011.\begin{table}
%\begin{adjustbox}{width=1.1\textwidth}
\small\centering
\begin{tabulary}{\linewidth}{ C C C C C }
\hline... & Alimentação durante o trabalho de parto (\%) & Movimentação durante o
trabalho de parto (\%) & Uso de procedimentos não farmacológios para alívio da
dor
(\%) & Progresso do trabalho de parto monitorado por partograma
(\%)\\ \hline
Região
& ...
& ...
& ...
& ...
\\ \hline

Norte
& 18,4
& 54,2
& 17,7
& 20,7
\\ \hline

Nordeste
& 16,6
& 39,1
& 19,1
& 30,4
\\ \hline

Sudeste
& 35,7
& 47,0
& 37,5
& 59,4
\\ \hline

Sul
& 22,0
& 56,3
& 30,5
& 51,1
\\ \hline

Centro-oeste
& 18,4
& 45,1
& 17,6
& 32,0
\\ \hline

Fonte de pagamento
& ...
& ...
& ...
& ...
\\ \hline

Pública
& 27,2
& 48,1
& 29,3
& 46,1
\\ \hline

Privada
& 10,4
& 29,1
& 15,3
& 25,7
\\ \hline

Idade (anos)
& ...
& ...
& ...
& ...
\\ \hline

10-19
& 27,3
& 49,3
& 31,7
& 48,1
\\ \hline

20-34
& 25,7
& 45,9
& 27,7
& 43,6
\\ \hline

≥ 35
& 18,9
& 38,9
& 16,5
& 35,4
\\ \hline

Anos de estudo
& ...
& ...
& ...
& ...
\\ \hline

≤ 7
& 21,4
& 43,2
& 22,8
& 41,4
\\ \hline

8-10
& 26,5
& 48,1
& 30,1
& 48,6
\\ \hline

11-14
& 29,4
& 48,4
& 31,7
& 45,1
\\ \hline

≥ 15
& 17,9
& 37,8
& 18,0
& 27,4
\\ \hline

Cor da pele
& ...
& ...
& ...
& ...
\\ \hline

Branca
& 29,7
& 48,5
& 31,5
& 47,5
\\ \hline

Preta
& 26,0
& 46,0
& 27,5
& 45,7
\\ \hline

Parda
& 23,7
& 45,6
& 26,3
& 42,3
\\ \hline

Amarela
& 26,0
& 37,4
& 30,2
& 40,4
\\ \hline

Indígena
& 17,5
& 38,7
& 23,2
& 56,3
\\ \hline

Paridade
& ...
& ...
& ...
& ...
\\ \hline

0
& 28,9
& 48,3
& 32,9
& 44,5
\\ \hline

1-2
& 23,2
& 44,2
& 24,6
& 45,4
\\ \hline

≥ 3
& 19,9
& 45,3
& 18,4
& 37,8
\\ \hline

Brasil
& 25,6
& 46,3
& 28,0
& 44,2
\\ \hline

\end{tabulary}
%\end{adjustbox}
\caption*{\footnotesize }
\caption{}
\end{table}

Tabela 3Odds ratio (OR) brutos e ajustados * para os determinantes
sociodemográficos das boas práticas durante o trabalho de parto em
mulheres de risco obstétrico habitual. Brasil, 2011.\begin{table}
%\begin{adjustbox}{width=1.1\textwidth}
\small\centering
\begin{tabulary}{\linewidth}{ C C C C C C C C C C C C C C C C C }
\hline... & Alimentação durante o trabalho de parto & Movimentação durante o trabalho
de parto & Uso de procedimentos não farmacológios para alívio
da dor & Progresso do trabalho de parto monitorado por
partograma\\ \hline
OR bruta & OR ajustada & IC95\% & OR bruta & OR ajustada & IC95\% & OR bruta &
OR ajustada & IC95\% & OR bruta & OR ajustada & IC95\%\\ \hline
Região [Ref: Sudeste]
& ...
& ...
& ...
& ...
& ...
& ...
& ...
& ...
& ...
& ...
& ...
& ...
\\ \hline

Norte
& 0,40
& 0,39
& 0,23-0,66
& 1,33
& 1,30
& 0,81-2,10
& 0,36
& 0,34
& 0,18-0,67
& 0,18
& 0,13
& 0,06-0,29
\\ \hline

Nordeste
& 0,36
& 0,36
& 0,24-0,54
& 0,72
& 0,74
& 0,53-1,03
& 0,39
& 0,40
& 0,27-0,58
& 0,30
& 0,27
& 0,16-0,43
\\ \hline

Sul
& 0,51
& 0,48
& 0,30-0,78
& 1,45
& 1,44
& 1,03-2,03
& 0,73
& 0,73
& 0,49-1,10
& 0,71
& 0,75
& 0,46-1,23
\\ \hline

Centro-oeste
& 0,40
& 0,38
& 0,22-0,68
& 0,93
& 0,95
& 0,63-1,44
& 0,35
& 0,34
& 0,20-0,59
& 0,32
& 0,24
& 0,13-0,45
\\ \hline

Fonte de pagamento ** [Ref: privada]
& ...
& ...
& ...
& ...
& ...
& ...
& ...
& ...
& ...
& ...
& ...
& ...
\\ \hline

Pública
& 3,23
& 4,52
& 3,04-6,71
& 2,26
& 2,48
& 1,65-3,73
& 2,29
& 2,75
& 1,97-3,86
& 2,48
& 3,12
& 1,91-5,09
\\ \hline

Idade (anos) [Ref: 20-34]
& ...
& ...
& ...
& ...
& ...
& ...
& ...
& ...
& ...
& ...
& ...
& ...
\\ \hline

10-19
& 1,09
& 1,01
& 0,85-1,21
& 1,14
& 1,10
& 0,92-1,31
& 1,21
& 1,06
& 0,84-1,33
& 1,20
& 1,28
& 1,08-1,53
\\ \hline

≥ 35
& 0,68
& 0,83
& 0,58-1,18
& 0,75
& 0,83
& 0,66-1,04
& 0,52
& 0,64
& 0,43-0,96
& 0,71
& 0,75
& 0,58-0,97
\\ \hline

Anos de estudos [Ref: ≥15]
& ...
& ...
& ...
& ...
& ...
& ...
& ...
& ...
& ...
& ...
& ...
& ...
\\ \hline

≤ 7
& 1,25
& 0,84
& 0,56-1,26
& 1,25
& 0,78
& 0,57-1,08
& 1,35
& 1,08
& 0,76-1,52
& 1,87
& 1,36
& 0,91-2,03
\\ \hline

8-10
& 1,65
& 0,92
& 0,63-1,34
& 1,52
& 0,91
& 0,67-1,23
& 1,96
& 1,27
& 0,93-1,72
& 2,51
& 1,47
& 1,01-2,14
\\ \hline

11-14
& 1,91
& 1,06
& 0,71-1,57
& 1,54
& 1,02
& 0,74-1,42
& 2,11
& 1,36
& 1,02-1,81
& 2,18
& 1,35
& 0,95-1,90
\\ \hline

Cor da pele [Ref: branca]
& ...
& ...
& ...
& ...
& ...
& ...
& ...
& ...
& ...
& ...
& ...
& ...
\\ \hline

Preta
& 0,83
& 0,79
& 0,59-1,05
& 0,90
& 0,92
& 0,75-1,14
& 0,83
& 0,85
& 0,64-1,14
& 0,93
& 0,96
& 0,71-1,31
\\ \hline

Parda
& 0,73
& 0,76
& 0,58-1,00
& 0,89
& 0,91
& 0,79-1,06
& 0,78
& 0,89
& 0,73-1,08
& 0,81
& 1,01
& 0,83-1,24
\\ \hline

Amarela
& 0,83
& 0,94
& 0,54-1,62
& 0,63
& 0,65
& 0,42-1,03
& 0,94
& 1,16
& 0,68-1,98
& 0,75
& 1,05
& 0,56-1,99
\\ \hline

Indígena
& 0,50
& 0,53
& 0,19-1,43
& 0,67
& 0,68
& 0,32-1,47
& 0,66
& 0,80
& 0,25-2,58
& 1,43
& 1,92
& 0,89-4,15
\\ \hline

Paridade [Ref: 1-2)
& ...
& ...
& ...
& ...
& ...
& ...
& ...
& ...
& ...
& ...
& ...
& ...
\\ \hline

0
& 1,35
& 1,36
& 1,16-1,58
& 1,18
& 1,14
& 1,02-1,29
& 1,50
& 1,49
& 1,1-2,00
& 0,96
& 0,89
& 0,75-1,06
\\ \hline

≥ 3
& 0,82
& 0,97
& 0,74-1,27
& 1,05
& 1,13
& 0,94-1,36
& 0,69
& 0,84
& 0,66-1,08
& 0,73
& 0,89
& 0,68-1,17
\\ \hline

\end{tabulary}
%\end{adjustbox}
\caption*{\footnotesize }
\caption{}
\end{table}

IC95\%: intervalo de 95\% de confiança; Ref: referência.

* Modelos ajustados por todas as variáveis apresentadas e pela
localização do hospital (capital, não capital);

** Parto pago pelo plano de saúde ou pela própria paciente.

IC95\%: intervalo de 95\% de confiança; Ref: referência.

* Modelos ajustados por todas as variáveis apresentadas e pela
localização do hospital (capital, não capital);

** Parto pago pelo plano de saúde ou pela própria paciente.

Após o controle por variáveis de confundimento potencial, o uso de ocitocina foi
estatisticamente menos frequente nas regiões Norte, Nordeste e Centro-oeste.
Para as
mulheres sem plano de saúde o uso de cateter venoso e analgesia raqui/epidural
foi
menos frequente, enquanto que o uso de ocitocina e amniotomia foi mais
frequente.
Para as adolescentes, o uso de cateter venoso e analgesia raqui/epidural foi
menor
do que nas adultas. Nas mulheres com o menor nível de escolaridade (≤ 7 anos de
estudos) foi mais frequente o uso de ocitocina (\textit{odds ratio}
– OR =
1,53; intervalo de 95\% de confiança – IC95\%: 1,03-2,28) e amniotomia (OR =
1,98;
IC95\%: 1,38-2,84) e menos frequente o uso de analgesia raqui/epidural (OR =
0,48;
IC95\%: 0,32-0,71). Nas primíparas observou-se uma maior frequência do uso de
todas
as intervenções durante o trabalho de parto, exceto amniotomia, e nas multíparas
foi
menor o uso de cateter venoso e analgesia raqui/epidural (Tabelas 4 e 5).

Tabela 4Incidência de intervenções durante o trabalho de parto em mulheres de
risco obstétrico habitual, de acordo com características
sociodemográficas. Brasil, 2011.\begin{table}
%\begin{adjustbox}{width=1.1\textwidth}
\small\centering
\begin{tabulary}{\linewidth}{ C C C C C }
\hline... & Cateter venoso periférico (\%) & Ocitocina (\%) & Analgesia epidural (\%)
& Amniotomia (\%) *\\ \hline
Região
& ...
& ...
& ...
& ...
\\ \hline

Norte
& 72,1
& 22,8
& 28,8
& 40,4
\\ \hline

Nordeste
& 71,5
& 30,9
& 26,8
& 35,8
\\ \hline

Sudeste
& 76,0
& 47,2
& 34,9
& 43,4
\\ \hline

Sul
& 72,9
& 46,1
& 28,7
& 47,9
\\ \hline

Centro-oeste
& 73,7
& 23,7
& 39,3
& 32,0
\\ \hline

Fonte de pagamento **
& ...
& ...
& ...
& ...
\\ \hline

Pública
& 72,8
& 39,5
& 27,1
& 42,4
\\ \hline

Privada **
& 83,2
& 25,8
& 73,7
& 27,1
\\ \hline

Idade (anos)
& ...
& ...
& ...
& ...
\\ \hline

10-19
& 73,0
& 41,4
& 27,3
& 46,5
\\ \hline

20-34
& 74,0
& 37,4
& 32,2
& 39,2
\\ \hline

≥ 35
& 74,8
& 34,5
& 39,7
& 34,3
\\ \hline

Anos de estudo
& ...
& ...
& ...
& ...
\\ \hline

≤ 7
& 70,4
& 37,0
& 21,5
& 46,1
\\ \hline

8-10
& 73,0
& 41,9
& 28,2
& 41,8
\\ \hline

11-14
& 76,6
& 37,9
& 37,8
& 37,9
\\ \hline

≥ 15
& 79,5
& 24,7
& 67,5
& 24,0
\\ \hline

Cor da pele
& ...
& ...
& ...
& ...
\\ \hline

Branca
& 74,8
& 41,5
& 37,3
& 37,8
\\ \hline

Preta
& 71,8
& 37,6
& 27,0
& 40,5
\\ \hline

Parda
& 73,8
& 37,1
& 29,2
& 42,1
\\ \hline

Amarela
& 74,5
& 24,8
& 39,8
& 41,4
\\ \hline

Indígena
& 54,3
& 31,7
& 26,6
& 57,4
\\ \hline

Paridade
& ...
& ...
& ...
& ...
\\ \hline

0
& 78,1
& 40,0
& 37,8
& 41,5
\\ \hline

1-2
& 70,9
& 37,0
& 28,0
& 40,0
\\ \hline

≥ 3
& 65,0
& 34,7
& 15,3
& 40,0
\\ \hline

Brasil
& 73,8
& 38,2
& 31,5
& 40,7
\\ \hline

\end{tabulary}
%\end{adjustbox}
\caption*{\footnotesize }
\caption{}
\end{table}

* Também foram excluídas as mulheres com ruptura espontânea de
membranas anterior à hospitalização;

** Parto pago pelo plano de saúde ou pela própria paciente.

* Também foram excluídas as mulheres com ruptura espontânea de
membranas anterior à hospitalização;

** Parto pago pelo plano de saúde ou pela própria paciente.

Tabela 5Odds ratio (OR) brutos e ajustados * para os determinantes
sociodemográficos das intervenções durante o trabalho de parto em
mulheres de risco obstétrico habitual. Brasil, 2011.\begin{table}
%\begin{adjustbox}{width=1.1\textwidth}
\small\centering
\begin{tabulary}{\linewidth}{ C C C C C C C C C C C C C C C C C }
\hline... & Cateter venoso periférico & Ocitocina & Analgesia epidural & Amniotomia
**\\ \hline
OR bruta & OR ajustada & IC95\% & OR bruta & OR ajustada & IC95\% & OR bruta &
OR ajustada & IC95\% & OR bruta & OR ajustada & IC95\%\\ \hline
Região [Ref: Sudeste]
& ...
& ...
& ...
& ...
& ...
& ...
& ...
& ...
& ...
& ...
& ...
& ...
\\ \hline

Norte
& 0,82
& 0,94
& 0,57-1,57
& 0,33
& 0,30
& 0,20-0,45
& 0,75
& 1,09
& 0,68-1,75
& 0,88
& 0,74
& 0,47-1,17
\\ \hline

Nordeste
& 0,79
& 0,82
& 0,55-1,21
& 0,50
& 0,49
& 0,33-0,71
& 0,68
& 0,75
& 0,52-1,09
& 0,73
& 0,64
& 0,48-0,86
\\ \hline

Sul
& 0,85
& 0,87
& 0,58-1,29
& 0,96
& 0,95
& 0,71-1,28
& 0,75
& 0,76
& 0,51-1,12
& 1,20
& 1,24
& 0,95-1,61
\\ \hline

Centro-oeste
& 0,88
& 0,94
& 0,57-1,54
& 0,35
& 0,33
& 0,22-0,48
& 1,21
& 1,23
& 0,68-2,21
& 0,61
& 0,61
& 0,42-0,88
\\ \hline

Fonte de pagamento *** [Ref: privada)
& ...
& ...
& ...
& ...
& ...
& ...
& ...
& ...
& ...
& ...
& ...
& ...
\\ \hline

Pública
& 0,54
& 0,59
& 0,41-0,83
& 1,88
& 1,94
& 1,26-2,99
& 0,13
& 0,19
& 0,11-0,32
& 1,98
& 1,60
& 1,03-2,49
\\ \hline

Idade (anos) [Ref: 20-34]
& ...
& ...
& ...
& ...
& ...
& ...
& ...
& ...
& ...
& ...
& ...
& ...
\\ \hline

10-19
& 0,95
& 0,77
& 0,64-0,93
& 1,18
& 1,08
& 0,93-1,25
& 0,79
& 0,73
& 0,61-0,88
& 1,35
& 1,19
& 0,95-1,49
\\ \hline

≥ 35
& 1,05
& 1,21
& 0,94-1,57
& 0,88
& 0,99
& 0,77-1,26
& 1,38
& 1,56
& 1,21-2,02
& 0,81
& 0,91
& 0,70-1,17
\\ \hline

Anos de estudo [Ref: ≥ 15]
& ...
& ...
& ...
& ...
& ...
& ...
& ...
& ...
& ...
& ...
& ...
& ...
\\ \hline

≤ 7
& 0,61
& 1,08
& 0,76-1,53
& 1,79
& 1,53
& 1,03-2,28
& 0,13
& 0,48
& 0,32-0,71
& 2,71
& 1,98
& 1,38-2,84
\\ \hline

8-10
& 0,70
& 1,08
& 0,76-1,53
& 2,20
& 1,65
& 1,08-2,52
& 0,19
& 0,54
& 0,36-0,8
& 2,28
& 1,53
& 1,11-2,12
\\ \hline

11-14
& 0,84
& 1,11
& 0,75-1,64
& 1,86
& 1,38
& 1,00-1,91
& 0,29
& 0,62
& 0,44-0,88
& 1,93
& 1,42
& 1,01-2,01
\\ \hline

Cor da pele [Ref: branca)
& ...
& ...
& ...
& ...
& ...
& ...
& ...
& ...
& ...
& ...
& ...
& ...
\\ \hline

Preta
& 0,86
& 0,95
& 0,71-1,26
& 0,85
& 0,86
& 0,70-1,06
& 0,62
& 0,85
& 0,65-1,12
& 1,12
& 1,07
& 0,86-1,33
\\ \hline

Parda
& 0,95
& 1,05
& 0,81-1,35
& 0,83
& 0,96
& 0,81-1,12
& 0,69
& 0,87
& 0,7-1,09
& 1,20
& 1,26
& 1,07-1,47
\\ \hline

Amarela
& 0,99
& 1,04
& 0,61-1,78
& 0,46
& 0,56
& 0,35-0,91
& 1,11
& 1,25
& 0,75-2,09
& 1,16
& 1,31
& 0,78-2,17
\\ \hline

Indígena
& 0,40
& 0,47
& 0,22-1,00
& 0,65
& 0,75
& 0,36-1,53
& 0,61
& 0,80
& 0,28-2,29
& 2,22
& 2,20
& 1,12-4,33
\\ \hline

Paridade [Ref: 1-2]
& ...
& ...
& ...
& ...
& ...
& ...
& ...
& ...
& ...
& ...
& ...
& ...
\\ \hline

0
& 1,46
& 1,63
& 1,38-1,93
& 1,13
& 1,16
& 1,00-1,33
& 1,56
& 1,77
& 1,5-2,08
& 1,07
& 1,07
& 0,90-1,27
\\ \hline

≥ 3
& 0,76
& 0,76
& 0,62-0,93
& 0,90
& 1,00
& 0,83-1,20
& 0,46
& 0,55
& 0,4-0,73
& 1,00
& 0,92
& 0,76-1,12
\\ \hline

\end{tabulary}
%\end{adjustbox}
\caption*{\footnotesize }
\caption{}
\end{table}

IC95\%: intervalo de 95\% de confiança; Ref: referência.

* Modelos ajustados por todas as variáveis apresentadas e pela
localização do hospital (capital, não capital);

** Também foram excluídas as mulheres com ruptuta espontânea de
membranas anterior à hospitalização;

*** Parto pago pelo plano de saúde ou pela própria paciente.

IC95\%: intervalo de 95\% de confiança; Ref: referência.

* Modelos ajustados por todas as variáveis apresentadas e pela
localização do hospital (capital, não capital);

** Também foram excluídas as mulheres com ruptuta espontânea de
membranas anterior à hospitalização;

*** Parto pago pelo plano de saúde ou pela própria paciente.

Nas Tabelas 6 e 7, descrevemos a incidência e os determinantes das intervenções
durante o parto. A litotomia foi mais frequente na Região Centro-oeste e menos
frequente nas mulheres que se declararam pretas. A manobra de Kristeller foi
mais
frequente na Região Centro-oeste, nas mulheres mais velhas e nas primíparas, e a
episiotomia foi muito mais frequente na Região Centro-oeste e em primíparas, e
um
pouco menor nas mulheres de baixa escolaridade.

Tabela 6Incidência de intervenções durante o parto em mulheres de risco
obstétrico habitual, de acordo com características sociodemográficas.
Brasil, 2011.\begin{table}
%\begin{adjustbox}{width=1.1\textwidth}
\small\centering
\begin{tabulary}{\linewidth}{ C C C C C }
\hline... & Litotomia (\%) & Manobra de Kristeler (\%) & Episiotomia (\%) & Cesariana
(\%)\\ \hline
Região
& ...
& ...
& ...
& ...
\\ \hline

Norte
& 90,3
& 33,9
& 48,6
& 43,3
\\ \hline

Nordeste
& 89,2
& 40,6
& 52,5
& 44,8
\\ \hline

Sudeste
& 92,0
& 36,1
& 56,7
& 44,7
\\ \hline

Sul
& 95,3
& 32,3
& 62,9
& 49,1
\\ \hline

Centro-oeste
& 97,3
& 45,5
& 69,2
& 50,2
\\ \hline

Fonte de pagamento *
& ...
& ...
& ...
& ...
\\ \hline

Pública
& 92,0
& 37,3
& 55,5
& 35,6
\\ \hline

Privada *
& 86,7
& 37,9
& 67,1
& 85,0
\\ \hline

Idade (anos)
& ...
& ...
& ...
& ...
\\ \hline

10-19
& 91,6
& 43,5
& 69,5
& 31,7
\\ \hline

20-34
& 91,7
& 35,2
& 52,3
& 47,6
\\ \hline

≥ 35
& 91,8
& 34,4
& 40,0
& 63,2
\\ \hline

Anos de estudo
& ...
& ...
& ...
& ...
\\ \hline

≤ 7
& 91,1
& 36,4
& 47,4
& 30,3
\\ \hline

8-10
& 92,3
& 37,1
& 57,7
& 36,7
\\ \hline

11-14
& 92,0
& 38,3
& 61,9
& 52,8
\\ \hline

≥ 15
& 92,2
& 40,0
& 74,1
& 83,0
\\ \hline

Cor da pele
& ...
& ...
& ...
& ...
\\ \hline

Branca
& 93,4
& 34,0
& 60,7
& 55,1
\\ \hline

Preta
& 87,7
& 38,4
& 52,1
& 34,3
\\ \hline

Parda
& 91,5
& 38,5
& 54,8
& 41,6
\\ \hline

Amarela
& 90,7
& 44,1
& 55,7
& 46,8
\\ \hline

Indígena
& 97,6
& 38,3
& 45,8
& 24,1
\\ \hline

Paridade
& ...
& ...
& ...
& ...
\\ \hline

0
& 91,6
& 49,6
& 74,6
& 49,2
\\ \hline

1-2
& 92,2
& 28,4
& 46,7
& 44,8
\\ \hline

≥ 3
& 89,9
& 22,4
& 18,8
& 28,5
\\ \hline

Brasil
& 91,7
& 37,3
& 56,1
& 45,5
\\ \hline

\end{tabulary}
%\end{adjustbox}
\caption*{\footnotesize }
\caption{}
\end{table}

* Parto pago pelo plano de saúde ou pela própria paciente.

* Parto pago pelo plano de saúde ou pela própria paciente.

Tabela 7Odds ratio (OR) brutos e ajustados * para os determinantes
sociodemográficos das intervenções durante o parto em mulheres de risco
habitual. Brasil, 2011.\begin{table}
%\begin{adjustbox}{width=1.1\textwidth}
\small\centering
\begin{tabulary}{\linewidth}{ C C C C C C C C C C C C C C C C C }
\hline... & Litotomia & Manobra de Kristeler & Episiotomia & Cesariana\\ \hline
OR bruta & OR ajustada & IC95\% & OR bruta & OR ajustada & IC95\% & OR bruta &
OR ajustada & IC95\% & OR bruta & OR ajustada & IC95\%\\ \hline
Região [Ref: Sudeste)
& ...
& ...
& ...
& ...
& ...
& ...
& ...
& ...
& ...
& ...
& ...
& ...
\\ \hline

Norte
& 0,81
& 0,86
& 0,17-4,30
& 0,91
& 1,04
& 0,67-1,59
& 0,72
& 0,97
& 0,58-1,61
& 0,94
& 1,69
& 1,14-2,51
\\ \hline

Nordeste
& 0,72
& 0,72
& 0,37-1,39
& 1,21
& 1,15
& 0,88-1,51
& 0,84
& 0,88
& 0,54-1,44
& 1,00
& 1,24
& 0,95-1,62
\\ \hline

Sul
& 1,76
& 1,49
& 0,86-2,58
& 0,85
& 0,81
& 0,59-1,11
& 1,30
& 1,27
& 0,75-2,15
& 1,19
& 1,07
& 0,80-1,43
\\ \hline

Centro-oeste
& 3,12
& 3,45
& 1,60-7,43
& 1,48
& 1,83
& 1,21-2,75
& 1,72
& 2,43
& 1,35-4,36
& 1,25
& 1,56
& 1,10-2,21
\\ \hline

Fonte de pagamento [Ref: sim]
& ...
& ...
& ...
& ...
& ...
& ...
& ...
& ...
& ...
& ...
& ...
& ...
\\ \hline

Não
& 1,76
& 2,10
& 0,63-7,00
& 0,97
& 0,90
& 0,54-1,49
& 0,61
& 0,75
& 0,49-1,15
& 0,10
& 0,14
& 0,10-0,20
\\ \hline

Idade (anos) [Ref: 20-34]
& ...
& ...
& ...
& ...
& ...
& ...
& ...
& ...
& ...
& ...
& ...
& ...
\\ \hline

10-19
& 0,98
& 0,90
& 0,66-1,25
& 1,42
& 0,78
& 0,64-0,96
& 2,08
& 1,10
& 0,86-1,39
& 0,51
& 0,63
& 0,55-0,72
\\ \hline

≥ 35
& 1,01
& 1,06
& 0,71-1,59
& 0,97
& 1,43
& 1,08-1,89
& 0,61
& 1,03
& 0,75-1,41
& 1,89
& 1,63
& 1,32-2,00
\\ \hline

Anos de estudo [ref: ≥15]
& ...
& ...
& ...
& ...
& ...
& ...
& ...
& ...
& ...
& ...
& ...
& ...
\\ \hline

≤ 7
& 0,86
& 0,74
& 0,31-1,75
& 0,86
& 1,17
& 0,77-1,77
& 0,31
& 0,48
& 0,3-0,77
& 0,09
& 0,40
& 0,30-0,53
\\ \hline

8-10
& 1,01
& 0,80
& 0,34-1,87
& 0,89
& 1,04
& 0,67-1,61
& 0,48
& 0,52
& 0,33-0,81
& 0,12
& 0,46
& 0,35-0,61
\\ \hline

11-14
& 0,96
& 0,77
& 0,38-1,56
& 0,93
& 0,92
& 0,59-1,44
& 0,57
& 0,55
& 0,33-0,92
& 0,23
& 0,58
& 0,46-0,73
\\ \hline

Cor da pele [Ref: branca)
& ...
& ...
& ...
& ...
& ...
& ...
& ...
& ...
& ...
& ...
& ...
& ...
\\ \hline

Preta
& 0,50
& 0,57
& 0,35-0,92
& 1,21
& 1,17
& 0,87-1,56
& 0,70
& 0,77
& 0,57-1,03
& 0,43
& 0,72
& 0,58-0,89
\\ \hline

Parda
& 0,76
& 0,82
& 0,52-1,29
& 1,22
& 1,18
& 0,94-1,47
& 0,78
& 0,87
& 0,67-1,14
& 0,58
& 0,82
& 0,70-0,95
\\ \hline

Amarela
& 0,69
& 0,69
& 0,23-2,10
& 1,53
& 1,43
& 0,76-2,68
& 0,81
& 0,76
& 0,39-1,46
& 0,72
& 0,87
& 0,53-1,43
\\ \hline

Indígena
& 2,84
& 3,70
& 0,63-21,78
& 1,20
& 1,37
& 0,59-3,18
& 0,55
& 0,68
& 0,24-1,92
& 0,26
& 0,35
& 0,18-0,69
\\ \hline

Paridade [Ref: 1-2]
& ...
& ...
& ...
& ...
& ...
& ...
& ...
& ...
& ...
& ...
& ...
& ...
\\ \hline

0
& 0,92
& 1,04
& 0,82-1,32
& 2,48
& 2,97
& 2,49-3,54
& 3,35
& 3,54
& 2,80-4,49
& 1,19
& 1,29
& 1,16-1,44
\\ \hline

≥ 3
& 0,75
& 0,72
& 0,47-1,12
& 0,73
& 0,58
& 0,46-0,72
& 0,26
& 0,27
& 0,21-0,35
& 0,49
& 0,65
& 0,54-0,78
\\ \hline

\end{tabulary}
%\end{adjustbox}
\caption*{\footnotesize }
\caption{}
\end{table}

IC95\%: intervalo de 95\% de confiança; Ref: referência.

* Modelos ajustados por todas as variáveis apresentadas e pela
localização do hospital (capital, não capital).

IC95\%: intervalo de 95\% de confiança; Ref: referência.

* Modelos ajustados por todas as variáveis apresentadas e pela
localização do hospital (capital, não capital).

A Região Centro-oeste registrou a maior incidência de cesariana (50,2\%), que
também
se verificou na Região Norte (OR = 1,69; IC95\%: 1,14-2,51). Após o controle por
variáveis confundidoras potenciais, as mulheres mais velhas (> 35 anos) e
primíparas também tiveram maior chance de realizar uma cesariana. Mulheres sem
plano
de saúde, adolescentes, nas categorias mais baixas de escolaridade (< 7 anos de
estudo), não brancas (exceto as amarelas) e multíparas, tiveram menos chances de
realizar uma cesariana (Tabelas 6 e 7).

\section{Discussão}

Neste artigo utilizou-se um critério abrangente para a definição das gestações
de
risco habitual. Visou-se a excluir qualquer situação de risco que pudesse
justificar
a utilização de intervenções durante o trabalho de parto e o parto.

O uso das boas práticas (alimentação durante o trabalho de parto, movimentação
durante o primeiro estágio de trabalho de parto, uso de métodos não
farmacológicos
para o alívio da dor e monitoramento da evolução do trabalho de parto com
partograma) apresentou prevalência variada e, de um modo geral, com valores que
não
alcançavam 50\% das gestantes, sendo mais frequente no grupo de baixo risco. Por
outro lado, as intervenções obstétricas durante o trabalho de parto e parto
apresentaram prevalência elevada, sendo a litotomia e o uso de cateter venoso os
mais frequentes.

O Brasil é conhecido mundialmente pela elevada incidência de cesarianas, o que
foi
confirmado neste artigo que mostrou uma proporção de 45,5\% em mulheres de risco
obstétrico habitual. Entretanto, também se observou que as intervenções médicas
foram excessivas sobre o trabalho de parto e o parto vaginal, tendo apenas 5,6\%
das
parturientes de risco habitual e 3,2\% das primíparas nesse grupo (dados não
mostrados) dado à luz de forma natural, sem qualquer tipo de intervenção na
fisiologia do trabalho de parto. Na Austrália, país onde as intervenções sobre o
parto também vêm aumentando nas últimas décadas, 15\% das mulheres pariram de
forma
natural no sistema privado e 35\% no sistema público de saúde\textsuperscript{[}\textsuperscript{7}\textsuperscript{]}. Já no Reino Unido, o parto natural correspondeu a
41,8 \% do total de nascimentos em 2011, sendo 97\% deles em instituições dos
serviços
públicos de saúde\textsuperscript{[}\textsuperscript{9}\textsuperscript{]}

Deve-se ressaltar que a frequência das intervenções foi maior em mulheres de
risco
habitual, exceto o uso de cateter venoso, analgesia obstétrica e cesariana,
demonstrando que, em grande parte, foram desnecessárias e cumpriram o papel de
repetição de uma rotina que parece não considerar nem a demanda clínica das
pacientes nem as evidências científicas do campo.

O modelo de atenção ao parto com uso excessivo de intervenções não encontra
respaldo
em estudos internacionais. A episiotomia foi observada em mais de 50\% das
mulheres
deste estudo e em quase 75\% das primíparas. A prática da episiotomia se
incorporou à
rotina da assistência ao parto desde o início do século passado com a intenção
de
reduzir o dano causado pela laceração natural do períneo, reduzir o risco de uma
posterior incontinência urinária e fecal, e proteger o neonato do trauma do
parto.
Essa prática foi incorporada à rotina da assistência obstétrica sem que nenhum
trabalho que avaliasse seus riscos e benefícios tivesse sido realizado.

Contudo, estudos controlados demonstram que a episiotomia aumenta o risco de
laceração perineal de terceiro e quarto graus, de infecção e hemorragia, sem
diminuir as complicações a longo prazo de dor e incontinência urinária e fecal\textsuperscript{[}\textsuperscript{10}\textsuperscript{]}. Por essas razões, as novas
diretrizes clínicas, baseadas em estudos adequadamente desenhados para essa
avaliação, desestimulam o seu uso rotineiro na assistência obstétrica\textsuperscript{[}\textsuperscript{11}\textsuperscript{]}.

A Organização Mundial da Saúde (\textsc{oms}) recomenda que a taxa de episiotomia seja
entre
10\% e 30\%\textsuperscript{[}\textsuperscript{1}\textsuperscript{]}. No Canadá, as taxas de
episiotomia apresentaram uma redução de 38\% para 24\% no período de 1993 a 2001\textsuperscript{[}\textsuperscript{12}\textsuperscript{]}
; na Finlândia, de 42\% em 1997
para 25\% em 2009\textsuperscript{[}\textsuperscript{10}\textsuperscript{]}
; na França, em
primíparas diminuiu de 71\% em 2003 para 44\% em 2010, e em multíparas de 36\%
para 14\%\textsuperscript{[}\textsuperscript{13}\textsuperscript{]}
; e nos Estados Unidos, em
mulheres de parto vaginal e gestação única foi de 17\% em 2012\textsuperscript{[}\textsuperscript{14}\textsuperscript{]}.

Durante o parto, a posição de litotomia foi uma regra, alcançando mais de 90\%
das
parturientes de risco habitual, apesar dos benefícios das posições
verticalizadas
para a mulher e para o feto\textsuperscript{[}\textsuperscript{15}\textsuperscript{]}.

No nosso estudo, verificou-se que a infusão de ocitocina e ruptura artificial da
membrana amniótica foi uma técnica muito utilizada para provocar a aceleração do
trabalho de parto. Ambas ocorreram em cerca de 40\% das mulheres de risco
habitual,
sendo mais frequente nas mulheres do setor publico, de mais baixa escolaridade.
Foi
também bastante elevada a taxa de manobra de Kristeller nos partos vaginais, de
37\%.
A ocitocina e a amniotomia têm sido utilizadas para o que se denomina “manejo
ativo
do parto”, visando à redução da duração do segundo estágio do trabalho de parto
e da
taxa de parto instrumental\textsuperscript{[}\textsuperscript{16}\textsuperscript{]}\textsuperscript{,}\textsuperscript{[}\textsuperscript{17}\textsuperscript{]}\textsuperscript{,}\textsuperscript{[}\textsuperscript{18}\textsuperscript{]}. As revisões sistemáticas da Cochrane apontam uma redução
modesta do número de cesarianas quando o manejo ativo do trabalho de parto é
implementado, entretanto, reconhece-se que os benefícios desta pequena redução
deve
ser ponderado contra os riscos de intervenções em mulheres de risco habitual,
requerendo novos trabalhos\textsuperscript{[}\textsuperscript{19}\textsuperscript{]}\textsuperscript{,}\textsuperscript{[}\textsuperscript{20}\textsuperscript{]}\textsuperscript{,}\textsuperscript{[}\textsuperscript{21}\textsuperscript{]}. Para a manobra de Kristeller\textsuperscript{[}\textsuperscript{22}\textsuperscript{]}, estudos não conseguiram demonstrar os benefícios
desta prática, tendo uma forte recomendação de evitá-la como uso rotineiro.

Uma questão-chave que tem sido levantada na literatura cientifica é a relação
entre a
analgesia epidural e intervenções no nascimento. Existem evidências fortes de
que a
analgesia epidural se associa ao aumento no risco de parto instrumental\textsuperscript{[}\textsuperscript{23}\textsuperscript{]}. Apesar de uma revisão
sistemática recente da Cochrane que compara analgesia epidural com outros
métodos
analgésicos não ter encontrado evidência de diferença estatisticamente
significativa
para o risco de cesariana, há sugestões de que altas taxas de
\textit{cross-over}
nos ensaios clínicos podem mascarar uma real
associação entre analgesia peridural e cesariana\textsuperscript{[}\textsuperscript{24}\textsuperscript{]}. Um artigo de Kotaska et al.\textsuperscript{[}\textsuperscript{25}\textsuperscript{]}
e um editorial de Klein\textsuperscript{[}\textsuperscript{26}\textsuperscript{]}
argumentam que essa revisão tem pouca validade
externa e que existem evidências suficientes para sugerir que a analgesia
epidural,
associada a baixas doses de ocitocina para a aceleração do trabalho de parto,
aumentam as taxas de cesariana. O único ensaio clínico randomizado não
contaminado
que demonstrou essa associação entre epidural e cesariana foi publicado 16 anos
atrás\textsuperscript{[}\textsuperscript{27}\textsuperscript{]}. Esse pequeno ensaio
clínico mostrou que as mulheres primíparas alocadas no grupo para receber a
epidural, tiveram uma chance 11,4 vezes maior de terem uma cesariana devido à
distócia do que as mulheres alocadas para receberem analgesia narcótica. O
estudo
foi interrompido com base na falta de ética em se continuar a randomizar as
mulheres
para receberem a epidural. A analgesia epidural também se associa com um aumento
na
taxa de indução ou aceleração do trabalho de parto\textsuperscript{[}\textsuperscript{28}\textsuperscript{]}
e com o uso de antibióticos para febre materna\textsuperscript{[}\textsuperscript{29}\textsuperscript{]}.

As recomendações atuais da \textsc{oms} e do Ministério da Saúde brasileiro para o manejo
do
trabalho de parto incluem a oferta de líquidos; o estímulo à adoção de posições
verticalizadas e à liberdade de movimentação, buscando aumentar o conforto
materno e
facilitar a progressão do trabalho de parto; e o uso de métodos não
farmacológicos
para alívio da dor, tais como banho de aspersão ou imersão em água quente,
massagens
e outros. Essas são tecnologias acessíveis, não invasivas e de baixo custo,
sendo
possíveis de serem ofertadas por todos os serviços de saúde\textsuperscript{[}\textsuperscript{1}\textsuperscript{]}\textsuperscript{,}\textsuperscript{[}\textsuperscript{30}\textsuperscript{]}\textsuperscript{,}\textsuperscript{[}\textsuperscript{31}\textsuperscript{]}.

Revisões sistemáticas indicam os benefícios das práticas, avaliadas neste
estudo,
para a atenção ao parto de risco obstétrico habitual. A adoção de posições
verticalizadas durante a primeira fase do trabalho de parto, verificada neste
trabalho como liberdade de movimentação da gestante durante este período, reduz
o
tempo de trabalho de parto e não parece estar associada com o aumento da
intervenção
ou efeitos negativos sobre o bem-estar das mães e bebês\textsuperscript{[}\textsuperscript{32}\textsuperscript{]}\textsuperscript{,}\textsuperscript{[}\textsuperscript{33}\textsuperscript{]}. Também não existem evidências que apóiem a restrição
de líquidos e alimentos durante o trabalho de parto para as mulheres com baixo
risco
de complicações\textsuperscript{[}\textsuperscript{34}\textsuperscript{]}. Já os métodos
não farmacológicos para alívio da dor são métodos não invasivos e parecem ser
seguros para a mãe e o bebê. Embora as evidências disponíveis não permitam a
recomendação do uso rotineiro do partograma como estratégia para a redução de
cesarianas e de desfechos negativos, devendo seu uso ser definido localmente,
estudos conduzidos em países de menor desenvolvimento econômico mostraram menor
proporção de cesarianas associadas ao uso do partograma, e identificação precoce
de
gestantes com progressão lenta do trabalho de parto\textsuperscript{[}\textsuperscript{35}\textsuperscript{]}.

A menor prevalência de boas práticas foi verificada nas regiões Norte e
Nordeste,
áreas menos desenvolvidas do país. Nessas regiões, a frequência de algumas
intervenções também foi menor, o que não significa necessariamente a adoção de
um
modelo menos intervencionista e mais “natural”. É provável que esses dados sejam
mais o reflexo de uma prática de abandono das mulheres à própria sorte do que um
modelo humanizado de atendimento, já que todas as boas práticas apresentaram
menor
frequência nessas regiões e os indicadores obstétricos e perinatais são os
piores do
país.

Mulheres atendidas pelo setor público apre- sentaram as maiores frequências de
boas
práticas e menor frequência de uso de acesso venoso. Esses resultados são
provavelmente decorrentes dos esforços feitos pelo Ministério da Saúde para a
promoção do parto humanizado e normal, por meio da divulgação de manuais
técnicos,
publicação de portarias, qualificação dos profissionais de saúde e adequação da
ambiência do parto\textsuperscript{[}\textsuperscript{3}\textsuperscript{]}\textsuperscript{,}\textsuperscript{[}\textsuperscript{36}\textsuperscript{]}. Como o instrumento utilizado não permitiu a identificação
de mulheres que tiveram parto pago por desembolso direto, é possível que algumas
mulheres atendidas em unidades mistas, e que foram classificadas como tendo
financiamento público da assistência, tenham pago pela assistência de seu parto.
Como essas mulheres apresentaram características socioeconômicas muito
semelhantes a
das mulheres atendidas em unidades públicas, é provável que esse erro de
classificação tenha ocorrido em poucos casos. Como se trata de erro de
classificação
não-diferencial em relação aos desfechos estudados, espera-se que tenha ocorrido
atenuamento da magnitude das associações observadas.

Por outro lado, a persistência do uso de intervenções dolorosas e
desnecessárias,
tais como a episiotomia e a manobra de Kristeller, demonstra que melhorias ainda
são
necessárias. O uso de ocitocina durante o trabalho de parto também foi mais
frequente entre as usuárias do setor público e nas gestantes de menor
escolaridade,
os mesmos grupos que apresentaram menor frequência do uso de analgesia
obstétrica,
reforçando a prática do parto doloroso, que pode trazer como consequência o
temor do
parto vaginal e o aumento do prestígio da cesariana entre as mulheres
brasileiras.

Um trabalho de avaliação da satisfação do parto de mulheres que participaram da
pesquisa \textit{Nascer no Brasil}\textsuperscript{[}\textsuperscript{37}\textsuperscript{]}, verificou menores níveis de
satisfação e maiores níveis de violência entre as mulheres que entraram em
trabalho
de parto. Todos os aspectos associados à relação do profissional com a gestante
estiveram significativamente associados à satisfação, mostrando que as mulheres
valorizam muito a forma como foram atendidas. O descaso com as gestantes no
setor
público de saúde tem sido amplamente divulgado pela imprensa leiga e mesmo pela
ouvidoria do Ministério da Saúde, que encontrou um percentual de 12,7\% das
mulheres
referindo terem sido submetidas a tratamentos desrespeitosos, tais como ser mal
atendidas, não serem ouvidas ou mesmo sofrerem agressões verbais e físicas\textsuperscript{[}\textsuperscript{38}\textsuperscript{]}. Esses dados demonstram que não
basta a modificação das práticas assistenciais se estas não forem acompanhadas
de
mudanças na relação entre profissionais e usuárias.

Neste estudo, todas as boas práticas e intervenções foram mais frequentes em
primíparas, que costumam ter um trabalho de parto mais longo. Verificou-se
também
que essas mulheres, em comparação às multíparas, foram com mais frequência
admitidas
precocemente, com três ou menos centímetros de dilatação (dados não mostrados),
ficando mais expostas às rotinas hospitalares. As intervenções sofridas, tais
como
episiotomia, manobra de Kristeller e cesariana, dão início a uma história
reprodutiva marcada por cicatrizes e perda da integridade do tecido perineal e
uterino.

Mulheres com maior escolaridade e aquelas com parto financiado pelo setor
privado,
também apresentaram proporção mais elevada de cesariana, de uso de analgesia e
de
realização de episiotomia, embora tenham sido menos expostas ao uso de
ocitocina.
Outros autores já demostraram a valorização da intervenção, em especial das
cesarianas, como um bom padrão de atendimento\textsuperscript{[}\textsuperscript{39}\textsuperscript{]}.

A análise por cor/raça não mostrou um padrão definido. De um modo geral, as
puérperas
amarelas foram mais parecidas com as brancas, e as indígenas e pardas mais
parecidas
com as pretas. Uma possível explicação para a falta de qualquer padrão pode ser
a
subjetividade da classificação da cor/raça que foi autorreferida.

Todas as boas práticas foram mais frequentes nas mulheres de risco habitual.
Surpreendentemente, as intervenções também o foram, com exceção do uso de
cateter
venoso e analgesia peridural. De todas as intervenções a cesariana foi a que
mostrou
maior diferença entre os grupos de risco, sendo ainda assim muito elevada no
grupo
de risco habitual, composto por gestantes sem as condições que mais
frequentemente
justificam a indicação de cesárea.

É conhecido que a prevalência de cesáreas no Brasil é a mais alta do mundo,
ficando
próxima dos valores da China (46,2\%), Turquia (42,7\%), México (42\%), Itália
(38,4\%)
e Estados Unidos (32,3\%) e muito superior à Inglaterra (23,7\%), França (20\%)
e
Finlândia (15,7\%)\textsuperscript{[}\textsuperscript{40}\textsuperscript{]}.

As taxas de cesariana foram muito menos frequentes no setor público, nas
mulheres
menos escolarizadas e nas não brancas, repetindo um padrão conhecido no Brasil\textsuperscript{[}\textsuperscript{41}\textsuperscript{]}\textsuperscript{,}\textsuperscript{[}\textsuperscript{42}\textsuperscript{]}
e em outros países\textsuperscript{[}\textsuperscript{7}\textsuperscript{]}\textsuperscript{,}\textsuperscript{[}\textsuperscript{8}\textsuperscript{]}, mostrando que o excesso de cesáreas acomete principalmente
as mulheres brasileiras de mais alto padrão de escolaridade, alcançando 89,9\%
no
setor privado de saúde.

São conhecidos os prejuízos da cesariana para a mãe e o recém-nascido. Um estudo
multicêntrico realizado em uma amostra de hospitais de oito países da América
Latina\textsuperscript{[}\textsuperscript{44}\textsuperscript{]}
avaliou os riscos e
benefícios da cesárea em relação ao parto vaginal, e concluiu que na
apresentação
cefálica a cesárea aumentou o risco de morbidade grave e mortalidade materna e
neonatal.

Resultados semelhantes foram encontrados por Souza et al em 2010\textsuperscript{[}\textsuperscript{45}\textsuperscript{]}
em um trabalho que incluiu, além
da América Latina, países da África e Ásia, levando os autores a concluírem que
a
cesárea deve ser realizada quando for identificado um benefício que compense os
custos e riscos adicionais desta cirurgia.

Hansen et al.\textsuperscript{[}\textsuperscript{46}\textsuperscript{]}, estudando a
coorte de Aarhus, verificaram que os nascidos de cesárea eletiva, quando
comparados
com os nascidos de parto vaginal, apresentaram risco mais elevado de morbidade
respiratória leve e grave, que aumentava à medida que diminuía a idade
gestacional.
Os resultados sugeriram que o trabalho de parto cumpre uma função importante na
maturação pulmonar da criança\textsuperscript{[}\textsuperscript{46}\textsuperscript{]}.

A cesariana também é um fator de risco bem estabelecido para o desenvolvimento
subsequente de uma placentação anormal\textsuperscript{[}\textsuperscript{47}\textsuperscript{]}. Os resultados da análise da base de dados da Universidade
de Chicago mostraram que a elevação da prática de cesariana foi um dos fatores
associados ao aumento de cinco vezes na taxa de placenta prévia\textsuperscript{[}\textsuperscript{48}\textsuperscript{]}.

Dentro do país, a análise por região geográfica mostrou taxa mais elevada de
cesariana na Região Centro-oeste e mais baixa na Região Norte. Entretanto, após
a
regressão logística, a Região Norte inverteu a posição e apresentou a maior OR
de
todas as regiões, mostrando que a chance de uma gestação resultar em uma
cesariana
foi maior do que em qualquer outro local do país. Essa modificação, após ajuste
para
as demais variáveis do modelo, foi decorrente da menor cobertura por planos de
saúde
na Região Norte e da maior taxa de cesariana nos serviços desta região, tanto
nos
públicos quanto nos privados (taxa de cesáreas duas vezes e meia maior que no
serviço público).

Após controle de variáveis intervenientes, um dado surpreendente foi a menor
chance
de cesárea na Região Sudeste, a mais rica e que tem as mais elevadas proporções
de
população com seguro privado de saúde. Esse achado pode refletir o início de uma
reversão do modelo “cesarista” e de excessivas intervenções, pois nessa região
foi
também maior a frequência de boas práticas obstétricas e é onde são mais
intensos os
movimentos de ativistas do parto normal, profissionais de saúde e das mulheres
em
favor do parto humanizado.

Um dos aspectos mais impressionantes da prática obstétrica brasileira é a pressa
em
provocar o nascimento das crianças, sem respeito à autonomia das mulheres no
processo de parturição. O controle do tempo e a imposição da dinâmica do
trabalho de
parto e parto explicam o índice excessivo de intervenções, incluindo as
cesarianas,
fazendo com que a assistência ao parto no Brasil seja focada na decisão do
médico e
não na dinâmica do corpo da mulher. Esse processo inicia durante a atenção
pré-natal
quando as mulheres não são informadas sobre as boas práticas e cuidados
obstétricos
adequados, sobre os benefícios do parto vaginal, e não são preparadas para
conduzirem o seu parto. No hospital, esse processo tem continuidade com a
imposição
de uma cascata de intervenções que não se baseiam em evidência científica e
produzem
um parto ruim.

Não é fácil interpretar os dados brasileiros, dada a extensão geográfica e
diversidade do país, mas as diferenças significativas nas taxas de cesáreas
entre as
regiões geográficas, mais altas na Região Norte e mais baixas na Região Sudeste,
levam à hipótese de movimentos contraditórios de expansão e retração do modelo
de
intervenção obstétrica aqui apresentado. Por se tratar de um estudo transversal,
os
dados mostrados não permitem fazer análises temporais e de tendência, apenas
constatar as diferenças entre as áreas geográficas e os grupos de mulheres.

Em conclusão, as mulheres brasileiras de todos os grupos socioeconômicos e de
risco
obstétrico habitual estão sendo desnecessariamente expostas aos riscos de
iatrogenia
no parto. Muitas intervenções desnecessárias foram realizadas, principalmente
nas
mulheres de grupos socioeconômicos mais elevados, as quais podem estar mais
propensas a sofrer os efeitos adversos do uso da tecnologia médica.

Para as mulheres socioeconomicamente desfavorecidas houve uma maior utilização
de
procedimentos dolorosos, como a aceleração do trabalho de parto e de baixo uso
de
analgesia obstétrica. Por outro lado, esses grupos tinham mais acesso às boas
práticas no trabalho de parto e parto. Empoderar as mulheres e promover o uso de
cuidados baseados em evidências são estratégias para cultivar o parto
humanizado,
principalmente para reduzir as desigualdades entre os ricos e os pobres.

Finalmente, para melhorar ainda mais o padrão de saúde das mães e crianças, o
\textsc{sus}
brasileiro precisa, urgentemente, melhorar o modelo de assistência obstétrica,
tanto
no setor privado quanto no público, promover práticas baseadas em evidências e
aprimorar a qualidade de vida e saúde da sua população.

Agradecimentos
Aos coordenadores regionais e estaduais, supervisores, entrevistadores e equipe
técnica do estudo, e às mães participantes que tornaram este estudo possível.

\section*{Referências}
\begin{itemize}

\item[1] World Health Organization, Maternal and Newborn Health/Safe
Motherhood Unit. Care in normal birth: a practical guide. Geneva: World Health
Organization; 1996.

\item[2] Victora CG, Aquino EM, do Carmo Leal M, Monteiro CA, Barros FC,
Szwarcwald CL. Maternal and child health in Brazil: progress and challenges.
Lancet 2011; 377:1863-76.

\item[3] Ministério da Saúde. Manual técnico pré-natal e puerpério – atenção
qualificada e humanizada. Brasília: Ministério da Saúde; 2006. (Série A Normas e
Manuais Técnicos/Série Direitos Sexuais e Direitos Reprodutivos,
5).

\item[4] Barros FC, Victora CG, Barros AJ, Santos IS, Albernaz E,
Matijasevich A, et al. The challenge of reducing neonatal mortality in
middle-income countries: findings from three Brazilian birth cohorts in 1982,
1993, and 2004. Lancet 2005; 365:847-54.

\item[5] do Carmo Leal M, da Silva AA, Dias MA, da Gama SG, Rattner D,
Moreira ME, et al. Birth in Brazil: national survey into labour and birth.
Reprod Health 2012; 9:15.

\item[6] Vasconcellos \textsc{mtl}, Silva \textsc{pln}, Pereira \textsc{ape}, Schi-lithz \textsc{aoc}, Souza
Junior \textsc{prb}, Szwarcwald CL. Desenho da amostra \textit{Nascer no Brasil}.
Pesquisa Nacional sobre Parto e Nascimento. Cad Saúde Pública 2014; 30
Suppl:49-58.

\item[7] Dahlen HG, Tracy S, Tracy M, Bisits A, Brown C, Thornton C. Rates
of obstetric intervention among low-risk women giving birth in private and
public hospitals in \textsc{nsw}: a population-based descriptive study. \textsc{bmj} Open 2012;
2(5). pii: e001723.

\item[8] Diniz \textsc{csg}, d’Orsi E, Domingues \textsc{rmsm}, Torres JA, Dias \textsc{mab}, Schneck
CA, et al. Implementação da presença de acompanhantes durante a internação para
o parto: dados da pesquisa \textit{Nascer no Brasil}. Cad Saúde Pública
2014; 30 Suppl:140-53.

\item[9] Health \& Social Care Information Centere. \textsc{nhs} Maternity
Statistics, England, 2011-2012: summary report. London: Health \& Social Care
Information Centre; 2012.

\item[10] Raisanen S, Vehvilainen-Julkunen K, Gisler M, Heinonen S. A
population-based register study to determine indications for episiotomy in
Finland. Int J Gynaecol Obstet 2011; 115:26-30.

\item[11] Carroli G, Mignini L. Episiotomy for vaginal birth. Cochrane
Database Syst Rev 2009; (1):CD000081.

\item[12] Graham ID, Carroli G, Davies C, Medves JM. Episiotomy rates around
the world: an update. Birth 2005; 32:219-23.

\item[13] Blondel B, Lelong N, Kermarrec M, Goffinet F. Trends in perinatal
health in France from 1995 to 2010. Results from the French National Perinatal
Surveys. J Gynecol Obstet Biol Reprod 2012; 41: e1-e15.

\item[14] Declercq ER, Sakala C, Corry MP, Applebaum S, Herrlich A. Listening
to mothers \textsc{iii}: pregnancy and childbirth. New York: Childbirth Connection;
2013.

\item[15] Priddis H, Dahlen H, Schmied V. What are the facilitators,
inhibitors, and implications of birth positioning? A review of the literature.
Women Birth 2012; 25:100-6.

\item[16] Schmitz T, Meunier E. Mesures à prendre pendant le travail pour
réduire le nombre d’extractions instrumentales. J Gynecol Obstet Biol Reprod
(Paris) 2008; 37 Suppl 8:179-87.

\item[17] Api O, Balcin ME, Ugurel V, Api M, Turan C, Unal O. The effect of
uterine fundal pressure on the duration of the second stage of labor: a
randomized controlled trial. Acta Obstet Gynecol Scand 2009;
88:320-4.

\item[18] Matsuo K, Shiki Y, Yamasaki M, Shimoya K. Use of uterine fundal
pressure maneuver at vaginal delivery and risk of severe perineal laceration.
Arch Gynecol Obstet 2009; 280:781-6.

\item[19] Wei S, Wo BL, Qi HP, Xu H, Luo ZC, Roy C, et al. Early amniotomy
and early oxytocin for prevention of, or therapy for, delay in first stage
spontaneous labour compared with routine care. Cochrane Database Syst Rev 2012;
9:CD006794.

\item[20] Fraser WD, Turcot L, Krauss I, Brisson-Carrol G. \textsc{withdrawn}:
Amniotomy for shortening spontaneous labour. Cochrane Database Syst Rev 2006;
(3):CD000015.

\item[21] Brown HC, Paranjothy S, Dowswell T, Thomas J. Package of care for
active management in labour for reducing caesarean section rates in low-risk
women. Cochrane Database Syst Rev 2008; (4):CD004907.

\item[22] Verheijen EC, Raven JH, Hofmeyr GJ. Fundal pressure during the
second stage of labour. Cochrane Database Syst Rev 2009;
(4):CD006067.

\item[23] Anim-Somuah M, Smyth R, Howell C. Epidural versus non-epidural or
no analgesia in labour. Cochrane Database Syst Rev 2005;
(4):CD000331.

\item[24] Lieberman E. Epidemiology of epidural analgesia and cesarean
delivery. Clin Obstet Gynecol 2004; 47:317-31.

\item[25] Kotaska AJ, Klein MC, Liston RM. Epidural analgesia associated with
low-dose oxytocin augmentation increases cesarean births: a critical look at the
external validity of randomized trials. Am J Obstet Gynecol 2006;
194:809-14.

\item[26] Klein MC. Does epidural analgesia increase rate of cesarean
section? Can Fam Physician 2006; 52: 419-21.

\item[27] Thorp JA, Hu DH, Albin RM, McNitt J, Meyer BA, Cohen GR, et al. The
effect of intrapartum epidural analgesia on nulliparous labor: a randomized,
controlled, prospective trial. Am J Obstet Gynecol 1993;
169:851-8.

\item[28] Mayberry LJ, Strange LB, Suplee PD, Gennaro S. Use of upright
positioning with epidural analgesia: findings from an observational study. \textsc{mcn}
Am J Matern Child Nurs 2003; 28:152-9.

\item[29] Goetzl L, Cohen A, Frigoletto Jr. F, Lang JM, Lieberman E. Maternal
epidural analgesia and rates of maternal antibiotic treatment in a low-risk
nulliparous population. J Perinatol 2003; 23:457-61.

\item[30] Secretaria de Políticas de Saúde, Ministério da Saúde. Parto,
aborto e puerpério: assistência humanizada à mulher. Brasília: Ministério da
Saúde; 2001.

\item[31] Ministério de Sanidad y política Social de España. Guía de práctica
clínica sobre la atención al parto normal. Madri: Ministerio de Sanidad y
Política Social de España; 2010.

\item[32] Gupta JK, Hofmeyr GJ, Shehmar M. Position in the second stage of
labour for women without epidural anaesthesia. Cochrane Database Syst Rev 2012;
5:CD002006.

\item[33] Lawrence A, Lewis L, Hofmeyr GJ, Dowswell T, Styles C. Maternal
positions and mobility during first stage labour. Cochrane Database Syst Rev
2009; (2):CD003934.

\item[34] Singata M, Tranmer J, Gyte GM. Restricting oral fluid and food
intake during labour. Cochrane Database Syst Rev 2010;
(1):CD003930.

\item[35] Lavender T, Hart A, Smyth RM. Effect of partogram use on outcomes
for women in spontaneous labour at term. Cochrane Database Syst Rev 2012;
8:CD005461.

\item[36] Brasil. Portaria n\textsuperscript{o}
1.459/GM/MS
de 24 de junho de 2011, que instituiu, no âmbito do \textsc{sus}, a Rede Cegonha. Diário
Oficial da União 2011; 27 jun.

\item[37] d’Orsi E, Brüggemann OM, Diniz \textsc{csg}, Aguiar JM, Gusman CR, Torres
JA, et al. Desigualdades sociais e satisfação das mulheres com o atendimento ao
parto no Brasil: estudo nacional de base hospitalar. Cad Saúde Publica 2014; 30
Suppl:154-68.

\item[38] Secretaria de Gestão Estratégica e Participativa, Ministério da
Saúde. Resultados preliminares da pesquisa de satisfação com mulheres puérperas
atendidas no Sistema Único de Saúde – \textsc{sus} entre maio de 2012 e fevereiro de
2013. Brasília: Ministério da Saúde; 2013.

\item[39] Behague DP, Victora CG, Barros FC. Consumer demand for caesarean
sections in Brazil: informed decision making, patient choice, or social
inequality? A population based birth cohort study linking ethnographic and
epidemiological methods. \textsc{bmj} 2002; 324:942-5.

\item[40] Organisation for Economic Co-operation and Development. Health at a
glance 2011. Paris: \textsc{oecd} Publishing; 2011.

\item[41] Domingues \textsc{rmsm}, Dias \textsc{mab}, Nakamura-Pereira M, Torres JA, d’Orsi E,
Pereira \textsc{ape}, et al. Processo de decisão pelo tipo de parto no Brasil: da
preferência inicial das mulheres à via de parto final. Cad Saúde Pública 2014;
30 Suppl:101-16.

\item[42] Barros AJ, Santos IS, Matijasevich A, Domingues MR, Silveira M,
Barros FC, et al. Patterns of deliveries in a Brazilian birth cohort: almost
universal cesarean sections for the better-off. Rev Saúde Pública 2011;
45:635-43.

\item[43] Barbadoro P, Chiatti C, D’ Errico MM, Di Stanislao F, Prospero E.
Caesarean delivery in South Italy: women without choice. A cross sectional
survey. PLoS One 2012; 7:e43906.

\item[44] Villar J, Carroli G, Zavaleta N, Donner A, Wojdyla D, Faundes A, et
al. Maternal and neonatal individual risks and benefits associated with
caesarean delivery: multicentre prospective study. \textsc{bmj} 2007;
335:1025.

\item[45] Souza JP, Gulmezoglu A, Lumbiganon P, Laopaiboon M, Carroli G,
Fawole B, et al. Caesarean section without medical indications is associated
with an increased risk of adverse short-term maternal outcomes: the 2004-2008
\textsc{who} Global Survey on Maternal and Perinatal Health. \textsc{bmc} Med 2010;
8:71.

\item[46] Hansen AK, Wisborg K, Uldbjerg N, Henriksen TB. Risk of respiratory
morbidity in term infants delivered by elective caesarean section: cohort study.
\textsc{bmj} 2008; 336:85-7.

\item[47] Lee YM, D’Alton ME. Cesarean delivery on maternal request: the
impact on mother and newborn. Clin Perinatol 2008; 35:505-18.

\item[48] Wu S, Kocherginsky M, Hibbard JU. Abnormal placentation:
twenty-year analysis. Am J Obstet Gynecol 2005; 192:1458-61.

\end{itemize}

   \section*{Metadados não aplicados}
    \begin{itemize}
    \ifdef{\lingua}{\item[\textbf{língua do artigo}] \lingua}{}
    \ifdef{\journalid}{\item[\textbf{journalid}] \journalid}{}
    \ifdef{\journaltitle}{\item[\textbf{journaltitle}] \journaltitle}{}
    \ifdef{\journalsubtitle}{\item[\textbf{journalsubtitle}] \journalsubtitle}{}
    \ifdef{\historydateaccepted}{\item[\textbf{historydateaccepted}] \historydateaccepted}{}
    \ifdef{\historydatereceived}{\item[\textbf{historydatereceived}] \historydatereceived}{}
    \ifdef{\ack}{\item[\textbf{ack}] \ack}{}
    \ifdef{\transjournaltitle}{\item[\textbf{journaltitle}] \journaltitle}{}
    \ifdef{\transjournalsubtitle}{\item[\textbf{journalsubtitle}] \journaltitle}{}
    \ifdef{\abbrevjournaltitle}{\item[\textbf{abbrevjournaltitle}] \abbrevjournaltitle}{}
    \ifdef{\issnppub}{\item[\textbf{issnppub}] \issnppub}{}
    \ifdef{\issnepub}{\item[\textbf{issnepub}] \issnepub}{}
    \ifdef{\alttitleauthor}{\item[\textbf{alttitle}] \alttitleauthor}{}
    \ifdef{\alttitle}{\item[\textbf{alttitleauthor}] \alttitle}{}
    \ifdef{\publishername}{\item[\textbf{publishername}] \publishername}{}
    \ifdef{\publisherid}{\item[\textbf{publisherid}] \publisherid}{}
    \ifdef{\subject}{\item[\textbf{subject}] \subject}{} 
    \ifdef{\transtitle}{\item[\textbf{transtitle}] \transtitle}{}
    \ifdef{\authornotes}{\item[\textbf{authornotes}] \authornotes}{}
    \ifdef{\articleid}{\item[\textbf{articleid}] \articleid}{}
    \ifdef{\articledoi}{\item[\textbf{articledoi}] \articledoi}{}
    \ifdef{\volume}{\item[\textbf{volume}] \volume}{}
    \ifdef{\issue}{\item[\textbf{issue}] \issue}{}
    \ifdef{\fpage}{\item[\textbf{fpage}] \fpage}{}
    \ifdef{\lpage}{\item[\textbf{lpage}] \lpage}{}
    \ifdef{\permissions}{\item[\textbf{permissions}] \permissions}{}
    \ifdef{\copyrightyear}{\item[\textbf{copyrightyear}] \copyrightyear}{}

    \end{itemize}
\end{document}
