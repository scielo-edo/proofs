% Generated by jats2tex@0.11.1.0
\documentclass{article}
\usepackage[T1]{fontenc}
\usepackage[utf8]{inputenc} %% *
\usepackage[portuges,spanish,english,german,italian,russian]{babel} %% *
\usepackage{amstext}
\usepackage{authblk}
\usepackage{unicode-math}
\usepackage{multirow}
\usepackage{graphicx}
\usepackage{etoolbox}
\usepackage{xtab}
\usepackage{enumerate}
\usepackage{hyperref}
\usepackage{penalidades}
\usepackage[footnotesize,bf,hang]{caption}
\usepackage[nodayofweek,level]{datetime}
\usepackage[top=0.85in,left=2.75in,footskip=0.75in]{geometry}
\newlength\savedwidth
\newcommand\thickcline[1]{\noalign{\global
\savedwidth
\arrayrulewidth
\global\arrayrulewidth 2pt}
\cline{#1}
\noalign{\vskip\arrayrulewidth}
\noalign{\global\arrayrulewidth\savedwidth}}
\newcommand\thickhline{\noalign{\global
\savedwidth\arrayrulewidth
\global\arrayrulewidth 2pt}
\hline
\noalign{\global\arrayrulewidth\savedwidth}}
\usepackage{lastpage,fancyhdr}
\usepackage{epstopdf}
\pagestyle{myheadings}
\pagestyle{fancy}
\fancyhf{}
\setlength{\headheight}{27.023pt}
\lhead{\includegraphics[width=10mm]{logo.png}}
\rhead{\ifdef{\journaltitle}{\journaltitle}{}
\ifdef{\volume}{vol.\,\volume}{}
\ifdef{\issue}{(\issue)}{}
\ifdef{\fpage}{\fpage--\lpage\,pp.}}
\rfoot{\thepage/\pageref{LastPage}}
\renewcommand{\footrule}{\hrule height 2pt \vspace{2mm}}
\fancyheadoffset[L]{2.25in}
\fancyfootoffset[L]{2.25in}
\lfoot{\sf \ifdef{\articledoi}{\articledoi}{}}
\setmainfont{Linux Libertine O}
\renewcommand*{\thefootnote}{\alph{footnote}}
\makeatletter
\newcommand{\fn}{\afterassignment\fn@aux\count0=}
\newcommand{\fn@aux}{\csname fn\the\count0\endcsname}
\makeatother

\newcommand{\journalid}{Cad Saude Publica}
\newcommand{\journaltitle}{Cadernos de Saúde Pública}
\newcommand{\abbrevjournaltitle}{Cad. Saúde Pública}
\newcommand{\issnppub}{0102-311X}
\newcommand{\issnepub}{1678-4464}
\newcommand{\publishername}{Fundação Oswaldo Cruz}
\newcommand\articledoi{\textsc{doi} 10.1590/0102-311\textsc{xco}06S114}
\def\subject{Debate}\newcommand{\subtitlestyle}[1]{-- \emph{#1}\medskip}
\newcommand{\transtitlestyle}[1]{\par\medskip\Large #1}
\newcommand{\transsubtitlestyle}[1]{-- \Large\emph{ #1}}

\newcommand{\titlegroup}{
\ifdef{\subtitle}{\subtitlestyle{\subtitle}}{}
\ifdef{\transtitle}{\transtitlestyle{\transtitle}}{}
\ifdef{\transsubtitle}{\transsubtitlestyle{\transsubtitle}}{}}

\title{Compromisso com a mudança\titlegroup{}}
\author{Gomes, Maria A. S. Mendes}
\affil{Instituto Nacional de Saúde da Mulher da Criança e do
AdolescenteFundação Oswaldo Cruz}
\date{ 08 2014}
\def\volume{30}
\def\issue{Suppl 1}
\def\fpage{S41}
\def\lpage{S42}
\def\permissions{This is an Open Access article distributed under the terms of
the Creative
Commons Attribution Non-Commercial License, which permits unrestricted
non-commercial
use, distribution, and reproduction in any medium, provided the original work is
properly cited.}

\begin{document}
\selectlanguage{portuges}
\section*{Metadados não aplicados}
\begin{itemize}
\item[\textbf{língua do artigo}]{Português}
\ifdef{\journalid}{\item[\textbf{journalid}] \journalid}{}
\ifdef{\journaltitle}{\item[\textbf{journaltitle}] \journaltitle}{}

\ifdef{\journalsubtitle}{\item[\textbf{journalsubtitle}] \journaltitle}{}
\ifdef{\transjournaltitle}{\item[\textbf{journaltitle}] \journaltitle}{}
\ifdef{\transjournalsubtitle}{\item[\textbf{journalsubtitle}] \journaltitle}{}

\ifdef{\abbrevjournaltitle}{\item[\textbf{abbrevjournaltitle}]
\abbrevjournaltitle}{}
\ifdef{\issnppub}{\item[\textbf{issnppub}] \issnppub}{}
\ifdef{\issnepub}{\item[\textbf{issnepub}] \issnepub}{}
\ifdef{\publishername}{\item[\textbf{publishername}] \publishername}{}
\ifdef{\publisherid}{\item[\textbf{publisherid}] \publisherid}{}
\ifdef{\subject}{\item[\textbf{subject}] \subject}{}
\ifdef{\transtitle}{\item[\textbf{transtitle}] \transtitle}{}
\ifdef{\authornotes}{\item[\textbf{authornotes}] \authornotes}{}
\ifdef{\articleid}{\item[\textbf{articleid}] \articleid}{}
\ifdef{\articledoi}{\item[\textbf{articledoi}] \articledoi}{}
\ifdef{\volume}{\item[\textbf{volume}] \volume}{}
\ifdef{\issue}{\item[\textbf{issue}] \issue}{}
\ifdef{\fpage}{\item[\textbf{fpage}] \fpage}{}
\ifdef{\lpage}{\item[\textbf{lpage}] \lpage}{}
\ifdef{\permissions}{\item[\textbf{permissions}] \permissions}{}
\end{itemize}
\maketitle

A preocupação com a forma de nascer no Brasil não é recente. Desde a década de
90, o
Ministério da Saúde vem buscando garantir boas práticas obstétricas e neonatais
e a ambiência
adequada para um momento que deve ser de celebração e prazer. A Rede Cegonha
(lançada em 2011
como uma das cinco redes prioritárias nas políticas de saúde para o país)
reafirmou esse
compromisso e enfatizou a urgência na revisão dos processos de cuidado em
maternidades
brasileiras.

Entretanto, numa perspectiva nacional, a análise de indicadores perinatais e das
experiências
vividas por mulheres brasileiras identifica a manutenção de um cenário
desafiador. Foi sobre
esse cenário que o grupo de pesquisadores liderados pela Prof\textsuperscript{a}
Maria do Carmo Leal se debruçou com a pesquisa \textit{Nascer no Brasil}.
Gestores, pesquisadores, profissionais de saúde e movimentos sociais passam a
ter em mãos um
estudo inédito, de enorme vulto e que documenta uma prática assistencial
inexplicável na
perspectiva clínica e fisiológica dos processos de parturição.

Em minha contribuição para esse debate, compartilho da mesma inquietação
demonstrada pelos
autores que destacam a dificuldade na interpretação dos dados brasileiros e os
movimentos
contraditórios nos processos de cuidado em suas diferentes variáveis (região do
país, acesso à
rede privada, idade, anos de educação, raça/cor e parição). E é nessa
perspectiva
contraditória que apresento meu posicionamento: apesar do cenário encontrado,
estamos vivendo
um período absolutamente fértil e promissor no que se refere à possibilidade de
revisão de
conceitos, valores e, consequentemente, de práticas assistenciais.

Identifico esse momento mediante um conjunto de situações e fatores com
potencial de agir
sinergicamente para uma efetiva mudança na atenção ao parto e nascimento no
Brasil. Destaco
nesse processo: (1) acúmulo considerável de estudos sobre a inadequação do
cuidado atualmente
prestado e sobre a violência institucional; (2) experiências sustentadas em
cidades como Belo
Horizonte (Minas Gerais) e Rio de Janeiro que assumiram, há mais de uma década,
a
implementação de boas práticas perinatais como política pública, incluindo
revisão de rotinas
e monitoramento de indicadores de processo e resultado, intervenções na
ambiência e inserção
da enfermagem obstétrica na atenção ao parto de risco habitual; (3) ênfase das
ações do
governo federal em maternidades definidas como prioritárias no Norte e Nordeste,
por
intermédio do Plano de Qualificação de Maternidades; (4) consolidação da
Política Nacional de
Humanização como eixo condutor nas redes de atenção que garante apoio
institucional para a
implementação das boas práticas; (5) expansão da formação da enfermagem
obstétrica,
disponibilizando cada vez mais profissionais habilitadas para atenção ao parto
de risco
habitual; (6) valorização, por parte de atores estratégicos na obstetrícia
brasileira, da
prática clínica baseada em evidências em substituição à repetição de práticas
assistenciais
desnecessárias e prejudiciais à fisiologia do parto e nascimento de risco
habitual; (7)
crescente e potente manifestação de desagrado de mulheres com suas experiências
em
maternidades públicas e privadas.

São movimentos que vêm ocorrendo em diferentes matizes e com diferenças sociais
e regionais,
mas, não por isso, menos significativos. Os estudos sobre a inadequação do
cuidado e violência
institucional em maternidades vêm fortalecendo os movimentos sociais e
suscitando discussões e
ações em espaços como o Ministério Público e o Legislativo.

As experiências citadas em duas capitais do Sudeste reforçam a análise dos
autores sobre a
maior adesão às boas práticas nessa região, reconhecida pelo seu importante
papel na
disseminação nacional de conhecimentos e práticas. Da mesma forma, as
características da
implantação da Rede Cegonha (cuidado com “o modo de fazer” e oferta de apoio
institucional nos
territórios) demonstram que seu objetivo não se limita à expansão da capacidade
instalada, o
que, inevitavelmente, contribuiria para “mais do mesmo”.

Embora ainda incipiente em termos nacionais, a atuação crescente da enfermagem
obstétrica,
reconhecida pelo seu impacto positivo na melhoria dos resultados\textsuperscript{[}\textsuperscript{1}\textsuperscript{]}, é decisiva para a mudança na forma de nascer no Brasil. As
experiências de mais de uma década em maternidades e casas de parto sob gestão
municipal nas
cidades do Rio de Janeiro e Belo Horizonte têm servido como importantes espaços
de produção de
conhecimento e sistematização de práticas. Desde 2012, várias iniciativas
envolvendo o governo
federal, universidades e instituições científicas da enfermagem estão em curso
objetivando
acelerar a formação profissional em diferentes modalidades (residência,
especialização,
atualização) para atuação em maternidades de todo o país

A valorização da “obstetrícia baseada em evidências”, mundialmente reconhecida e
legitimada,
assim como sua influência positiva na formação de novas gerações de obstetras
vem sendo
percebida em maternidades que têm se destacado por indicadores bem mais próximos
aos de países
com os melhores índices perinatais. Parte desses profissionais já vem
experimentando, desde a
sua formação, ambientes institucionais que, da estrutura física às rotinas
(incluindo a
atuação da enfermagem obstétrica\textsuperscript{[}\textsuperscript{1}\textsuperscript{]}
), buscam
favorecer a fisiologia do parto de risco habitual.

O descontentamento com rotinas institucionais pouco centradas na mulher e sua
família vem
sendo fortalecido pela progressiva consciência de direitos que vêm permeando a
sociedade
brasileira. Nas unidades públicas, o direito ao acompanhante de escolha da
mulher é um
importante exemplo e vem sendo conquistado nacionalmente de forma irreversível.
Por outro
lado, dentre as mulheres usuárias do setor privado, observamos também movimentos
que
reivindicam de forma contundente o direito ao parto normal. Por mais estranheza
que essa
reivindicação possa causar em países como Inglaterra, França ou Canadá, no
contexto
brasileiro, com índices de cesárea no setor privado em torno de 90\%, ela pode
ser compreendida
com razoável facilidade. Inúmeros depoimentos de usuárias e médicos do setor
privado apontam
elementos externos ao processo clínico como definidores da realização da
cesárea. Esse
movimento também me parece irreversível e a veiculação, em circuito comercial,
do filme
\textit{O Renascimento do Parto}
é um instigante sinalizador de tal processo.

Na perspectiva dessas reflexões e da minha confiança nas possibilidades de
mudanças, partilho
também com os autores e leitores minha clareza sobre o longo caminho a
percorrer. Dentre as
inúmeras e complexas questões que exigem muita dedicação apresento três que
identifico como
centrais: (1) rediscussão e readequação dos processos de cuidado perinatal em
hospitais
universitários. A participação limitada da enfermagem obstétrica na cena do
parto nesses
espaços, reconhecidos como formadores de opinião, prejudica e retarda, no âmbito
acadêmico, a
revisão de práticas e rotinas clínicas, a pesquisa, a produção de conhecimento,
e a formação e
educação permanente das diferentes categorias profissionais; (2) assim como os
autores,
destaco a atenção pré-natal na rede pública como agente fundamental para maior
apropriação
pelas mulheres da fisiologia e importância do parto normal, desconstruindo o
senso comum das
experiências negativas frequentemente veiculadas em nosso meio; (3) revisão dos
processos de
produção do cuidado no setor privado minimizando o impacto de fatores externos
ao cuidado
clínico na definição do tipo de parto.

Finalizo destacando e valorizando o papel de uma instituição absolutamente
estratégica para
as revisões necessárias e urgentes no cuidado obstétrico no Brasil. Em recente e
histórico
posicionamento sobre o parto e o nascimento de risco habitual, a Federação
Brasileira de
Ginecologia e Obstetrícia (\textsc{febrasgo}. http://www.febrasgo.org.br/site/?p=7069,
acessado em
20/Jan/2014) defende práticas reconhecidas por aumentar a satisfação da
parturiente com parto
vaginal (dieta, privacidade, conforto, deambulação, relaxamento, liberdade de
escolha da
posição no segundo período, uso restrito da episiotomia), apoia os centros de
parto, sem
distocia, intra-hospitalares, a inserção da enfermagem obstétrica em serviços
públicos e
privados e a criação de equipes de plantão para assistência ao parto normal na
rede
suplementar. É um posicionamento que reforça minha confiança em um novo tempo
para quem nasce
no Brasil.

\section*{}
\begin{itemize}

\item[1] Sandall J, Soltani H, Gates S, Shennan A, Devane D. Midwife-led
continuity
models versus other models of care for childbearing women. Cochrane Database
Syst Rev
2013; 8: CD004667.

\end{itemize}

\end{document}
