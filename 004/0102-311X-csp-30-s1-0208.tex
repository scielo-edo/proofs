% Generated by jats2tex@0.11.1.0
\documentclass{article}
\usepackage[T1]{fontenc}
\usepackage[utf8]{inputenc} %% *
\usepackage[portuges,spanish,english,german,italian,russian]{babel} %% *
\usepackage{amstext}
\usepackage{authblk}
\usepackage{unicode-math}
\usepackage{multirow}
\usepackage{graphicx}
\usepackage{etoolbox}
\usepackage{xtab}
\usepackage{enumerate}
\usepackage{hyperref}
\usepackage{penalidades}
\usepackage[footnotesize,bf,hang]{caption}
\usepackage[nodayofweek,level]{datetime}
\usepackage[top=0.85in,left=2.75in,footskip=0.75in]{geometry}
\newlength\savedwidth
\newcommand\thickcline[1]{\noalign{\global
\savedwidth
\arrayrulewidth
\global\arrayrulewidth 2pt}
\cline{#1}
\noalign{\vskip\arrayrulewidth}
\noalign{\global\arrayrulewidth\savedwidth}}
\newcommand\thickhline{\noalign{\global
\savedwidth\arrayrulewidth
\global\arrayrulewidth 2pt}
\hline
\noalign{\global\arrayrulewidth\savedwidth}}
\usepackage{lastpage,fancyhdr}
\usepackage{epstopdf}
\pagestyle{myheadings}
\pagestyle{fancy}
\fancyhf{}
\setlength{\headheight}{27.023pt}
\lhead{\includegraphics[width=10mm]{logo.png}}
\rhead{\ifdef{\journaltitle}{\journaltitle}{}
\ifdef{\volume}{vol.\,\volume}{}
\ifdef{\issue}{(\issue)}{}
\ifdef{\fpage}{\fpage--\lpage\,pp.}}
\rfoot{\thepage/\pageref{LastPage}}
\renewcommand{\footrule}{\hrule height 2pt \vspace{2mm}}
\fancyheadoffset[L]{2.25in}
\fancyfootoffset[L]{2.25in}
\lfoot{\sf \ifdef{\articledoi}{\articledoi}{}}
\setmainfont{Linux Libertine O}
\renewcommand*{\thefootnote}{\alph{footnote}}
\makeatletter
\newcommand{\fn}{\afterassignment\fn@aux\count0=}
\newcommand{\fn@aux}{\csname fn\the\count0\endcsname}
\makeatother

\newcommand{\journalid}{Cad Saude Publica}
\newcommand{\journaltitle}{Cadernos de Saúde Pública}
\newcommand{\abbrevjournaltitle}{Cad. Saúde
Pública}
\newcommand{\issnppub}{0102-311X}
\newcommand{\issnepub}{1678-4464}
\newcommand{\publishername}{Fundação Oswaldo Cruz}
\newcommand\articledoi{\textsc{doi} 10.1590/0102-311X00176913}
\def\subject{\textsc{artigo}}\newcommand{\subtitlestyle}[1]{-- \emph{#1}\medskip}
\newcommand{\transtitlestyle}[1]{\par\medskip\Large #1}
\newcommand{\transsubtitlestyle}[1]{-- \Large\emph{ #1}}

\newcommand{\titlegroup}{
\ifdef{\subtitle}{\subtitlestyle{\subtitle}}{}
\ifdef{\transtitle}{\transtitlestyle{\transtitle}}{}
\ifdef{\transsubtitle}{\transsubtitlestyle{\transsubtitle}}{}}

\title{Estrutura das maternidades: aspectos relevantes para a
qualidade da atenção ao parto e nascimento\titlegroup{}}
\newcommand{\transtitle}{Estructura de los hospitales de maternidad: aspectos
relevantes
para la calidad de la atención durante el parto y el
nacimiento}
\author[{1}]{Bittencourt, Sonia Duarte de Azevedo}
\author[{1}]{Reis, Lenice Gnocchi da Costa}
\author[{2}]{Ramos, Márcia Melo}
\author[{3}]{Rattner, Daphne}
\author[{1}]{Rodrigues, Patrícia Lima}
\author[{4}]{Neves, Dilma Costa Oliveira}
\author[{5}]{Arantes, Sandra Lúcia}
\author[{1}]{Leal, Maria do Carmo}
\affil[1]{Fundação Oswaldo Cruz}
\affil[2]{Secretaria Municipal de Saúde}
\affil[3]{Universidade de Brasília}
\affil[4]{Centro Universitário do Estado do
Pará}
\affil[5]{Universidade Federal de Mato Grosso do
Sul}
\def\authornotes{Correspondência\textit{S. A. Bittencourt}
\textit{Departamento de Epidemiologia e Métodos Quantitativos em Saúde, Escola
Nacional de Saúde Pública Sergio Arouca, Fundação Oswaldo Cruz.}
\textit{Rua Leopoldo Bulhões 1480, Rio de Janeiro, RJ}
\textit{21041-210, Brasil.}
\textit{sonia@ensp.fiocruz.br}

Colaboradores

S. D. A. Bittencourt foi da coordenação nacional do estudo e colaborou na
análise e redação do artigo. L. G. C. Reis colaborou na elaboração do
questionário, na análise e redação do artigo. M. M. Melo e P. L. Rodrigues
colaboraram na elaboração do questionário e na redação do artigo. D.
Rattner, D. O. Neves e S. L. Arantes foram coordenadoras estaduais do estudo
e colaboraram na análise e redação do artigo. M. C. Leal foi a coordenadora
nacional do estudo e colaborou na análise e redação do artigo.}
\date{ 08 2014}
\def\volume{30}
\def\issue{Suppl 1}
\def\fpage{S208}
\def\lpage{S219}
\def\permissions{This is an Open Access article distributed under the terms of
the
Creative Commons Attribution Non-Commercial License, which permits
unrestricted non-commercial use, distribution, and reproduction in any
medium, provided the original work is properly cited.}
\newcommand{\kwdgroup}{Maternidades, Estrutura dos Serviços, Qualidade da
Assistência à Saúde}
\newcommand{\kwdgroupes}{Maternidades, Estructura de los Servicios, Calidade de
la Atención de Saúde}
%%% Nota %%%%%%%%%%%%%%%%%%%%%%%%%%%%%%%%%%%%%%%%%%%%%%%%%%%%%%%%
\expandafter\newcommand\csname \endcsname{
Financiamento

Conselho Nacional de Desenvolvimento Científico e Tecnológico (\textsc{cnp}q);
Departamento de Ciência e Tecnologia, Secretaria de Ciência, Tecnologia e
Insumos Estratégicos, Ministério da Saúde; Escola Nacional de Saúde Pública
Sergio Arouca, Fundação Oswaldo Cruz (Projeto \textsc{inova}) e Fundação de Amparo à
Pesquisa do Estado do Rio de Janeiro (Faperj).}

\begin{document}
\selectlanguage{portuges}
\section*{Metadados não aplicados}
\begin{itemize}
\item[\textbf{língua do artigo}]{Português}
\ifdef{\journalid}{\item[\textbf{journalid}] \journalid}{}
\ifdef{\journaltitle}{\item[\textbf{journaltitle}] \journaltitle}{}

\ifdef{\journalsubtitle}{\item[\textbf{journalsubtitle}] \journaltitle}{}
\ifdef{\transjournaltitle}{\item[\textbf{journaltitle}] \journaltitle}{}
\ifdef{\transjournalsubtitle}{\item[\textbf{journalsubtitle}] \journaltitle}{}

\ifdef{\abbrevjournaltitle}{\item[\textbf{abbrevjournaltitle}]
\abbrevjournaltitle}{}
\ifdef{\issnppub}{\item[\textbf{issnppub}] \issnppub}{}
\ifdef{\issnepub}{\item[\textbf{issnepub}] \issnepub}{}
\ifdef{\publishername}{\item[\textbf{publishername}] \publishername}{}
\ifdef{\publisherid}{\item[\textbf{publisherid}] \publisherid}{}
\ifdef{\subject}{\item[\textbf{subject}] \subject}{}
\ifdef{\transtitle}{\item[\textbf{transtitle}] \transtitle}{}
\ifdef{\authornotes}{\item[\textbf{authornotes}] \authornotes}{}
\ifdef{\articleid}{\item[\textbf{articleid}] \articleid}{}
\ifdef{\articledoi}{\item[\textbf{articledoi}] \articledoi}{}
\ifdef{\volume}{\item[\textbf{volume}] \volume}{}
\ifdef{\issue}{\item[\textbf{issue}] \issue}{}
\ifdef{\fpage}{\item[\textbf{fpage}] \fpage}{}
\ifdef{\lpage}{\item[\textbf{lpage}] \lpage}{}
\ifdef{\permissions}{\item[\textbf{permissions}] \permissions}{}
\end{itemize}
\maketitle

\begingroup

\begin{abstract}

Avaliar aspectos da estrutura de uma amostra de maternidades do Brasil. A
estrutura foi avaliada tendo como referências as normas do Ministério da Saúde e
englobou: localização geográfica, volume de partos, existência de \textsc{uti}, atividade
de ensino, qualificação de recursos humanos, disponibilidade de equipamentos e
medicamentos. Os resultados evidenciam diferenças na qualificação e na
disponibilidade de equipamentos e insumos dos serviços de atenção ao parto e
nascimento segundo o tipo de financiamento, regiões do país e grau de
complexidade. As regiões Norte/Nordeste e Centro-oeste apresentaram os maiores
problemas. No Sul/Sudeste, os hospitais estavam melhores estruturados, atingindo
proporções satisfatórias em vários dos aspectos estudados, próximas ou mesmo
superiores ao patamar da rede privada. O presente estudo traz para o debate a
qualidade da estrutura dos serviços hospitalares ofertados no país, e sublinha a
necessidade de desenvolvimento de estudos analíticos que considerem o processo e
os resultados da assistência.

\iflanguage{portuges}{\medskip\noindent\textbf{Palavras-chave:} \kwdgroup}{}
\iflanguage{english}{\medskip\noindent\textbf{Keywords:} \kwdgroupen}{}
\iflanguage{spanish}{\medskip\noindent\textbf{Palavras claves:} \kwdgroupes}{}
\iflanguage{french}{\medskip\noindent\textbf{Mots clés:} \kwdgroupfr}{}
\end{abstract}
\endgroup

\begingroup
\renewcommand{\section}[1]{\subsection*{#1}}
\begin{otherlanguage}{spanish}

\begin{abstract}

El presente estudio evalúa aspectos en cuanto a la estructura de una muestra de
hospitales de maternidad en Brasil. El marco ha sido evaluado en función de
patrones de referencia del Ministerio de Salud y abarca: ubicación geográfica,
volumen de nacimientos, presencia de IU, actividades de aprendizaje, formación
de recursos humanos, disponibilidad de equipos y medicamentos. Los resultados
muestran diferencias en la cualificación y disponibilidad de equipos y servicios
de suministros para el parto, según regiones, y su grado de complejidad. El
Norte/Nordeste y Centro-oeste mostraron los mayores problemas. En el
Sur/Sudeste, los hospitales estaban mejor estructurados, alcanzando proporciones
satisfactorias en diversos aspectos del estudio, cercanos o justo por encima del
nivel de la red privada. Este estudio aporta al debate la cuestión la calidad
estructural de los servicios hospitalarios que se ofrecen en el país, y hace
hincapié en la necesidad de desarrollo de estudios de análisis que tengan en
cuenta los procesos y resultados de la atención.

\ifdef{\kwdgroupes}{\medskip\noindent\textbf{Palavras claves:} \kwdgroupes}{}
\end{abstract}
\end{otherlanguage}
\endgroup
\section{Introdução}

Nas últimas décadas houve inúmeros avanços na atenção à saúde da mulher em
consequência de esforços coletivos, com marcada presença de movimentos sociais.
A
inscrição da morte materna como grave violação dos direitos humanos, sem dúvida,
contribuiu para que a diminuição da mortalidade materna fosse considerada como
um
dos \textit{Objetivos de Desenvolvimento do Milênio}\textsuperscript{[}\textsuperscript{1}\textsuperscript{]}.

Nesse período, a mortalidade materna teve significativa diminuição no Brasil,
embora
a meta de redução em 75\%, até 2015, tomando como base os indicadores de 1990,
não
será atingi- da\textsuperscript{[}\textsuperscript{2}\textsuperscript{]}. A mortalidade
infantil também obteve uma importante redução, principalmente à custa do
componente
pós-neonatal\textsuperscript{[}\textsuperscript{2}\textsuperscript{]}. É sabido que a
maior parte dessas mortes, tanto de mães como de recém-nascidos, é evitável\textsuperscript{[}\textsuperscript{3}\textsuperscript{]}
e ocorre dentro dos hospitais\textsuperscript{[}\textsuperscript{4}\textsuperscript{]}.

Desse modo, a qualidade dos serviços obstétricos assume um importante papel para
o
alcance de melhorias na saúde materna e infantil. No entanto, avaliar a
qualidade
dos serviços obstétricos não é simples, pois são dois pacientes envolvidos, que
por
vezes podem ter necessidades conflitantes, e esse balanço requer um cálculo
complexo
e cuidadoso\textsuperscript{[}\textsuperscript{5}\textsuperscript{]}.

Para mensurar a qualidade da atenção à saúde, Donabedian\textsuperscript{[}\textsuperscript{6}\textsuperscript{]}
propôs um arcabouço teórico baseado na tríade
estrutura, processo e resultado, que tem sido bastante utilizada na investigação
de
serviços de saúde\textsuperscript{[}\textsuperscript{7}\textsuperscript{]}. A estrutura se
refere às características relativamente mais estáveis dos serviços, incluindo
desde
a disponibilidade de recursos humanos, financeiros, equipamentos e insumos, até
o
modo como estão organizados. Por si só, a estrutura não determina a qualidade da
atenção, no entanto, há muito já se aponta que suas deficiências podem
interferir
nos resultados. Stilwell et al.\textsuperscript{[}\textsuperscript{8}\textsuperscript{]}

analisaram as maternidades de uma região da Inglaterra e demonstraram haver
relação
entre o número de pediatras e a taxa de mortalidade perinatal.

Pesquisas realizadas nas maternidades do país mostraram deficiências na
disponibilidade de equipamentos, instrumentais cirúrgicos, capacitação de
recursos
humanos e na presença de Unidades de Terapia Intensiva (\textsc{uti})\textsuperscript{[}\textsuperscript{9}\textsuperscript{]}\textsuperscript{,}\textsuperscript{[}\textsuperscript{10}\textsuperscript{]}\textsuperscript{,}\textsuperscript{[}\textsuperscript{11}\textsuperscript{]}\textsuperscript{,}\textsuperscript{[}\textsuperscript{12}\textsuperscript{]}\textsuperscript{,}\textsuperscript{[}\textsuperscript{13}\textsuperscript{]}, evidenciando lacunas e potencialidades do sistema de
saúde para atender de forma resolutiva a assistência durante o parto e
nascimento.

Espera-se com este estudo traçar um panorama abrangente de aspectos relacionados
à
estrutura da amostra de estabelecimentos de saúde participantes da pesquisa
\textit{Nascer no Brasil}\textsuperscript{[}\textsuperscript{14}\textsuperscript{]}.

\section{Método}

O estudo \textit{Nascer no Brasil}
: pesquisa nacional sobre parto e
nascimento”\textsuperscript{[}\textsuperscript{14}\textsuperscript{]}, é uma coorte de
base hospitalar, cujo objetivo foi avaliar as condições de assistência ao parto
e
nascimento no país, realizado no período de fevereiro de 2011 a outubro de 2012.

Foram elegíveis para o estudo todos os estabelecimentos de saúde que realizaram
mais
de 500 partos no ano 2007, segundo o Sistema de Informações sobre Nascidos Vivos
(\textsc{sinasc}).

A amostra foi estratificada de acordo com as cinco grandes regiões do país,
localização (capital, fora da capital) e tipo de estabelecimento conforme o
financiamento dos partos realizados (privado, público ou misto). Como
estabelecimentos mistos foram classificados aqueles que constavam no Cadastro
Nacional de Estabelecimentos de Saúde como privados, mas que também tinham
leitos
contratados pelo setor público. Esses, juntamente com os públicos, têm como
fonte de
financiamento o Sistema Único de Saúde (\textsc{sus}).

Em cada uma das cinco macrorregiões foram gerados seis estratos: capital
(privado/misto/público) e fora da capital (privado/misto/público). Ao final, a
amostra foi composta por trinta estratos. Em cada um selecionou-se uma amostra
probabilística em dois estágios. No primeiro, foram selecionados os
estabelecimentos
de saúde e, no segundo, as puérperas e seus conceptos.

Os pesos amostrais foram baseados no inverso das probabilidades de inclusão na
amostra. Para assegurar que as estimativas totais fossem iguais ao número de
hospitais da amostra, em 2011, um processo de calibração foi utilizado em cada
estrato de seleção. Os resultados apresentados são as estimativas para o
universo
hospitalar da pesquisa (1.402) com base na amostra de 266 hospitais visitados.

Para alcançar os objetivos do estudo, além dos questionários aplicados às 23.940
puérperas selecionadas, um questionário, com dados para conhecer a estrutura do
hospital, foi preenchido pelos supervisores de campo durante a entrevista
pessoal
com os gestores dos estabelecimentos de saúde amostrados.

Esse instrumento de coleta de dados foi desenvolvido com base na legislação
vigente:
\textit{\textsc{rdc}/Anvisa
n}\textsuperscript{\textit{o}}
\textit{
36}, de 3 de junho de 2008\textsuperscript{[}\textsuperscript{15}\textsuperscript{]}
; \textit{\textsc{rdc}/ Anvisa}

n\textsuperscript{o}
50, de 21 de fevereiro de 2002\textsuperscript{[}\textsuperscript{16}\textsuperscript{]}
; \textit{Portaria GM/MS}

n\textsuperscript{o}
1.091, de 25 de agosto de 1999\textsuperscript{[}\textsuperscript{17}\textsuperscript{]}
; \textit{Portaria GM/MS}
n\textsuperscript{o}
3.432, de 12 de agosto de 1998\textsuperscript{[}\textsuperscript{18}\textsuperscript{]}
; \textit{Portaria
GM/MS}
n\textsuperscript{o}
2.048, de 5 de novembro de
2002\textsuperscript{[}\textsuperscript{19}\textsuperscript{]}
; \textit{Portaria
GM/MS}
n\textsuperscript{o}
1.071, de 4 de julho de 2005\textsuperscript{[}\textsuperscript{20}\textsuperscript{]}
; \textit{Portaria
GM/MS}
n\textsuperscript{o}
2.418 de 2 de dezembro de
2005\textsuperscript{[}\textsuperscript{21}\textsuperscript{]}.

Os hospitais foram classificados: segundo o volume de partos realizados,
categorizado\textsuperscript{[}\textsuperscript{22}\textsuperscript{]}
em baixo (até 999 partos),
médio (de 1.000 a 2.999) e alto (de 3.000 e mais partos); a existência de \textsc{uti} de
adulto e/ou de neonatal; a realização de atividade de ensino e ser de
referência,
via central de leitos, para parto de alto risco.

As questões relativas aos recursos humanos limitaram-se à verificação da
existência
de coordenador médico e de enfermagem com título de especialista nos serviços de
obstetrícia e de neonatologia.

Dentre os requisitos de estrutura exigidos pela legislação brasileira foi
verificada
a existência de equipamentos de emergência para o cuidado da mãe
(respirador/ventilador mecânico, reanimador manual, laringoscópico e tubo
orotraqueal) e para os neonatos (laringoscópio e tubo endotraqueal neonatal,
sondas
traqueais neonatais sem válvula, dispositivo para aspirar mecônio e aspirador
com
manômetro e oxigênio, sonda de aspiração gástrica e material para ventilação).
Ademais, foi perguntado sobre a existência de: banco de sangue e unidade
transfusional, laboratório de patologia clínica e disponibilidade de ambulância
para
a mulher e para o recém-nascido.

Também foi avaliada a disponibilidade dos seguintes medicamentos no hospital:
anti-hipertensivos, ansiolíticos/hipnóticos, corticosteroides, ocitócitos,
inibidores da contratilidade uterina, coagulantes/hemostáticos para a mulher e
para
o recém-nascido e, especificamente, sulfato de magnésio (anticonvulsivante) e
surfactante (indutor da maturação do pulmão do recém- nascido), além do colírio
profilático da oftalmia gonocócica e da imunoglobulina anti-D para mulher Rh
negativo.

A análise compreendeu a distribuição de frequência relativa das variáveis
estudadas,
de acordo com o tipo de financiamento dos estabelecimentos (público, misto e
privado). E, internamente a cada um desses três estratos, no primeiro momento,
os
hospitais foram agrupados por similaridade em três grandes regiões:
Norte/Nordeste;
Sul/Sudeste e Centro-oeste. Por último, os dados de estrutura foram observados
segundo dois grupos de hospitais, o de maior complexidade, classificados como
aqueles cuja \textsc{uti} neonatal tinha seis ou mais leitos, além de leitos de \textsc{uti} para
adultos, ficando os restantes como de menor complexidade.

O projeto de pesquisa foi aprovado pelo Comitê de Ética em Pesquisa da Escola
Nacional de Saúde Pública Sergio Arouca/Fiocruz (parecer
n\textsuperscript{o}
92/10). Não há conflito de interesse em
relação aos métodos utilizados como parte da investigação ou interesse
financeiro
dos pesquisadores.

\section{Resultados}

Do total de estabelecimentos de saúde investigados, 36,1\% eram públicos, 45,7\%
mistos
e o restante privados (18,2\%). Ao analisar as três grandes regiões estudadas,
observa-se que no Norte/Nordeste pouco mais da metade do total dos hospitais
eram
públicos, no Centro-oeste este percentual foi de 43\% e no Sul/Sudeste equivaleu
a
23,5\%. Os hospitais mistos representaram 24,6\% no Norte/Nordeste, 34\% na
Região
Centro-oeste e no Sul/Sudeste, a participação chegou a 60,9\% dos
estabelecimentos. A
presença dos hospitais privados variou de 15,5\% no Sul/Sudeste, o valor mais
baixo,
a 23\% na Região Centro-oeste, o valor mais alto.

Na Tabela 1, nota-se que quase 30\% dos
estabelecimentos do tipo público e privado estão localizados na capital dos
estados,
já para os classificados como misto o percentual foi de 13,4\%. Observa-se
também que
no Centro-oeste o padrão se altera, a maior parte dos hospitais públicos e
mistos
está localizada nas capitais (63\% e 68\%, respectivamente), sugerindo que há
problemas de cobertura para usuárias residentes fora das capitais desta região.

Tabela 1Frequência relativa dos estabelecimentos de saúde por tipo de
financiamento e regiões geográficas, segundo localização na capital e
algumas características de estrutura. Brasil, 2010 *.
\begin{table}
\begin{xtabular}{ l | l | l | l l | l | l | l | l | l | l | l | l | l | l | l }
\hline
Público & Misto & Privado & Brasil\\ \hline
N/NE (\%) & S/SE (\%) & CO (\%) & N/NE (\%) & S/SE (\%) & CO (\%) & N/NE (\%) &
S/SE (\%) & CO (\%) & Público (\%) & Misto (\%) & Privado (\%)\\ \hline
Localização na capital
& 16,9
& 38,7
& 62,8
& 18,4
& 8,1
& 67,6
& 32,4
& 23,0
& 39,1
& 28,8
& 13,4
& 28,3
\\ \hline

Volume de parto
&
&
&
&
&
&
&
&
&
&
&
&
\\ \hline

Alto
& 15,1
& 36,0
& 7,1
& 28,0
& 10,0
& 17,6
& 3,7
& 14,8
& 0,0
& 22,3
& 13,9
& 8,7
\\ \hline

Médio
& 46,4
& 53,8
& 76,2
& 52,8
& 51,8
& 26,5
& 49,1
& 50,8
& 39,1
& 51,7
& 50,6
& 48,8
\\ \hline

Baixo
& 38,5
& 10,2
& 16,7
& 19,2
& 38,3
& 55,9
& 47,2
& 34,4
& 60,9
& 26,0
& 35,5
& 42,5
\\ \hline

Com \textsc{uti}
& 32,9
& 69,2
& 48,8
& 55,6
& 67,2
& 42,4
& 76,1
& 97,6
& 69,6
& 47,7
& 63,7
& 86,2
\\ \hline

Tipo de \textsc{uti}
&
&
&
&
&
&
&
&
&
&
&
&
\\ \hline

\textsc{uti} adulto
& 29,7
& 3,9
& 23,8
& 57,1
& 38,0
& 14,3
& 0,0
& 15,8
& 0,0
& 15,4
& 40,4
& 8,7
\\ \hline

\textsc{uti} neonatal
& 29,7
& 15,6
& 0,0
& 2,9
& 3,1
& 7,1
& 9,6
& 14,2
& 31,3
& 19,6
& 3,2
& 13,7
\\ \hline

\textsc{uti} adulto e neonatal
& 40,7
& 80,5
& 76,2
& 40,0
& 59,0
& 78,6
& 90,4
& 70,0
& 68,8
& 65,0
& 56,4
& 77,6
\\ \hline

Realiza atividade de ensino
& 69,4
& 82,8
& 100,0
& 52,4
& 79,2
& 85,3
& 0,0
& 21,1
& 13,0
& 76,9
& 74,1
& 11,4
\\ \hline

Referência para parto de alto risco
& 20,1
& 56,1
& 32,6
& 7,1
& 30,1
& 20,6
& ..
& ..
& ..
& 34,5
& 25,2
& ..
\\ \hline

\end{xtabular}
\end{table}

CO: Centro-oeste; N/NE: Norte/Nordeste; S/SE: Sul/Sudeste; \textsc{uti}:
unidade de terapia intensiva. * Valores ponderados segundo plano
amostral. ..: não se aplica aos hospitais privados.

CO: Centro-oeste; N/NE: Norte/Nordeste; S/SE: Sul/Sudeste; \textsc{uti}:
unidade de terapia intensiva. * Valores ponderados segundo plano
amostral. ..: não se aplica aos hospitais privados.

O volume de partos por maternidade foi outro aspecto abordado neste estudo. Para
o
país predomina os estabelecimentos com volume de parto médio (entre 1.000 e
2.999
partos ao ano), com exceção da Região Centro-oeste, onde preponderaram hospitais
com
baixos volumes de partos, tanto entre hospitais mistos (56\%) como privados
(61\%).

Observa-se, ainda na Tabela 1, que hospitais
com disponibilidade de leitos de \textsc{uti} foram mais frequentes no Sul/Sudeste (69\%
dos
públicos, 67\% dos mistos e 98\% dos privados) e, em relação ao financiamento, é
marcante a predominância em estabelecimentos privados (86\%); é mais frequente a
oferta de ambos os leitos de \textsc{uti}, tanto neonatal como de adulto.

Atividades de ensino são realizadas majoritariamente em hospitais públicos
(77\%) e
nos mistos (74\%), e prevalece nos hospitais estudados no Centro-oeste (100\%
dos
públicos e 85\% dos mistos).

Especificamente para os estabelecimentos públicos e mistos foi perguntado se a
maternidade era referência para parto de alto risco e se participavam da central
de
regulação de leitos. Entre os hospitais públicos é maior a proporção de
referências
para alto risco (35\%), havendo 25\% de estabelecimentos de referência entre os
hospitais mistos; no Sul/Sudeste, 56\% dos hospitais públicos e 30\% dos mistos
serviam de referência.

A responsabilidade técnica pela execução das atividades de assistência nas
especialidades, em geral, deveria ser de pessoas com liderança e preparo, para
manter os serviços atualizados em termos de conhecimento, tecnologia, e outros
aspectos referentes à manutenção da qualidade da assistência. Espera-se que, com
a
especialização, os profissionais possam manejar melhor esses aspectos. Como se
vê na
Tabela 2, para todos os tipos de
financiamento há menor proporção de coordenadores médicos e de enfermagem em
obstetrícia no Norte/Nordeste. Dentre os médicos, a maioria tinha
especialização,
havendo maior déficit de coordenadores de enfermagem com especialização. Para a
Neonatologia, a variação é maior, desde 32\% de disponibilidade de médicos
coordenadores em hospitais públicos do Norte/Nordeste e mistos do Centro-oeste,
até
100\% nos hospitais privados do Norte/Nordeste. Quanto a enfermeiros
coordenadores,
variou de 35\% em hospitais públicos do Norte/Nordeste a 82\% em mistos do
Centro-Oeste. Observa-se que a proporção de hospitais com os quatro
coordenadores
com especialização predomina no Sul/Sudeste e em hospitais públicos, e é muito
baixa
principalmente no Norte/Nordeste, possivelmente por ausência de especialistas.

Tabela 2Frequência relativa dos estabelecimentos de saúde por tipo de
financiamento e regiões geográficas, segundo existência de coordenador
especialista. Brasil, 2010 *.
\begin{table}
\begin{xtabular}{ l | l | l | l l | l | l | l | l | l | l | l | l | l | l | l }
\hline
Público & Misto & Privado & Brasil\\ \hline
N/NE (\%) & S/SE (\%) & CO (\%) & N/NE (\%) & S/SE (\%) & CO (\%) & N/NE (\%) &
S/SE (\%) & CO (\%) & Público (\%) & Misto (\%) & Priva- do (\%)\\ \hline
Serviço
&
&
&
&
&
&
&
&
&
&
&
&
\\ \hline

Obstetrícia
&
&
&
&
&
&
&
&
&
&
&
&
\\ \hline

Médico
&
&
&
&
&
&
&
&
&
&
&
&
\\ \hline

Coordenador
& 50,4
& 95,7
& 93,0
& 73,6
& 91,9
& 54,6
& 69,4
& 73,8
& 73,9
& 70,6
& 86,3
& 71,8
\\ \hline

Com especiali-zação
& 85,7
& 100,0
& 100,0
& 82,6
& 100,0
& 88,9
& 100,0
& 100,0
& 100,0
& 94,4
& 96,7
& 100,0
\\ \hline

Enfermeiro
&
&
&
&
&
&
&
&
&
&
&
&
\\ \hline

Coordenador
& 63,3
& 71,5
& 83,7
& 74,6
& 92,1
& 100,0
& 61,1
& 63,4
& 73,9
& 68,0
& 89,1
& 63,4
\\ \hline

Com espe-cialização
& 55,7
& 67,7
& 36,1
& 45,7
& 51,7
& 23,5
& 90,9
& 47,4
& 23,5
& 58,0
& 49,0
& 62,7
\\ \hline

Neonatologia
&
&
&
&
&
&
&
&
&
&
&
&
\\ \hline

Médico
&
&
&
&
&
&
&
&
&
&
&
&
\\ \hline

Coordenador
& 31,5
& 72,0
& 76,2
& 54,0
& 64,0
& 32,4
& 100,0
& 80,5
& 82,6
& 50,2
& 60,3
& 89,0
\\ \hline

Com espe-cialização
& 85,2
& 100,0
& 93,8
& 64,7
& 91,2
& 100,0
& 52,3
& 100,0
& 100,0
& 94,1
& 86,8
& 77,0
\\ \hline

Enfermeiro
&
&
&
&
&
&
&
&
&
&
&
&
\\ \hline

Coordenador
& 34,9
& 72,0
& 69,0
& 52,0
& 50,7
& 81,8
& 64,2
& 68,0
& 73,9
& 51,4
& 52,6
& 66,9
\\ \hline

Com espe-cialização
& 47,4
& 79,1
& 55,2
& 32,3
& 43,0
& 18,5
& 61,4
& 75,9
& 54,1
& 64,4
& 39,2
& 71,8
\\ \hline

Todos os coordenadores com especialização
& 9,4
& 29,2
& 18,6
& 7,2
& 18,7
& 8,8
& 4,5
& 11,4
& 12,5
& 17,4
& 15,9
& 8,7
\\ \hline

\end{xtabular}
\end{table}

CO: Centro-oeste; N/NE: Norte/Nordeste; S/SE: Sul/Sudeste. * Valores
ponderados segundo plano amostral.

CO: Centro-oeste; N/NE: Norte/Nordeste; S/SE: Sul/Sudeste. * Valores
ponderados segundo plano amostral.

A Tabela 3 mostra a disponibilidade de
equipamentos essenciais e estratégicos para viabilizar a sobrevivência materna e
do
recém-nascido nas emergências. Para as emergências maternas, a disponibilidade é
maior nos estabelecimentos privados (99\%) e mistos (89\%), e menor nos públicos
(71\%), havendo maior carência no Norte/Nordeste: apenas 56\% dos hospitais
públicos
destas regiões contavam com todos os equipamentos considerados essenciais.
Também
para as emergências do recém-nascido prevalece a disponibilidade maior nos
estabelecimentos privados (88\%), havendo 82\% nos mistos e 68\% nos públicos.
Novamente, a deficiência é maior nos hospitais do Norte/Nordeste: apenas 45\%
dos
hospitais públicos e 64\% dos mistos contavam com todos os equipamentos. A
disponibilidade de banco de sangue ou agência transfusional variou de 48\%, nos
estabelecimentos mistos do Norte/Nordeste, a 84\% nos mistos do Sul/Sudeste,
sendo
75\% nos mistos, 69\% nos públicos e 67\% nos privados; laboratórios de análises
clínicas foram encontrados entre 70\% dos mistos do Norte/Nordeste e 100\% dos
públicos do Centro-oeste, sendo 92\% nos públicos, 87\% nos privados e 85\% nos
mistos;
a disponibilidade de ambulância para a mulher variou de 50\% nos privados do
Norte/Nordeste a 100\% em várias regiões e tipos de financiamento, sendo 97\%
nos
públicos, 88\% nos mistos e 61\% nos privados; e ambulância para o recém-nascido
variou de zero, nos hospitais privados do Centro-oeste, a 100\% nos hospitais
públicos do Centro-oeste, sendo 67\% nos hospitais públicos, 51\% nos mistos e
17\% nos
privados.

Tabela 3Frequência relativa dos estabelecimentos de saúde por tipo de
financiamento e regiões geográficas, segundo disponibilidade de
equipamentos de emergência, banco de sangue, laboratório de análises
clínicas e ambulâncias. Brasil, 2010 *.
\begin{table}
\begin{xtabular}{ l | l | l | l | l l | l | l | l | l | l | l | l | l | l | l |
l }
\hline
Equipamento de
emergência & Público & Misto & Privado & Brasil\\ \hline
N/NE (\%) & S/SE (\%) & CO (\%) & N/NE (\%) & S/SE (\%) & CO (\%) & N/NE (\%) &
S/SE (\%) & CO (\%) & Público (\%) & Misto (\%) & Privado (\%)\\ \hline
Materna
&
&
&
&
&
&
&
&
&
&
&
&
\\ \hline

Respirador/Ventilador mecânico
& 62,9
& 91,4
& 74,4
& 69,0
& 95,0
& 97,1
& 97,2
& 100,0
& 100,0
& 74,4
& 90,0
& 98,8
\\ \hline

Laringoscópico e tubo orotraqueal
& 85,6
& 100,0
& 100,0
& 89,6
& 100,0
& 100,0
& 100,0
& 100,0
& 100,0
& 92,1
& 98,0
& 100,0
\\ \hline

Ressuscitador manual
& 94,2
& 100,0
& 100,0
& 98,4
& 100,0
& 100,0
& 100,0
& 100,0
& 100,0
& 96,8
& 99,7
& 100,0
\\ \hline

Todos os equipamentos
& 56,3
& 91,4
& 74,4
& 64,3
& 95,0
& 97,1
& 97,2
& 100,0
& 100,0
& 70,6
& 89,1
& 98,8
\\ \hline

Recém-nascido
&
&
&
&
&
&
&
&
&
&
&
&
\\ \hline

Laringoscópico e tubo endotraqueal neonatal
& 83,8
& 100,0
& 100,0
& 85,7
& 100,0
& 100,0
& 100,0
& 95,9
& 100,0
& 91,1
& 97,2
& 98,0
\\ \hline

Sondas traqueais neonatais sem válvula
& 73,2
& 99,5
& 100,0
& 86,5
& 97,2
& 100,0
& 100,0
& 95,9
& 100,0
& 85,3
& 95,2
& 98,0
\\ \hline

Dispositivo para aspirar mecônio e aspirador com
manômetro e oxigênio)
& 55,0
& 94,1
& 100,0
& 86,5
& 90,6
& 73,5
& 100,0
& 76,2
& 95,7
& 73,2
& 88,9
& 88,2
\\ \hline

Material para ventilação ressuscitador
manual)
& 88,5
& 100,0
& 100,0
& 92,1
& 100,0
& 100,0
& 100,0
& 95,9
& 100,0
& 93,8
& 98,4
& 98,0
\\ \hline

Todos os equipamentos
& 44,8
& 93,5
& 100,0
& 64,3
& 87,7
& 73,5
& 100,0
& 75,6
& 95,7
& 67,7
& 82,1
& 87,8
\\ \hline

Banco de sangue ou Unidade transfusional
& 62,2
& 77,0
& 74,4
& 47,6
& 83,6
& 58,8
& 56,5
& 74,6
& 75,0
& 68,8
& 75,2
& 66,9
\\ \hline

Laboratório de análises clínicas
& 91,7
& 89,8
& 100,0
& 69,6
& 87,6
& 97,1
& 79,8
& 91,1
& 95,7
& 91,9
& 84,6
& 86,6
\\ \hline

Ambulância para mulher
& 95,3
& 100,0
& 100,0
& 77,0
& 90,9
& 88,2
& 49,5
& 62,6
& 100,0
& 97,4
& 88,1
& 60,6
\\ \hline

Ambulância para recém-nascido
& 64,7
& 63,1
& 100,0
& 60,3
& 46,4
& 87,9
& 28,7
& 8,9
& 0,0
& 67,1
& 51,3
& 16,5
\\ \hline

\end{xtabular}
\end{table}

CO: Centro-oeste; N/NE: Norte/Nordeste; S/SE: Sul/Sudeste. * Valores
ponderados segundo plano amostral.

CO: Centro-oeste; N/NE: Norte/Nordeste; S/SE: Sul/Sudeste. * Valores
ponderados segundo plano amostral.

Já no que se refere a medicamentos essenciais, como mostra a Tabela 4, a
situação reverte sendo as proporções de
disponibilidade no setor privado menores, com exceção de surfactante e
coagulantes/hemostáticos para a mulher. Todavia, quando verifica-se a
disponibilidade de todos os medicamentos listados como essenciais, constata-se
que
há uma inversão, havendo um gradiente privado (71\%), misto (59\%) e público
(43\%).
Novamente é no Norte/Nordeste que são encontradas as maiores carências, havendo
completude dessa medicação em apenas 37\% dos públicos e 35\% dos mistos.

Tabela 4Frequência relativa dos estabelecimentos de saúde por tipo de
financiamento e regiões geográficas, segundo disponibilidade de
medicamentos. Brasil, 2010 *.
\begin{table}
\begin{xtabular}{ l | l | l | l l | l | l | l | l | l | l | l | l | l | l | l }
\hline
Público & Misto & Privado & Brasil\\ \hline
N/NE (\%) & S/SE (\%) & CO (\%) & N/NE (\%) & S/SE (\%) & CO (\%) & N/NE (\%) &
S/SE (\%) & CO (\%) & Público (\%) & Misto (\%) & Privado\\ \hline
Medicamento
&
&
&
&
&
&
&
&
&
&
&
&
\\ \hline

Anti-Hipertensivos
& 100,0
& 99,5
& 100,0
& 100,0
& 100,0
& 100,0
& 92,7
& 89,4
& 100,0
& 99,8
& 100,0
& 92,1
\\ \hline

Ansiolíticos/hipnóticos
& 97,1
& 94,7
& 88,4
& 92,9
& 95,8
& 100,0
& 92,7
& 87,7
& 100,0
& 95,7
& 95,5
& 90,9
\\ \hline

Corticosteróides
& 97,1
& 100,0
& 100,0
& 93,6
& 100,0
& 100,0
& 97,2
& 89,4
& 100,0
& 98,4
& 98,8
& 94,1
\\ \hline

Ocitócitos
& 100,0
& 100,0
& 100,0
& 100,0
& 100,0
& 100,0
& 90,7
& 89,4
& 100,0
& 100,0
& 100,0
& 90,9
\\ \hline

Inibidores de contrabilidade uterina
& 100,0
& 97,3
& 100,0
& 98,4
& 97,7
& 100,0
& 97,2
& 89,4
& 100,0
& 99,0
& 98,0
& 94,1
\\ \hline

Sulfato de magnésio
& 100,0
& 98,4
& 95,3
& 100,0
& 97,7
& 100,0
& 97,2
& 89,4
& 100,0
& 99,0
& 98,3
& 94,1
\\ \hline

Surfactante
& 58,6
& 88,2
& 83,3
& 39,2
& 73,7
& 64,7
& 97,2
& 87,0
& 87,0
& 71,6
& 66,6
& 91,3
\\ \hline

Coagulantes/Hemostáticos para a mulher
& 87,5
& 70,4
& 76,2
& 92,9
& 89,2
& 100,0
& 97,2
& 85,2
& 95,7
& 80,3
& 90,5
& 91,3
\\ \hline

Coagulantes/Hemostáticos para o
recém-nascidos
& 98,6
& 100,0
& 100,0
& 100,0
& 100,0
& 100,0
& 97,2
& 89,4
& 100,0
& 99,2
& 100,0
& 94,1
\\ \hline

Colírio profilático
& 87,8
& 81,7
& 88,4
& 96,0
& 93,3
& 100,0
& 97,2
& 84,6
& 100,0
& 85,8
& 94,2
& 91,7
\\ \hline

Imunoglobulina anti-D
& 74,8
& 95,7
& 88,4
& 96,0
& 93,3
& 100,0
& 97,2
& 84,6
& 100,0
& 83,6
& 93,6
& 79,5
\\ \hline

Todos os medicamentos
& 37,3
& 48,1
& 53,5
& 34,9
& 64,2
& 64,7
& 66,1
& 71,8
& 83,3
& 42,6
& 58,6
& 70,6
\\ \hline

\end{xtabular}
\end{table}

CO: Centro-oeste; N/NE: Norte/Nordeste; S/SE: Sul/Sudeste. * Valores
ponderados segundo plano amostral.

CO: Centro-oeste; N/NE: Norte/Nordeste; S/SE: Sul/Sudeste. * Valores
ponderados segundo plano amostral.

A Tabela 5 evidencia que hospitais de maior
complexidade, aqui considerados como os que dispunham de seis ou mais leitos de
\textsc{uti}
neonatal e \textsc{uti} para adultos, são cerca de 30\% dos públicos e mistos e 59\% dos
privados. Em geral estão localizados nas capitais, principalmente os públicos
(64\%).
Encontra-se proporção mais alta de hospitais de maior complexidade no tipo de
financiamento misto (80\% para Norte/Nordeste e 64\% no Sul/Sudeste) e no
privado (68\%
no Sul/Sudeste e 57\% no Centro-oeste). Hospitais de maior complexidade tendem a
ter
volume médio de partos e, nos de menor complexidade, predomina o volume baixo.
Com
maior fre- quência, estabelecimentos de maior complexidade realizam atividades
de
ensino, são referência para alto risco e dispõem de coordenações médicas e de
enfermagem com especialização. Também são nesses hospitais que são encontrados
com
maior frequência os equipamentos de emergência materna e neonatal e medicamentos
essenciais. Com exceção dos estabelecimentos privados, também são mais
frequentes
nos hospitais de maior complexidade banco de sangue ou unidade transfusional,
laboratório de análises clínicas e ambulâncias para a mulher e para o
recém-nascido.

Tabela 5Distribuição dos estabelecimentos de saúde por tipo de financiamento
e nível de complexidade por localização geográfica e aspectos da
estrutura. Brasil, 2010*.
\begin{table}
\begin{xtabular}{ l l | l | l l | l | l | l | l | l }
\hline
\textsc{uti} neonatal com seis ou mais leitos e \textsc{uti} de
adulto\\ \hline
Público & Misto & Privado\\ \hline
Não (\%) & Sim (\%) & Não (\%) & Sim (\%) & Não (\%) & Sim (\%)\\ \hline
Localização na capital
& 13,6
& 63,6
& 9,9
& 20,3
& 25,0
& 30,7
\\ \hline

Volume de partos
&
&
&
&
&
&
\\ \hline

Alto
& 14,7
& 39,6
& 10,2
& 22,1
& 4,8
& 12,0
\\ \hline

Médio
& 50,5
& 54,6
& 41,3
& 69,7
& 23,8
& 66,0
\\ \hline

Baixo
& 34,8
& 5,8
& 48,5
& 8,2
& 71,4
& 22,0
\\ \hline

Total
& 100,0
& 100,0
& 100,0
& 100,0
& 100,0
& 100,0
\\ \hline

Realiza atividade de ensino
& 70,3
& 91,6
& 67,4
& 88,0
& 2,9
& 17,3
\\ \hline

Referência para parto de alto risco
& 12,7
& 84,4
& 5,8
& 65,7
& 0,0
& 0,0
\\ \hline

Coordenação médica e de enfermagem com
especialização
& 5,4
& 45,5
& 5,1
& 38,5
& 4,8
& 12,0
\\ \hline

Emergência materna
&
&
&
&
&
&
\\ \hline

Todos os equipamentos
& 53,3
& 100,0
& 76,1
& 94,2
& 97,1
& 100,0
\\ \hline

Emergência recém-nascido
&
&
&
&
&
&
\\ \hline

Todos os equipamentos
& 61,8
& 90,9
& 87,3
& 92,3
& 87,5
& 88,7
\\ \hline

Disponibilidade
&
&
&
&
&
&
\\ \hline

Banco de sangue ou unidade transfusional
& 57,2
& 94,8
& 65,8
& 94,7
& 71,4
& 63,3
\\ \hline

Laboratório de análises clínicas
& 89,0
& 98,7
& 83,6
& 86,5
& 73,1
& 96,0
\\ \hline

Ambulância para mulher
& 96,3
& 100,0
& 87,8
& 88,9
& 61,9
& 59,3
\\ \hline

Ambulância para o recém-nascido
& 70,8
& 58,4
& 48,8
& 56,9
& 34,3
& 4,0
\\ \hline

Medicamentos
&
&
&
&
&
&
\\ \hline

Todos os medicamentos
& 32,6
& 66,0
& 48,1
& 80,3
& 46,2
& 88,0
\\ \hline

\end{xtabular}
\end{table}

* Valores ponderados segundo plano amostral.

* Valores ponderados segundo plano amostral.

\section{Discussão}

Ao traçar um panorama de alguns aspectos da estrutura das maternidades do
Brasil,
este estudo busca conhecer potencialidades e deficiências do sistema de saúde na
assistência ao parto e nascimento. No Brasil, nas últimas décadas, esse tema vem
recebendo maior atenção de pesquisadores dada a persistência de níveis
inaceitáveis
dos indicadores maternos e perinatais do país, concomitantes às crescentes
coberturas de acesso no atendimento hospitalar ao parto\textsuperscript{[}\textsuperscript{4}\textsuperscript{]}\textsuperscript{,}\textsuperscript{[}\textsuperscript{10}\textsuperscript{]}\textsuperscript{,}\textsuperscript{[}\textsuperscript{22}\textsuperscript{]}\textsuperscript{,}\textsuperscript{[}\textsuperscript{23}\textsuperscript{]}\textsuperscript{,}\textsuperscript{[}\textsuperscript{24}\textsuperscript{]}\textsuperscript{,}\textsuperscript{[}\textsuperscript{25}\textsuperscript{]}.

Embora neste artigo não tenha sido considerada a qualidade do processo envolvido
na
assistência ao parto e nascimento das maternidades selecionadas, as evidências
da
associação entre oferta de profissionais e ambientes adequados no cuidado seguro
à
mulher e ao recém-nascido e a ocorrência de resultados favoráveis reafirmam a
importância da avaliação de recursos de estrutura, mesmo de forma isolada\textsuperscript{[}\textsuperscript{12}\textsuperscript{]}\textsuperscript{,}\textsuperscript{[}\textsuperscript{26}\textsuperscript{]}.

O delineamento amostral do estudo permitiu investigar mais profundamente as
variações
da estrutura dos estabelecimentos segundo o tipo de financiamento, e no interior
de
cada um destes grupos, segundo sua localização geográfica.

Este estudo evidenciou que a maior rede de estabelecimentos de assistência ao
parto e
nascimento é conveniada ao \textsc{sus}. Situação semelhante foi encontrada em pesquisa
realizada no Rio de Janeiro\textsuperscript{[}\textsuperscript{3}\textsuperscript{]}\textsuperscript{,}\textsuperscript{[}\textsuperscript{7}\textsuperscript{]}, na Região Metropolitana de São Paulo\textsuperscript{[}\textsuperscript{22}\textsuperscript{]}
e em Santa Catarina\textsuperscript{[}\textsuperscript{27}\textsuperscript{]}.

Embora a proporção de atendimentos aos usuários do \textsc{sus} e à clientela de planos
privados de saúde ou de pacientes particulares não tenha sido considerada nos
estabelecimentos de tipo misto, os resultados confirmam que a maior rede do \textsc{sus},
especialmente dos estabelecimentos públicos, localizada no Norte/Nordeste pode
ser
atribuída ao baixo contingente da população coberta por planos privados de saúde
residente nestas áreas. Da mesma forma, a concentração da clientela coberta por
planos privados de saúde ou particulares no Sul/Sudeste pode indicar padrão de
convênios diferenciados entre as unidades mistas e privadas, assim como
expressar a
organização da oferta por contar, em algumas localidades, com menos
estabelecimentos
públicos, a necessidade de contratar serviços privados, ou ainda a necessidade
dos
estabelecimentos privados de complementar a sua receita com convênios com o \textsc{sus}.

A maior disponibilidade de estabelecimentos de saúde conveniados ao \textsc{sus} fora das
capitais dos estados era esperada, dada a dispersão da população que vive em um
grande número de municípios, sobretudo no Norte/Nordeste. O padrão diferenciado
da
Região Centro-oeste, com excessiva concentração de maternidades nas capitais, é
preocupante. Diferentemente das outras regiões, no Sul/Sudeste a quase
totalidade
dos hospitais mistos estava localizada fora das capitais, sugerindo que em
cidades
menores a disponibilidade deve ser diversificada às duas clientelas para não
multiplicar serviços, o que seria pouco custo/efetivo; já os hospitais públicos
estavam concentrados na capital, com distribuição semelhante para o setor
privado.
Os percentuais de hospitais privados localizados fora das capitais apresentaram
pouca variação entre as regiões, o que é indicativo de uma rede organizada com
uma
lógica própria.

Ao analisar os hospitais de acordo com a sua complexidade, ou seja, os com \textsc{uti}
neonatal com seis ou mais leitos e \textsc{uti} de adultos, há outra evidência da
diferença
de organização dos três tipos de financiamento. A rede privada é mais bem
equipada,
e não há diferença de distribuição dos hospitais classificados segundo grau de
complexidade entre capital, fora das capitais e regiões estudadas. Sendo de
referência, em geral, a concentração dos hospitais públicos de maior
complexidade na
capital, com baixa participação nas regiões, sobretudo no Norte/Nordeste, aponta
possíveis carências para a população que tem o acesso exclusivamente aos
estabelecimentos de saúde do \textsc{sus}, que podem estar ou não sendo cobertas pelos
hospitais mistos, em que os de maior complexidade estão concentrados fora das
capitais e com representação relevante no Norte/Nordeste do país.

Apesar dos limites inerentes ao estudo, sobretudo aqueles relacionados à
ausência de
dados detalhados do número de leitos disponíveis para internação, o tamanho,
perfil
demográfico, social e necessidade de saúde da população em idade fértil e dos
recém-nascidos\textsuperscript{[}\textsuperscript{10}\textsuperscript{]}, os resultados
aqui apresentados sublinham desigualdade geográfica na oferta de serviços
hospitalares do \textsc{sus}, sendo ainda mais acentuada entre os hospitais de maior
complexidade, e indicam vazios assistenciais que impõem deslocamento geográfico
para
a internação para o parto, em um contexto de baixo suporte à assistência à
gestante;
este fato pode aumentar a probabilidade de morte infantil, como foi mostrado por
Almeida \& Szwarcwald\textsuperscript{[}\textsuperscript{28}\textsuperscript{]}, além
de apontar que a regionalização da assistência hospitalar ainda é um desafio.

Os indicadores indiretos do grau de complexidade dos estabelecimentos da
amostra,
empregados neste estudo, foram o volume de procedimentos realizados, a
existência de
\textsc{uti} neonatal com pelo menos seis leitos e/ou \textsc{uti} de adulto, realização de
atividade
de ensino, coordenação dos serviços de obstetrícia e neonatologia, e
especificamente
para os públicos e mistos, ser referência para partos de alto risco.

Em relação a essas características, os resultados reforçam que a rede hospitalar
é
heterogênea. Os hospitais públicos e mistos apresentaram uma maior oferta de
estabelecimentos com médio e alto volumes de parto no ano de 2007 onde se
concentram
os hospitais de maior complexidade, o que está de acordo com a tendência de que
com
maior volume de partos se justificam gastos com a manutenção de equipamentos e
de
recursos humanos habilitados no uso de alta tecnologia médica para atender
situações
de emergência\textsuperscript{[}\textsuperscript{23}\textsuperscript{]}\textsuperscript{,}\textsuperscript{[}\textsuperscript{29}\textsuperscript{]}. No entanto, também há uma grande
quantidade de estabelecimentos públicos e mistos que realizaram mais de mil
partos
em 2007 que não dispunham de \textsc{uti}. Em contraste, na rede privada, embora sejam
mais
frequentes os hospitais com baixo volume de partos, os estabelecimentos com \textsc{uti}
são
bem mais frequentes – o que pode ser indicativo da necessidade de cuidados
intensivos aos recém-nascidos, associados com as altas taxas de cesariana neste
setor ou às exigências da clientela.

Muitos hospitais públicos e mistos realizam atividades de ensino, o que pode ser
indicativo de uma equipe profissional mais experiente e, portanto, com maior
possibilidade de impacto positivo na qualidade da assistência. Com o pressuposto
de
que coordenadores médicos e enfermeiros especialistas em serviços de obstetrícia
e
de neonatologia podem apresentar maior competência clínica para a tomada de
decisão
quanto à realização de um procedimento adequa- do\textsuperscript{[}\textsuperscript{13}\textsuperscript{]}\textsuperscript{,}\textsuperscript{[}\textsuperscript{30}\textsuperscript{]}, o artigo se limitou a descrever a existência de
coordenador médico e de enfermagem e seu grau acadêmico. Ainda foi baixa a
presença
dos coordenadores médicos e de enfermagem nos serviços de obstetrícia e
neonatologia, mormente com especialização, mesmo nos hospitais de maior
complexidade. A situação mais dramática foi observada nas maternidades públicas
localizadas no Norte/Nordeste. Para as outras regiões, os coordenadores eram
quase
duas vezes mais frequentes nos estabelecimentos públicos e mistos em comparação
aos
da rede privada.

Outro aspecto avaliado cujo mecanismo permite ampliar o acesso aqueles que mais
necessitam de cuidados foi a regulação da internação hospitalar para o parto
pelo
\textsc{sus}, particularmente às gestantes e recém-nascidos de alto risco.

Entre as maternidades públicas e mistas o serviço de referência formal para o
parto
via central de regulação foi encontrado prioritariamente nas de maior
complexidade.
Mas foi surpreendente um percentual importante dessas maternidades que não
informaram ser referência para outras, expressando uma falta de organização da
rede
de atenção às gestantes e recém-nascidos de alto risco. Outro ponto a destacar é
a
existência de maternidades de baixa complexidade que se identificaram como de
referência para o parto de alto risco. Desse total, 33\% delas estavam
localizadas no
interior da região Nordeste.

O panorama indica grandes inadequações da estrutura hospitalar que podem
interferir
na qualidade do processo de assistência ao parto e ao nascimento, com potencial
para
aumentar a ocorrência de desfechos desfavoráveis para as mulheres e os
recém-nascidos\textsuperscript{[}\textsuperscript{12}\textsuperscript{]}.

O estudo apontou que o conjunto de equipamentos mínimos necessários para
atendimentos
de emergência para a mulher foi declarado como disponíveis em todos os hospitais
da
rede privada, e nos estabelecimentos públicos e mistos esta situação se
restringiu
aos de maior complexidade. Já em relação aos equipamentos de emergência para o
recém-nascido, parcela mais significativa de hospitais não apresentou o conjunto
completo dos equipamentos relacionados. A conjuntura é preocupante, sobretudo
nos
públicos e mistos de menor complexidade localizados no Norte/Nordeste, fato que
pode
refletir ainda nos níveis de mortalidade neonatal.

Em um contexto em que a hemorragia é uma das principais causas de morte materna
no
Brasil, é preocupante observar que 40\% das maternidades de maior complexidade
do
setor privado não possuem banco de sangue ou unidades transfusionais,
principalmente
levando em conta suas altas taxas de intervenções cirúrgicas. A
indisponibilidade
desse insumo no hospital retarda o atendimento desses casos\textsuperscript{[}\textsuperscript{13}\textsuperscript{]}.

Ainda que a disponibilidade de ambulância nas maternidades estudadas seja um
pré-requisito para garantir o acesso oportuno às internações para o parto no
nível
adequado de assistência, a situação encontrada foi crítica principalmente no
setor
privado, sendo pior ainda para transferência de recém-nascidos em maternidades
de
menor complexidade, fato que pode contribuir para a ocorrência de mortes
evitáveis
no período neonatal, pois o motivo mais comum para a transferência do
recém-nascidos
é a necessidade de assistência de neonatal intensiva\textsuperscript{[}\textsuperscript{4}\textsuperscript{]}\textsuperscript{,}\textsuperscript{[}\textsuperscript{13}\textsuperscript{]}.

No momento da entrevista, um percentual importante de maternidades declarou não
ter
disponível um ou mais dos medicamentos estudados. Entre eles destacaram aqueles
para
induzir maturação do pulmão do recém-nascido, para estancar hemorragia, ou para
prevenir isoimunização pelo fator Rh negativo e, para prevenir a oftalmia
gonocócica. O quadro é preocupante pois pode ter implicação direta na ocorrência
de
complicações como síndrome da angústia respiratória neonatal\textsuperscript{[}\textsuperscript{31}\textsuperscript{]}, óbito materno e infantil, síndrome de Sheehan\textsuperscript{[}\textsuperscript{32}\textsuperscript{]}, aborto e outras.

Há uma grande proporção de estabelecimentos muito mal equipados e sem
profissionais
especializados, e os resultados apontam ainda que a disponibilidade de hospitais
de
maior complexidade é mais iníqua em comparação com os de menor complexidade. De
todas as regiões, o Norte/Nordeste e em seguida a Centro-oeste apresentaram as
maiores lacunas e problemas, sobretudo entre os estabelecimentos públicos e
mistos.
Já no Sul/Sudeste, esses grupos de hospitais estavam mais bem estruturados,
atingindo proporções em vários dos critérios estudados próximas ou mesmo
superiores
ao patamar da rede privada. Os resultados são indicativos de que uma parcela
importante de mãe e recém- nascidos foram expostos a riscos desnecessários e
evitáveis.

Se por um lado há incertezas quanto à confiabilidade dos dados de estrutura
fornecidos pelo gestor das maternidades estudadas, visto que não houve a
verificação
direta dos itens presentes no instrumento de coleta de dados por parte dos
supervisores de campo, por outro lado esta escolha garantiu a participação de
todos
os hospitais selecionados da amostra e o baixo porcentual de não resposta.
Também é
importante destacar que a disponibilidade de equipamentos e insumos não garante
que
as necessidades de saúde das mulheres que buscaram assistência nos
estabelecimentos
estudados foram atendidas.

Mesmo considerando as limitações deste trabalho, os resultados trazem elementos
para
o debate sobre a qualidade dos serviços hospitalares. Apontam para a necessidade
de
dar continuidade à avaliação da estrutura dos hospitais e de desenvolver estudos
analíticos, a fim de explorar a questão da variação do desempenho hospitalar, o
que
exigirá informações mais detalhadas sobre outros aspectos da estrutura dos
hospitais, do perfil socioeconômico e da gravidade da clientela, bem como do
processo de cuidado durante a assistência ao parto e nascimento, provenientes da
aplicação de questionários junto à puérpera e de resgate de dados dos
prontuários da
pesquisa \textit{Nascer no Brasil}.

Finalmente, sugere-se que futuros trabalhos adotem como unidade de análise para
o
estudo da estrutura as redes de atenção perinatal regionalizadas, uma vez que a
questão da complexidade, da regulação, da disponibilidade de bancos de sangue e
serviços transfuncionais e outras deverão estar dimensionadas para as
necessidades
de saúde regionais e poderão contribuir com propostas de melhoria da qualidade,
assim como indicar direções para a organização de redes regionais de atenção à
saúde\textsuperscript{[}\textsuperscript{14}\textsuperscript{]}, na perspectiva de
subsidiar a organização e o funcionamento do \textsc{sus}.

Agradecimentos
Aos coordenadores regionais e estaduais, supervisores, entrevistadores e equipe
técnica do trabalho, e às mães participantes que tornaram este estudo possível.
As
Maternidade Municipais Carmela Dutra e a de Xerém pela confiança em permitir que
suas instalações servissem para testagem do questionário sobre estrutura das
maternidades. As bolsistas do Programa Institucional de Bolsas de Iniciação
Científica (\textsc{pibic}) Stella Lenz e Katherine Knust pelo apoio na organização das
referências bibliográficas e elaboração das tabelas.

\section*{Referências}
\begin{itemize}

\item[1] Ministério da Saúde. Objetivos de desenvolvimento do milênio:
relatório nacional de acompanhamento. Brasília, DF, 2010a.
http://www.portal.saude.gov.br/portal/arquivos/pdf/relatorio\_{}na
cional\_{}acompanhamento\_{}220910.pdf (acessado em Set/2013).

\item[2] Victora CG, Aquino EM, do Carmo Leal M, Monteiro CA, Barros FC,
Szwarcwald CL. Maternal and child health in Brazil: progress and challenges.
Lancet 2011; 377:1863-76.

\item[3] Leal MC, Gama \textsc{sgn}, Campos MR, Cavalini LT, Garbayo LS, Brasil \textsc{clp},
et al. Fatores associados à morbi-mortalidade perinatal em uma amostra de
maternidades públicas e privadas do Município do Rio de Janeiro, 1999-2001. Cad
Saúde Pública 2004; 20 Suppl 1:S20-33.

\item[4] Schramm \textsc{jma}, Szwarcwald CL, Esteves \textsc{map}. Assistência obstétrica e
risco de internação na rede de hospitais do Estado do Rio de Janeiro. Rev Saúde
Pública 2002; 36:590-7.

\item[5] Korst LM, Gregory KD, Lu MC, Reyes C, Hobel CJ, Chavez GF. A
framework for the development of maternal quality of care indicators. Matern
Child Health J 2005; 9:317-41.

\item[6] Donabedian A. Basic approaches to assessment: structure, process
and outcome. In: In: Donabedian A, editor. Explorations in quality assessment
and monitoring. v. I. Ann Arbor: Health Adiministration Press; 1980. pp.
77-125.

\item[7] Hearld LR, Alexander JA, Fraser I, Jiang HJ. How do hospital
organizational structure and processes affect quality of care? A critical review
of research methods. Med Care Res Rev 2008; 65:259-99.

\item[8] Stilwell J, Szczepura A, Mugford M. Factors affecting the outcome
of maternity care. 1. Relationship between staffing and perinatal deaths at the
hospital of birth. J Epidemiol Community Health 1988; 42:
57-69.

\item[9] Magluta C, Noronha MF, Gomes \textsc{mam}, Aquino LA, Alves CA, Silva RS.
Estrutura de maternidades do Sistema Único de Saúde do Rio de Janeiro: desafio à
qualidade do cuidado à saúde. Rev Bras Saúde Matern Infant 2009;
9:319-29.

\item[10] Leal MC, Viacava F. Maternidades do Brasil. Radis 2002;
2:8-26.

\item[11] Conselho Regional de Medicina do Estado de São Paulo. Avaliação
dos serviços de assistência ao parto e ao neonato no Estado de São Paulo,
1997-1998. São Paulo: Conselho Regional de Medicina do Estado de São Paulo;
2000.

\item[12] Costa JO, Xavier CC, Proietti FA, Delgado MS. Avaliação dos
recursos hospitalares para assistência perinatal em Belo Horizonte, Minas
Gerais. Rev Saúde Pública 2004; 38:701-8.

\item[13] Rosa \textsc{mlg}, Hortale VA. Óbitos perinatais evitáveis e estrutura de
atendimento obstétrico na rede pública: estudo de caso de um município da região
metropolitana do Rio de Janeiro. Cad Saúde Pública 2000;
16:773-83.

\item[14] do Carmo Leal M, da Silva AA, Dias MA, da Gama SG, Rattner D,
Moreira ME, et al. Birth in Brazil: national survey into labour and birth.
Reprod Health 2012; 9:15.

\item[15] Agência Nacional de Vigilância Sanitária. Resolução \textsc{rdc}
n\textsuperscript{o}
36, de 3 de junho de 2008. Dispõe
sobre Regulamento Técnico para Funcionamento dos Serviços de Atenção Obstétrica
e Neonatal. Diário Oficial da União 2008; 4 jun.

\item[16] Agência Nacional de Vigilância Sanitária. Resolução \textsc{rdc}
n\textsuperscript{o}
50, de 21 de fevereiro de 2002.
Regulamento técnico para planejamento, programação, elaboração e avaliação de
projetos físicos de estabelecimentos assistenciais de saúde. Diário Oficial da
União 2002; 20 mar.

\item[17] Ministério da Saúde. Portaria GM/MS
n\textsuperscript{o}
1.091 de 25 de agosto de 1999. Cria a
Unidade de Cuidados Intermediários Neonatal, no âmbito do \textsc{sus}, para o
atendimento ao recém-nascido de médio risco. Diário Oficial da União 1999; 26
ago.

\item[18] Ministério da Saúde. Portaria GM/MS
n\textsuperscript{o}
3.432, de 12 de agosto de 1998.
Estabelece critérios de classificação para as Unidades de Tratamento Intensivo –
\textsc{uti}. Diário Oficial da União 1998; 13 ago.

\item[19] Ministério da Saúde. Portaria GM/MS
n\textsuperscript{o}
2.048, de 05 de novembro de 2002.
Regulamento técnico dos sistemas estaduais de urgência e emergência. Diário
Oficial da União 2002; 5 nov.

\item[20] Ministério da Saúde. Portaria GM/MS
n\textsuperscript{o}
1.071, de 4 de julho de 2005. Política
Nacional de Atenção ao Paciente Crítico. Diário Oficial da União 2005; 8
jul.

\item[21] Ministério da Saúde. Portaria GM/MS
n\textsuperscript{o}
2.418, de 6 de dezembro de 2005.
Garante às parturientes o direito à presença de acompanhante durante o trabalho
de parto, parto e pós-parto imediato, no âmbito do Sistema Único de Saúde – \textsc{sus}.
Diário Oficial da União 2005; 2 dez.

\item[22] Silva \textsc{zpd}, Almeida MF, Ortiz LP, Alencar GP, Alencar AA, Schoesp,
et al. Morte neonatal precoce segundo complexidade hospitalar e rede \textsc{sus} e
não-\textsc{sus} na Região Metropolitana de São Paulo, Brasil. Cad Saúde Pública 2010;
26:123-34.

\item[23] Novaes \textsc{mhd}. Mortalidade neonatal e avaliação da qualidade de
atenção ao parto e ao recém-nascido no Município de São Paulo. São Paulo, 1999
[Tese de Livre-Docência]. São Paulo: Faculdade de Medicina, Universidade de São
Paulo.

\item[24] Almeida \textsc{mfd}, Novaes \textsc{hmd}, Alencar GP, Rodrigues LC. Mortalidade
neonatal no Município de São Paulo: influência do peso ao nascer e de fatores
sócio-demográficos e assistenciais. Rev Bras Epidemiol 2002;
5:93-107.

\item[25] Barros \textsc{ajd}, Matijasevich A, Santos IS, Albernaz EP, Victora CG.
Neonatal mortality: description and effect of hospital of birth after risk
adjustment. Rev Saúde Pública 2008; 42:1-9.

\item[26] Machado JP, Martins \textsc{acm}, Martins MS. Avaliação da qualidade do
cuidado hospitalar no Brasil: uma revisão sistemática. Cad Saúde Pública 2013;
29:1063-82.

\item[27] Neumann NA, Tanaka OY, Victora CG, Cesar JA. Qualidade e equidade
de atenção ao pré-natal e ao parto em Criciúma, Santa Catarina, Sul do Brasil.
Rev Bras Epidemiol 2003; 6:307-18.

\item[28] Almeida WS, Szwarcwald CL. Mortalidade infantil e acesso
geográfico ao parto dos municípios brasileiros. Rev Saúde Pública 2012;
46:68-76.

\item[29] Mayfield JA, Rosenblatt RA, Baldwin LM, Chu J, Logerfo JP. The
relation of obstetrical volume and nursery level to perinatal mortality. Am J
Public Health 1990; 80:819-23.

\item[30] Rosa \textsc{mlg}, Hortale VA. Óbitos perinatais evitáveis e ambiente
externo ao sistema de assistência: estudo de caso em município da Região
Metropolitana do Rio de Janeiro. Cad Saúde Pública 2002;
18:623-31.

\item[31] Barría-Pailaquilén RM1, Mendoza-Maldonado Y, Urrutia-Toro Y,
Castro-Mora C, Santander-Manríquez G. Trends in infant mortality rate and
mortality for neonates born at less than 32 weeks and with very low birth
weight. Rev Latinoam Enferm 2011; 19:977-84.

\item[32] Ruano R, Yoshizaki CT, Martinelli S, Pereira PP. Intercorrências
clínico-cirúrgicas. In:Zugaib M, organizador. \textsc{zugaib} obstetrícia. Barueri:
Editora Manole; 2008. p. 851-82.

\end{itemize}

\end{document}
