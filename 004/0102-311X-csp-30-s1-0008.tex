% Generated by jats2tex@0.11.1.0
\documentclass{article}
\usepackage[T1]{fontenc}
\usepackage[utf8]{inputenc} %% *
\usepackage[portuges,spanish,english,german,italian,russian]{babel} %% *
\usepackage{amstext}
\usepackage{authblk}
\usepackage{unicode-math}
\usepackage{multirow}
\usepackage{graphicx}
\usepackage{etoolbox}
\usepackage{xtab}
\usepackage{enumerate}
\usepackage{hyperref}
\usepackage{penalidades}
\usepackage[footnotesize,bf,hang]{caption}
\usepackage[nodayofweek,level]{datetime}
\usepackage[top=0.85in,left=2.75in,footskip=0.75in]{geometry}
\newlength\savedwidth
\newcommand\thickcline[1]{\noalign{\global
\savedwidth
\arrayrulewidth
\global\arrayrulewidth 2pt}
\cline{#1}
\noalign{\vskip\arrayrulewidth}
\noalign{\global\arrayrulewidth\savedwidth}}
\newcommand\thickhline{\noalign{\global
\savedwidth\arrayrulewidth
\global\arrayrulewidth 2pt}
\hline
\noalign{\global\arrayrulewidth\savedwidth}}
\usepackage{lastpage,fancyhdr}
\usepackage{epstopdf}
\pagestyle{myheadings}
\pagestyle{fancy}
\fancyhf{}
\setlength{\headheight}{27.023pt}
\lhead{\includegraphics[width=10mm]{logo.png}}
\rhead{\ifdef{\journaltitle}{\journaltitle}{}
\ifdef{\volume}{vol.\,\volume}{}
\ifdef{\issue}{(\issue)}{}
\ifdef{\fpage}{\fpage--\lpage\,pp.}}
\rfoot{\thepage/\pageref{LastPage}}
\renewcommand{\footrule}{\hrule height 2pt \vspace{2mm}}
\fancyheadoffset[L]{2.25in}
\fancyfootoffset[L]{2.25in}
\lfoot{\sf \ifdef{\articledoi}{\articledoi}{}}
\setmainfont{Linux Libertine O}
\renewcommand*{\thefootnote}{\alph{footnote}}
\makeatletter
\newcommand{\fn}{\afterassignment\fn@aux\count0=}
\newcommand{\fn@aux}{\csname fn\the\count0\endcsname}
\makeatother

\newcommand{\journalid}{Cad Saude Publica}
\newcommand{\journaltitle}{Cadernos de Saúde Pública}
\newcommand{\abbrevjournaltitle}{Cad. Saúde
Pública}
\newcommand{\issnppub}{0102-311X}
\newcommand{\issnepub}{1678-4464}
\newcommand{\publishername}{Fundação Oswaldo Cruz}
\newcommand\articledoi{\textsc{doi} 10.1590/0102-311\textsc{xpe}01S114}
\def\subject{\textsc{perspectivas}}\newcommand{\subtitlestyle}[1]{-- \emph{#1}\medskip}
\newcommand{\transtitlestyle}[1]{\par\medskip\Large #1}
\newcommand{\transsubtitlestyle}[1]{-- \Large\emph{ #1}}

\newcommand{\titlegroup}{
\ifdef{\subtitle}{\subtitlestyle{\subtitle}}{}
\ifdef{\transtitle}{\transtitlestyle{\transtitle}}{}
\ifdef{\transsubtitle}{\transsubtitlestyle{\transsubtitle}}{}}

\title{Para reinventar o parto e o nascimento no Brasil: de
volta ao futuro\titlegroup{}}
\newcommand{\transtitle}{Para reinventar el parto y el nacimiento en Brasil:
regreso al
futuro}
\author[{1}]{Aquino, Estela M. L.}
\affil[1]{Universidade Federal da Bahia}
\def\authornotes{Correspondência E. M. L. Aquino. Musa – Programa Integrado em
Gênero e Saúde, Instituto de Saúde Coletiva, Universidade Federal da Bahia. Rua
Basílio da Gama
s/n\textsuperscript{o},2\textsuperscript{o}

andar, Salvador, BA 40110-040, Brasil. estela@ufba.br}
\date{ 08 2014}
\def\volume{30}
\def\issue{Suppl 1}
\def\fpage{S8}
\def\lpage{S10}
\def\permissions{This is an Open Access article distributed under the terms of
the
Creative Commons Attribution Non-Commercial License, which permits
unrestricted non-commercial use, distribution, and reproduction in any
medium, provided the original work is properly cited.}

\begin{document}
\selectlanguage{portuges}
\section*{Metadados não aplicados}
\begin{itemize}
\item[\textbf{língua do artigo}]{Português}
\ifdef{\journalid}{\item[\textbf{journalid}] \journalid}{}
\ifdef{\journaltitle}{\item[\textbf{journaltitle}] \journaltitle}{}

\ifdef{\journalsubtitle}{\item[\textbf{journalsubtitle}] \journaltitle}{}
\ifdef{\transjournaltitle}{\item[\textbf{journaltitle}] \journaltitle}{}
\ifdef{\transjournalsubtitle}{\item[\textbf{journalsubtitle}] \journaltitle}{}

\ifdef{\abbrevjournaltitle}{\item[\textbf{abbrevjournaltitle}]
\abbrevjournaltitle}{}
\ifdef{\issnppub}{\item[\textbf{issnppub}] \issnppub}{}
\ifdef{\issnepub}{\item[\textbf{issnepub}] \issnepub}{}
\ifdef{\publishername}{\item[\textbf{publishername}] \publishername}{}
\ifdef{\publisherid}{\item[\textbf{publisherid}] \publisherid}{}
\ifdef{\subject}{\item[\textbf{subject}] \subject}{}
\ifdef{\transtitle}{\item[\textbf{transtitle}] \transtitle}{}
\ifdef{\authornotes}{\item[\textbf{authornotes}] \authornotes}{}
\ifdef{\articleid}{\item[\textbf{articleid}] \articleid}{}
\ifdef{\articledoi}{\item[\textbf{articledoi}] \articledoi}{}
\ifdef{\volume}{\item[\textbf{volume}] \volume}{}
\ifdef{\issue}{\item[\textbf{issue}] \issue}{}
\ifdef{\fpage}{\item[\textbf{fpage}] \fpage}{}
\ifdef{\lpage}{\item[\textbf{lpage}] \lpage}{}
\ifdef{\permissions}{\item[\textbf{permissions}] \permissions}{}
\end{itemize}
\maketitle

As manifestações populares que tomaram as ruas do Brasil em junho de 2013
trouxeram a
saúde para o centro do debate público. Os protestos expuseram limites do Sistema
Único
de Saúde (\textsc{sus}), que apesar da grande ampliação da cobertura assistencial, não
consegue
assegurar serviços de qualidade e padece cronicamente de subfinanciamento e
problemas de
gestão. Ademais, o modelo de financiamento da saúde acentua desigualdades
sociais. A
maioria mais pobre enfrenta dificuldades para usufruir o direito universal aos
serviços
públicos de saúde, previsto na \textit{Constituição Federal}
de 1988.
Entretanto, 25\% dos brasileiros possuem seguros privados de saúde, mas são
beneficiados
com forte renúncia fiscal pelo Estado que, sustentado pelos tributos de todos,
financia
o acesso à assistência geralmente de melhor qualidade para um segmento
minoritário e
privilegiado\textsuperscript{[}\textsuperscript{1}\textsuperscript{]}. O \textsc{sus} sofre ainda
reflexos da (de)formação profissional, especialmente de médicos não preparados
para a
atenção básica, o que ficou exposto na reação corporativa à importação de
profissionais
estrangeiros para áreas descobertas no programa Mais Médicos.

Esses problemas assumem configurações específicas no modelo tecnocrático de
assistência
ao parto, caracterizado pela primazia da tecnologia sobre as relações humanas e
suposta
neutralidade de valores\textsuperscript{[}\textsuperscript{2}\textsuperscript{]}. Nele, subjaz
a ideia de passividade das mulheres, imobilizadas durante o parto, enquanto
sofrem
intervenções por profissionais desconhecidos para abreviar o tempo até o
nascimento\textsuperscript{[}\textsuperscript{3}\textsuperscript{]}. O uso sem controle de
procedimentos desnecessários e danosos é maximizado pela lógica mercantil e pela
(de)formação médica, e assume expressão mais visível na crescente epidemia de
cesáreas\textsuperscript{[}\textsuperscript{4}\textsuperscript{]}.

Os resultados da pesquisa \textit{Nascer no Brasil}
confirmam em âmbito nacional
o panorama descrito em estudos locais, e denunciado pelos movimentos de mulheres
e de
humanização do parto\textsuperscript{[}\textsuperscript{2}\textsuperscript{]}. Constituem
evidências contundentes das desigualdades socioeconômicas, raciais e regionais
na
atenção ao parto\textsuperscript{[}\textsuperscript{5}\textsuperscript{]}\textsuperscript{,}\textsuperscript{[}\textsuperscript{6}\textsuperscript{]}\textsuperscript{,}\textsuperscript{[}\textsuperscript{7}\textsuperscript{]}.

O modelo tecnocrático se manifesta distintamente no \textsc{sus} e na assistência
suplementar,
acentuando desigualdades na qualidade do parto hospitalar que atingiu cobertura
universal\textsuperscript{[}\textsuperscript{4}\textsuperscript{]}. Nos serviços públicos, é
frequente a desarticulação entre a atenção pré-natal e ao parto\textsuperscript{[}\textsuperscript{8}\textsuperscript{]}, a peregrinação em busca de internação\textsuperscript{[}\textsuperscript{8}\textsuperscript{]}, e o uso rotineiro de episiotomia e
ocitocina\textsuperscript{[}\textsuperscript{3}\textsuperscript{]}\textsuperscript{,}\textsuperscript{[}\textsuperscript{7}\textsuperscript{]}. Nos serviços privados, a cesariana
agendada previamente mesmo entre primíparas alcança a maioria dos partos\textsuperscript{[}\textsuperscript{4}\textsuperscript{]}\textsuperscript{,}\textsuperscript{[}\textsuperscript{9}\textsuperscript{]}\textsuperscript{,}\textsuperscript{[}\textsuperscript{10}\textsuperscript{]}. Em ambos os setores, não se garante o direito à
informação nem se respeita a autonomia das mulheres, fere-se a integridade
corporal e
nega-se o direito previsto em lei ao acompanhante\textsuperscript{[}\textsuperscript{6}\textsuperscript{]}, tornando o parto solitário, inseguro e doloroso\textsuperscript{[}\textsuperscript{11}\textsuperscript{]}.

As iniciativas para modificar tal quadro refletem lutas políticas e ideológicas
no campo
da saúde quanto às alternativas de modelo de atenção, desde a criação do
Programa de
Assistência Integral à Saúde da Mulher (\textsc{paism}), em 1983, resultante da
convergência de
propostas do movimento sanitário e do feminismo\textsuperscript{[}\textsuperscript{12}\textsuperscript{]}.

O feminismo visa superar a perspectiva materno-infantil e incorporar a noção de
mulher
como sujeito, ultrapassar sua especificidade reprodutiva, e assumir abordagem
ampliada
de saúde. A humanização do parto se situa no marco mais geral dos direitos
sexuais e
reprodutivos, os quais incluem a garantia à maternidade segura, à contracepção e
ao
aborto\textsuperscript{[}\textsuperscript{12}\textsuperscript{]}.

A saúde é central na agenda feminista, questionando-se a biomedicina, que
fornece as
bases para justificar relações hierárquicas de gênero. As lutas têm se
concretizado na
ocupação de instâncias de controle social e monitoramento de políticas públicas
e na
atuação militante em postos de gestão\textsuperscript{[}\textsuperscript{12}\textsuperscript{]}.

Todavia, registra-se o crescimento da ação organizada de forças conservadoras e
religiosas, no Parlamento e sobre o Governo, ameaçando a laicidade do Estado. No
Ministério da Saúde, a influência desses grupos tem resultado em recuo político
e
fortalecimento do chamado materno- infantilismo\textsuperscript{[}\textsuperscript{13}\textsuperscript{]}.

A adoção da estratégia de Rede Cegonha representa simbólica e materialmente o
encolhimento da agenda feminista e de construção do \textsc{sus}. Carrega no nome a
dessexualização da reprodução, conferindo ênfase ao concepto\textsuperscript{[}\textsuperscript{13}\textsuperscript{]}. Desvincula a atenção ao parto da Política Nacional
de Assistência Integral à Saúde das Mulheres (\textsc{pnaism}) e reforça o
materno-infantilismo
na definição de prioridades políticas. Obscurece o aborto inseguro em contexto
de
redução marcante da fecundidade.

A dimensão simbólica não é questão menor, se mantido o atual modelo de
financiamento
público. A redução da pobreza promove a inclusão de segmentos sociais, ansiosos
pelo
consumo de bens e serviços, a exemplo do parto cesáreo. As distorções podem se
acentuar
pela incorporação de “novas consumidoras”, que percebem o acesso à tecnologia
como sinal
de prestígio social e modernidade\textsuperscript{[}\textsuperscript{11}\textsuperscript{]}.

Estão em disputa diferentes projetos de sociedade, quanto à construção do \textsc{sus} e
à
equidade de gênero em saúde. Em tal cenário se inserem as oportunidades de
mudança da
atenção ao parto. As soluções não são apenas técnicas, mas essencialmente
políticas. Não
há como alcançar transformações sem retomar o projeto de \textsc{sus} democrático e os
pressupostos da Reforma Sanitária de universalidade, integralidade e equidade,
que a
\textit{Constituição}
incorporou. É imprescindível defender a laicidade do
Estado democrático e plural e ressaltar a intersecção entre gênero, classe
social,
raça/etnia e sexualidade na produção/reprodução de desigualdades sociais em
saúde. É
imperativo assegurar a humanização da atenção baseada em evidências científicas,
mas
também baseada em direitos das mulheres, redefinindo práticas e relações
interpessoais\textsuperscript{[}\textsuperscript{2}\textsuperscript{]}.

É inspirador revisitarmos os anos 1980 e sua radicalidade, repolitizando as
necessidades
de saúde e voltando a ampliar a agenda para a elaboração e implementação de
políticas
públicas voltadas às mulheres.

\section*{}
\begin{itemize}

\item[1] Paim J, Almeida-Filho N, Vieira-da-Silva L. Saúde coletiva:
futuros possíveis. In:Paim J, Almeida-Filho N, organizadores. Saúde coletiva:
teoria e prática. Rio de Janeiro: Medbook Editora Científica; 2013. p.
669-86.

\item[2] Diniz \textsc{csg}. Humanização da assistência ao parto no Brasil: os
muitos sentidos de um movimento. Ciênc Saúde Coletiva 2005;
10:627-37.

\item[3] Diniz SG, Chacham AS. “The cut above” and “the cut below”: the
abuse of caesareans and episiotomy in Sao Paulo, Brazil. Reprod Health Matters
2004; 12:100-10.

\item[4] Victora CG, Aquino EM, Leal MC, Monteiro CA, Barros FC, Szwarcwald
CL. Maternal and child health in Brazil: progress and challenges. Lancet 2011;
377:1863-76.

\item[5] d’Orsi E, Brüggemann OM, Diniz SG, Aguiar JM, Gusman CR, Torres
JA, et al. Desigualdades sociais e satisfação das mulheres com o atendimento ao
parto no Brasil: estudo nacional de base hospitalar. Cad Saúde Pública;
submetido.

\item[6] Diniz \textsc{csg}, d’Orsi E, Domingues \textsc{rmsm}, Torres JA, Schneck CA, Dias
\textsc{mab}, et al. Implementation of continuous support during labor and childbirth:
data from the Born in Brazil National Survey. Cad Saúde Pública;
submetido.

\item[7] Leal MC, Pereira \textsc{ape}, Domingues \textsc{rmsm}, Theme Filha MM, Dias \textsc{mab},
Nakamura-Pereira M, et al. Intervenções obstétricas durante o trabalho de parto
e parto em mulheres brasileiras de risco habitual. Cad Saúde Pública 2014; 30
Suppl:17-47.

\item[8] Viellas EF, Domingues \textsc{rmsm}, Dias \textsc{mab}, Gama \textsc{sgn}, Theme Filha MM,
Costa JV, et al. Assistência pré-natal no Brasil: Cad Saúde Pública 2014; 30
Suppl:85-100.

\item[9] Gama \textsc{sgn}, Viellas EF, Schilithz \textsc{aoc}, Theme Filha MM, Carvalho ML,
Gomes \textsc{kro}, et al. Fatores associados à cesariana entre primíparas adolescentes
no Brasil, 2011-2012. Cad Saúde Pública 2014; 30 Suppl:117-27.

\item[10] Domingues \textsc{rmsm}, Dias \textsc{mab}, Nakamura-Pereira M, Torres JA, d’Orsi E,
Pereira \textsc{ape}, et al. Processo de decisão pelo tipo de parto no Brasil: da
preferência inicial das mulheres à via de parto final. Cad Saúde Pública 2014;
30 Suppl: 101-16.

\item[11] Diniz SG. Gênero, saúde materna e o paradoxo perinatal. Rev Bras
Crescimento Desenvolv Hum 2009; 19:313-26.

\item[12] Costa AM. Participação social na conquista das políticas de saúde
para mulheres no Brasil. Ciênc Saúde Coletiva 2009; 14:1073-83.

\item[13] Diniz S. Materno-infantilism, feminism and maternal health policy
in Brazil. Reprod Health Matters 2012; 20:125-32.

\end{itemize}

\end{document}
