<!DOCTYPE article PUBLIC "-//NLM//DTD Journal Publishing DTD v3.0 20080202//EN" "journalpublishing3.dtd">
<article xmlns:xlink="http://www.w3.org/1999/xlink" xmlns:mml="http://www.w3.org/1998/Math/MathML"
  dtd-version="3.0" article-type="brief-report" xml:lang="pt">
  <front>
    <journal-meta>
      <journal-id journal-id-type="nlm-ta">Rev Saude Publica</journal-id>
      <journal-title-group>
        <journal-title>Revista de Saúde Pública</journal-title>
        <abbrev-journal-title abbrev-type="publisher">Rev. Saúde Pública</abbrev-journal-title>
      </journal-title-group>
      <issn pub-type="ppub">0034-8910</issn>
      <issn pub-type="epub">1518-8787</issn>
      <publisher>
        <publisher-name>Faculdade de Saúde Pública da Universidade de São Paulo</publisher-name>
      </publisher>
    </journal-meta>
    <article-meta>
      <article-id pub-id-type="publisher-id">S0034-8910.2013047004402</article-id>
      <article-id pub-id-type="doi">10.1590/S0034-8910.2013047004402</article-id>
      <article-categories>
        <subj-group subj-group-type="heading">
          <subject>Prática de Saúde Pública</subject>
        </subj-group>
      </article-categories>
      <title-group>
        <article-title xml:lang="pt">Necessidade de implantar programa nacional de segurança do
          paciente no Brasil</article-title>
        <trans-title-group xml:lang="es">
          <trans-title>Necesidad de implantar programa nacional de seguridad del paciente en
            Brasil</trans-title>
        </trans-title-group>
      </title-group>
      <contrib-group>
        <contrib contrib-type="author">
          <name>
            <surname>Capucho</surname>
            <given-names>Helaine Carneiro</given-names>
          </name>
          <xref ref-type="aff" rid="aff1">I</xref>
        </contrib>
        <contrib contrib-type="author">
          <name>
            <surname>Cassiani</surname>
            <given-names>Silvia Helena De Bortoli</given-names>
          </name>
          <xref ref-type="aff" rid="aff2">II</xref>
        </contrib>
      </contrib-group>
      <aff id="aff1">
        <label>I</label>
        <institution content-type="orgdiv2">Departamento de Gestão e Incorporação de Tecnologias em
          Saúde</institution>
        <institution content-type="orgdiv1">Secretaria de Ciência, Tecnologia e Insumos
          Estratégicos</institution>
        <institution content-type="orgname">Ministério da Saúde</institution>
        <addr-line>
          <named-content content-type="city">Brasília</named-content>
          <named-content content-type="state">DF</named-content>
        </addr-line>
        <country>Brasil</country> Departamento de Gestão e Incorporação de Tecnologias em Saúde.
        Secretaria de Ciência, Tecnologia e Insumos Estratégicos. Ministério da Saúde. Brasília,
        DF, Brasil</aff>
      <aff id="aff2">
        <label>II</label>
        <institution content-type="orgdiv2">Departamento de Enfermagem Geral e
          Especializada</institution>
        <institution content-type="orgdiv1">Escola de Enfermagem de Ribeirão Preto</institution>
        <institution content-type="orgname">Universidade de São Paulo</institution>
        <addr-line>
          <named-content content-type="city">Ribeirão Preto</named-content>
          <named-content content-type="state">SP</named-content>
        </addr-line>
        <country>Brasil</country> Departamento de Enfermagem Geral e Especializada. Escola de
        Enfermagem de Ribeirão Preto. Universidade de São Paulo. Ribeirão Preto, SP,
        Brasil</aff>
      <author-notes>
        <corresp>
          <bold>Correspondência | Correspondence</bold> : Helaine Carneiro Capucho. Esplanada dos
          Ministérios. Bloco G Edifício Sede 9º andar Sala 949. 70058-900 Brasília, DF, Brasil.
          E-mail: helaine.capucho@saude.gov.br</corresp>
        <fn fn-type="conflict">
          <p>Os autores declaram não haver conflito de interesses.</p>
        </fn>
      </author-notes>
      <pub-date pub-type="epub-ppub">
        <day>21</day>
        <month>08</month>
        <year>2013</year>
      </pub-date>
      <volume>47</volume>
      <issue>4</issue>
      <fpage>791</fpage>
      <lpage>798</lpage>
      <history>
        <date date-type="received">
          <day>4</day>
          <month>6</month>
          <year>2012</year>
        </date>
        <date date-type="accepted">
          <day>21</day>
          <month>4</month>
          <year>2013</year>
        </date>
      </history>
      <permissions>
        <license license-type="open-access"
          xlink:href="http://creativecommons.org/licenses/by-nc/3.0/">
          <license-p>This is an Open Access article distributed under the terms of the Creative
            Commons Attribution Non-Commercial License, which permits unrestricted non-commercial
            use, distribution, and reproduction in any medium, provided the original work is
            properly cited.</license-p>
        </license>
      </permissions>
      <abstract xml:lang="pt">
        <p>O objetivo do estudo foi suscitar reflexão acerca da necessidade de se criar um sistema
          nacional de notificações sobre incidentes como base para um programa de segurança do
          paciente. Incidentes em saúde acarretam danos aos pacientes e oneram o sistema de saúde.
          Embora tenha lançado recentemente um programa de avaliação da qualidade nas instituições
          de saúde, o Ministério da Saúde, Brasil, ainda não possui um programa que avalie
          sistematicamente os resultados negativos da assistência. Discute-se a necessidade de se
          implementar programa brasileiro de segurança do paciente, a fim de promover a cultura pela
          segurança do paciente e da qualidade em saúde no Sistema Único de Saúde.</p>
      </abstract>
      <trans-abstract xml:lang="es">
        <p>El objetivo del estudio fue suscitar reflexión sobre la necesidad de crear un sistema
          nacional de notificaciones sobre incidentes como base para un programa brasileño de
          seguridad del paciente. Incidentes en salud generaron daños a los pacientes y sobrecargan
          el sistema de salud. A pesar de que se haya lanzado recientemente un programa de
          evaluación de la calidad en las instituciones de salud, el Ministerio de Salud Brasileño
          no posee aún un programa que evalúe sistemáticamente los resultados negativos de la
          asistencia. Se discute la necesidad de implementar un programa brasileño de seguridad del
          paciente, con el fin de promover la cultura por la seguridad del paciente y la calidad de
          la salud en el Sistema Único de Salud.</p>
      </trans-abstract>
      <kwd-group xml:lang="pt">
        <kwd>Segurança do Paciente</kwd>
        <kwd>Avaliação de Programas e Projetos de Saúde</kwd>
        <kwd>Sistema Único de Saúde</kwd>
        <kwd>Garantia da Qualidade dos Cuidados de Saúde</kwd>
      </kwd-group>
      <kwd-group xml:lang="es">
        <kwd>Segurança do Paciente</kwd>
        <kwd>Avaliação de Programas e Projetos de Saúde</kwd>
        <kwd>Sistema Único de Saúde</kwd>
        <kwd>Garantia da Qualidade dos Cuidados de Saúde</kwd>
      </kwd-group>
      <kwd-group xml:lang="en">
        <kwd>Segurança do Paciente</kwd>
        <kwd>Avaliação de Programas e Projetos de Saúde</kwd>
        <kwd>Sistema Único de Saúde</kwd>
        <kwd>Garantia da Qualidade dos Cuidados de Saúde</kwd>
      </kwd-group>
      <counts>
        <fig-count count="2"/>
        <ref-count count="23"/>
        <page-count count="8"/>
      </counts>
    </article-meta>
  </front>
  <body>
    <sec sec-type="intro">
      <title>INTRODUÇÃO</title>
      <p>Desde há muito tempo os resultados da assistência são utilizados para avaliação da
        qualidade dos serviços de saúde. Os babilônicos pagavam pelos serviços médicos mediante os
        resultados obtidos e, na Idade Média, os médicos que obtivessem resultados negativos na
        prestação de assistência tinham parte de seus corpos mutilados. <xref ref-type="bibr"
          rid="B17">
          <sup>17</sup>
        </xref>
      </p>
      <p>Os resultados negativos em saúde são conhecidos principalmente como eventos adversos ou
        qualquer tipo de incidente com potencial para causar danos aos pacientes <xref
          ref-type="bibr" rid="B20">
          <sup>20</sup>
        </xref> e que pode fornecer importantes informações para a construção de um sistema de saúde
        mais seguro. <xref ref-type="bibr" rid="B14">
          <sup>14</sup>
        </xref> Os incidentes podem ser sem dano, com dano (evento adverso), ou <italic>near
          misses</italic> , também denominado de potencial evento adverso. <xref ref-type="bibr"
          rid="B4">
          <sup>4</sup>
        </xref>
        <sup>,</sup>
        <xref ref-type="bibr" rid="B23">
          <sup>23</sup>
        </xref>
      </p>
      <p>Resultados negativos em saúde foram relatados pelo <italic>Institute of Medicine</italic>
        (IOM) em 1999, <xref ref-type="bibr" rid="B11">
          <sup>11</sup>
        </xref> que estimou entre 44.000 a 98.000 mortes por ano nos Estados Unidos devido a erros
        na assistência ao paciente. Desde então, os resultados ou desfechos em saúde têm sido objeto
        de estudo, pois estão relacionados diretamente à qualidade e à segurança do paciente. A
        segurança do paciente é definida como o ato de evitar, prevenir ou melhorar os resultados
        adversos ou as lesões originadas no processo de atendimento médico-hospitalar. <xref
          ref-type="bibr" rid="B21">
          <sup>21</sup>
        </xref>
      </p>
      <p>Diante da mobilização mundial após a publicação desse impactante relatório, a Organização
        Mundial da Saúde (OMS) lançou a Aliança Mundial para a Segurança do Paciente em 2004. <xref
          ref-type="fn" rid="fn1">
          <sup>a</sup>
        </xref> Isso despertou os países membros para o compromisso de desenvolver políticas
        públicas e práticas voltadas para a segurança do paciente, incluindo o Brasil.</p>
      <p>Na Europa, estimou-se que 10,8% dos pacientes hospitalizados foram acometidos por eventos
        adversos, dos quais 46% poderiam ter sido prevenidos. <xref ref-type="bibr" rid="B22">
          <sup>22</sup>
        </xref> No Brasil, estudo conduzido em hospitais do Rio de Janeiro estimou incidência de
        7,6% desses eventos. <xref ref-type="bibr" rid="B13">
          <sup>13</sup>
        </xref>
      </p>
      <p>Apesar de o Ministério da Saúde e a Agência Nacional de Vigilância Sanitária (ANVISA)
        promoverem iniciativas da Aliança Mundial para a Segurança do Paciente da OMS, como a
        campanha para introdução do protocolo de cirurgia segura nos hospitais, a adesão por parte
        dos serviços é baixa, justamente por não terem uma cultura institucional voltada para a
        segurança do paciente. Isso se reflete na alta ocorrência de eventos adversos evitáveis em
        hospitais brasileiros, que corresponde a cerca de 67% de todos os eventos adversos. <xref
          ref-type="bibr" rid="B13">
          <sup>13</sup>
        </xref>
        <sup>,</sup>
        <xref ref-type="bibr" rid="B20">
          <sup>20</sup>
        </xref>
      </p>
      <p>Embora o sistema de saúde brasileiro tenha aspectos positivos como a cobertura universal de
        vacinação e o sistema nacional de transplantes, a alta frequência de eventos adversos
        relacionados a medicamentos e infecções hospitalares é motivo de preocupação. <xref
          ref-type="bibr" rid="B13">
          <sup>13</sup>
        </xref>
        <sup>,</sup>
        <xref ref-type="bibr" rid="B20">
          <sup>20</sup>
        </xref> Esses eventos são atribuídos à falta de políticas governamentais <xref ref-type="fn"
          rid="fn2">
          <sup>b</sup>
        </xref>
        <sup>,</sup>
        <xref ref-type="fn" rid="fn3">
          <sup>c</sup>
        </xref>
        <sup>,</sup>
        <xref ref-type="fn" rid="fn4">
          <sup>d</sup>
        </xref> que incentivem as instituições de saúde a participar de programas de qualidade e
        acreditação. <xref ref-type="bibr" rid="B15">
          <sup>15</sup>
        </xref>
        <sup>,</sup>
        <xref ref-type="bibr" rid="B16">
          <sup>16</sup>
        </xref> Atualmente, há hospitais brasileiros que ainda são prestadores de serviços que atuam
        sem avaliar seus processos de trabalho ou usar seus resultados para a melhoria contínua da
        qualidade. <xref ref-type="fn" rid="fn4">
          <sup>d</sup>
        </xref>
      </p>
      <p>Faz-se necessário, portanto, conhecer a realidade brasileira quanto à ocorrência de
        incidentes, o que pode ser obtido com o envolvimento das instituições de saúde para que
        monitorizem essa ocorrência e o tratamento das informações pertinentes, além de notificá-las
        aos órgãos governamentais. Entretanto, a simples existência de um fluxo de informações
        organizado não gera conhecimento por si só. Esse só se dará por meio da ação de atores
        interdisciplinares e que interajam entre si. <xref ref-type="bibr" rid="B14">
          <sup>14</sup>
        </xref>
      </p>
      <p>O objetivo deste estudo foi suscitar reflexão acerca da utilização de um sistema nacional
        de notificações sobre incidentes como base para um programa brasileiro de segurança do
        paciente.</p>
    </sec>
    <sec>
      <title>Qualidade no SUS e segurança do paciente</title>
      <p>Em 2011, o Ministério da Saúde lançou um Projeto de Formação e Melhoria da Qualidade de
        Rede de Atenção à Saúde, o QualiSUS Rede. <xref ref-type="fn" rid="fn5">
          <sup>e</sup>
        </xref> Apesar de ser um importante avanço para o desenvolvimento da qualidade do Sistema
        Único de Saúde (SUS), o projeto não contempla incentivo à adoção de um programa de
        acreditação hospitalar e também não contempla objetivo estratégico diretamente relacionado à
        segurança do paciente, item considerado essencial para a qualidade, segundo o IOM e a
        OMS.</p>
      <p>Outra iniciativa do Ministério da Saúde é a monitorização do Índice de Desempenho do SUS
        (IDSUS), que tem como objetivo aferir o desempenho do sistema de saúde quanto ao acesso –
        potencial ou obtido – e à efetividade da atenção básica, das atenções ambulatorial e
        hospitalar, e de urgências e emergência na esfera nacional. <xref ref-type="fn" rid="fn6">
          <sup>f</sup>
        </xref> Essa aferição é feita por meio de indicadores de qualidade.</p>
      <p>Dentre os indicadores estabelecidos no IDSUS, não há nenhum relacionado diretamente à
        segurança do paciente, como a taxa de incidentes ocorridos no atendimento de urgência e
        emergência. Por outro lado, no IDSUS os indicadores são abordados como a proporção de óbitos
        nas internações por infarto agudo do miocárdio, que calcula o risco de morrer por essa
        condição após a internação por tal causa, e estima, indiretamente, o atraso do atendimento
        pré-hospitalar e no diagnóstico. <xref ref-type="fn" rid="fn4">
          <sup>d</sup>
        </xref> Ainda que sejam poucos, pode-se considerar um avanço que esse tipo de indicador
        esteja sendo utilizado em um programa oficial do governo brasileiro.</p>
      <p>O projeto prevê repasse de verba diferenciado para aquelas regiões que atingirem níveis
        mais elevados de qualidade. Esse tipo de programa já é realizado com sucesso em outros
        países. Na Inglaterra e nos Estados Unidos, por exemplo, além de compartilhar os indicadores
        sobre segurança do paciente entre as instituições do país, com o objetivo de conhecer e
        estabelecer níveis de qualidade e segurança nas organizações hospitalares, aquelas que
        atingem níveis mais elevados são recompensadas com remuneração diferenciada. <xref
          ref-type="bibr" rid="B3">
          <sup>3</sup>
        </xref>
        <sup>,</sup>
        <xref ref-type="bibr" rid="B9">
          <sup>9</sup>
        </xref>
      </p>
      <p>Esse modelo de pagamento por qualidade é conhecido como <italic>pay for
          performance</italic> (P4P) <xref ref-type="bibr" rid="B7">
          <sup>7</sup>
        </xref> e é alternativo àquele amplamente utilizado no Brasil, a remuneração
          <italic>fee-for-service</italic> (pagamento por serviço executado), que estimula a
        sobreutilização de recursos, especialmente as tecnologias em saúde, e que não traz garantias
        de que o custo adicional e a facilidade de acesso resultem numa efetiva melhoria da
        qualidade do nível de saúde da população atendida. <xref ref-type="bibr" rid="B7">
          <sup>7</sup>
        </xref>
      </p>
      <p>O P4P está em desenvolvimento em muitos países, incluindo o Brasil. Na Inglaterra, país
        modelo para o uso dessa ferramenta, os pagamentos contribuem com 30% da renda de algumas
        clínicas. <xref ref-type="bibr" rid="B7">
          <sup>7</sup>
        </xref>
        <sup>,</sup>
        <xref ref-type="bibr" rid="B12">
          <sup>12</sup>
        </xref> O que se espera do P4P é que os próprios usuários passem a escolher o serviço pelo
        qual desejam ser atendidos, com base em relatórios públicos de indicadores de desempenho.
        Constance et al <xref ref-type="bibr" rid="B6">
          <sup>6</sup>
        </xref> têm mostrado que a publicação desses relatórios é um bom mecanismo para a melhoria
        da qualidade na saúde.</p>
      <p>O Brasil ainda tem como desafios a alta rotatividade de profissionais de saúde nos serviços
        públicos, além da limitação qualitativa dos recursos humanos, do uso indevido das
        tecnologias e da baixa continuidade da atenção prestada aos pacientes. <xref ref-type="bibr"
          rid="B1">
          <sup>1</sup>
        </xref>
        <sup>,</sup>
        <xref ref-type="bibr" rid="B20">
          <sup>20</sup>
        </xref> Ainda, pequeno número de hospitais brasileiros se dedica ao ensino e à pesquisa e
        não influencia a melhoria das práticas assistenciais em razão da desarticulação entre
        ensino, pesquisa e assistência, e da discreta utilização da saúde baseada em evidências na
        assistência ao paciente e das pesquisas sobre segurança do paciente que estão restritas a
        ilhas de excelência. <xref ref-type="bibr" rid="B1">
          <sup>1</sup>
        </xref>
        <sup>,</sup>
        <xref ref-type="bibr" rid="B16">
          <sup>16</sup>
        </xref>
        <sup>,</sup>
        <xref ref-type="bibr" rid="B20">
          <sup>20</sup>
        </xref>
      </p>
      <p>Diante do cenário exposto, no qual as políticas implementadas pelo Ministério da Saúde não
        têm sido suficientes para estimular o olhar crítico para a segurança do paciente, com
        estabelecimento de metas específicas para prevenir danos evitáveis e minimizar riscos de
        incidentes, propõe-se o desenvolvimento de um programa nacional de segurança do paciente que
        esteja vinculada aos programas de qualidade do governo federal. Tal programa deve envolver,
        no mínimo, o Ministério da Saúde, a ANVISA, a Agência Nacional de Saúde Suplementar (ANS) e
        o Ministério da Educação, sendo o último um importante aliado para a formação de
        profissionais de saúde, especialmente nos hospitais de ensino.</p>
      <p>O programa nacional de segurança do paciente faz-se necessário porque vem ao encontro do
        moderno conceito em saúde de prevenção quaternária, que objetiva a detecção de indivíduos em
        risco de intervencionismo excessivo em saúde, que implica atividades desnecessárias, e
        sugerir-lhes alternativas eticamente aceitáveis, atenuando ou evitando efeitos adversos.
          <xref ref-type="bibr" rid="B2">
          <sup>2</sup>
        </xref>
        <sup>,</sup>
        <xref ref-type="bibr" rid="B19">
          <sup>19</sup>
        </xref>
      </p>
      <p>Essa abordagem é particularmente importante no Brasil, que teve o crescimento exponencial
        de novas tecnologias disponíveis no mercado de saúde na última década, especialmente após a
        criação da ANVISA, tem marco legal muito recente sobre incorporação de tecnologias baseada
        em evidências <xref ref-type="fn" rid="fn7">
          <sup>g</sup>
        </xref> e está investindo em um modelo humanizado e orientado para a saúde. <xref
          ref-type="bibr" rid="B1">
          <sup>1</sup>
        </xref>
      </p>
      <p>O programa de segurança do paciente deve ser difundido nas diferentes instituições que
        compõem o sistema de saúde em todos os estados da federação a fim de que conheçam e
        compartilhem o conhecimento acerca dos resultados obtidos na assistência, incluindo os
        resultados negativos. Portanto, a implantação de um sistema nacional de notificações de
        incidentes deve ser uma das ações prioritárias de um programa nacional de segurança do
        paciente que contemple, minimamente, metas para gestão dos riscos envolvendo a assistência à
        saúde, tais como a identificação correta de pacientes, redução de infecções hospitalares,
        erros em procedimentos como cirurgias e medicação, que estão entre as chamadas nove soluções
        para a segurança do paciente, segundo a OMS. <xref ref-type="fn" rid="fn8">
          <sup>h</sup>
        </xref>
      </p>
    </sec>
    <sec>
      <title>Sistema de notificações de incidentes</title>
      <p>As notificações por parte dos profissionais de saúde, pacientes e seus cuidadores são
        importantes para a identificação de incidentes em saúde, especialmente por ser um método de
        baixo custo e, principalmente, por envolver os profissionais que prestam assistência em uma
        política de melhoria contínua centrada no paciente.</p>
      <p>Para garantir a produção de informação nas instituições de saúde para a tomada de decisões
        e a responsabilização com a melhoria de qualidade, é condição essencial que sejam feitos
        investimentos no desenvolvimento de capacidades locais e nos sistemas de informação já
        existentes. <xref ref-type="bibr" rid="B10">
          <sup>10</sup>
        </xref>
      </p>
      <p>A experiência da ANVISA com a Rede Sentinela é um bom exemplo de que a interação entre
        governo e instituições de saúde é possível e pode promover o desenvolvimento dos serviços em
        prol da segurança e da qualidade, seja pela cultura do relato voluntário, seja pela adesão
        aos programas de qualidade.</p>
      <p>O número de notificações encaminhadas pelos hospitais integrantes da Rede Sentinela ao
        Sistema de Notificações em Vigilância Sanitária (NOTIVISA) aumentou em 48,8% após o primeiro
        ano de implantação do sistema, quando comparado ao ano anterior. O estímulo da ANVISA para
        que os hospitais da Rede participassem de programas de qualidade, acreditação ou similar
        influenciou para que 30% desses hospitais estivessem participando de algum programa desse
        tipo em 2008. <xref ref-type="fn" rid="fn3">
          <sup>c</sup>
        </xref>
      </p>
      <p>Embora a iniciativa da ANVISA tenha sido importante para o estímulo da qualidade nos
        hospitais, ela é uma pequena parcela entre as mais de 8.000 instituições hospitalares
        brasileiras, correspondendo apenas a cerca de 13% dos leitos hospitalares no País. <xref
          ref-type="fn" rid="fn3">
          <sup>c</sup>
        </xref>
      </p>
      <p>Por esse motivo, o papel dos órgãos governamentais que recebem as informações sobre os
        resultados em saúde é fundamental, cabendo a eles ações que promovam a melhoria em curto
        espaço de tempo a fim de evitar danos aos pacientes. A utilização de sistemas informatizados
        em plataforma web, ou seja, disponíveis na internet para envio e recebimento imediatos, é
        passo fundamental para que um país de grande extensão territorial como o Brasil desenvolva
        um programa nacional de segurança do paciente. Além disso, deve-se estabelecer um modelo
        brasileiro de pagamento por desempenho a fim de beneficiar as instituições que estejam
        comprometidas como o modelo de melhoria contínua da qualidade.</p>
      <p>O modelo para o sistema nacional de notificações de incidentes pode ser útil no
        desenvolvimento da cultura de segurança do paciente no SUS, conforme mostra a <xref
          ref-type="fig" rid="f01">Figura</xref> .</p>
      <p>
        <fig id="f01">
          <label>Figura</label>
          <caption>
            <title>Fluxo simplificado para o Sistema Nacional de Notificações de Incidentes em
              Saúde.</title>
          </caption>
          <graphic xlink:href="0034-8910-rsp-47-04-0791-gf01"/>
        </fig>
      </p>
      <p>O desenvolvimento e implementação de um sistema informatizado único para receber
        notificações de todas as instituições de saúde deverá ter como objetivos facilitar e
        agilizar o processo de envio e de tomada de decisões a partir da notificação, minimizando
        riscos e evitando eventos adversos, ampliando a qualidade da assistência e a segurança dos
        pacientes em todos os níveis, da menor clínica de atenção à saúde ao sistema de saúde
        brasileiro; ampliar o conhecimento sobre os riscos e incidentes que ocorrem nas instituições
        brasileiras, direcionando o planejamento de ações dos gestores de saúde; melhorar a
        qualidade dos dados encaminhados; garantir a legibilidade das informações disponíveis;
        preservar a confidencialidade dos notificadores e dados relatados; e, por fim, reduzir
        custos do processo de notificação.</p>
      <p>À medida que o sistema seja utilizado com maior frequência e eficácia para o envio
        voluntário de notificações de incidentes, especialmente os potenciais eventos adversos, a
        tomada de decisões quanto às intervenções necessárias para evitar a ocorrência de danos
        poderá ser mais rápida, o que pode reduzir gastos desnecessários com tratamento de eventos
        adversos que poderiam ter sido prevenidos.</p>
      <p>Para tanto, a autonomia e pró-atividade das instituições de saúde deve ser estimulada. Até
        que haja a tomada de decisão por parte do governo, as instituições também devem realizar
        ações de melhoria internas, visando à promoção da segurança do paciente e à qualidade da
        atenção. Nesse sentindo, os estabelecimentos de saúde deverão ter acesso ao sistema
        informatizado não somente para o envio de notificações da gerência de riscos para o sistema
        nacional, mas também para que essa gerência se utilize do sistema para receber as
        notificações da equipe de saúde de sua instituição. Adicionalmente, o sistema deve permitir
        que a instituição acompanhe o andamento da análise das informações por ela encaminhadas ao
        sistema nacional. IMAGEMINLINE.</p>
          <inline-formula>
          <inline-graphic xlink:href="0034-8910-rsp-47-04-0791-gf01"/>
          </inline-formula>
      <p>O sistema informatizado é uma importante estratégia de promoção da qualidade aliada à
        sustentabilidade, pois, ao deixar de utilizar papéis, reduz gastos com materiais de consumo
        e geração de resíduos como os próprios papéis, cartuchos de impressoras e canetas.
        Adicionalmente, há outros aspectos que justificam a implantação de sistemas informatizados
        de notificação, a saber: <xref ref-type="bibr" rid="B5">
          <sup>5</sup>
        </xref>
      </p>
      <list list-type="bullet">
        <list-item>
          <p>elimina a necessidade de utilização de sistemas de envio de documentos internos às
            instituições e destas para o sistema nacional de notificações, o que reduz o tempo da
            chegada da informação e reduz gastos com o envio delas;</p>
        </list-item>
        <list-item>
          <p>pode-se eliminar a possibilidade de extravio e perda de informações, especialmente se
            estas forem preservadas em bancos de dados redundantes e cópias de segurança, sem
            necessidade de espaço para arquivo físico, além de permitir o manuseio ágil das
            informações e a análise de indicadores de gestão;</p>
        </list-item>
        <list-item>
          <p>é possível requerer mais informações sobre os incidentes sem dificultar a coleta dos
            dados, melhorando a qualidade das informações e ampliando a participação dos
            profissionais de saúde, o que não é possível com o sistema manuscrito.</p>
        </list-item>
      </list>
      <p>Quanto aos aspectos sociais da sustentabilidade, reduzir o tempo gasto para o envio do
        relato aumenta a participação dos profissionais de saúde com as notificações, bem como sua
        disponibilidade junto aos pacientes possibilitam o envolvimento do paciente e seus
        cuidadores no processo de monitorização de riscos e incidentes em saúde. Em uma política
        nacional, esses atores são fontes importantes de notificação voluntária. Com o sistema
        informatizado de notificações em plataforma web, é possível que qualquer pessoa com acesso à
        internet faça uma notificação.</p>
      <p>Países que já possuem política nacional de segurança do paciente, como Inglaterra, Estados
        Unidos, Austrália e Canadá, já permitem que os usuários do sistema e seus cuidadores façam
        notificações sobre riscos e incidentes que vivenciaram ou perceberam em serviços de saúde,
        sendo fundamentais para promoção da qualidade da assistência.</p>
      <p>O sistema nacional de notificações deveria, ainda, estar hospedado em um site interativo
        que disponibilizasse gratuitamente notícias, dicas de segurança, informações sobre eventos
        adversos, protocolos sobre como implementar um programa de segurança nos serviços de saúde,
        cursos e palestras <italic>online</italic> , como o <italic>Institute for Health
          Improvement</italic> tem feito nos Estados Unidos para os hospitais americanos. Esse
        portal teria como dupla função manter e estimular a adesão das instituições, e difundir e
        estimular a adoção de práticas seguras por meio do intercâmbio entre elas.</p>
      <p>A implantação do sistema informatizado de notificações sobre incidentes na saúde como base
        para a cultura de segurança do paciente no sistema de saúde brasileiro parece ser uma
        estratégia viável e necessária para a qualificação da assistência, com a qual os gestores
        conhecerão os incidentes que ocorrem na prestação de assistência aos usuários do sistema, em
        instituições públicas e privadas, de forma sistematizada, sem depender de que pesquisas
        sejam realizadas exclusivamente para esse fim. Desse modo, nortear-se-á o delineamento de
        estratégias de gestão de riscos para a segurança do paciente, ampliando a qualidade dos
        serviços ofertados à população brasileira.</p>
    </sec>
  </body>
  <back>
    <app-group>
      <p>DESTAQUES</p>
      <p>Recentemente, foi lançado pelo Ministério da Saúde o Programa Nacional de Segurança do
        Paciente para que ações de segurança do paciente fossem promovidas no âmbito do Sistema
        Único de Saúde. É louvável essa iniciativa e foi o maior foco de discussão do artigo, o qual
        aborda importantes pontos que não foram incluídos no citado Programa. Esses pontos
        referem-se à forma com que o Governo pretende remunerar as instituições que obtiverem os
        melhores resultados na prestação de serviços e também como serão tratadas as informações
        advindas de notifi cações, de forma que o conhecimento gerado promova a melhoria efetiva dos
        serviços prestados pelo Sistema Único de Saúde.</p>
      <p>O artigo faz refl exão à luz da melhoria da política de saúde para a segurança do paciente,
        onde foi possível verificar que ações para a cultura de segurança podem proliferar e gerar
        bons resultados também em hospitais públicos. Assim, traz à luz discussões importantes para
        todas as instituições e especialmente para gestores do sistema de forma a aprimorar o
        Programa Nacional e desencadear a melhoria contínua dos serviços centrada no paciente.</p>
      <p>Profa. Rita de Cássia Barradas Barata</p>
      <p>Editora Científica</p>
    </app-group>
    <ref-list>
      <title>REFERÊNCIAS</title>
      <ref id="B1">
        <label>1</label>
        <mixed-citation>. Almeida-Filho N. Ensino superior e os serviços de saúde no Brasil.
            <italic>Lancet</italic> . 2011;6-7.</mixed-citation>
        <element-citation publication-type="journal">
          <person-group>
            <name>
              <surname>Almeida</surname>
              <given-names>N</given-names>
              <suffix>Filho</suffix>
            </name>
          </person-group>
          <article-title xml:lang="pt">Ensino superior e os serviços de saúde no
            Brasil</article-title>
          <source>Lancet</source>
          <year>2011</year>
          <fpage>6</fpage>
          <lpage>7</lpage>
        </element-citation>
      </ref>
      <ref id="B2">
        <label>2</label>
        <mixed-citation>. Bentzen N. WONCA dictionary of general/family practice. Copenhagen:
          Maanedskift Lager; 2003.</mixed-citation>
        <element-citation publication-type="book">
          <person-group>
            <name>
              <surname>Bentzen</surname>
              <given-names>N</given-names>
            </name>
          </person-group>
          <source xml:lang="en">WONCA dictionary of general/family practice</source>
          <publisher-loc>Copenhagen</publisher-loc>
          <publisher-name>Maanedskift Lager</publisher-name>
          <year>2003</year>
        </element-citation>
      </ref>
      <ref id="B3">
        <label>3</label>
        <mixed-citation>. Berlowitz D, Burgess Jr JF, Young GJ. Improving quality of care: emerging
          evidence on pay-for-performance. <italic>Med Care Res Rev</italic> . 2006;63(1
          Suppl):73S-95S.</mixed-citation>
        <element-citation publication-type="journal">
          <person-group>
            <name>
              <surname>Berlowitz</surname>
              <given-names>D</given-names>
            </name>
            <name>
              <surname>Burgess</surname>
              <given-names>JF</given-names>
              <suffix>Jr</suffix>
            </name>
            <name>
              <surname>Young</surname>
              <given-names>GJ</given-names>
            </name>
          </person-group>
          <article-title xml:lang="en">Improving quality of care: emerging evidence on
            pay-for-performance</article-title>
          <source>Med Care Res Rev</source>
          <year>2006</year>
          <volume>63</volume>
          <issue>1</issue>
          <supplement>Suppl</supplement>
          <fpage>73S</fpage>
          <lpage>95S</lpage>
        </element-citation>
      </ref>
      <ref id="B4">
        <label>4</label>
        <mixed-citation>. Capucho HC. Near miss: quase erro ou potencial evento adverso? <italic>Rev
            Latino-Am Enferm</italic> . 2011;19(5):1272-3.
          DOI:10.1590/S0104-11692011000500027</mixed-citation>
        <element-citation publication-type="journal">
          <person-group>
            <name>
              <surname>Capucho</surname>
              <given-names>HC</given-names>
            </name>
          </person-group>
          <article-title xml:lang="pt">Near miss: quase erro ou potencial evento
            adverso?</article-title>
          <source>
            <italic>Rev Latino-Am Enferm</italic>
          </source>
          <year>2011</year>
          <volume>19</volume>
          <issue>5</issue>
          <fpage>1272</fpage>
          <lpage>1273</lpage>
          <pub-id pub-id-type="doi">10.1590/S0104-11692011000500027</pub-id>
        </element-citation>
      </ref>
      <ref id="B5">
        <label>5</label>
        <mixed-citation>. Capucho HC, Arnas ER, Cassiani SHBD. Segurança do paciente: comparação
          entre notificações voluntárias manuscritas e informatizadas sobre incidentes em saúde.
            <italic>Rev Gaucha Enferm</italic> . 2013;34(1):164-72.
          DOI:10.1590/S1983-14472013000100021</mixed-citation>
        <element-citation publication-type="journal">
          <person-group>
            <name>
              <surname>Capucho</surname>
              <given-names>HC</given-names>
            </name>
            <name>
              <surname>Arnas</surname>
              <given-names>ER</given-names>
            </name>
            <name>
              <surname>Cassiani</surname>
              <given-names>SHBD</given-names>
            </name>
          </person-group>
          <article-title xml:lang="pt">Segurança do paciente: comparação entre notificações
            voluntárias manuscritas e informatizadas sobre incidentes em saúde</article-title>
          <source>Rev Gaucha Enferm</source>
          <year>2013</year>
          <volume>34</volume>
          <issue>1</issue>
          <fpage>164</fpage>
          <lpage>172</lpage>
          <pub-id pub-id-type="doi">10.1590/S1983-14472013000100021</pub-id>
        </element-citation>
      </ref>
      <ref id="B6">
        <label>6</label>
        <mixed-citation>. Constance HF, Yee Wei L, Mattke S, Damberg C, Shekelle PG. Systematic
          Review: The Evidence That Publishing Patient Care Performance Data Improves Quality of
          Care. <italic>Ann Intern Med</italic> . 2008;15(148):111-123.</mixed-citation>
        <element-citation publication-type="journal">
          <person-group>
            <name>
              <surname>Constance</surname>
              <given-names>HF</given-names>
            </name>
            <name>
              <surname>Yee Wei</surname>
              <given-names>L</given-names>
            </name>
            <name>
              <surname>Mattke</surname>
              <given-names>S</given-names>
            </name>
            <name>
              <surname>Damberg</surname>
              <given-names>C</given-names>
            </name>
            <name>
              <surname>Shekelle</surname>
              <given-names>PG</given-names>
            </name>
          </person-group>
          <article-title xml:lang="en">Systematic Review: The Evidence That Publishing Patient Care
            Performance Data Improves Quality of Care</article-title>
          <source>Ann Intern Med</source>
          <year>2008</year>
          <volume>15</volume>
          <issue>148</issue>
          <fpage>111</fpage>
          <lpage>123</lpage>
        </element-citation>
      </ref>
      <ref id="B7">
        <label>7</label>
        <mixed-citation>. Escrivao Jr A, Koyama MF. O relacionamento entre hospitais e operadoras de
          planos de saúde no âmbito do Programa de Qualificação da Saúde Suplementar da ANS.
            <italic>Cienc Saude Coletiva</italic> . 2007;12(4):903-14.
          DOI:10.1590/S1413-81232007000400012</mixed-citation>
        <element-citation publication-type="journal">
          <person-group>
            <name>
              <surname>Escrivao</surname>
              <given-names>A</given-names>
              <suffix>Jr</suffix>
            </name>
            <name>
              <surname>Koyama</surname>
              <given-names>MF</given-names>
            </name>
          </person-group>
          <article-title xml:lang="pt">O relacionamento entre hospitais e operadoras de planos de
            saúde no âmbito do Programa de Qualificação da Saúde Suplementar da ANS</article-title>
          <source>Cienc Saude Coletiva</source>
          <year>2007</year>
          <volume>12</volume>
          <issue>4</issue>
          <fpage>903</fpage>
          <lpage>914</lpage>
          <pub-id pub-id-type="doi">10.1590/S1413-81232007000400012</pub-id>
        </element-citation>
      </ref>
      <ref id="B8">
        <label>8</label>
        <mixed-citation>. Fisher ES. Paying for Performance - Risks and Recommendations. <italic>New
            Eng J Med</italic> . 2006;355(18):1845-7. DOI:10.1056/NEJMp068221</mixed-citation>
        <element-citation publication-type="journal">
          <person-group>
            <name>
              <surname>Fisher</surname>
              <given-names>ES</given-names>
            </name>
          </person-group>
          <article-title xml:lang="en">Paying for Performance - Risks and
            Recommendations</article-title>
          <source>New Eng J Med</source>
          <year>2006</year>
          <volume>355</volume>
          <issue>18</issue>
          <fpage>1845</fpage>
          <lpage>1847</lpage>
          <pub-id pub-id-type="doi">10.1056/NEJMp068221</pub-id>
        </element-citation>
      </ref>
      <ref id="B9">
        <label>9</label>
        <mixed-citation>. Fung CH, Lim YW, Mattke S, Damberg C, Shekelle PG. Systematic Review: The
          Evidence That Publishing Patient Care Performance Data Improves Quality of Care.
            <italic>Ann Intern Med</italic> . 2008;148(2):111-23.
          DOI:10.7326/0003-4819-148-2-200801150-00006</mixed-citation>
        <element-citation publication-type="journal">
          <person-group>
            <name>
              <surname>Fung</surname>
              <given-names>CH</given-names>
            </name>
            <name>
              <surname>Lim</surname>
              <given-names>YW</given-names>
            </name>
            <name>
              <surname>Mattke</surname>
              <given-names>S</given-names>
            </name>
            <name>
              <surname>Damberg</surname>
              <given-names>C</given-names>
            </name>
            <name>
              <surname>Shekelle</surname>
              <given-names>PG</given-names>
            </name>
          </person-group>
          <article-title xml:lang="en">Systematic Review: The Evidence That Publishing Patient Care
            Performance Data Improves Quality of Care</article-title>
          <source>Ann Intern Med</source>
          <year>2008</year>
          <volume>148</volume>
          <issue>2</issue>
          <fpage>111</fpage>
          <lpage>123</lpage>
          <pub-id pub-id-type="doi">10.7326/0003-4819-148-2-200801150-00006</pub-id>
        </element-citation>
      </ref>
      <ref id="B10">
        <label>10</label>
        <mixed-citation>. Gouvêa CSDD, Travassos C. Indicadores de segurança do paciente para
          hospitais de pacientes agudos: revisão sistemática. <italic>Cad Saude Publica</italic> .
          2010;26(6):1061-78. DOI:10.1590/S0102-311X2010000600002</mixed-citation>
        <element-citation publication-type="journal">
          <person-group>
            <name>
              <surname>Gouvêa</surname>
              <given-names>CSDD</given-names>
            </name>
            <name>
              <surname>Travassos</surname>
              <given-names>C</given-names>
            </name>
          </person-group>
          <article-title xml:lang="pt">Indicadores de segurança do paciente para hospitais de
            pacientes agudos: revisão sistemática</article-title>
          <source>Cad Saude Publica</source>
          <year>2010</year>
          <volume>26</volume>
          <issue>6</issue>
          <fpage>1061</fpage>
          <lpage>1078</lpage>
          <pub-id pub-id-type="doi">10.1590/S0102-311X2010000600002</pub-id>
        </element-citation>
      </ref>
      <ref id="B11">
        <label>11</label>
        <mixed-citation>. Kohn LT, Corrigan JM, Donaldson MS. To err is human: building a safer
          health system. 2.ed. Washington: National Academy of Sciences; 1999.</mixed-citation>
        <element-citation publication-type="book">
          <person-group>
            <name>
              <surname>Kohn</surname>
              <given-names>LT</given-names>
            </name>
            <name>
              <surname>Corrigan</surname>
              <given-names>JM</given-names>
            </name>
            <name>
              <surname>Donaldson</surname>
              <given-names>MS</given-names>
            </name>
          </person-group>
          <source xml:lang="en">To err is human: building a safer health system</source>
          <edition>2.ed</edition>
          <publisher-loc>Washington</publisher-loc>
          <publisher-name>National Academy of Sciences</publisher-name>
          <year>1999</year>
        </element-citation>
      </ref>
      <ref id="B12">
        <label>12</label>
        <mixed-citation>. McDonald R, Roland M. Pay for performance in primary care in England and
          California: comparison of unintended consequences. <italic>Ann Fam Med</italic> .
          2009;7(2):121-7. DOI:10.1370/afm.946</mixed-citation>
        <element-citation publication-type="journal">
          <person-group>
            <name>
              <surname>McDonald</surname>
              <given-names>R</given-names>
            </name>
            <name>
              <surname>Roland</surname>
              <given-names>M</given-names>
            </name>
          </person-group>
          <article-title xml:lang="en">Pay for performance in primary care in England and
            California: comparison of unintended consequences</article-title>
          <source>Ann Fam Med</source>
          <year>2009</year>
          <volume>7</volume>
          <issue>2</issue>
          <fpage>121</fpage>
          <lpage>127</lpage>
          <pub-id pub-id-type="doi">10.1370/afm.946</pub-id>
        </element-citation>
      </ref>
      <ref id="B13">
        <label>13</label>
        <mixed-citation>. Mendes W, Martins M, Rozenfeld S, Travassos C. The assessment of adverse
          events in hospitals in Brazil. <italic>Int J Qual Health Care</italic> .
          2009;21(4):279-84. DOI:10.1093/intqhc/mzp022</mixed-citation>
        <element-citation publication-type="journal">
          <person-group>
            <name>
              <surname>Mendes</surname>
              <given-names>W</given-names>
            </name>
            <name>
              <surname>Martins</surname>
              <given-names>M</given-names>
            </name>
            <name>
              <surname>Rozenfeld</surname>
              <given-names>S</given-names>
            </name>
            <name>
              <surname>Travassos</surname>
              <given-names>C</given-names>
            </name>
          </person-group>
          <article-title xml:lang="en">The assessment of adverse events in hospitals in
            Brazil</article-title>
          <source>Int J Qual Health Care</source>
          <year>2009</year>
          <volume>21</volume>
          <issue>4</issue>
          <fpage>279</fpage>
          <lpage>284</lpage>
          <pub-id pub-id-type="doi">10.1093/intqhc/mzp022</pub-id>
        </element-citation>
      </ref>
      <ref id="B14">
        <label>14</label>
        <mixed-citation>. Miasso, AI, Grou CR, Cassiani SHB, Silva AEBC, Fakih FT. Erros de
          medicação: tipos, fatores causais e providencias em quatro hospitais brasileiros.
            <italic>Rev Esc Enferm USP</italic> . 2006:40(4):524-32.
          DOI:10.1590/S0080-62342006000400011</mixed-citation>
        <element-citation publication-type="journal">
          <person-group>
            <name>
              <surname>Miasso</surname>
              <given-names>AI</given-names>
            </name>
            <name>
              <surname>Grou</surname>
              <given-names>CR</given-names>
            </name>
            <name>
              <surname>Cassiani</surname>
              <given-names>SHB</given-names>
            </name>
            <name>
              <surname>Silva</surname>
              <given-names>AEBC</given-names>
            </name>
            <name>
              <surname>Fakih</surname>
              <given-names>FT</given-names>
            </name>
          </person-group>
          <article-title xml:lang="pt">Erros de medicação: tipos, fatores causais e providencias em
            quatro hospitais brasileiros</article-title>
          <source>Rev Esc Enferm USP</source>
          <year>2006</year>
          <volume>40</volume>
          <issue>4</issue>
          <fpage>524</fpage>
          <lpage>532</lpage>
          <pub-id pub-id-type="doi">10.1590/S0080-62342006000400011</pub-id>
        </element-citation>
      </ref>
      <ref id="B15">
        <label>15</label>
        <mixed-citation>. Novaes HM. O processo de acreditação dos serviços de saúde. <italic>Rev
            Adm Saude</italic> . 2007;9(37):133-40.</mixed-citation>
        <element-citation publication-type="journal">
          <person-group>
            <name>
              <surname>Novaes</surname>
              <given-names>HM</given-names>
            </name>
          </person-group>
          <article-title xml:lang="pt">O processo de acreditação dos serviços de
            saúde</article-title>
          <source>Rev Adm Saude</source>
          <year>2007</year>
          <volume>9</volume>
          <issue>37</issue>
          <fpage>133</fpage>
          <lpage>140</lpage>
        </element-citation>
      </ref>
      <ref id="B16">
        <label>16</label>
        <mixed-citation>. Paim J, Travassos C, Almeida C, Bahia L, Macinko J. O sistema de saúde
          brasileiro: história, avanços e desafios. <italic>Lancet</italic> .
          2011;11-31.</mixed-citation>
        <element-citation publication-type="journal">
          <person-group>
            <name>
              <surname>Paim</surname>
              <given-names>J</given-names>
            </name>
            <name>
              <surname>Travassos</surname>
              <given-names>C</given-names>
            </name>
            <name>
              <surname>Almeida</surname>
              <given-names>C</given-names>
            </name>
            <name>
              <surname>Bahia</surname>
              <given-names>L</given-names>
            </name>
            <name>
              <surname>Macinko</surname>
              <given-names>J</given-names>
            </name>
          </person-group>
          <article-title xml:lang="pt">O sistema de saúde brasileiro: história, avanços e
            desafios</article-title>
          <source>Lancet</source>
          <year>2011</year>
          <fpage>11</fpage>
          <lpage>31</lpage>
        </element-citation>
      </ref>
      <ref id="B17">
        <label>17</label>
        <mixed-citation>. Shoyer AL, London MJ, VillaNueva CB, Sethi GK, Marshall G, Moritz TE, et
          al. The processes, structures, and outcomes of care in cardiac surgery study an overview.
            <italic>Med Care</italic> . 1995;33(10):OS1-4.
          DOI:10.1097/00005650-199510001-00001</mixed-citation>
        <element-citation publication-type="journal">
          <person-group>
            <name>
              <surname>Shoyer</surname>
              <given-names>AL</given-names>
            </name>
            <name>
              <surname>London</surname>
              <given-names>MJ</given-names>
            </name>
            <name>
              <surname>VillaNueva</surname>
              <given-names>CB</given-names>
            </name>
            <name>
              <surname>Sethi</surname>
              <given-names>GK</given-names>
            </name>
            <name>
              <surname>Marshall</surname>
              <given-names>G</given-names>
            </name>
            <name>
              <surname>Moritz</surname>
              <given-names>TE</given-names>
            </name>
            <etal>et al</etal>
          </person-group>
          <article-title xml:lang="en">The processes, structures, and outcomes of care in cardiac
            surgery study an overview</article-title>
          <source>Med Care</source>
          <year>1995</year>
          <volume>33</volume>
          <issue>10</issue>
          <fpage>OS1</fpage>
          <lpage>OS4</lpage>
          <pub-id pub-id-type="doi">10.1097/00005650-199510001-00001</pub-id>
        </element-citation>
      </ref>
      <ref id="B18">
        <label>18</label>
        <mixed-citation>. Thomas AN, Panchagnula U. Medication-related patient safety incidents in
          critical care: a review of reports to the UK National Patient Safety Agency.
            <italic>Anaesthesia</italic> . 2008;63(7):726-33.
          DOI:10.1111/j.1365-2044.2008.05485.x</mixed-citation>
        <element-citation publication-type="journal">
          <person-group>
            <name>
              <surname>Thomas</surname>
              <given-names>AN</given-names>
            </name>
            <name>
              <surname>Panchagnula</surname>
              <given-names>U</given-names>
            </name>
          </person-group>
          <article-title xml:lang="en">Medication-related patient safety incidents in critical care:
            a review of reports to the UK National Patient Safety Agency</article-title>
          <source>Anaesthesia</source>
          <year>2008</year>
          <volume>63</volume>
          <issue>7</issue>
          <fpage>726</fpage>
          <lpage>733</lpage>
          <pub-id pub-id-type="doi">10.1111/j.1365-2044.2008.05485.x</pub-id>
        </element-citation>
      </ref>
      <ref id="B19">
        <label>19</label>
        <mixed-citation>. Unruh LY, Zhang NJ. Nurse Staffing and patient safety in hospitals: new
          variable and longitudinal approaches. <italic>Nurs Res</italic> . 2012;61(1):3-12.
          DOI:10.1097/NNR.0b013e3182358968</mixed-citation>
        <element-citation publication-type="journal">
          <person-group>
            <name>
              <surname>Unruh</surname>
              <given-names>LY</given-names>
            </name>
            <name>
              <surname>Zhang</surname>
              <given-names>NJ</given-names>
            </name>
          </person-group>
          <article-title xml:lang="en">Nurse Staffing and patient safety in hospitals: new variable
            and longitudinal approaches</article-title>
          <source>Nurs Res</source>
          <year>2012</year>
          <volume>61</volume>
          <issue>1</issue>
          <fpage>3</fpage>
          <lpage>12</lpage>
          <pub-id pub-id-type="doi">10.1097/NNR.0b013e3182358968</pub-id>
        </element-citation>
      </ref>
      <ref id="B20">
        <label>20</label>
        <mixed-citation>. Victora CG, Barreto ML, Leal MC, Monteiro CA, Schmidt MI, Paim J, et al.
          Condições de saúde e inovações nas políticas de saúde no Brasil: o caminho a percorrer.
            <italic>Lancet</italic> . 2011;90-102.</mixed-citation>
        <element-citation publication-type="journal">
          <person-group>
            <name>
              <surname>Victora</surname>
              <given-names>CG</given-names>
            </name>
            <name>
              <surname>Barreto</surname>
              <given-names>ML</given-names>
            </name>
            <name>
              <surname>Leal</surname>
              <given-names>MC</given-names>
            </name>
            <name>
              <surname>Monteiro</surname>
              <given-names>CA</given-names>
            </name>
            <name>
              <surname>Schmidt</surname>
              <given-names>MI</given-names>
            </name>
            <name>
              <surname>Paim</surname>
              <given-names>J</given-names>
            </name>
            <etal>et al</etal>
          </person-group>
          <article-title xml:lang="pt">Condições de saúde e inovações nas políticas de saúde no
            Brasil: o caminho a percorrer</article-title>
          <source>Lancet</source>
          <year>2011</year>
          <fpage>90</fpage>
          <lpage>102</lpage>
        </element-citation>
      </ref>
      <ref id="B21">
        <label>21</label>
        <mixed-citation>. Vincent C. Segurança do paciente. Orientações para evitar eventos
          adversos. São Caetano do Sul: Editora Yendis; 2009.</mixed-citation>
        <element-citation publication-type="book">
          <person-group>
            <name>
              <surname>Vincent</surname>
              <given-names>C</given-names>
            </name>
          </person-group>
          <source xml:lang="pt">Segurança do paciente. Orientações para evitar eventos
            adversos</source>
          <publisher-loc>São Caetano do Sul</publisher-loc>
          <publisher-name>Editora Yendis</publisher-name>
          <year>2009</year>
        </element-citation>
      </ref>
      <ref id="B22">
        <label>22</label>
        <mixed-citation>. Vincent C, Woloshynowych M. Adverse events in British hospitals:
          preliminary retrospective record review. <italic>BMJ</italic> . 2001;322(7285):517-9.
          DOI:10.1136/bmj.322.7285.517</mixed-citation>
        <element-citation publication-type="journal">
          <person-group>
            <name>
              <surname>Vincent</surname>
              <given-names>C</given-names>
            </name>
            <name>
              <surname>Woloshynowych</surname>
              <given-names>M</given-names>
            </name>
          </person-group>
          <article-title xml:lang="en">Adverse events in British hospitals: preliminary
            retrospective record review</article-title>
          <source>BMJ</source>
          <year>2001</year>
          <volume>322</volume>
          <issue>7285</issue>
          <fpage>517</fpage>
          <lpage>519</lpage>
          <pub-id pub-id-type="doi">10.1136/bmj.322.7285.517</pub-id>
        </element-citation>
      </ref>
      <ref id="B23">
        <label>23</label>
        <mixed-citation>. World Health Organization. The conceptual framework for the international
          classification for patient safety. Version 1.1. Final technical report. Chapter 3. The
          international classification for patient safety. Key concepts and preferred terms. Geneva;
          2009 [citado 2011 jul 04]. Disponível em:
          http://www.who.int/patientsafety/taxonomy/icps_chapter3.pdf</mixed-citation>
        <element-citation publication-type="report">
          <person-group>
            <collab>World Health Organization</collab>
          </person-group>
          <source xml:lang="en">The conceptual framework for the international classification for
            patient safety. Version 1.1. Final technical report. Chapter 3. The international
            classification for patient safety. Key concepts and preferred terms</source>
          <publisher-loc>Geneva</publisher-loc>
          <year>2009</year>
          <date-in-citation content-type="cited-date">2011 jul 04</date-in-citation>
          <ext-link ext-link-type="uri"
            xlink:href="http://www.who.int/patientsafety/taxonomy/icps_chapter3.pdf"
            >http://www.who.int/patientsafety/taxonomy/icps_chapter3.pdf</ext-link>
        </element-citation>
      </ref>
    </ref-list>
    <fn-group>
      <fn id="fn1">
        <label>a</label>
        <p>World Health Organization. APPS web-based registration mechanism open. Geneva; 2012
          [citado 2012 jun 2]. Disponível em: http://www.who.int/patientsafety/en/</p>
      </fn>
      <fn id="fn2">
        <label>b</label>
        <p>Capucho HC. Sistemas manuscrito e informatizado de notificação voluntária de incidentes
          em saúde como base para a cultura de segurança do paciente [tese de doutorado]. São Paulo:
          Escola de Enfermagem de Ribeirão Preto da USP; 2012.</p>
      </fn>
      <fn id="fn3">
        <label>c</label>
        <p>Ministério da Saúde. Portaria GM/MS nº 529, de 1 de abril de 2013. Institui o Programa
          Nacional de Segurança do Paciente (PNSP). <italic>Diario Oficial Uniao</italic> . 2 abr
          2013;Seção1:43-4.</p>
      </fn>
      <fn id="fn4">
        <label>d</label>
        <p>Petramale CA. O projeto dos hospitais sentinela e a gerência de risco sanitário
          hospitalar. In: Capucho HC, Carvalho FD, Cassiani SHB. Farmacovigilância - Gerenciamento
          de Riscos da Terapia Medicamentosa para a Segurança do Paciente. São Caetano do Sul:
          Editora Yendis. 2001. p. 191-224.</p>
      </fn>
      <fn id="fn5">
        <label>e</label>
        <p>Ministério da Saúde. Portaria nº 396, de 4 de março de 2011 Institui o projeto de
          formação e melhoria da qualidade de rede de saúde (Quali-SUS-Rede) e suas diretrizes
          operacionais gerais. <italic>Diario Oficial Uniao</italic> . 9 mar 2011.</p>
      </fn>
      <fn id="fn6">
        <label>f</label>
        <p>Ministério da Saúde. Índice de Desempenho do SUS – IDSUS. Brasília (DF); 2011 [citado
          2012 jun 2]. Disponível em:
          http://portal.saude.gov.br/portal/saude/area.cfm?id_area=1080</p>
      </fn>
      <fn id="fn7">
        <label>g</label>
        <p>Brasil. Lei nº 12.401, de 28 de abril de 2011. Altera a Lei nº 8.080, de 19 de setembro
          de 1990, para dispor sobre a assistência terapêutica e a incorporação de tecnologia em
          saúde no âmbito do Sistema Único de Saúde - SUS. <italic>Diario Oficial Uniao</italic> .
          29 abr 2011:1.</p>
      </fn>
      <fn id="fn8">
        <label>h</label>
        <p>World Health Organization. WHO launches ‘Nine patient safety solutions. Geneva; 2007
          [citado 2012 jun 2]. Disponível em:
          http://www.who.int/mediacentre/news/releases/2007/pr22/en/index.html</p>
      </fn>

    </fn-group>
  </back>
  <sub-article article-type="translation" id="TR01" xml:lang="en">
    <front-stub>
      <article-categories>
        <subj-group subj-group-type="heading">
          <subject>Public Health Practice</subject>
        </subj-group>
      </article-categories>
      <title-group>
        <article-title xml:lang="en">The need to establish a national patient safety program in
          Brazil</article-title>
      </title-group>
      <contrib-group>
        <contrib contrib-type="author">
          <name>
            <surname>Capucho</surname>
            <given-names>Helaine Carneiro</given-names>
          </name>
          <xref ref-type="aff" rid="aff3">I</xref>
        </contrib>
        <contrib contrib-type="author">
          <name>
            <surname>Cassiani</surname>
            <given-names>Silvia Helena De Bortoli</given-names>
          </name>
          <xref ref-type="aff" rid="aff4">II</xref>
        </contrib>
      </contrib-group>
      <aff id="aff3">
        <label>I</label> Departamento de Gestão e Incorporação de Tecnologias e m Saúde. Secretaria
        de Ciência, Tecnologia e Insumos Estratégicos. Ministério da Saúde. Brasília, DF,
        Brasil</aff>
      <aff id="aff4">
        <label>II</label> Departamento de Enfermagem Geral e Especializada. Escola de Enfermagem de
        Ribeirão Preto. Universidade de São Paulo. Ribeirão Preto, SP, Brasil</aff>
      <author-notes>
        <corresp>
          <label>Correspondence</label> : Helaine Carneiro Capucho Esplanada dos Ministérios Bloco G
          Edifício Sede 9º andar Sala 949 70058-900 Brasília, DF, Brasil E-mail:
          helaine.capucho@saude.gov.br</corresp>
        <fn fn-type="conflict">
          <p>The authors declare that there are no conflicts of interest.</p>
        </fn>
      </author-notes>
      <abstract xml:lang="en">
        <p>The aim of the study was to promote reflection on the need to create a national incident
          notification system based on a brazilian patient safety program. Incidents in health care
          harm patients and encumber the health care system. Although a quality assessment program
          has been recently launched in health care institutions, the Brazilian Ministry of Health
          does not yet have a program which systematically assesses negative outcomes of health
          care. This article discusses the need to establish a national patient safety program in
          Brazil, aiming to promote a culture of patient safety and quality health care in the
          Brazilian Unified Health System.</p>
      </abstract>
      <kwd-group xml:lang="en">
        <kwd>Patient Safety</kwd>
        <kwd>Program Evaluation</kwd>
        <kwd>Unified Health System</kwd>
        <kwd>Quality Assurance, Health Care</kwd>
      </kwd-group>
    </front-stub>
    <body>
      <sec sec-type="intro">
        <title>INTRODUCTION</title>
        <p>For a long time the results of health care have been used to assess the quality of health
          care services. The Babylonians paid for medical services according to results and, in the
          Middle Ages, doctors whose services had negative outcomes had parts of their bodies
          mutilated. <xref ref-type="bibr" rid="B17">
            <sup>17</sup>
          </xref>
        </p>
        <p>Negative health care results are mainly known as any type of adverse event with the
          potential to harm patients <xref ref-type="bibr" rid="B20">
            <sup>20</sup>
          </xref> and can furnish significant data for creating a safer health care system. <xref
            ref-type="bibr" rid="B14">
            <sup>14</sup>
          </xref> The incident may harm (adverse event), or not, the patient or be a near miss,
          which is also classed as a potential adverse event. <xref ref-type="bibr" rid="B4">
            <sup>4</sup>
          </xref>
          <sup>,</sup>
          <xref ref-type="bibr" rid="B23">
            <sup>23</sup>
          </xref>
        </p>
        <p>Negative health care results were reported by the Institute of Medicine (IOM) in 1999,
            <xref ref-type="bibr" rid="B11">
            <sup>11</sup>
          </xref> when an estimated 44,000 to 98,000 deaths in the United States were caused by
          errors in patient care. Since then, health care results or outcomes have been subject to
          scrutiny as they are directly related to patients’ health care quality and safety. Patient
          safety is defined as the act of avoiding, preventing or improving adverse results or
          injuries resulting from the medical-hospital care process. <xref ref-type="bibr" rid="B21">
            <sup>21</sup>
          </xref>
        </p>
        <p>Faced with increased worldwide awareness after the publication of this shocking report,
          the World Health Organization (WHO) launched the World Alliance for Patient Safety in
          2004. <xref ref-type="fn" rid="fn9">
            <sup>a</sup>
          </xref> This awakened member countries, including Brazil, to their commitment to
          developing public policies and practices aimed as patient safety.</p>
        <p>In Europe, it is estimated that 10.8% of patients in hospitals were subject to adverse
          events, 46% of which were avoidable. <xref ref-type="bibr" rid="B22">
            <sup>22</sup>
          </xref> In Southeastern Brazil, a study of hospitals in Rio de Janeiro, RJ, estimated the
          incidence of these events at 7.6%. <xref ref-type="bibr" rid="B13">
            <sup>13</sup>
          </xref>
        </p>
        <p>Although the Ministry of Health and the <italic>Agência Nacional de Vigilância
            Sanitária</italic> (ANVISA – National Health Monitoring Agency) promoted World Alliance
          for Patient Safety initiatives, such as the campaign to introduce safe surgery protocols
          in hospitals, adherence on the part of the health care services is low, for the very
          reason that there is no institutional culture of patient safety. This is reflected in the
          high incidence of avoidable adverse events in Brazilian hospitals, which account for
          around 67% of all adverse events. <xref ref-type="bibr" rid="B13">
            <sup>13</sup>
          </xref>
          <sup>,</sup>
          <xref ref-type="bibr" rid="B20">
            <sup>20</sup>
          </xref>
        </p>
        <p>Although there are some positive aspects to the Brazilian health care system, such as
          universal vaccination coverage and the national transplant system, the high frequency of
          adverse events related to medications and hospital infections is a cause for concern.
            <xref ref-type="bibr" rid="B13">
            <sup>13</sup>
          </xref>
          <sup>,</sup>
          <xref ref-type="bibr" rid="B20">
            <sup>20</sup>
          </xref> These events are attributed to a lack of government policies <xref ref-type="fn"
            rid="fn10">
            <sup>b</sup>
          </xref>
          <sup>,</sup>
          <xref ref-type="fn" rid="fn11">
            <sup>c</sup>
          </xref>
          <sup>,</sup>
          <xref ref-type="fn" rid="fn12">
            <sup>d</sup>
          </xref> incentivizing health care institutions to participate in quality and accreditation
          programs. <xref ref-type="bibr" rid="B15">
            <sup>15</sup>
          </xref>
          <sup>,</sup>
          <xref ref-type="bibr" rid="B16">
            <sup>16</sup>
          </xref> There are currently Brazilian hospitals providing health care services without
          evaluating their processes or using such results to improve quality. <xref ref-type="fn"
            rid="fn12">
            <sup>d</sup>
          </xref>
        </p>
        <p>It is, therefore, necessary to be aware of the reality of such incidents in Brazil, which
          can be obtained by health care institutions becoming involved in monitoring them and in
          dealing appropriately with the data, in addition to reporting to government bodies.
          However, the fact that an organized flow of information exists does not in itself generate
          knowledge. This can only come about through the interdisciplinary interaction of those
          involved. <xref ref-type="bibr" rid="B14">
            <sup>14</sup>
          </xref>
        </p>
        <p>The aim of this study was to provoke reflection on using a national incident notification
          system as the basis for a Brazilian Patient Safety Program.</p>
      </sec>
      <sec>
        <title>BRAZILIAN UNIFIED HEALTH SYSTEM QUALITY AND PATIENT SAFETY</title>
        <p>In 2011, the Ministry of Health launched the Health Care Network Training and Quality
          Improvement Program, the QualiSUS Rede. <xref ref-type="fn" rid="fn13">
            <sup>e</sup>
          </xref> Although this was an important step forward in the Brazilian Unified Health System
          (SUS) quality development, the project did not include incentives to adopt a hospital
          accreditation program, nor a strategic objective directly linked to patient safety,
          something the IOM and WHO considered essential to quality.</p>
        <p>Another Ministry of Health initiative was to monitor the SUS performance rate (IDSUS),
          which aims to measure the health care system’s performance with regards access – potential
          and actual – and the effectiveness of primary health care, outpatient and hospital care
          and emergency care on a national level. <xref ref-type="fn" rid="fn14">
            <sup>f</sup>
          </xref> This measurement is achieved through quality indicators.</p>
        <p>None of the indicators established in the IDSUS are directly linked to patient safety,
          such as the rate of incidents in emergency care. On the other hand, in the IDSUS, the
          indicators are dealt with as the percentage of in-hospital deaths from acute myocardial
          infarction, which calculates the risk of dying from this condition after admission for
          this reason, and thus indirectly estimate delays in pre-hospital care and in diagnosis.
            <xref ref-type="fn" rid="fn12">
            <sup>d</sup>
          </xref> The use of this indicator in the Brazilian government’s program can be viewed as a
          step forward, albeit a small one.</p>
        <p>The project envisages a differentiated allocation of funds for those regions which reach
          higher levels of quality. This type of program has been successfully carried out in other
          countries. In England and the United States, e.g., in addition to sharing safety
          indicators between institutions in the country, with the aim of recognizing and
          establishing quality and safety levels in hospitals, those which achieve the highest
          levels are compensated with differentiated remuneration. <xref ref-type="bibr" rid="B3">
            <sup>3</sup>
          </xref>
          <sup>,</sup>
          <xref ref-type="bibr" rid="B9">
            <sup>9</sup>
          </xref>
        </p>
        <p>This model of payment by quality is known as pay for performance (P4P) <xref
            ref-type="bibr" rid="B7">
            <sup>7</sup>
          </xref> and is an alternative to the fee-for-service system widely used in Brazil, which
          promotes the overuse of resources, especially health care technologies, and gives no
          guarantee that the additional cost and ease of access result in effective improvements in
          the level of health care provided to the population cared for. <xref ref-type="bibr"
            rid="B7">
            <sup>7</sup>
          </xref>
        </p>
        <p>O P4P is being developed in many countries, including Brazil. In England, the model for
          using this system, the payments make up 30% of the income of some clinics. <xref
            ref-type="bibr" rid="B7">
            <sup>7</sup>
          </xref>
          <sup>,</sup>
          <xref ref-type="bibr" rid="B12">
            <sup>12</sup>
          </xref> It is expected that P4P leads to the service users themselves choosing the service
          they want to treat them, based on publicly reported performance indicators. Constance et
          al <xref ref-type="bibr" rid="B6">
            <sup>6</sup>
          </xref> showed that publishing those reports is an effective mechanism for improving
          health care quality.</p>
        <p>Brazil still faces the challenges of high turnover of health care professionals in public
          services, as well as qualitative limitations of human resources, inappropriate use of
          technology and poor continuity of care provided to patients. <xref ref-type="bibr"
            rid="B1">
            <sup>1</sup>
          </xref>
          <sup>,</sup>
          <xref ref-type="bibr" rid="B20">
            <sup>20</sup>
          </xref> Even so, a small number of hospitals in Brazil are dedicated to teaching and
          researching and do not influence improvements in health care practices due to the
          disconnection between teaching, research and health care, the poor use of evidence based
          health care in patient care and the fact that research into patient safety is limited to
          centers of excellence. <xref ref-type="bibr" rid="B1">
            <sup>1</sup>
          </xref>
          <sup>,</sup>
          <xref ref-type="bibr" rid="B16">
            <sup>16</sup>
          </xref>
          <sup>,</sup>
          <xref ref-type="bibr" rid="B20">
            <sup>20</sup>
          </xref>
        </p>
        <p>Faced with the situation exposed in the report, in which policies implanted by the
          Ministry of Health proved to be insufficient to stimulate critical examination of patient
          safety, establishing specific goals to prevent avoidable harm and minimize the risk of
          incidents, it was proposed that a national patient safety program, linked to federal
          government quality programs, be set up. Such a program should involve the Ministry of
          Health, ANVISA, <italic>Agência Nacional de Saúde Suplementar</italic> (ANS – National
          Supplementary Health Agency) and the Ministry of Education, the latter being an important
          ally in training health care professionals, especially in teaching hospitals.</p>
        <p>The National Patient Safety Program is necessary as it is in line with modern perceptions
          of quaternary prevention health care, which aims to detect individuals at risk of
          excessive health care interventionism, which implies unnecessary actions and suggests
          ethically acceptable alternatives, attenuating or avoiding adverse effects. <xref
            ref-type="bibr" rid="B2">
            <sup>2</sup>
          </xref>
          <sup>,</sup>
          <xref ref-type="bibr" rid="B19">
            <sup>19</sup>
          </xref>
        </p>
        <p>This approach is especially important in Brazil, which has experienced exponential growth
          in the new technologies available in the health care market over the last ten years,
          especially after the establishment of ANVISA, and has a very recent evidence-based legal
          framework on the incorporation of technology <xref ref-type="fn" rid="fn15">
            <sup>g</sup>
          </xref> and is investing in a humanized health care model. <xref ref-type="bibr" rid="B1">
            <sup>1</sup>
          </xref>
        </p>
        <p>The Patient Safety Program should be disseminated in the various institutions which make
          up the health care system, in all of the states, with the aim of obtaining and sharing
          knowledge of health care results, including negative results. Therefore, establishing a
          national incident notification system should be a priority of a national patient safety
          program which includes, at the very least, goals for managing the risks involved in health
          care, such as correctly identifying patients, reducing hospital infections, reducing
          errors in surgical procedures and medication, which are included in the WHO nine solutions
          for patient safety. <xref ref-type="fn" rid="fn16">
            <sup>h</sup>
          </xref>
        </p>
        <sec>
          <title>Incident notification system</title>
          <p>Notifications by health care professionals, patients and their carers are important in
            identifying incidents in health care, it is a low cost method, and it involves
            professionals providing health care in a policy of patient-centered continuous
            improvement.</p>
          <p>In order to guarantee that health care institutions produce data for decision making
            and taking responsibility for improvements in quality, investment in developing local
            capabilities and already existing information systems is essential. <xref
              ref-type="bibr" rid="B10">
              <sup>10</sup>
            </xref>
          </p>
          <p>The experience of ANVISA with the Sentry Network is a good example of how cooperation
            between the government and health care institutions is possible and can promote the
            development of safety and quality services, either by a culture of voluntary reporting
            or by adhering to quality programs.</p>
          <p>The number of notifications from hospitals belonging to the Sentry Network to the
              <italic>Sistema de Notificações em Vigilância Sanitária</italic> (NOTIVISA – Health
            Monitoring Notification System) increased by 48.8% in the first year after the system
            was established, compared to the previous year. ANVISA stimulated the hospitals in the
            network to participate in quality and accreditation programs and 30% of these hospitals
            were participating in programs of this type in 2008. <xref ref-type="fn" rid="fn11">
              <sup>c</sup>
            </xref>
          </p>
          <p>Although the ANVISA initiative was important in encouraging quality in hospitals, it
            affected only a small percentage of the more than 8,000 Brazilian hospital institutions,
            corresponding to around 13% of hospital beds in the country. <xref ref-type="fn"
              rid="fn11">
              <sup>c</sup>
            </xref>
          </p>
          <p>For this reason, the role played by the government bodies that receive the data on
            health care results is essential, as it is their responsability to encourage
            improvements at short-term with the aim of avoiding harm coming to the patients. Using
            an online computerized systems, i.e., data can be sent and received instantly, is an
            essential step for a country the size of Brazil to develop a national patient safety
            program. Moreover, a Brazilian pay per performance model must be established to benefit
            those institutions that are committed to the continuous quality improvement model.</p>
          <p>The model for the national incident notification system could be useful in developing a
            culture of patient safety within the SUS, as shown in the <xref ref-type="fig" rid="f02"
              >Figure</xref> .</p>
          <p>
            <fig id="f02">
              <label>Figure</label>
              <caption>
                <title>Simplified flowchart of the National Health Care Incident Notification
                  System.</title>
              </caption>
              <graphic xlink:href="0034-8910-rsp-47-04-0791-gf01-en"/>
            </fig>
          </p>
          <p>Developing and establishing a single computerized system receiving notifications from
            all health care institutions should aim to facilitate the process of sending these
            notifications and making decisions based on them, to minimize risks and avoid adverse
            events, to improve the quality of care and increase patient safety at all levels, from
            the smallest health care clinic to the Brazilian public health system; to increase
            knowledge of the risks and incidents which occur in Brazilian institutions, to guide
            health care managers’ planning and actions; to improve the quality of data sent; to
            ensure the clarity of the information sent; to preserve confidentiality of those who
            report and of the data reported; and, finally, to reduce the cost of the notification
            process.</p>
          <p>As the system becomes more frequently and efficiently used for the voluntary
            notification of incidents, especially those that are potentially adverse, making
            decisions as to interventions necessary for avoiding harm will be quicker, which could
            reduce the unnecessary costs of treating adverse events which could have been
            avoided.</p>
          <p>Therefore, autonomy and being pro-active should be encouraged in health care
            institutions. While waiting for the government to take action, institutions themselves
            should also work on internal improvements, aiming to encourage patient safety and
            quality health care. Thus, health care establishments should not only have access to a
            computerized system for sending notifications of risk management to the national system
            but also for management to receive notifications from their institutions health care
            team. Moreover, the system should allow the institution to follow the progress of
            analysis of the data sent to the national system.</p>
          <p>The computerized system is an important strategy in encouraging quality together with
            sustainability as it obviates the need for paper, reduces spending on materials and
            reduces waste such as paper, ink cartridges and pens. In addition, other aspects justify
            the establishing of such a system, these being: <xref ref-type="bibr" rid="B5">
              <sup>5</sup>
            </xref>
          </p>
          <list list-type="bullet">
            <list-item>
              <p>it eliminates the institutions’ need for systems for sending internal documents and
                data to the national notification system, which reduces the time needed for
                information to arrive and the cost of sending them;</p>
            </list-item>
            <list-item>
              <p>it may prevent information being misplaced or lost, especially if it is backed up
                in a database with a security copies, avoiding the need for physical storage space,
                as well as making the data easier to handle and to analyze for management
                indicators;</p>
            </list-item>
            <list-item>
              <p>it is possible to request further information on incidents without making data
                collection more difficult, improving the quality of information and increasing
                participation on the part of health care professionals, which is not possible when
                using a paper system.</p>
            </list-item>
          </list>
          <p>As for social aspects of sustainability, it reduces the time spent on sending reports,
            which increases health care professionals’ participation in the notifications as well as
            increasing the time available for them to spend with patients, thus becoming more
            involved with patients and their carers in the process of monitoring risks and incidents
            in health care. These individuals are important sources of voluntary notification in a
            national policy. An online computerized system of notifications would make it possible
            for anyone with internet access to report an incident.</p>
          <p>Countries that already have a national patient safety policy, such as England, The
            United States, Australia and Canada allow service users and their carers to report risks
            and incidents they have experienced or observed in the health care system, and these are
            essential to encouraging quality health care.</p>
          <p>The national notification system should be hosted on an interactive site that provides
            free notifications, safety tips, information on adverse events, protocols on how to
            implement safety programs in the health care system, online courses and seminars, such
            as the Institute for Health Improvement for hospitals in the US. This portal would have
            the dual function of maintaining and encouraging adherence in institutions and
            disseminating and encouraging the adoption of best practice in patient safety through
            exchanges between themselves.</p>
          <p>Establishing a computerized notification system for health care incidents in the
            Brazilian Health Care System, based on a culture of patient safety, appears to be a
            viable and necessary strategy to characterize health care, through which managers will
            be systematically made aware of incidents occurring in the health care provided to
            service users in public and private institutions, without having to depend on research
            carried out exclusively to this purpose. Thus, guidance will be provided in outlining
            risk management strategies for patient safety, increasing the quality of services
            supplied to the Brazilian population.</p>

        </sec>
      </sec>
    </body>
    <back>
      <app-group>
        <p>HIGHLIGHTS</p>
        <p>The Ministry of Health recently launched the <italic>Programa Nacional de Segurança do
            Paciente</italic> (National Patient Safety Program) so that patient safety activities
          would be promoted within the Brazilian Unified Health System (SUS). This is a praiseworthy
          initiative and was the main focus of the article, which dealt with important issues that
          were not included in the abovementioned program. These issues refer to the way in which
          the Government aims to remunerate those institutions which obtain the best results for
          providing services and also to how data derived from notifications are treated, so as to
          promote the knowledge generated and to improve the services provided by the SUS.</p>
        <p>The article sheds light on improvements in health care policies for patient safety, in
          which it is possible to see that activities involved in the culture of safety also
          proliferate and generate good results in public hospitals. Thus, attention is drawn to
          important discussions for all institutions and especially for managers of the system so as
          to improve the national program and trigger continuous improvements in patient-centered
          services.</p>
        <p>Profa. Rita de Cássia Barradas Barata </p>
        <p>Scientific Editor</p>
      </app-group>
      <fn-group>
        <fn id="fn9">
          <label>a</label>
          <p>World Health Organization. APPS web-based registration mechanism open. Geneva; 2012
            [cited 2012 jun 2]. Available from: http://www.who.int/patientsafety/en/  </p>
        </fn>
        <fn id="fn10">
          <label>b</label>
          <p>Capucho HC. Sistemas manuscrito e informatizado de notificação voluntária de incidentes
            em saúde como base para a cultura de segurança do paciente [tese de doutorado]. São
            Paulo: Escola de Enfermagem de Ribeirão Preto da USP; 2012.  </p>
        </fn>
        <fn id="fn11">
          <label>c</label>
          <p>Ministério da Saúde. Portaria GM/MS nº 529, de 1 de abril de 2013. Institui o Programa
            Nacional de Segurança do Paciente (PNSP). <italic>Diario Oficial Uniao</italic> . 2 abr
            2013;Seção1:43-4.  </p>
        </fn>
        <fn id="fn12">
          <label>d</label>
          <p>Petramale CA. O projeto dos hospitais sentinela e a gerência de risco sanitário
            hospitalar. In: Capucho HC, Carvalho FD, Cassiani SHB. Farmacovigilância - Gerenciamento
            de Riscos da Terapia Medicamentosa para a Segurança do Paciente. São Caetano do Sul:
            Editora Yendis. 2001. p. 191-224.  </p>
        </fn>
        <fn id="fn13">
          <label>e</label>
          <p>Ministério da Saúde. Portaria nº 396, de 4 de março de 2011 Institui o projeto de
            formação e melhoria da qualidade de rede de saúde (Quali-SUS-Rede) e suas diretrizes
            operacionais gerais. <italic>Diario Oficial Uniao</italic> . 9 mar 2011.  </p>
        </fn>
        <fn id="fn14">
          <label>f</label>
          <p>Ministério da Saúde. Índice de Desempenho do SUS – IDSUS. Brasília (DF); 2011 [cited
            2012 jun 2]. Available from:
            http://portal.saude.gov.br/portal/saude/area.cfm?id_area=1080</p>
        </fn>
        <fn id="fn15">
          <label>g</label>
          <p>Brasil. Lei nº 12.401, de 28 de abril de 2011. Altera a Lei nº 8.080, de 19 de setembro
            de 1990, para dispor sobre a assistência terapêutica e a incorporação de tecnologia em
            saúde no âmbito do Sistema Único de Saúde - SUS. <italic>Diario Oficial Uniao</italic> .
            29 abr 2011:1.  </p>
        </fn>
        <fn id="fn16">
          <label>h</label>
          <p>World Health Organization. WHO launches ‘Nine patient safety solutions. Geneva; 2007
            [cited 2012 jun 2]. Available from:
            http://www.who.int/mediacentre/news/releases/2007/pr22/en/index.html</p>
        </fn>
      </fn-group>
    </back>
  </sub-article>
</article>
