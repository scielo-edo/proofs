% Generated by jats2tex@0.11.1.0
\documentclass{article}
\usepackage[T1]{fontenc}
\usepackage[utf8]{inputenc} %% *
\usepackage[portuges,spanish,english,german,italian,russian]{babel} %% *
\usepackage{amstext}
\usepackage{authblk}
\usepackage{unicode-math}
\usepackage{multirow}
\usepackage{graphicx}
\usepackage{etoolbox}
\usepackage{xtab}
\usepackage{enumerate}
\usepackage{hyperref}
\usepackage{penalidades}
\usepackage[footnotesize,bf,hang]{caption}
\usepackage[nodayofweek,level]{datetime}
\usepackage[top=0.85in,left=2.75in,footskip=0.75in]{geometry}
\newlength\savedwidth
\newcommand\thickcline[1]{\noalign{\global
\savedwidth
\arrayrulewidth
\global\arrayrulewidth 2pt}
\cline{#1}
\noalign{\vskip\arrayrulewidth}
\noalign{\global\arrayrulewidth\savedwidth}}
\newcommand\thickhline{\noalign{\global
\savedwidth\arrayrulewidth
\global\arrayrulewidth 2pt}
\hline
\noalign{\global\arrayrulewidth\savedwidth}}
\usepackage{lastpage,fancyhdr}
\usepackage{epstopdf}
\pagestyle{myheadings}
\pagestyle{fancy}
\fancyhf{}
\setlength{\headheight}{27.023pt}
\lhead{\includegraphics[width=10mm]{logo.png}}
\rhead{\ifdef{\journaltitle}{\journaltitle}{}
\ifdef{\volume}{vol.\,\volume}{}
\ifdef{\issue}{(\issue)}{}
\ifdef{\fpage}{\fpage--\lpage\,pp.}}
\rfoot{\thepage/\pageref{LastPage}}
\renewcommand{\footrule}{\hrule height 2pt \vspace{2mm}}
\fancyheadoffset[L]{2.25in}
\fancyfootoffset[L]{2.25in}
\lfoot{\sf \ifdef{\articledoi}{\articledoi}{}}
\setmainfont{Linux Libertine O}
\renewcommand*{\thefootnote}{\alph{footnote}}
\makeatletter
\newcommand{\fn}{\afterassignment\fn@aux\count0=}
\newcommand{\fn@aux}{\csname fn\the\count0\endcsname}
\makeatother

\newcommand{\journalid}{Interface (Botucatu)}
\newcommand{\publisherid}{icse}
\newcommand{\journaltitle}{Interface - Comunicação, Saúde, Educação}
\newcommand{\abbrevjournaltitle}{Interface}
\newcommand{\issnppub}{1414-3283}
\newcommand{\issnepub}{1807-5762}
\newcommand{\publishername}{Laboratório de Educação e Comunicação em Saúde,
Departamento de Saúde Pública, Faculdade de Medicina de Botucatu e Instituto de
Biociências de Botucatu - \textsc{unesp}}
\newcommand\articleid{1807-57622014.0427}
\newcommand\articledoi{\textsc{doi} 10.1590/1807-57622014.0427}
\def\subject{Espaço Aberto}\newcommand{\subtitlestyle}[1]{-- \emph{#1}\medskip}
\newcommand{\transtitlestyle}[1]{\par\medskip\Large #1}
\newcommand{\transsubtitlestyle}[1]{-- \Large\emph{ #1}}

\newcommand{\titlegroup}{
\ifdef{\subtitle}{\subtitlestyle{\subtitle}}{}
\ifdef{\transtitle}{\transtitlestyle{\transtitle}}{}
\ifdef{\transsubtitle}{\transsubtitlestyle{\transsubtitle}}{}}

\title{Lições aprendidas na avaliação de um programa brasileiro de atenção a
idosos vítimas de violência\titlegroup{}}
\newcommand{\transtitle}{Lecciones aprendidas en la evaluación de un programa
brasileño de atención a ancianos víctimas de la violencia}
\author[{a}]{Minayo, Maria Cecília de Souza}
\author[{b}]{Souza, Edinilsa Ramos de}
\author[{c}]{Ribeiro, Adalgisa Peixoto}
\author[{d}]{Figueiredo, Ana Elisa Bastos}
\affil[ad]{Fundação Oswaldo Cruz}
\def\authornotes{
Colaboradores: Os autores participaram, igualmente, de todas as etapas de
elaboração do artigo.}
\date{ 2015}
\def\volume{19}
\def\issue{52}
\def\fpage{171}
\def\lpage{182}
\def\permissions{This is an Open Access article distributed under the terms of
the Creative Commons Attribution Non-Commercial License, which permits
unrestricted non-commercial use, distribution, and reproduction in any medium,
provided the original work is properly cited.}
\newcommand{\kwdgroup}{Idoso, Violência contra o idoso, Direitos do idoso,
Prevenção de violência, Avaliação}
\newcommand{\kwdgroupes}{Anciano, Violencia contra el anciano, Derechos del
anciano, Prevención de violencia, Evaluación}

\begin{document}
\selectlanguage{portuges}
\section*{Metadados não aplicados}
\begin{itemize}
\item[\textbf{língua do artigo}]{Português}
\ifdef{\journalid}{\item[\textbf{journalid}] \journalid}{}
\ifdef{\journaltitle}{\item[\textbf{journaltitle}] \journaltitle}{}

\ifdef{\journalsubtitle}{\item[\textbf{journalsubtitle}] \journaltitle}{}
\ifdef{\transjournaltitle}{\item[\textbf{journaltitle}] \journaltitle}{}
\ifdef{\transjournalsubtitle}{\item[\textbf{journalsubtitle}] \journaltitle}{}

\ifdef{\abbrevjournaltitle}{\item[\textbf{abbrevjournaltitle}]
\abbrevjournaltitle}{}
\ifdef{\issnppub}{\item[\textbf{issnppub}] \issnppub}{}
\ifdef{\issnepub}{\item[\textbf{issnepub}] \issnepub}{}
\ifdef{\publishername}{\item[\textbf{publishername}] \publishername}{}
\ifdef{\publisherid}{\item[\textbf{publisherid}] \publisherid}{}
\ifdef{\subject}{\item[\textbf{subject}] \subject}{}
\ifdef{\transtitle}{\item[\textbf{transtitle}] \transtitle}{}
\ifdef{\authornotes}{\item[\textbf{authornotes}] \authornotes}{}
\ifdef{\articleid}{\item[\textbf{articleid}] \articleid}{}
\ifdef{\articledoi}{\item[\textbf{articledoi}] \articledoi}{}
\ifdef{\volume}{\item[\textbf{volume}] \volume}{}
\ifdef{\issue}{\item[\textbf{issue}] \issue}{}
\ifdef{\fpage}{\item[\textbf{fpage}] \fpage}{}
\ifdef{\lpage}{\item[\textbf{lpage}] \lpage}{}
\ifdef{\permissions}{\item[\textbf{permissions}] \permissions}{}
\end{itemize}
\maketitle

\begingroup

\begin{abstract}

Apresentam-se “lições aprendidas” no processo de avaliação e monitoramento dos
centros de prevenção de violência contra os idosos, programa criado em 2007 pela
Secretaria de Direitos Humanos da Presidência da República (\textsc{sdh}), cuja proposta
se baseia em atenção multidisciplinar oferecida por profissionais de direito,
saúde e assistência social. Fez-se uma pesquisa avaliativa do programa durante
três anos, utilizando-se triangulação de métodos quantitativos e qualitativos.
As lições aprendidas provêm da visão dos diferentes atores que compartilharam a
experiência, inclusive os idosos. A eficiência e a efetividade das ações
mostraram a importância da iniciativa. Mas aqui se apontam, também, seus
limites, particularmente problemas de sustentabilidade pela falta de
investimento dos gestores e pela descontinuidade do apoio e da orientação da
\textsc{sdh}. Em 2013, dos 18 projetos implantados, seis haviam encerrado suas
atividades, 12 continuavam ativos, desses, dois expandiram suas ações.

\iflanguage{portuges}{\medskip\noindent\textbf{Palavras-chave:} \kwdgroup}{}
\iflanguage{english}{\medskip\noindent\textbf{Keywords:} \kwdgroupen}{}
\iflanguage{spanish}{\medskip\noindent\textbf{Palavras claves:} \kwdgroupes}{}
\iflanguage{french}{\medskip\noindent\textbf{Mots clés:} \kwdgroupfr}{}
\end{abstract}
\endgroup

\begingroup
\renewcommand{\section}[1]{\subsection*{#1}}
\begin{otherlanguage}{spanish}

\begin{abstract}

Se presentan las “lecciones aprendidas” en el proceso de evaluación y monitoreo
de los centros de prevención de violencia contra los ancianos, programa creado
en 2007 por la Secretaría de Derechos Humanos de la Presidencia de la República
(\textsc{sdh}), cuya propuesta se basa en la atención multidisciplinaria ofrecida por
profesionales de derecho, salud y asistencia social. Se hizo una encuesta de
evaluación del programa durante tres años, utilizándose la triangulación de
métodos cuantitativos y cualitativos. Las lecciones aprendidas tienen origen en
la visión de los diferentes actores que compartieron la
experiencia. La eficiencia y efectividad de las acciones mostraron la
importancia de la iniciativa. Pero aquí se señalan también sus límites,
particularmente problemas de sostenibilidad por la falta de inversión de los
gestores y por la discontinuidad del apoyo y de la orientación de la \textsc{sdh}. En
2013, de los 18 proyectos implantados, seis habían cerrado sus actividades, 12
continuaban activos siendo que de ellos, dos ampliaron sus acciones.

\ifdef{\kwdgroupes}{\medskip\noindent\textbf{Palavras claves:} \kwdgroupes}{}
\end{abstract}
\end{otherlanguage}
\endgroup
\section{Introdução}

“Lições aprendidas” é uma expressão que, atualmente, titula um dos capítulos da
maioria dos trabalhos de avaliação de políticas
sociais\textsuperscript{[}\textsuperscript{1}\textsuperscript{]}\textsuperscript{,}\textsuperscript{[}\textsuperscript{2}\textsuperscript{]}. Nesse capítulo, se encontram os resultados positivos e negativos provenientes
do desenvolvimento dos projetos ou programas, evidenciando os ensinamentos que a
partir dele possam ser generalizados\textsuperscript{[}\textsuperscript{2}\textsuperscript{]}.

Neste artigo, focalizamos esse tema a partir da avaliação e monitoramento de 18
Centros Integrados de Atenção e Prevenção à Violência contra a Pessoa Idosa
(\textsc{ciapvi}), os quais seguimos passo a passo durante três anos (2007 a 2009), na
implantação, implementação e nos resultados das ações.

Na avaliação de programas sociais, é comum a referência às dimensões desejáveis
de desempenho traduzidas em termos de eficiência, eficácia e efetividade. A
eficiência diz respeito à competência no uso dos instrumentos adequados para se
alcançarem os resultados com menor dispêndio de recursos e esforços. A eficácia
pode ser entendida como o alcance de resultados desejados em condições
controladas\textsuperscript{[}\textsuperscript{3}\textsuperscript{]}
; e a efetividade se refere à obtenção dos resultados pretendidos.

Trabalhamos com os conceitos de monitoramento e de
avaliação\textsuperscript{[}\textsuperscript{4}\textsuperscript{]}\textsuperscript{-}\textsuperscript{[}\textsuperscript{10}\textsuperscript{]}. O monitoramento pode ser interpretado como etapa ou componente do processo de
avaliação. Monitora-se a eficiência na implantação e no desenvolvimento de uma
proposta e, também, a eficácia das ações e a efetividade dos resultados. O
monitoramento é realizado por meio de indicadores que mostram se determinada
proposta ocorre conforme o desejado ou se há necessidade de mudança de rumos no
seu desenvolvimento. Geralmente, esse acompanhamento é registrado em relatórios
curtos e diretos contendo: análise da situação; dados sobre os investimentos
feitos; identificação dos problemas e da busca de soluções; garantia de que as
atividades previstas estejam sendo executadas corretamente; e informações sobre
problemas no andamento da totalidade ou parte da
proposta\textsuperscript{[}\textsuperscript{7}\textsuperscript{]}.

Uma avaliação completa incorpora o ciclo inteiro de análise de uma intervenção,
começando antes que o processo se inicie (ex-ante), passando pelo acompanhamento
(monitoramento) da implantação, da implementação, e pela análise dos resultados
e efeitos produzidos (ex-post). Uma avaliação pode também focalizar apenas
determinado momento ou etapa da intervenção. Aqui focalizamos apenas o tópico
das lições aprendidas na implantação e na implementação dos \textsc{ciapvi}. No entanto,
mesmo quando se analisa apenas parte do processo, é preciso evidenciar a origem,
a história, os objetivos da ação e o contexto em que ela se
desenvolve\textsuperscript{[}\textsuperscript{7}\textsuperscript{]}\textsuperscript{-}\textsuperscript{[}\textsuperscript{9}\textsuperscript{]}. Apresentamos os ensinamentos que a experiência dos \textsc{ciapvi} traz para a
aplicação de políticas públicas semelhantes, precedidos de breve descrição da
proposta, seus objetivos e a metodologia de sua construção.

\section{A proposta}

Os \textsc{ciapvi} foram criados em 2007 pela Secretaria de Direitos Humanos, da
Presidência da República do Brasil (\textsc{sdh}), de forma cooperativa com prefeituras,
governos estaduais e instituições não governamentais. Foram instituídos como uma
estratégia de ação prevista no Plano Nacional de Enfrentamento da Violência e
Maus-Tratos contra a Pessoa Idosa\textsuperscript{[}\textsuperscript{11}\textsuperscript{]}.

Esses centros funcionam com uma equipe multiprofissional composta por
assistentes sociais, psicólogos e advogados, capacitados para atender à pessoa
idosa. Suas atividades priorizam: orientação sobre direitos humanos ao idoso,
vítima de violências; estudo e diagnóstico psicossocial das situações
demandadas; mediação de conflitos familiares e comunitários, visando a
sensibilização social sobre o envelhecimento e as violências; promoção de grupos
de convivência; atendimento ao agressor; encaminhamento dos casos específicos de
maus-tratos, abandono ou negligências aos serviços especializados, como
Defensoria Pública, Ministério Público, da Saúde, delegacias de polícia,
instituições de assistência social, entre outras; internação do idoso em
Instituições de Longa Permanência para idosos; notificação dos casos nas
delegacias especializadas de atendimento ao idoso, no Ministério Público ou na
Defensoria Pública; e investimento na formação de técnicos para lidar com essas
questões.

Entre 2007 e 2009 foram criados 18 \textsc{ciapvi} no país: oito na região Nordeste, três
na região Norte; quatro na Sudeste; dois na Sul e um na região Centro-Oeste. A
maioria deles não tem infraestrutura própria, estando sediados em secretarias de
estado de Justiça, de Segurança Pública, de Direitos Humanos e Cidadania e
defensorias públicas, de assistência e desenvolvimento social e de defesa de
direitos.

A avaliação foi construída a partir da compreensão dos objetivos e da missão dos
centros. Os parâmetros para condução desse processo foram: as Deliberações da I
Conferência Nacional dos Direitos da Pessoa
Idosa\textsuperscript{[}\textsuperscript{12}\textsuperscript{]}, o Estatuto do Idoso\textsuperscript{[}\textsuperscript{13}\textsuperscript{]}
e as diretrizes do Plano de Ação para o Enfrentamento da Violência Contra a
Pessoa Idosa, do quadriênio 2007/2010\textsuperscript{[}\textsuperscript{11}\textsuperscript{]}. Esses documentos apresentam: o diagnóstico dos principais problemas de
violência vivenciados pelos idosos e as diretrizes para a organização, o
controle, o acompanhamento e a avaliação das ações a serem realizadas pelo poder
público.

\section{Método de realização do monitoramento e da avaliação}

O estudo avaliativo baseou-se na triangulação de métodos quantitativos e
qualitativos\textsuperscript{[}\textsuperscript{14}\textsuperscript{]}.

O processo de monitoramento e avaliação foi paulatinamente construído a partir
de várias oficinas participativas e intersetoriais de trabalho, envolvendo todos
os participantes desse programa social. Esse procedimento corresponde ao que há
de mais atual em avaliação de políticas sociais, denominado ‘avaliação de quarta
geração’\textsuperscript{[}\textsuperscript{15}\textsuperscript{]}\textsuperscript{-}\textsuperscript{[}\textsuperscript{17}\textsuperscript{]}. O pressuposto dessa abordagem é que qualquer processo avaliativo é mais
eficiente e fidedigno quando pensado e construído em forma de aprendizagem
compartilhada que beneficia a todos, inclusive os pesquisadores.

Na primeira das oficinas, estabeleceram-se as estratégias para realização do
monitoramento e da avaliação, e articularam-se essas ações com assessoria à \textsc{sedh}
e às unidades locais para correção dos rumos no processo.

Nesta etapa, criaram-se instrumentos de caráter exploratório e gerencial para
acompanhar e analisar as ações de implantação e implementação dos serviços,
investigando-se: a história de implantação de cada centro; sua viabilidade
institucional e administrativa; a aceitação da proposta pelos gestores; a
compatibilidade entre a realidade, as políticas locais e o Plano de Ação para o
Enfrentamento da Violência contra a Pessoa Idosa; os obstáculos encontrados; as
estratégias de superação das dificuldades; a infraestrutura e os recursos
humanos disponíveis. Um formulário com questões abertas e fechadas, discutido
com os coordenadores dos centros, foi enviado por meio eletrônico para ser
preenchido pelos gestores e sua equipe. Os dados obtidos foram analisados
segundo cada unidade e para o conjunto delas.

Em duas outras oficinas, uma em 2007 e outra em 2008, foram pactuados os
indicadores quantitativos e qualitativos de avaliação, por meio da técnica
‘grupo nominal’\textsuperscript{[}\textsuperscript{18}\textsuperscript{]}\textsuperscript{,}\textsuperscript{[}\textsuperscript{19}\textsuperscript{]}. Nessas discussões, foram consensualizados indicadores: de estrutura:
sustentabilidade, estabilidade da equipe e formação continuada; de processo:
articulação, referência e contrarreferência, parcerias, redes e inclusão; e de
resultados: atividades de prevenção, atendimentos, casos resolvidos e casos
reincidentes\textsuperscript{[}\textsuperscript{4}\textsuperscript{]}\textsuperscript{,}\textsuperscript{[}\textsuperscript{8}\textsuperscript{]}.

Os resultados do monitoramento foram sendo publicados na página de um
Observatório em forma de boletins quadrimestrais, durante os anos de 2008 a
2010, destacando as situações locais e comparando as unidades.

A realização da avaliação propriamente dita, por meio de instrumentos
quantitativos e qualitativos, abrangeu o processo de implantação e a
implementação dos centros. O “formulário de avaliação da implantação” foi
aplicado a cada um dos serviços existentes e aos que foram criados a partir de
2008. As respostas foram analisadas e os resultados discutidos presencialmente
com os coordenadores dos centros numa oficina de trabalho.

A avaliação da implementação foi realizada em duas etapas: em 2008, enfatizamos
o estudo dos êxitos e dificuldades no desenvolvimento dos trabalhos. Em 2009, os
resultados da ação e as expectativas futuras. Nesse processo, aplicamos três
formulários com questões abertas e fechadas e fizemos visitas a todos os
centros. Essas visitas tiveram como finalidade observar as atividades in loco e
entrevistar os vários atores envolvidos na proposta (gestores, profissionais,
parceiros e idosos). Ao todo, realizamos: 48 entrevistas individuais com
gestores e parceiros, 42 grupos focais com os profissionais e 14 com os idosos.

Na última etapa avaliativa, submetemos as análises que fizemos, sobre todo o
processo de monitoramento e avaliação, ao exame crítico dos principais atores
envolvidos, num seminário preparado para essa finalidade. Um relatório
preliminar, contendo a síntese dos resultados, foi enviado antecipadamente aos
coordenadores, profissionais dos centros, representantes da \textsc{sdh} e de
instituições parceiras. Após a validação desses atores, o relatório final com os
resultados foi concluído.

\section{Resultados}

Apesar de o foco do artigo ser sobre “lições aprendidas”, entendemos que será
útil ao leitor ter informações mínimas sobre os resultados da avaliação a partir
dos indicadores de produtividade e relevância dos centros. Em seguida,
discutimos o aprendizado que o processo de implantação e implementação traz para
a realização de programas semelhantes, destacando a relevância desse serviço, os
êxitos e as dificuldades, mas, sobretudo, os motivos que conduziram a um e a
outro.

\section{Indicadores de produtividade e relevância dos centros}

O Quadro 1 apresenta alguns indicadores de produtividade dos centros nos anos de
2008 e 2009.

Quadro 1. Indicadores avaliativos da produtividade dos Centros, 2008 e 2009
\begin{table}
\begin{xtabular}{ l | l | l | l | l | l | l | l | l | l }
\hline
Centros & Ano de implantação & N &\textsuperscript{o}
& ações de prevenção & N &\textsuperscript{o}
& atendimentos & \% casos resolvidos & \% casos reincidentes\\ \hline
São Luís/MA
& 2006
& 47
& 2.173
& 73,9
& 26,1
\\ \hline

Manaus/AM
& 2007
& 3
& 1.832
& 80,0
& 5,0
\\ \hline

Rio Branco/AC
& 2007
& 17
& 1.680
& 47,2
& 3,1
\\ \hline

São Cristóvão/SE
& 2007
& 38
& 470
& 24,0
& 0,2
\\ \hline

Maceió/AL
& 2007
& 40
& 54
& 35,2
& ...
\\ \hline

Natal/RN
& 2007
& 16
& 37
& 97,3
& 2,7
\\ \hline

Teresina/PI
& 2007
& 8
& 2.812
& 66,2
& 6,4
\\ \hline

Marília/SP
& 2007
& 2
& 235
& 17,0
& 6,4
\\ \hline

Florianópolis/SC
& 2007
& 5
& 1.822
& 13,5
& 1,3
\\ \hline

Palmas/TO
& 2008
& 9
& 1.362
& 33,7
& 0,9
\\ \hline

Recife/PE
& 2008
& 8
& 1.002
& 12,9
& 26,4
\\ \hline

Salvador/BA
& 2008
& 8
& 507
& 70,4
& 0,4
\\ \hline

Belo Horizonte/MG
& 2008
& 11
& 591
& 18,4
& ...
\\ \hline

Cornélio Procópio/PR
& 2008
& 15
& 230
& 70,9
& 2,6
\\ \hline

Campinas/SP
& 2008
& 1
& 2.432
& 26,8
& 4,9
\\ \hline

Rio de Janeiro/RJ
& 2009
& 5
& 540
& 86,0
& 0
\\ \hline

Fortaleza/CE
& 2009
& 31
& 696
& 23,4
& 0,9
\\ \hline

Goiânia/GO
& 2009
& 0
& 16
& 6,2
& ...
\\ \hline

Total
&
& 264
& 18.459
& 47,0
& 7,3
\\ \hline

\end{xtabular}
\end{table}

As ações de prevenção, que poderiam ser classificadas como de sensibilização e
tomada de consciência, geralmente, incluíram: grandes contingentes de população
em feiras, festividades cívicas e religiosas, escolas, programas de rádio,
televisão, artigos em jornais, seminários abertos, todas elas focadas no
processo de envelhecimento, na condição dos idosos e nas violências que sofrem.

Quanto ao número de atendimentos, é importante esclarecer que cada centro foi
criado em período diferente, talvez, por isso, os mais antigos, como São Luís,
Manaus, Rio Branco, Teresina e Florianópolis, tendam a mostrar índices de
eficiência maior. Entretanto, centros criados posteriormente, como Palmas,
Recife e Campinas, também apresentaram elevada produtividade. Outros fatores
influenciam nessa quantidade de atendimentos, como o tamanho da cidade (Cornélio
Procópio) e o fechamento dos centros durante o monitoramento (Natal, Belo
Horizonte).

Foram considerados “casos resolvidos” aqueles em que o idoso foi integralmente
atendido em sua necessidade. A resolutividade foi alta em 38,9\% das unidades.
No entanto, na maioria, ficou abaixo de 50\%. Os encaminhamentos constituíram as
maiores dificuldades para os profissionais. A rede pública de serviços não está
preparada para acolher a pessoa idosa e nem costuma dar retorno quando atende. O
idoso com dependências físicas, mentais e sociais e com necessidade de respostas
imediatas encontra maior dificuldade no
atendimento\textsuperscript{[}\textsuperscript{20}\textsuperscript{]}. Verificamos percentuais baixos de reincidência dos casos resolvidos pelos
centros, o que demonstra a efetividade dos atendimentos.

Os indicadores quantitativos de produtividade foram complementados pela análise
das percepções dos diferentes atores envolvidos no trabalho dos \textsc{ciapvi}. Ouvir as
pessoas sobre os programas nos quais estão envolvidas é importante, pois o que
elas pensam sobre a situação influencia seu comportamento.

Alguns gestores e profissionais destacaram que os idosos saem dos centros mais
tranquilos, cientes de seus direitos, se sentem protegidos, seguros, respaldados
e satisfeitos. Muitos disseram que o centro é um serviço que veio para ficar em
seus municípios, e ressaltam os ganhos conquistados por meio dele: a mobilização
da opinião pública e da mídia, a articulação da rede de proteção local e o
envolvimento da comunidade. Os representantes de instituições parceiras também
destacaram que, na maioria das vezes, o centro consegue atender à demanda com
respostas efetivas. Ressaltaram, ainda, a importância da participação ativa e
comprometida dos técnicos, muitos deles voluntários, e que o trabalho
multiprofissional tornou o grupo coeso e integrado.

Para os parceiros das \textsc{ong} e de outras instituições públicas, o serviço oferecido
pelos centros se tornou referência para a cidadania dos idosos e uma iniciativa
de vanguarda em relação ao tema, nas localidades em que foram criados.
Promotores de justiça, delegados e defensores públicos assinalaram que houve
redução nas queixas e denúncias que chegam a suas instituições, o que é
indicativo do êxito dos centros.

Os idosos ouvidos consideraram que receberam bom atendimento, pois os
profissionais os escutaram, deram encaminhamento correto e seus problemas foram
resolvidos. Comentaram que se sentem respeitados, à vontade e tranquilos após o
atendimento. No entanto, vários deles criticaram: a burocracia e filas nos
serviços para onde são encaminhados; o espaço físico dos centros, que,
frequentemente, ainda não é adequado, dificultando a privacidade dos
atendimentos, e a morosidade dos processos que endereçam à justiça. Por isso,
muitos idosos sugeriram melhorias para os centros, como: adequação do espaço
físico visando à privacidade dos atendimentos; abolição das filas para
atendimento nos serviços a que são referidos; aumento do número de visitas dos
técnicos às suas casas quando há denúncia de violência; penalidades mais duras
para seus agressores; justiça mais ágil; estabelecimento dos centros em locais
acessíveis; e meio de transporte para que os idosos dependentes sejam atendidos
em casa ou levados para atendimento nos serviços.

\section{Lições aprendidas na implantação dos centros}

O momento de implantação é crucial porque dele se originará uma prática coletiva
eficiente ou não. Cada centro teve sua história particular de criação, mas os
pontos comuns ressaltados como facilitadores foram: (1) oficinas oferecidas pela
\textsc{sedh} para os profissionais que atuariam nos centros para discutir a viabilidade
do trabalho, criar linguagem comum e oferecer orientações para atuação de rotina
e frente a problemas mais complexos. Essa estratégia tornou possível: a
identificação e a sinergia das pessoas com a proposta, a obtenção de informações
precisas e objetivas para a implantação, garantias para a operacionalização e
discussão dos indicadores com os quais os serviços seriam avaliados; (2)
constituição de um grupo de profissionais e gestores dedicados, com formação
básica e experiência pessoal e institucional anterior para trabalhar com idosos;
(3) apoio financeiro e orientação técnica da \textsc{sedh} para prover o centro com
condições básicas de funcionamento nos locais onde esse trabalho foi pioneiro;
(4) acolhimento institucional de algum órgão que lhe deu abrigo, cobertura e
condições de atuação; (5) criação de infraestrutura básica para o projeto:
espaço físico adequado; equipamentos; acesso à internet e telefone, transporte
para visitas domiciliares e locomoção de idosos dependentes, e salas para
atendimento individual; (6) localização do centro em lugar de fácil acesso para
os idosos que utilizam transporte público; (7) autonomia do coordenador para
adaptar e organizar o espaço físico e para selecionar sua equipe; (8) construção
de parcerias ativas com a rede de serviços de saúde, assistência social e
segurança, instituições não governamentais e voluntariado.

Dentre os fatores que dificultaram a implantação de alguns centros, gestores e
técnicos destacaram: (1) escassez de recursos financeiros para realizar a
quantidade de ações previstas, pois algumas unidades dependiam exclusivamente do
subsídio da \textsc{sdh}; (2) problemas burocráticos que provocaram morosidade nos
processos licitatórios para aquisição de materiais e equipamentos, contratação
de técnicos e de empresas envolvidas nas ações; (3) problemas na gestão do órgão
ao qual a unidade estava vinculada: troca de chefias, de gestores dos próprios
centros, tornando descontínuos os atendimentos e encaminhamentos; (4)
inexistência, precariedade ou inadequação de serviços e programas de suporte na
rede de proteção para os quais os idosos deveriam ser encaminhados. É o caso de
centros de cuidados diurnos e de vagas em Instituições de Longa Permanência para
Idosos (\textsc{ilpi}) com elevado grau de dependência; (5) problemas de integração entre
os profissionais das distintas áreas (direito, saúde, assistência social,
segurança e transporte) e dos órgãos que deveriam atuar em rede; (6)
dificuldades de sensibilização da sociedade para as questões do envelhecimento e
da violência contra a pessoa idosa.

\section{Lições aprendidas na implementação dos centros}

Apesar dos problemas desafiantes de se criar um trabalho multiprofissional, o
desenvolvimento das atividades nos \textsc{ciapvi} mostrou conquistas importantes,
consensualmente valorizadas por gestores, profissionais, parceiros e pelos
próprios idosos.

Mostrou ser altamente eficiente: (1) ter foco na missão; (2) ter um grupo de
profissionais capacitados, motivados, coesos e comprometidos com o tema; (3)
insistir no atendimento multidisciplinar incluindo apoio jurídico, psicológico e
social; (4) fazer parcerias com instituições públicas responsáveis pelo
atendimento (como secretarias de saúde, de assistência social, de transporte, de
mobilidade urbana) ou com instituições apoiadoras, como: universidades,
conselhos de direitos, Ministério Público, Ministério da Justiça, Secretaria da
Justiça, Rede Nacional de Atendimento dos Direitos do Idoso e grupos de
voluntariado; (5) oferecer atendimento resolutivo, personalizado e direto, que
significou acolher e escutar o idoso, realizar os encaminhamentos necessários,
envolver a família na solução dos problemas e reforçar seus vínculos por meio da
mediação compreensiva de seus conflitos; (6) acompanhar as atividades do centro
com cursos que promovessem o bom nível dos profissionais e a orientação dos
familiares; (7) acompanhar o desenvolvimento do centro com a produção e
divulgação de materiais informativos para sensibilização da sociedade; (8)
integrar e fortalecer fóruns em defesa dos direitos da pessoa idosa; (9) quando
necessário, descentralizar atividades para municípios e bairros com grande
número de idosos; (10) contar com apoio político e gerencial e aporte financeiro
do governo local.

Esse decálogo poderia ser completado com a constatação da importância de contar
com a participação dos idosos e de suas famílias no desenvolvimento das ações.
Isso ajudou na adequação das atividades dos centros e permitiu a redução da
judicialização dos conflitos que costumam protagonizar ou dos quais são vítimas.

Foram apontados os problemas que emperraram o trabalho. Alguns são de ordem
institucional, outros de gestão de pessoal e do trabalho, e outros se referem ao
desempenho das atividades cotidianas.

Do ponto de vista institucional, as principais falhas foram: (1) descontinuidade
dos convênios com a \textsc{sdh} que orientava as ações e financiava os centros. Em
função dessa interrupção, algumas unidades encerraram definitivamente suas
atividades, pois não conseguiram ser assumidas pelos governos locais. Outras
continuam existindo, mas perderam o foco multiprofissional e passaram a atuar em
uma lógica setorial, desconfigurando a proposta inicial. (2) Em alguns \textsc{ciapvi},
persistiram problemas: de infraestrutura (falta de espaço físico adequado à
privacidade dos atendimentos); de equipamentos imprescindíveis, como telefone,
transporte para idosos dependentes e para realização de visitas domiciliares; e
de pessoal, persistindo quadro de profissionais reduzido, insuficiente e de
elevada rotatividade; (3) em várias unidades, a excessiva lentidão burocrática
dos órgãos a que estavam subordinadas prejudicou a realização de ações
previstas, a contratação e pagamento de profissionais, e licitações para compra
de materiais e equipamentos.

Do ponto de vista da gestão dos centros, foram assinalados: (1) em alguns
locais, evidente desarticulação da rede de apoio e insuficiência de serviços que
deveriam garantir atenção integral aos idosos; (2) falhas no sistema de
referência e contrarreferência, comprometendo a resolutividade dos casos de
violência; (3) inexistência ou insuficiência de vagas em \textsc{ilpi}, deixando, sem
apoio, idosos dependentes em extrema condição de pobreza ou que não tinham
familiares cuidadores à disposição; (4) dificuldades de inclusão dos idosos e
dos abusadores com transtornos psiquiátricos na rede de atendimento de saúde
mental; (5) falta de formação dos profissionais para abordar idosos e familiares
com transtornos psiquiátricos ou envolvidos em atos criminosos; (6) ausência de
instâncias para trabalhar com os agressores; (7) morosidade no processo de
interdição e abrigamento compulsório para idosos com demência e sem cuidadores
adequados; (8) inexistência de bancos de dados unificados da rede de proteção.

Do ponto de vista das equipes e suas atividades, os profissionais mencionaram
como problemáticas: (1) inexperiência para lidar com a violência contra a pessoa
idosa; (2) descontinuidade nos processos de contratação e capacitação para atuar
nos Centros; (3) e rotatividade dos profissionais e estagiários.

Sobre os idosos e suas famílias, os profissionais destacaram: (1) resistência da
pessoa idosa em relatar a violência que sofre, seja porque o abusador é um
familiar com o qual terá de conviver quando regressar para casa, ou por temor de
possível desfecho policial ou jurídico-penal da situação que narrar; (2)
comprometimento mental e dependência química de alguns idosos e de seus
abusadores; (3) frequente omissão, falta de compreensão e de compromisso de
familiares em relação às necessidades de seus parentes idosos.

\section{Discussão}

Há evidência de acertos e dificuldades no desenvolvimento dos \textsc{ciapvi} que foram
concebidos como uma estratégia para construir uma base de conhecimentos e
práticas para lidar com a violência contra a pessoa idosa, garantindo seus
direitos. Ressalta-se que, embora falemos da pessoa idosa de forma genérica, na
verdade, os maiores frequentadores dessas unidades são pessoas pobres e que têm
algum grau de dificuldade de se defender por si só. É em relação a esse grupo
específico que as instituições públicas e a sociedade brasileira vêm
demonstrando maior dificuldade em atuar protetivamente. Mas esse não é um
problema apenas brasileiro, pois, nos documentos do Conselho
Europeu\textsuperscript{[}\textsuperscript{21}\textsuperscript{]}, que reúne o pensamento da Comunidade Europeia, o foco são os idosos
dependentes, considerados os mais vulneráveis. Na Europa, pretende-se criar um
fundo específico para que esse grupo social seja contemplado, uma vez que os
serviços a ele destinados ora são mantidos por órgãos públicos ora por
organizações não governamentais. O principal desafio tem sido equilibrar as
responsabilidades da família e do Estado nos cuidados aos idosos e ampliar a
oferta e o acesso aos serviços especializados, assim como proteger e aliviar o
desgaste dos cuidadores familiares.

Se os \textsc{ciapvi} pareciam apontar um caminho de possibilidades no sentido citado
acima, seu papel se esmaeceu. Como soe acontecer no serviço público brasileiro,
as mudanças na gestão da \textsc{sdh} estabeleceram outras prioridades, tornando esses
centros apenas uma experiência a mais. Em levantamento realizado em 2013,
constatamos que, dos 18 avaliados, seis encerraram suas atividades em função da
falta de recursos financeiros e do apoio da gestão local. Dos 12 restantes, nove
ainda funcionavam conforme o modelo multiprofissional preconizado originalmente.
Os demais continuavam oferecendo seus serviços com foco apenas setorial. Dois
centros (Rio de Janeiro, São Luís), ao contrário, ampliaram suas atividades
dentro da filosofia proposta: o primeiro criou uma ouvidoria e implantou uma
unidade em Niterói. O segundo criou um atendimento móvel que contempla a
periferia de São Luís.

A proposta de criação dos \textsc{ciapvi} foi pensada como uma dentre várias outras
estratégias do Plano Nacional de Enfrentamento da Violência contra a Pessoa
Idosa. Nesse sentido, ela é parte de um todo e, como tal, ela cumpriu sua
missão. Foi com o intuito de generalizar essa estratégia que a \textsc{sedh} contratou a
pesquisa de monitoramento e avaliação para acompanhar o percurso do experimento.

A pergunta que aqui cabe é se iniciativas como os \textsc{ciapvi} têm sentido se não
forem verdadeiramente institucionalizadas. A experiência internacional mostra
que cada país vai encontrando formas de lidar com os problemas gerados pelo
envelhecimento da população. Entretanto, observamos que a questão dos idosos
dependentes e as violências que acometem os velhos em geral, continuam a ser
temas sobre os quais se preocupam governantes, profissionais de saúde,
pesquisadores e, sobretudo, familiares. Os \textsc{ciapvi}, inicialmente criados como uma
experiência-piloto de qualidade, eficiência e baixo custo, com pretensão de
estender-se nacionalmente, poderiam constituir um exemplo para os países que
estão buscando respostas aos problemas mencionados.

Artigos que avaliam propostas de prevenção da violência contra a pessoa idosa
mostram que um programa efetivo deve atender a cada idoso e abranger os vários
componentes de sua vida, como saúde, educação, interações sociais e condições de
vida, provendo serviços relevantes que diminuam os seus riscos de violência.
Também deve incluir intervenções como campanhas na mídia e estratégias de
mudanças que possam melhorar sua condição de
vida\textsuperscript{[}\textsuperscript{22}\textsuperscript{]}\textsuperscript{-}\textsuperscript{[}\textsuperscript{25}\textsuperscript{]}. Tais objetivos coincidem com os da proposta original dos \textsc{ciapvi}, embora não se
tenha encontrado nenhuma iniciativa igual. Os Estados Unidos, por exemplo,
criaram, em 2007, o “Nacional Center of Elder Abuse” (\textsc{ncea}) dentro do
“Department of Health and Human Services”, para assegurar recursos para
pesquisa, aplicação de políticas sociais, de saúde, de justiça; para formação de
cuidadores; para defesa de direitos dos idosos e atendimento aos familiares
necessitados. Esse centro atua em colaboração com os Estados e com iniciativas
locais\textsuperscript{[}\textsuperscript{26}\textsuperscript{]}. No Canadá, existe: um Departamento do Idoso no Sistema Público de Saúde; uma
Rede Nacional de Cuidados; um Centro de Direito dos Idosos e uma Rede de
Prevenção de Abusos\textsuperscript{[}\textsuperscript{27}\textsuperscript{]}.

\section{Considerações finais}

Do estudo desse caso avaliado, aprendemos algumas lições:

A primeira delas é que, quando se faz um arranjo estratégico novo como
dispositivo para o desenvolvimento de uma política pública, a instituição que o
cria precisa ter foco, objetivos claros, trabalhar coletivamente com todos os
atores que o implementarão ou que serão destinatários da iniciativa. No entanto,
não basta isso, é preciso ter recursos financeiros, orientação técnica e
previsão de sustentabilidade. No caso dos \textsc{ciapvi}, houve grande investimento
inicial da \textsc{sdh}, seguido por descontinuidade.

Concluímos que os 66\% dos centros que persistem têm apoio local e caminham por
conta própria. Mas a meta da \textsc{sdh}, de implantar, pelo menos, um em cada estado,
não foi atingida. Entendemos, portanto, que o papel orientador e indutor dessa
secretaria não deveria ter cessado até que a proposta fosse inteiramente
implantada.

Um dos problemas cruciais relativos à continuidade dos centros foi que a
proposta de sua criação nunca foi consensual na \textsc{sdh} e nos governos locais.
Durante o processo de avaliação, percebeu-se, ao contrário, muita resistência
interna e opiniões ambivalentes quanto a assumir esses arranjos e a
incorporá-los às instâncias locais. Observamos que, na prática, criar
estratégias para o cuidado do idoso não é prioridade para alguns governantes e
para parte da sociedade brasileira, a não ser para as famílias.

Ressalta-se a importância, caso a proposta seja retomada, de garantir, aos
centros, suas características iniciais multiprofissionais e de atuação em rede.
Os resultados conseguidos por meio de uma pequena equipe formada por psicólogo,
advogado e assistente social, evidenciados nos indicadores, por si só falam da
relevância e adequação desse formato.

O engajamento dos técnicos e gestores foi um ponto alto do sucesso dos centros.
Ao final do processo avaliativo, muitos coordenadores e profissionais
ressaltaram que, apesar de ser um projeto recente, o esforço e o empenho dos
funcionários permitiram que os serviços se transformassem em referência e fossem
potencializadores de ações efetivas e significativas.

Foi fundamental o papel dos centros na dinamização de uma rede
interinstitucional de serviços necessários e imprescindíveis, como os de saúde e
de justiça que, infelizmente, constatou-se serem os de mais difícil acesso e os
que apresentam demora e falta de qualidade no atendimento. Ressalta-se a
insuficiência de oferta de atenção especializada para idosos dependentes e com
problemas graves de saúde física e de acesso aos serviços de saúde mental.

Quatro problemas ligados ao cotidiano dos centros põem em risco o trabalho:
falta de garantia de institucionalidade e sustentabilidade pelos órgãos locais
em colaboração com a \textsc{sdh}; algumas iniciativas em curso, de tornar o serviço
apenas setorial; inadequado e insuficiente financiamento para garantir as
atividades; e falta de investimento em pessoas competentes e sua substituição
por funcionários inexperientes.

\begin{enumerate}
\item
A primeira delas é que, quando se faz um arranjo estratégico novo como
dispositivo para o desenvolvimento de uma política pública, a instituição que o
cria precisa ter foco, objetivos claros, trabalhar coletivamente com todos os
atores que o implementarão ou que serão destinatários da iniciativa. No entanto,
não basta isso, é preciso ter recursos financeiros, orientação técnica e
previsão de sustentabilidade. No caso dos \textsc{ciapvi}, houve grande investimento
inicial da \textsc{sdh}, seguido por descontinuidade.

\item
Concluímos que os 66\% dos centros que persistem têm apoio local e caminham por
conta própria. Mas a meta da \textsc{sdh}, de implantar, pelo menos, um em cada estado,
não foi atingida. Entendemos, portanto, que o papel orientador e indutor dessa
secretaria não deveria ter cessado até que a proposta fosse inteiramente
implantada.

\item
Um dos problemas cruciais relativos à continuidade dos centros foi que a
proposta de sua criação nunca foi consensual na \textsc{sdh} e nos governos locais.
Durante o processo de avaliação, percebeu-se, ao contrário, muita resistência
interna e opiniões ambivalentes quanto a assumir esses arranjos e a
incorporá-los às instâncias locais. Observamos que, na prática, criar
estratégias para o cuidado do idoso não é prioridade para alguns governantes e
para parte da sociedade brasileira, a não ser para as famílias.

\item
Ressalta-se a importância, caso a proposta seja retomada, de garantir, aos
centros, suas características iniciais multiprofissionais e de atuação em rede.
Os resultados conseguidos por meio de uma pequena equipe formada por psicólogo,
advogado e assistente social, evidenciados nos indicadores, por si só falam da
relevância e adequação desse formato.

\item
O engajamento dos técnicos e gestores foi um ponto alto do sucesso dos centros.
Ao final do processo avaliativo, muitos coordenadores e profissionais
ressaltaram que, apesar de ser um projeto recente, o esforço e o empenho dos
funcionários permitiram que os serviços se transformassem em referência e fossem
potencializadores de ações efetivas e significativas.

\item
Foi fundamental o papel dos centros na dinamização de uma rede
interinstitucional de serviços necessários e imprescindíveis, como os de saúde e
de justiça que, infelizmente, constatou-se serem os de mais difícil acesso e os
que apresentam demora e falta de qualidade no atendimento. Ressalta-se a
insuficiência de oferta de atenção especializada para idosos dependentes e com
problemas graves de saúde física e de acesso aos serviços de saúde mental.

\item
Quatro problemas ligados ao cotidiano dos centros põem em risco o trabalho:
falta de garantia de institucionalidade e sustentabilidade pelos órgãos locais
em colaboração com a \textsc{sdh}; algumas iniciativas em curso, de tornar o serviço
apenas setorial; inadequado e insuficiente financiamento para garantir as
atividades; e falta de investimento em pessoas competentes e sua substituição
por funcionários inexperientes.

\end{enumerate}

\section*{Referências}
\begin{itemize}

\item[1] The \textsc{dac} Network on Development. Evaluation. Joint evaluations: recent
experiences, lessons learned and options for the future [Internet] [acesso 2014
Abr 4]. Disponível em: http://www.oecd.org/dac/evaluation/dcdndep/35353699.pdf

\item[2] Fundo das Nações Unidas para a Infância (\textsc{unicef}). Innovations, lessons
learned and good practices [Internet] [acesso 2013 Abr 4]. Disponível em:
http://www.unicef.org/innovations

\item[3] Marinho A, Façanha LO. Programas sociais: efetividade, eficiência e
eficácia como dimensões operacionais da avaliação. Texto para Discussão nº 787
[Internet]. Rio de Janeiro: Instituto de Pesquisa Econômica Aplicada; 2001
[acesso 2013 Abr 5]. Disponível em:
http://www.ipea.gov.br/portal/images/stories/\textsc{pdf}s/TDs/td\_{}0787.pdf

\item[4] Tamaki EM, Tanaka OU, Felisberto E, Alves \textsc{cka}, Drumond Junior M,
Bezerra \textsc{lca}, et al. Metodologia de construção de um painel de indicadores para o
monitoramento e a avaliação da gestão do \textsc{sus}. Cienc Saude Colet. 2012;
17(4):839-49.

\item[5] Carvalho \textsc{alb}, Souza MF, Shimizu HE, Senra \textsc{imvb}, Oliveira KC. A gestão
do \textsc{sus} e as práticas de monitoramento e avaliação: possibilidades e desafios
para a construção de uma agenda estratégica. Cienc Saude Colet. 2012;
17(4):901-11.

\item[6] Miranda AS, Carvalho \textsc{alb}, Cavalcante \textsc{cgcs}. Subsídios sobre práticas de
monitoramento e avaliação sobre gestão governamental em Secretarias Municipais
de Saúde. Cienc Saude Colet. 2012; 17(4):913-20.

\item[7] Minayo \textsc{mcs}. Importância da avaliação qualitativa combinada com outras
modalidades de avaliação. Saude Transform Soc. 2011; 1(3):2-11.

\item[8] Contandriopoulos AP. Avaliando a institucionalização da pesquisa. Cienc
Saude Colet. 2006; 11(3):705-11.

\item[9] Knaap P. Theory-based evaluation and learning: possibilities and
challenges. Evaluation. 2004; 10(1):16-34.

\item[10] Weiss CH. Evaluation, methods for studying programs and policies. New
Jersey: Prentice Hall; 1998.

\item[11] Secretaria de Direitos Humanos da Presidência da República. Plano de
ação para enfrentamento da violência contra a pessoa idosa (2007-2010). Brasília
(DF): \textsc{sdh}; 2007.

\item[12] Secretaria de Direitos Humanos da Presidência da República. Relatório
da Primeira Conferência Nacional dos Direitos da Pessoa Idosa. Construindo a
Rede Nacional de Proteção e Defesa da Pessoa Idosa. Brasília: \textsc{sdh}; 2006
[Internet] [acesso 2014 Mar 12]. Disponível em: www.sdh.gov.br

\item[13] Lei nº 10.741, de 01 de outubro de 2003. Dispõe sobre o Estatuto do
Idoso e dá outras providências. Estatuto do Idoso. Diário Oficial da União. 3
Out 2003.

\item[14] Minayo \textsc{mcs}, Assis SG, Souza ER, organizadoras. Avaliação por
triangulação de métodos. Rio de Janeiro: Fiocruz; 2005.

\item[15] Kantorski LP, Wetzel C, Olschowsky A, Jardim \textsc{vmr}, Bielemann \textsc{vlm},
Schneider JF. Avaliação de quarta geração: contribuições metodológicas para
avaliação de serviços de saúde mental. Interface (Botucatu). 2009;
13(31):343-55.

\item[16] Guba EG, Lincoln YS. Fourth generation evaluation. J Adv Nurs. 2000;
31(1):117-25.

\item[17] Koch T. Having a say: negotiation in fourth-generation evaluation. Adv
Nurs. 2000; 31(1):125-31.

\item[18] Lancaster T, Hart R, Gardner S. Literature and medicine: evaluating a
special study module using the nominal group technique. Med Educ. 2002;
36(11):1071-86.

\item[19] Gallangher M, Hares T, Spencer J, Bradshaw C, Webb I. The nominal
group technique: a research tool for general practice. Fam Pract. 1993;
10(1):76-91.

\item[20] Lima-Costa MF, Matos DL, Camarano AA. Evolução das desigualdades
sociais em saúde entre idosos e adultos brasileiros: um estudo baseado na
Pesquisa Nacional por Amostra de Domicílios (\textsc{pnad} 1998, 2003). Cienc Saude
Colet. 2006; 11(4):941-50.

\item[21] Garcia AM. La protección social en la Unión Europea, un modelo
homogéneo? \textsc{ice} Rev Econ [Internet]. 2005 [acesso 2014 Abr 25]; 820:1-27.
Disponível em: http://www.revistasice.com/Cache\textsc{pdf}/\textsc{ice}\_{}820\_{}195219\_{}\_{}9
282FA86374DE5D092216C7C85F66803.pdf

\item[22] Nation M, Crusto C, Wandersman A, Kumpfer KL, Seybolt D,
Morrissey-kane E, et al. What works in prevention: principles of effective
prevention programs. Am Psychol. 2003; 58(6-7):449-56.

\item[23] Gates S, Ficher JD, Cooke MW, Lamb SE. Multifactorial assessment and
targeted intervention for preventing falls and injuries among older people in
community and emergency care settings: systematic review and meta-analysis. \textsc{bmj}.
2008; 336(7636):130-3.

\item[24] Chan KL. Comparative review on national strategies in the prevention
of domestic violence. Open Soc Sci J. 2011; 4:1-8.

\item[25] World Health Organization. Missing voices: views of older persons on
elder abuse. Genève: World Health Organization; 2002.

\item[26] National Center on elder abuse: administration on aging. Department of
Health and Human Services [Internet] [acesso 2014 Maio 20]. Disponível em:
http://www.ncea.aoa.gov/index.aspx

\item[27] Réseau canadien pour la prévention des mauvais traitements envers les
aîné(e)s. Canadian Network for prevention on elder abuse [Internet] [acesso 2014
Maio 30]. Disponível em: http://www.cnpea.ca/ageism.pdf

\end{itemize}

\end{document}
