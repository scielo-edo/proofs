% Generated by jats2tex@0.11.1.0
\documentclass{article}
\usepackage[T1]{fontenc}
\usepackage[utf8]{inputenc} %% *
\usepackage[portuges,spanish,english,german,italian,russian]{babel} %% *
\usepackage{amstext}
\usepackage{authblk}
\usepackage{unicode-math}
\usepackage{multirow}
\usepackage{graphicx}
\usepackage{etoolbox}
\usepackage{xtab}
\usepackage{enumerate}
\usepackage{hyperref}
\usepackage{penalidades}
\usepackage[footnotesize,bf,hang]{caption}
\usepackage[nodayofweek,level]{datetime}
\usepackage[top=0.85in,left=2.75in,footskip=0.75in]{geometry}
\newlength\savedwidth
\newcommand\thickcline[1]{\noalign{\global
\savedwidth
\arrayrulewidth
\global\arrayrulewidth 2pt}
\cline{#1}
\noalign{\vskip\arrayrulewidth}
\noalign{\global\arrayrulewidth\savedwidth}}
\newcommand\thickhline{\noalign{\global
\savedwidth\arrayrulewidth
\global\arrayrulewidth 2pt}
\hline
\noalign{\global\arrayrulewidth\savedwidth}}
\usepackage{lastpage,fancyhdr}
\usepackage{epstopdf}
\pagestyle{myheadings}
\pagestyle{fancy}
\fancyhf{}
\setlength{\headheight}{27.023pt}
\lhead{\includegraphics[width=10mm]{logo.png}}
\rhead{\ifdef{\journaltitle}{\journaltitle}{}
\ifdef{\volume}{vol.\,\volume}{}
\ifdef{\issue}{(\issue)}{}
\ifdef{\fpage}{\fpage--\lpage\,pp.}}
\rfoot{\thepage/\pageref{LastPage}}
\renewcommand{\footrule}{\hrule height 2pt \vspace{2mm}}
\fancyheadoffset[L]{2.25in}
\fancyfootoffset[L]{2.25in}
\lfoot{\sf \ifdef{\articledoi}{\articledoi}{}}
\setmainfont{Linux Libertine O}
\renewcommand*{\thefootnote}{\alph{footnote}}
\makeatletter
\newcommand{\fn}{\afterassignment\fn@aux\count0=}
\newcommand{\fn@aux}{\csname fn\the\count0\endcsname}
\makeatother

\newcommand{\journalid}{Interface (Botucatu)}
\newcommand{\publisherid}{icse}
\newcommand{\journaltitle}{Interface - Comunicação, Saúde, Educação}
\newcommand{\abbrevjournaltitle}{Interface}
\newcommand{\issnppub}{1414-3283}
\newcommand{\issnepub}{1807-5762}
\newcommand{\publishername}{Laboratório de Educação e Comunicação em Saúde,
Departamento de Saúde Pública, Faculdade de Medicina de Botucatu e Instituto de
Biociências de Botucatu - \textsc{unesp}}
\newcommand\articleid{1807-57622014.0628}
\newcommand\articledoi{\textsc{doi} 10.1590/1807-57622014.0628}
\def\subject{Artigos}\newcommand{\subtitlestyle}[1]{-- \emph{#1}\medskip}
\newcommand{\transtitlestyle}[1]{\par\medskip\Large #1}
\newcommand{\transsubtitlestyle}[1]{-- \Large\emph{ #1}}

\newcommand{\titlegroup}{
\ifdef{\subtitle}{\subtitlestyle{\subtitle}}{}
\ifdef{\transtitle}{\transtitlestyle{\transtitle}}{}
\ifdef{\transsubtitle}{\transsubtitlestyle{\transsubtitle}}{}}

\title{Leituras holísticas: de Tchékhov à Medicina Narrativa\footnote{\fn19}
\titlegroup{}}
\newcommand{\transtitle}{Lecturas holísticas: de Tchékhov a la Medicina
Narrativa}
\author[{a}]{Fernandes, Isabel}
\affil[a]{Universidade de Lisboa}
\date{ 2015}
\def\volume{19}
\def\issue{52}
\def\fpage{71}
\def\lpage{82}
\def\permissions{This is an Open Access article distributed under the terms of
the Creative Commons Attribution Non-Commercial License, which permits
unrestricted non-commercial use, distribution, and reproduction in any medium,
provided the original work is properly cited.}
\newcommand{\kwdgroup}{Palavras-chaveAnton Tchékhov, Medicina baseada na prova,
Medicina baseada na narrativa, Medicina narrativa}
\newcommand{\kwdgroupes}{Palabras-claveAnton Tchékhov, Medicina basada en
pruebas, Medicina basada en la narrativa, Medicina narrativa}
%%% Nota fn1 %%%%%%%%%%%%%%%%%%%%%%%%%%%%%%%%%%%%%%%%%%%%%%%%%%%%%%%%
\expandafter\newcommand\csname fn1\endcsname{
O conto foi também publicado no Brasil, com o título “Um caso clínico.” Segui a
tradução em português europeu\textsuperscript{[}\textsuperscript{1}\textsuperscript{]}. Sempre que esta se manifestou menos satisfatória, recorri à tradução inglesa
constante da antologia de Jack Coulehan, com o título “A Doctor’s
Visit”\textsuperscript{[}\textsuperscript{2}\textsuperscript{]}.}
%%% Nota fn2 %%%%%%%%%%%%%%%%%%%%%%%%%%%%%%%%%%%%%%%%%%%%%%%%%%%%%%%%
\expandafter\newcommand\csname fn2\endcsname{
Sobre a importância dos chamados “incipit,” veja-se, por exemplo, Andréa Del
Lungo\textsuperscript{[}\textsuperscript{3}\textsuperscript{]}.}
%%% Nota fn3 %%%%%%%%%%%%%%%%%%%%%%%%%%%%%%%%%%%%%%%%%%%%%%%%%%%%%%%%
\expandafter\newcommand\csname fn3\endcsname{
Sobre isto, uso os termos e conceitos utilizados na palestra “Bioética e
Humanização”, proferida por Diego Gracia, em 31 de março de 2014, no Congresso
Internacional Humanidades e Humanização em Saúde, promovido pela \textsc{fmusp}, em São
Paulo, Brasil. Qualquer incorreção ou deturpação de leitura é, porém, da minha
total responsabilidade.}
%%% Nota fn4 %%%%%%%%%%%%%%%%%%%%%%%%%%%%%%%%%%%%%%%%%%%%%%%%%%%%%%%%
\expandafter\newcommand\csname fn4\endcsname{
Faço-me eco da expressão sartriana: “être en situation,” no sentido de sermos o
resultado dum feixe de circunstâncias e pertenças mas não, em última instância,
determinados por elas.}
%%% Nota fn5 %%%%%%%%%%%%%%%%%%%%%%%%%%%%%%%%%%%%%%%%%%%%%%%%%%%%%%%%
\expandafter\newcommand\csname fn5\endcsname{
Estamos perante o artifício dialógico que F. K. Stanzel designa de
reflectorização e que origina o que chama de situação narrativa figural, em que
uma personagem é usada como reflectora\textsuperscript{[}\textsuperscript{5}\textsuperscript{]}.}
%%% Nota fn6 %%%%%%%%%%%%%%%%%%%%%%%%%%%%%%%%%%%%%%%%%%%%%%%%%%%%%%%%
\expandafter\newcommand\csname fn6\endcsname{
Cf. Stanzel, segundo o qual, a situação narrativa figural convida à empatia com
a personagem-reflectora ou detentora da
focalização\textsuperscript{[}\textsuperscript{5}\textsuperscript{]}.}
%%% Nota fn7 %%%%%%%%%%%%%%%%%%%%%%%%%%%%%%%%%%%%%%%%%%%%%%%%%%%%%%%%
\expandafter\newcommand\csname fn7\endcsname{
Dorrit Cohn usa o termo “quoted monologue” para designar este artifício
narrativo, em narrativas de terceira
pessoa\textsuperscript{[}\textsuperscript{6}\textsuperscript{]}.}
%%% Nota fn8 %%%%%%%%%%%%%%%%%%%%%%%%%%%%%%%%%%%%%%%%%%%%%%%%%%%%%%%%
\expandafter\newcommand\csname fn8\endcsname{
Veja-se uma obra como Soi même comme un autre, por
exemplo\textsuperscript{[}\textsuperscript{7}\textsuperscript{]}.}
%%% Nota fn9 %%%%%%%%%%%%%%%%%%%%%%%%%%%%%%%%%%%%%%%%%%%%%%%%%%%%%%%%
\expandafter\newcommand\csname fn9\endcsname{
Esta necessária interdependência e colaboração faz lembrar o célebre triângulo
hipocrático: “The [medical] Art consists of three factors, the disease, the
patient and the physician. The physician is the servant of the Art. The patient
must cooperate with the physician in combating the
disease”\textsuperscript{[}\textsuperscript{9}\textsuperscript{]}
(p. 1).}
%%% Nota fn10 %%%%%%%%%%%%%%%%%%%%%%%%%%%%%%%%%%%%%%%%%%%%%%%%%%%%%%%%
\expandafter\newcommand\csname fn10\endcsname{
A única personagem que, neste conto, tem a pretensão de usar uma linguagem
médico-científica é a governanta, que macaqueia o idioleto médico, procurando,
assim, representar o papel que julga se espera dela. Trata-se de mais uma
comparação, neste caso, irónica.}
%%% Nota fn11 %%%%%%%%%%%%%%%%%%%%%%%%%%%%%%%%%%%%%%%%%%%%%%%%%%%%%%%%
\expandafter\newcommand\csname fn11\endcsname{
As designações variam, com os \textsc{eua} privilegiando a designação “Narrative
Medicine” (Medicina Narrativa - MN), e outros países, como o Reino Unido,
preferindo a designação mais abrangente: “Medical Humanities” (Humanidades
Médicas), a que correspondem, em França, as “humanités médicales.” Daí, por
exemplo, o nome do programa pioneiro nos \textsc{eua}: Program in Narrative Medicine,
sediado no New York’s College of Physicians and Surgeons da Universidade de
Columbia.}
%%% Nota fn12 %%%%%%%%%%%%%%%%%%%%%%%%%%%%%%%%%%%%%%%%%%%%%%%%%%%%%%%%
\expandafter\newcommand\csname fn12\endcsname{
Preferimos traduzir “Evidence-based Medicine” por “Medicina baseada na prova,”
dado que a tradução corrente: “Medicina baseada na evidência” atraiçoa o sentido
do termo inglês “evidence,” o qual reclama, ao contrário do português, para ser
incontestado, ser antes provado.}
%%% Nota fn13 %%%%%%%%%%%%%%%%%%%%%%%%%%%%%%%%%%%%%%%%%%%%%%%%%%%%%%%%
\expandafter\newcommand\csname fn13\endcsname{
Note-se que já o aparecimento dos Grupos Balint na década de cinquenta do século
XX sinalizava a existência dum mal-estar e da necessidade de o colmatar:
reconhecendo que a consulta médica é, no essencial, um ato de relação, visavam
amparar o clínico, fazendo-o partilhar inter pares os dilemas e dificuldades da
sua relação com o doente concreto e singular. O objetivo seria o de fomentar uma
atitude mais empática e mais autoconsciente, abrindo caminho a uma relação de
confiança entre ambos. (Fonte:
http://www.apmgf.pt/ficheiros/Boletim\_{}Inscricao\_{}Balint.pdf)}
%%% Nota fn14 %%%%%%%%%%%%%%%%%%%%%%%%%%%%%%%%%%%%%%%%%%%%%%%%%%%%%%%%
\expandafter\newcommand\csname fn14\endcsname{
A expressão “medicina defensiva” é usada, a este propósito, por
Antunes\textsuperscript{[}\textsuperscript{10}\textsuperscript{]}
(p.24).}
%%% Nota fn15 %%%%%%%%%%%%%%%%%%%%%%%%%%%%%%%%%%%%%%%%%%%%%%%%%%%%%%%%
\expandafter\newcommand\csname fn15\endcsname{
A este respeito, além de Antunes\textsuperscript{[}\textsuperscript{10}\textsuperscript{]}, veja-se, também, David H. Newman\textsuperscript{[}\textsuperscript{14}\textsuperscript{]}\textsuperscript{,}\textsuperscript{[}\textsuperscript{15}\textsuperscript{]}.}
%%% Nota fn16 %%%%%%%%%%%%%%%%%%%%%%%%%%%%%%%%%%%%%%%%%%%%%%%%%%%%%%%%
\expandafter\newcommand\csname fn16\endcsname{
Cf. Aristóteles, Ética a Nicómaco\textsuperscript{[}\textsuperscript{21}\textsuperscript{]}. Agradeço ao Professor Diego Gracia a chamada de atenção para a pertinência
desta distinção aristotélica.}
%%% Nota fn17 %%%%%%%%%%%%%%%%%%%%%%%%%%%%%%%%%%%%%%%%%%%%%%%%%%%%%%%%
\expandafter\newcommand\csname fn17\endcsname{
O autor continua: “Politicamente, pensar global tornou-se mais uma necessidade
do que uma opção ideológica. As fronteiras já não protegem ninguém. A nossa
relação com a terra tornou-se mais completa e inevitável. O mesmo se passa com a
nossa saúde”\textsuperscript{[}\textsuperscript{24}\textsuperscript{]}
(p. 356-7, tradução minha).}
%%% Nota fn18 %%%%%%%%%%%%%%%%%%%%%%%%%%%%%%%%%%%%%%%%%%%%%%%%%%%%%%%%
\expandafter\newcommand\csname fn18\endcsname{
“More so than any other subject, literature lends itself to powerful
teaching”\textsuperscript{[}\textsuperscript{25}\textsuperscript{]}
(p. 74).}
%%% Nota fn19 %%%%%%%%%%%%%%%%%%%%%%%%%%%%%%%%%%%%%%%%%%%%%%%%%%%%%%%%
\expandafter\newcommand\csname fn19\endcsname{
O artigo apresentado resulta de investigação desenvolvida no âmbito do projecto
“Medicina \& Narrativa – (Con)textos e práticas interdisciplinares” [Ref.
\textsc{ptdc}/\textsc{cpc}-\textsc{elt}/3719/2012], do \textsc{ceaul}/\textsc{ulices} – Centro de Estudos Anglísticos da
Universidade de Lisboa, financiado pela \textsc{fct} (Fundação para a Ciência e a
Tecnologia).}

\begin{document}
\selectlanguage{portuges}
\section*{Metadados não aplicados}
\begin{itemize}
\item[\textbf{língua do artigo}]{Português}
\ifdef{\journalid}{\item[\textbf{journalid}] \journalid}{}
\ifdef{\journaltitle}{\item[\textbf{journaltitle}] \journaltitle}{}

\ifdef{\journalsubtitle}{\item[\textbf{journalsubtitle}] \journaltitle}{}
\ifdef{\transjournaltitle}{\item[\textbf{journaltitle}] \journaltitle}{}
\ifdef{\transjournalsubtitle}{\item[\textbf{journalsubtitle}] \journaltitle}{}

\ifdef{\abbrevjournaltitle}{\item[\textbf{abbrevjournaltitle}]
\abbrevjournaltitle}{}
\ifdef{\issnppub}{\item[\textbf{issnppub}] \issnppub}{}
\ifdef{\issnepub}{\item[\textbf{issnepub}] \issnepub}{}
\ifdef{\publishername}{\item[\textbf{publishername}] \publishername}{}
\ifdef{\publisherid}{\item[\textbf{publisherid}] \publisherid}{}
\ifdef{\subject}{\item[\textbf{subject}] \subject}{}
\ifdef{\transtitle}{\item[\textbf{transtitle}] \transtitle}{}
\ifdef{\authornotes}{\item[\textbf{authornotes}] \authornotes}{}
\ifdef{\articleid}{\item[\textbf{articleid}] \articleid}{}
\ifdef{\articledoi}{\item[\textbf{articledoi}] \articledoi}{}
\ifdef{\volume}{\item[\textbf{volume}] \volume}{}
\ifdef{\issue}{\item[\textbf{issue}] \issue}{}
\ifdef{\fpage}{\item[\textbf{fpage}] \fpage}{}
\ifdef{\lpage}{\item[\textbf{lpage}] \lpage}{}
\ifdef{\permissions}{\item[\textbf{permissions}] \permissions}{}
\end{itemize}
\maketitle

\begingroup
\renewcommand{\abstractname}{Resumo}
\begin{abstract}

Neste artigo, procura-se evidenciar como, no conto “Um caso da prática médica”,
o escritor russo, Anton Tchékhov, ele próprio médico, ficcionaliza e chama a
atenção para aspetos da prática clínica hoje tantas vezes descurados,
designadamente a observação e avaliação do ambiente em que vivem os doentes; e
aspetos de índole familiar, social e até sexual, bem como a importância crucial
da relação interpessoal de carácter dialógico entre médico e doente.
Relacionar-se-á esta chamada de atenção indireta de Tchékhov com a situação
atual no domínio da prática clínica, recorrendo-se ao conceito de Medicina
Baseada na Prova (\textsc{ebm}), como paradigma atualmente dominante nas práticas
médicas, e à necessidade de o complementar com o da Medicina Baseada na
Narrativa (\textsc{nbm}) ou Medicina Narrativa.

\iflanguage{portuges}{\medskip\noindent\textbf{Palavras-chave:} \kwdgroup}{}
\iflanguage{english}{\medskip\noindent\textbf{Keywords:} \kwdgroupen}{}
\iflanguage{spanish}{\medskip\noindent\textbf{Palavras claves:} \kwdgroupes}{}
\iflanguage{french}{\medskip\noindent\textbf{Mots clés:} \kwdgroupfr}{}
\end{abstract}
\endgroup

\begingroup
\renewcommand{\section}[1]{\subsection*{#1}}
\begin{otherlanguage}{spanish}
\renewcommand{\abstractname}{Resumen}
\begin{abstract}

En este artículo se busca mostrar como en el cuento “Un caso de práctica
médica”, el escritor ruso Anton Tchékhov, que era médico, presenta como ficción
y llama la atención para aspectos de la práctica clínica tantas veces
descuidados, específicamente la observación y la evaluación del ambiente en que
viven los enfermos y también sobre aspectos de índole familiar, social e incluso
sexual, así como la importancia crucial de la relación inter-personal del
carácter dialógico entre médico y paciente. Se relacionará esta llamada de
atención indirecta de Tchékhov con la situación actual en el dominio de la
práctica clínica, recurriéndose al concepto de \textsc{ebm} (Evidence based medicine)
como paradigma dominante en la actualidad en las prácticas médicas y la
necesidad de complementarlo con el de \textsc{nbm} (Narrative based medicine) o Medicina
narrativa.

\ifdef{\kwdgroupes}{\medskip\noindent\textbf{Palavras claves:} \kwdgroupes}{}
\end{abstract}
\end{otherlanguage}
\endgroup
Fundação para a Ciência e a Tecnologia\textsc{ptdc}/\textsc{cpc}-\textsc{elt}/3719/2012
Anton Tchékhov (1860-1904), escritor e médico russo, é autor de muitos contos
famosos, de entre os quais o texto intitulado “Um caso da prática médica,”
publicado em 1898, será objeto da primeira parte deste artigo\textsuperscript{(}\footnote{\fn1}\textsuperscript{)}.

Para benefício dos que possam não conhecer o texto, fornece-se uma breve
paráfrase. Um prestigiado médico de Moscovo recebe um telegrama solicitando-lhe
uma visita até a fábrica dos Liálikovs, para consultar a filha dos
proprietários, e resolve enviar o seu assistente, o Dr. Koroliov. Este não
consegue detetar nenhum problema sério de saúde na jovem Lisa – que se queixa de
nervosismo, insónias e palpitações, e, não vendo motivos para se demorar,
dispõe-se a partir, mas é instado a pernoitar na casa da família pela mãe, que
teme que se repita outra noite de pesadelo. Como não consegue dormir, o médico
vagueia de madrugada pela fábrica e zonas adjacentes, e tem como que uma
epifania, ao aperceber-se duma conexão entre o opressivo meio envolvente e a
doença da jovem. A perceção duma enfermidade social e espiritual que afeta
aquele microcosmos justifica, aos seus olhos, os sintomas da paciente e leva-o a
encará-la de modo diferente: da visão distanciada e profissional que lhe
dispensara antes e que de nada lhe valera, passa a uma atenção mais generosa à
pessoa e à situação de Lisa, com quem conversa. Agora, além dos batimentos
cardíacos, torna-se atento à expressão dos seus sentimentos, medos e ansiedades,
com os quais, de resto, se solidariza. Ela responde-lhe com a revelação de estar
também convencida de não ter nenhuma doença física, antes sentir “inquietação” e
“medo.” A conversa constitui o momento de viragem decisivo para a jovem, que, na
manhã seguinte, aparece a despedir-se do médico, sorridente e vestida de branco,
como que para uma festa.

Proponho que atentemos no/s parágrafo/s de abertura, geralmente cruciais para o
entendimento do que verdadeiramente está em causa no texto, e, também, à
identificação de qual/quais a/s figura/s de retórica mais
recorrente/s\textsuperscript{(}\footnote{\fn2}\textsuperscript{)}. Começando pela abertura da narrativa, verificamos que nos fala duma figura que
irá estar ausente do conto: o anónimo professor. E, de imediato, nos
perguntamos: por que incluir, neste início, uma personagem sem interferência no
desenrolar dos acontecimentos?
\begin{quote}

O professor recebeu um telegrama da fábrica dos Liálikov: pediam que fosse lá
com urgência. Adoecera a filha de uma tal senhora Lialikova, pelos vistos a
proprietária da fábrica, e mais nada se podia perceber do telegrama longo e
desconexo. Então, o professor não quis ir, mandando em seu lugar um dos seus
médicos, o doutor Koroliov.\textsuperscript{[}\textsuperscript{1}\textsuperscript{]}
(p.261)

\end{quote}

Aparentemente inconsequente para o desenvolvimento da história, a menção à
atitude do professor é, contudo, significativa, porque ele baseia a sua decisão
de não responder ao apelo que lhe é feito num ato interpretativo, num ato de
leitura. O facto de o telegrama ser “longo e desconexo” deprecia, aos seus
olhos, aqueles que o escreveram, e determina que se esquive à visita,
delegando-a no seu assistente. Chama-se aqui a atenção para o ato de leitura
como qualquer coisa que precede e acompanha as nossas ações e decisões, e que,
como veremos, assinala o tipo de saber e decisões próprios do ato clínico como
ato deliberativo, isto é, não meramente cognitivo, dado não se limitar a
equacionar factos, mas envolver também emoções e valores, bem como ponderar
deveres. Isto é, chama-se indiretamente a atenção para o carácter hermenêutico
complexo da prática médica, bem como da sua essência
deliberativa\textsuperscript{(}\footnote{\fn3}\textsuperscript{)}. Para que não restem dúvidas, mais abaixo, ao descrever o que Koroliov vê no
seu percurso até à fábrica, lê-se na tradução inglesa: “And now when the
workpeople timidly and respectfully made way for the carriage, in their faces,
their caps, their walk, he read physical impurity, drunkenness, nervous
exhaustion, bewilderment”\textsuperscript{[}\textsuperscript{2}\textsuperscript{]}
(p.174, ênfase minha).

Tchékhov está, obviamente, apostado em apresentar os médicos como leitores, como
descodificadores dos sistemas sígnicos que os rodeiam, e disso mesmo o texto irá
tratar.

Quanto ao recurso retórico central neste conto, e não apenas por ser o mais
repetido, mas por ser, ele próprio, transformado em princípio composicional da
narrativa, como adiante se demonstrará, torna-se inescapável a comparação.
Veja-se, por exemplo, logo no segundo parágrafo, a referência ao cocheiro da
carruagem que espera Koroliov: “o cocheiro tinha uma pena de pavão no chapéu e
respondia a todas as perguntas em voz alta, como um
soldado…”\textsuperscript{[}\textsuperscript{1}\textsuperscript{]}
(p.261, ênfase minha). Aqui, o primeiro contacto com um habitante local aponta,
por meio da comparação, para o mundo castrense e sua inerente disciplina e
relações de poder, aspetos que serão reverberados na apresentação da fábrica.

Ora, a essência da comparação (que, como se sabe, está na base de toda a
construção metafórica) consiste numa aproximação de objectos ou planos
diferentes, num esforço integrativo que, segundo Percy Bysshee Shelley, na sua
célebre Defesa da Poesia (escrita em 1821, mas publicada postumamente, em 1840),
caracteriza o próprio modo poético. A partir duma distinção entre razão e
imaginação, em que, segundo defende: “A razão diz respeito às diferenças e a
imaginação às semelhanças entre as coisas”, Shelley vai mais longe e afirma que
a poesia, sendo a expressão desse poder integrativo da imaginação, é uma
linguagem “vitalmente metafórica, ou seja, assinala as relações até aí
inapreendidas entre as coisas e perpetua essa
apreensão”\textsuperscript{[}\textsuperscript{4}\textsuperscript{]}
(p. 227, traduções minhas).

Que seja a comparação o recurso retórico mais flagrante e recorrente no conto é
significativo, dado que, por ele, se assinala a necessidade dum olhar e duma
atitude integradoras das forças e relações múltiplas que governam a nossa
existência enquanto seres humanos e, mais do que nunca, em momentos críticos
como o duma crise ou duma doença, como é o caso neste texto.

Além disso, em termos menos microscópicos, o conto vai progredindo por meio de
outras comparações ou aproximações, ainda que não de natureza
retórico-estilística mas, antes, de teor composicional. Constrói-se e
desdobra-se por aproximações sucessivas entre personagens e entre estas e o meio
ambiente. Desde logo, temos a comparação entre a governanta e os patrões,
seguida da que assimila Lisa à mãe, e, finalmente, a comparação entre o médico e
Lisa, neste último caso, unidos por um conspícuo e inesperado “nós,” que, não
sendo meramente sintoma duma sensibilidade geracional comum, vai mais fundo e
sinaliza a partilha da vulnerabilidade humana que une sempre, afinal, médico e
doente.

Se associarmos estes dois aspetos, o início do texto e o recurso retórico nele
central, poderemos arriscar dizer que Tchékhov estará interessado em propor-nos
um tipo de médico que designaremos de hermeneuta ou leitor competente, e um modo
de exercer a prática clínica que tenha em conta a pessoa do paciente, no seu
contexto doméstico, familiar e social; em suma, ao invés duma medicina que
disseque ou separe, propõe-nos uma medicina integrativa, restauradora duma
imagem global do paciente “em situação”\textsuperscript{(}\footnote{\fn4}\textsuperscript{)}.

Vejamos como o faz no conto em análise. Trata-se duma narrativa de narrador
heterodiegético omnisciente, vulgo de terceira pessoa, em que o ponto de vista é
cedido ao jovem Koroliov – é o seu olhar e a sua consciência do mundo que
acompanhamos\textsuperscript{(}\footnote{\fn5}\textsuperscript{)}. Tal escolha merece reparo – o convite ao leitor para partilhar o ponto de
vista do jovem clínico aponta no sentido de ilustrar um comportamento e uma
visão que se dá à consideração como digna de atenção, ou, por outras palavras,
convida à empatia entre o leitor e esta personagem\textsuperscript{(}\footnote{\fn6}\textsuperscript{)}.

O momento climático ocorre quando, de noite, vagueando, primeiro, no perímetro
fabril e, em seguida, nas imediações, o médico se dá conta dum ambiente hostil,
infernal, semelhante ao duma prisão – comparação inescapável que nos é
comunicada significativamente em monólogo citado\textsuperscript{(}\footnote{\fn7}\textsuperscript{)}, na sequência duma descrição em que os sentidos da audição e da visão da
personagem são convocados, testemunhando o seguinte cenário:
\begin{quote}

Ouviu-se, ao lado do terceiro edifício: “jac… jac… jac…” E o mesmo nos restantes
blocos fabris, e depois por trás das barracas e no portão. E parecia que, no
meio do silêncio nocturno, era o próprio monstro com olhos rubros quem produzia
estes sons, que era o próprio Diabo quem ali reinava sobre os patrões e os
operários, enganando estes e aqueles.

Koroliov saiu para o campo.

- Quem é? – interpelou-o uma voz grosseira. “Como numa prisão…”, pensou Koroliov
e não respondeu.\textsuperscript{[}\textsuperscript{1}\textsuperscript{]}
(p. 268)

\end{quote}

Mas não é só o entorno da fábrica que afeta negativamente o médico, também o
interior da casa da família Liálikov, de gosto duvidoso e com um pretensiosismo
novo-rico, tem o dom de o irritar. Além disso, as primeiras impressões que tem
de Lisa também induzem nele a intuição de que algo falha ao nível da sexualidade
da jovem, que o leva a pensar: “Deveria casar-se, está na
altura…”\textsuperscript{[}\textsuperscript{1}\textsuperscript{]}
(p. 264).

Apercebe-se, então, do sem-sentido dum empreendimento que a ninguém beneficia:
nem aos trabalhadores, duramente explorados e alienados, nem aos patrões, também
eles, afinal, vítimas do mesmo sistema mecanizado e insensível e do mesmo
ambiente insalubre, como se prova pela morte prematura do pai de Lisa e pelo
sofrimento presente de mãe e filha. A única e irónica exceção a este estado de
coisas irracional parece ser a governanta, Khristina Dmitrievna, que sabe tirar
partido da boa cama e boa mesa ao seu dispor.

O que se torna evidente ao longo do conto são as qualidades de Koroliov, no que
diz respeito às suas capacidades de observação, de audição, de interpretação e à
sua sensibilidade em geral. Trata-se de alguém de mente aberta, que sabe ouvir,
ver e descodificar o que contempla, e que é afetado pela ausência de beleza e
vida em seu redor.

O diagnóstico que faz do microcosmos industrializado que observa leva-o a uma
nova e eloquente comparação: tal como a vida fabril é para ele enigmática,
assim, também, o são certas doenças incuráveis, cujas causas se revelam obscuras
mas incontornáveis. Por mais distração e entretenimento que se proporcione aos
operários, tal não logrará curá-los da doença que os aflige, como acontece com
os tratamentos vãos que se aplicam em casos de doenças sem cura.

Mas se a avaliação e a diagnose do contexto social e familiar por parte do
clínico será certeira e decisiva para o que se segue, é o diálogo com Lisa, a
doente, que vai constituir a peripécia conducente ao desenlace. Este diálogo
evidencia a dependência mútua entre o eu e o outro, explorada, entre outros
pensadores, por um Paul Ricoeur\textsuperscript{(}\footnote{\fn8}\textsuperscript{)}
ou por um Mikhail Bakhtin. Este último, por exemplo, num passo eloquente de Art
and Answerability (obra porventura menos conhecida), chama a atenção para a
complementaridade de olhares que caracteriza qualquer encontro interpessoal e
sem a qual a apreensão do todo envolvente seria impossível:
\begin{quote}

Quando eu contemplo um ser humano completo que está situado fora de mim e face a
mim, os horizontes concretos por nós efectivamente experimentados não coincidem.
Porque, a cada momento dado, independentemente da posição e da proximidade a que
esse outro ser humano, a quem eu estou a contemplar, se encontra de mim, eu
hei-de ver e saber sempre algo que ele, a partir da posição que ocupa, fora de
mim e face a mim, não pode, ele próprio, ver: partes do seu corpo que são
incessíveis ao seu próprio olhar (a sua cabeça, a sua face e respectiva
expressão), o mundo atrás de si e toda uma série de objectos e relações que,
quaisquer que sejam as nossas posições relativas, me é acessível a mim e não a
ele. Ao olharmos um para o outro, dois mundos diferentes estão reflectidos nas
pupilas dos nossos olhos. É possível, se se assumir uma posição apropriada,
reduzir esta diferença de horizontes a um mínimo, mas para anular tal diferença
completamente seria necessário fundirmo-nos num só, tornarmo-nos uma e a mesma
pessoa.\textsuperscript{[}\textsuperscript{8}\textsuperscript{]}
(p.23)

\end{quote}

O carácter conspicuamente óptico conferido a esta cena matricial definidora da
relação intersubjectiva, alerta-nos para a interdependência dos pontos de vista
irredutíveis que nela se cruzam, indiciando, ao mesmo tempo, que, para o
reconhecimento topográfico de qualquer situação, é indispensável a variação do
enfoque. Transpondo esta interdependência para o encontro clínico, postula-se a
necessidade de o olhar do médico ser complementado pelo do doente, de modo a
recuperar o contexto em que este se insere e, assim, alcançar o
diagnóstico\textsuperscript{(}\footnote{\fn9}\textsuperscript{)}.

Para se aperceber plenamente da sua situação e dado o seu isolamento relativo,
Lisa precisa do olhar de terceiros, neste caso do médico, mas este tão-pouco
está seguro de conseguir sozinho lidar com o enigmático mal que afeta Lisa e,
sobretudo, de ser capaz de lhe dar conta das suas próprias conclusões, sem que
ela com ele interaja. É interessante verificar que o tal diálogo crucial entre
médico e paciente é, contra as nossas expectativas, verbalmente enxuto, não
muito longo e transmite alguma insegurança de Koroliov: a linguagem técnica da
medicina não lhe facilita nem garante o acesso ao íntimo desta criatura que se
revelara, entretanto, inteligente e sensível; terá de recorrer a um outro tipo
de linguagem. Quase instintivamente, o médico senta-se na borda da cama, pega na
mão de Lisa, usa a primeira pessoa do plural, implicando-se no caso, partilhando
com ela o que sente ter de lhe transmitir, mesmo sem saber bem como. Por outras
palavras, trata-se dum verdadeiro diálogo, dado que ambos os interlocutores se
respeitam, se colocam no mesmo patamar, e se vão reposicionando à medida que a
conversa se desenrola, visando uma genuína troca de pontos de vista, um
entendimento autêntico.

Não há aqui paternalismo, nem o médico se refugia no pedestal do seu saber
técnico, antes, admite sentir-se embaraçado, constrangido, como sempre pode
acontecer quando o encontro verdadeiro de dois seres humanos ocorre e os
intervenientes procuram, em conjunto, uma verdade partilhável. Por isso, emergem
nele sinais duma interdependência que não se esgota na relação clínica, antes a
ultrapassa, já que Lisa admite que o que verdadeiramente lhe falta é um amigo,
alguém em quem confie e com quem possa falar. A capacidade de gerar confiança, a
abertura para o imponderável do que o outro tem para nos contar, a
disponibilidade para a escuta atenta, a empatia, eis o que, neste diálogo, se
encena e se insinua como ingredientes indispensáveis num encontro que se queira
digno desse nome e produtivo. Nele o médico revela-se, afinal, não apenas em
sintonia com a paciente, mas como um seu igual, e os sintomas de doença – a
insónia e a taquicardia –, transmutam-se em sinais duma saudável reação a um
contexto hostil e insalubre, e indiciam um instinto de vida de que ambos
comungam: “A sua insónia é respeitável: seja como for, é um bom sinal. De facto,
uma conversa como esta nossa teria sido impensável para os nossos pais; […] ora
nós, a nossa geração, dormimos mal, atormentamo-nos, falamos muito e duvidamos
sempre de termos ou não razão”\textsuperscript{[}\textsuperscript{1}\textsuperscript{]}
(p. 272).

Não interessa se o que Koroliov diz relativamente às diferenças geracionais é
verdadeiro ou não, mas, sim, que tal serve como via de acesso e como garantia da
necessária confiança viabilizadora do diálogo franco e aberto. O clínico
prescindiu transitoriamente da linguagem da ciência, arriscou-se na fragilidade
incerta da sua humanidade, cuidando de criar uma ponte para o
outro…\textsuperscript{(}\footnote{\fn10}\textsuperscript{)}.

Afinal, tudo se interliga e se nivela: o médico com a paciente, a governanta com
os donos da fábrica, estes com os operários e, finalmente, os elementos naturais
com o universo humano.

Tudo está, aqui, em estreita interdependência: ao avaliar o ecossistema que
envolve a doente, o médico é levado a um diagnóstico que, como tantas vezes
acontece, excede a sua competência clínica e o seu saber científico. A vida
mecanizada, insalubre e irracional gerada pela industrialização repercute-se nos
corpos e nos espíritos, em geral, mas especialmente nos mais sensíveis, e exige
uma abordagem holística, que tenha em conta a pessoa como um todo, na
multiplicidade das suas pertenças, constrangimentos, compromissos e
circunstâncias.

Em diagnose certeira do atual estado de coisas na área da saúde, João Lobo
Antunes, em obra recente, intitulada A Nova Medicina, traça o seguinte quadro:
\begin{quote}

Um facto surpreendente é que o progresso da biomedicina veio aumentar
paradoxalmente a incerteza da decisão médica. […]

Ainda hoje na prática clínica o diagnóstico começa com a colheita da história, o
exame físico – agora um pouco menosprezado pela abundância e rigor das técnica

s de imagem – a que se segue o recurso a exames auxiliares de diagnóstico. Na
Nova Medicina a imagem quase aboliu a narrativa da doença, em parte porque o
médico tem menos tempo para ouvir […], e porque o próprio doente tem dificuldade
em explicar as suas queixas e acha que a doença está claramente revelada nas
imagens obtidas.

A prevalência dos exames auxiliares de diagnóstico e, sobretudo, das imagens,
tenderam a negligenciar os aspectos verbais, ou melhor, narrativos, inerentes à
prática clínica: ouve-se menos o doente e dá-se-lhe menos espaço na consulta,
porque o seu testemunho verbal foi
desvalorizado.\textsuperscript{[}\textsuperscript{10}\textsuperscript{]}
(p. 29-30)

\end{quote}

Pode dizer-se que é neste contexto, neste equilíbrio difícil entre objetividade
e rigor científicos e o reconhecimento da necessidade das palavras, do relato
verbal, que emerge a Medicina Narrativa ou as Humanidades
Médicas\textsuperscript{(}\footnote{\fn11}\textsuperscript{)}, como também têm sido chamadas. Trata-se dum campo de estudo que, nas últimas
décadas, tem vindo a chamar a atenção para a necessidade de um recentramento na
relação intersubjetiva que caracteriza a prática clínica, atendendo e acudindo
às competências comunicativas e a uma preparação humanística por parte do
pessoal para/médico que coloque a relação com a pessoa do doente no centro dos
cuidados. Para tanto, a Medicina Narrativa, profundamente interdisciplinar, tira
partido do contributo de áreas do saber do âmbito das Humanidades,
designadamente, dos Estudos Literários, da Filosofia, da Ética, da Antropologia,
da Sociologia, entre outras. Se os anos oitenta do século passado testemunharam
a emergência e a afirmação quase incontestada da “Medicina baseada na prova”
(Evidence-based Medicine – \textsc{ebm}\textsuperscript{(}\footnote{\fn12}\textsuperscript{)}
), que abordaremos adiante, já o aparecimento desta nova área de investigação –
Narrative-based Medicine – \textsc{nbm} ou Narrative Medicine – NM – tem conhecido menos
protagonismo no âmbito das ciências médicas, muito embora não seja propriamente
mais recente\textsuperscript{(}\footnote{\fn13}\textsuperscript{)}. Foi, sobretudo, nas décadas de oitenta e noventa do século passado que se
fizeram notar, de forma mais decidida, publicações e iniciativas conducentes a
um movimento que hoje se manifesta a nível mundial, mas muito em particular no
mundo anglo-saxónico. Refiro-me ao aparecimento de programas académicos como o
da Universidade de Columbia, fundado em 2000, e a publicações como as de Richard
Zaner, Tricia Greenhalgh, Brian Hurwitz ou Rita Charon.

A Medicina Narrativa, nas palavras da sua principal defensora, Rita Charon: “[é
uma] medicina praticada com a competência narrativa para reconhecer, interpretar
e ser levado a agir pela situação crítica dos
outros”\textsuperscript{[}\textsuperscript{11}\textsuperscript{]}
(p. 83).

Em definição posterior e mais completa, Hurwitz contempla a diferença crucial
entre singularidade biológica e a individualidade do doente (esta última objeto
privilegiado da MN):
\begin{quote}

[a] Medicina Narrativa é uma prática e uma atitude intelectual que permite aos

médicos olhar para lá dos mecanismos biológicos no cerne das abordagens
convencionais à prática médica, e abarcar domínios de pensamento e modos de dizer que se focalizam na
linguagem e na representação, nas emoções e relações que iluminam a prática dos
cuidados de saúde.\textsuperscript{[}\textsuperscript{12}\textsuperscript{]}
(p. 73, tradução e ênfase minhas)

\end{quote}

Defende-se ser esta
\begin{quote}

a plataforma mais desenvolvida que existe para exercer clínica com as técnicas e
a sensibilidade necessárias, a fim de sermos capazes decombinar o conhecimento
clínico e científico com o entendimento interpessoal dos muitos e variados
relatos encontrados nos cuidados de saúde
individuais.\textsuperscript{[}\textsuperscript{12}\textsuperscript{]}
(p. 84, tradução e ênfase minhas)

\end{quote}

A questão central que hoje enfrentamos consiste em equilibrar os extraordinários
avanços registados na medicina como ciência, ou melhor, na/s ciência/s médica/s,
e a tendência para uma certa subalternização da relação interpessoal no encontro
clínico, da escuta atenta da história do doente, da consideração do seu ambiente
social e familiar, aspetos que, como se viu, alguns entendem ser consequência da
crença inquestionada e (quase) exclusiva no valor decisivo dos exames auxiliares
de diagnóstico e, sobretudo, no rigor das técnicas de imagem.

Esta nova área interdisciplinar, focando-se na linguagem e/ou na representação,
não rejeita os avanços científicos, antes, depende deles, mas combina-os com
outras disciplinas, derivando sustento intelectual de domínios que vão além da
estrita ciência, designadamente o das artes. Assume-se como atividade
interpretativa e provisória, inalienável do contexto situacional em que é
praticada\textsuperscript{[}\textsuperscript{12}\textsuperscript{]}.

Não sinalizando, propriamente, uma preocupação nova na medicina clínica, a MN é,
porém, um campo inovador no modo como, pela primeira vez, conjuga a prática
médica com áreas de investigação alheias a
ela\textsuperscript{[}\textsuperscript{12}\textsuperscript{]}. E responde, modernamente, ao crescente domínio do paradigma científico de base
positivista, aliado a um profissionalismo algo estreito no exercício da clínica.
A MN apareceu como o necessário complemento à institucionalização desse
paradigma duro no exercício da arte médica, de base tecnológica, instalado no
início dos anos oitenta do século XX e concretizado pela já referida
\textsc{ebm}\textsuperscript{[}\textsuperscript{10}\textsuperscript{]}.

O que devemos entender por \textsc{ebm}? Nas palavras de um dos seus defensores: “[a]
Medicina baseada na prova, ou medicina factual, define-se como a utilização
consciente e judiciosa dos melhores e mais atualizados dados (provas) da
investigação clínica, atendendo à especificidade de cada
doente”\textsuperscript{[}\textsuperscript{13}\textsuperscript{]}
(p. 71, tradução minha).

Tais provas advêm de estudos clínicos sistemáticos, submetidos a testes
extensivos de carácter aleatório, a meta-análises, a estudos transversais ou
estudos de acompanhamento fidedignos. A \textsc{ebm} preconiza o recurso a uma revisão
sistemática da melhor literatura médica disponível (sujeita a arbitragem
científica), análise de risco/benefício e testes de controlo randomizado.
Consiste, pois, em basear as decisões clínicas não apenas em conhecimentos
teóricos, no julgamento e na experiência (principais componentes da medicina
tradicional), mas, também, nas provas científicas, ou seja, os conhecimentos
dedutíveis a partir de investigação científica sistemática, realizada sobretudo
no plano do diagnóstico, do prognóstico e do tratamento de doenças e que se
baseiam em resultados validados e aplicáveis na prática médica corrente.

Ainda que os seus defensores aleguem não prescindir do julgamento e da
experiência médicas e afirmem que a \textsc{ebm} é complementar em relação à prática
médica tradicional, a verdade é que esta orientação acentuou o pendor
cientificizante da “nova medicina” e levou os médicos, em gesto
defensivo\textsuperscript{(}\footnote{\fn14}\textsuperscript{)}, a uma utilização excessiva de todo este tipo de resultados e decorrentes dados
estatísticos, duma multiplicidade de testes e técnicas auxiliares de
diagnóstico, o que determinou um aumento exponencial do número de exames (muitas
vezes desnecessários), com riscos acrescidos para os doentes e um aumento
significativo dos custos na saúde\textsuperscript{(}\footnote{\fn15}\textsuperscript{)}.

Apesar de tudo, julgo que não devemos encorajar uma visão contraditória entre
\textsc{ebm} e \textsc{nbm}/MN, mas, antes, como defende
Parker\textsuperscript{[}\textsuperscript{16}\textsuperscript{]}, afirmar que a \textsc{ebm}, devidamente concebida e complementada pela MN, pode
constituir a condição necessária para o exercício consciente da liberdade
clínica nos nossos dias.

Mas a MN não preconiza apenas que os profissionais de saúde estejam somente
atentos e disponíveis para acolher, na história clínica, as informações mais
salientes do enredo que, pelo seu relato, o doente veicula, mas, também, que
sejam recetivos e capazes de registar e descodificar a linguagem peculiar usada
para o narrar, linguagem essa que, por vezes, pode até entrar em conflito com o
ostensivo enredo da história relatada. Inspirando-se no método de leitura do New
Criticism – o close reading, Charon gosta de usar a expressão close listening,
para se referir ao tipo de atenção que o médico deve prestar ao que o doente lhe
transmite. Pela capacidade de registarem o tipo de metáforas utilizadas, a
repetição insistente de certas expressões, o ponto de vista e a sequência
temporal adotados, as ambiguidades, os silêncios, entre outras características
discursivas, os profissionais de saúde perceberão melhor e poderão, mais
cabalmente, diagnosticar a situação psicossomática particular do doente que têm
perante si. E isto só se consegue com treino. Charon defende que: “o treino
narrativo ao nível da leitura e da escrita contribui para a eficácia
clínica”\textsuperscript{[}\textsuperscript{17}\textsuperscript{]}
(p.107). E acrescenta: “o tipo de decisões terapêuticas que tomamos pode ser
marcadamente diferente do processo de decisão convencional, como resultado do
aprofundamento narrativo da relação
medico-doente”\textsuperscript{[}\textsuperscript{17}\textsuperscript{]}
(p.191).

À leitura dos sinais do corpo, deve aduzir-se uma descodificação das narrativas
e outros indícios verbais e não verbais dos doentes, e uma consciência dos
aspetos éticos e socioculturais envolvidos. Tal requer uma predisposição
cognitiva particular que só uma formação humanística – aliada ao conhecimento e
à experiência clínica – pode promover\textsuperscript{[}\textsuperscript{18}\textsuperscript{]}\textsuperscript{,}\textsuperscript{[}\textsuperscript{19}\textsuperscript{]}. Esta apreensão e decifração da situação humana específica que, num determinado
momento, caracteriza o doente, e que conduz ao diagnóstico, será tanto ou mais
competente e eficaz quanto o profissional de saúde dispuser dos instrumentos de
análise e interpretação capazes de lhe permitirem fazer sentido do relato que
ouve. Parte desses instrumentos pode ser-lhe facultada pelos estudos literários,
em geral, e, em particular, pelos contributos dados mais recentemente pela
narratologia\textsuperscript{[}\textsuperscript{20}\textsuperscript{]}.

O reconhecimento desta necessária complementaridade entre um conhecimento
científico de pendor universalizante, característico das ciências biológicas, e
uma abordagem atenta e aberta às ressonâncias peculiares do caso histórico
concreto que, no encontro clínico e na história clínica, vão sendo reveladas,
ecoa a distinção estabelecida por Aristóteles, na sua Ética a Nicómaco, entre
dois tipos de conhecimento. Refiro-me à díade sabedoria filosófica e sabedoria
prática que, fazendo, ambas, parte do lado racional da alma, são de índole bem
diferente. Orientada para a ação, a sabedoria prática não se ocupa apenas com
universais, como é característico da sabedoria filosófica, mas, antes, atende
aos factos particulares imediatos, conhecidos pela experiência. É este tipo de
sabedoria que permite corrigir o raciocínio e conduz à excelência da
deliberação, fim visado pelo ato clínico\textsuperscript{(}\footnote{\fn16}\textsuperscript{)}.

A importância do reconhecimento duma interdependência entre saber científico, de
carácter objetivo, com valor universalizante, e um outro tipo de saber, atento à
situação contextual particular e, nesse sentido, provisório e dependente dum ato
hermenêutico complexo, foi o que, em parte pelo menos, terá levado Lennard J.
Davis e David B. Morris a publicarem, em 2007, o seu manifesto a favor do que
designam de “biocultures” [“biocultura/s”], onde defendem o seguinte:
\begin{quote}

Pensar na ciência sem incluir uma análise histórica e cultural seria como pensar
o texto literário sem a teia contextual dos conhecimentos discursivos, ativos ou
dormentes, em momentos particulares. É igualmente limitado pensar na literatura
[…] sem considerar a rede de sentidos que podem ser aprendidos a partir duma
perspetiva científica. […]

O biológico sem o cultural, ou o cultural sem o biológico, está condenado a ser
reducionista, na melhor das hipóteses, e falho de rigor, na
pior.\textsuperscript{[}\textsuperscript{22}\textsuperscript{]}
(p. 411, tradução e ênfases minhas)

\end{quote}

Com este termo, “biocultura/s,” pretendem validar e consolidar uma
multiplicidade de experiências interdisciplinares que já estavam no terreno em
inúmeras universidades e que vão desde as “Humanidades Médicas” ou “Medicina
Narrativa” à história da medicina, passando pela: saúde pública, bioética,
epidemiologia, estudos de identidade e do corpo, antropologia médica, sociologia
médica, filosofia da medicina etc. Contestam, ainda, a legitimidade da
tradicional fronteira entre as chamadas ciências duras e as restantes, por
defenderem “a existência duma comunidade de intérpretes, transversal às
disciplinas, intérpretes que estão dispostos a aprender uns com os
outros”\textsuperscript{[}\textsuperscript{22}\textsuperscript{]}
(p. 416, tradução e ênfase minhas).

Na atual conjuntura, seria vantajoso que o pessoal de saúde e, em particular, os
médicos se encarassem como membros duma “comunidade de intérpretes,” e, nesse
sentido, valorizassem as competências interpretativas postuladas pelas
exigências da “nova medicina”. Mas tal só se concretizará por via duma mudança
de atitude, e esta só ocorrerá quando se abandonarem hábitos e uma mentalidade
ainda agora demasiado arreigados no exercício da medicina. Treinado para curar,
convicto de que a/s ciência/s médica/s e os seus inegáveis progressos são
garantia de sucesso, o aluno das atuais escolas médicas não está preparado para
admitir que não sabe ou que se enganou\textsuperscript{[}\textsuperscript{15}\textsuperscript{]}. Daí que, quando confrontados com situações que fogem ao formato tipificado ou
ao caso documentado e estatisticamente previsto, para já não falar do fracasso
dos tratamentos, estes candidatos a médicos experimentem situações de ansiedade
e desespero que nada deveria justificar. Conforme defende Newmann, a moderna
ciência e, sobretudo, a Medicina “continua a tratar o organismo humano como um
modelo de causa e efeito, uma máquina cartesiana com funções previsíveis e
mensuráveis no mundo físico”. Ora, tal pressuposto é claramente inadequado, e
não tem em conta a complexidade das “maquinações internas do corpo humano,” a
grande maioria das quais está ainda por
compreender\textsuperscript{[}\textsuperscript{15}\textsuperscript{]}
(p. 34). Além disso, ignora ainda um outro fator, para o qual a física quântica
chamou a nossa atenção: a importância do observador. Conforme argumenta Newmann:
\begin{quote}

[o] modelo estóico do médico moderno que apoiamos – um observador objectivo e
distante – não existe. A nossa presença altera a trajectória da doença e tem
impacto sobre a experiência humana. Não somos espectadores indiferentes, nem
devemos sê-lo; fazemos parte integrante e poderosa tanto da doença como da
cura.\textsuperscript{[}\textsuperscript{15}\textsuperscript{]}
(p. 223)

\end{quote}

É isso mesmo que nos ensina o conto de Tchékhov, ao propor-nos um
protagonista-médico que, confrontado com uma situação enigmática e atípica,
acaba por envolver-se, deixando-se impregnar pelo ambiente que rodeia a paciente
e tornando-se, assim, ‘parte integrante da cura’.

Jack Coulehan, observador atento do uso de narrativas e de metáforas em
medicina, mostrou-nos como a linguagem prevalecente na prática clínica reflete
uma cultura que torna o doente num objeto e, o médico, alguém investido de poder
para abordar esse doente-corpo,
subalternizado\textsuperscript{[}\textsuperscript{23}\textsuperscript{]}. Por exemplo, tome-se a expressão inglesa: “history taking” [em português,
“colheita da história”], também ela metafórica, em que o peso conceptual de
“history” sugere a objetivação do que o doente relata, ao mesmo tempo que está
implícito tratar-se duma entidade que encontraremos se a procurarmos de modo
suficientemente agressivo, tal como se procura uma caixa negra por entre os
destroços da vida do doente. O verbo “taking” sugere, implicitamente, uma
violação do doente. O médico como que arranca a história ao doente, como quem
arranca um dente ou remove um apêndice. A “colheita” em português, outra
metáfora, também ela sugere um acto agressivo, rapace, sublinhando a passividade
do doente.

Coulehan analisa, ainda, as metáforas mais recorrentes no discurso médico, e
chama a atenção para as três mais usuais: a da guerra (a doença é o inimigo e o
médico, qual comandante das tropas, propõe-se vencê-la), a paternal (a doença
vista como ameaça e o médico como protetor) e a da engenharia (a doença como
disfunção e o médico como técnico).

Com o declínio das metáforas paternalistas, que a medicina contemporânea baniu,
ganharam mais força as metáforas militares e mecânicas. Em ambos os casos,
porém, a subalternidade do doente e a redução deste ou do seu corpo à dimensão
de objeto são inescapáveis. Tal vai de par com a conceção do médico como um
observador objectivo e distante, detentor do saber, que não se envolve.

Numa época em que, constantemente, somos surpreendidos pelo reconhecimento da
nossa interdependência global, seja em termos climáticos, políticos ou
económicos, talvez fizesse mais sentido, como alguém sugeriu, substituir estas
metáforas por outras mais holísticas; refiro-me a metáforas ambientalistas,
cujos pressupostos remetem para a inter-relação entre as partes envolvidas e
entre estas e o todo, evocativas da integralidade dinâmica quer do corpo social
quer do corpo físico.
\begin{quote}

O tratamento bem-sucedido da dor exige mais comunicação entre doente e médico,
comunicação entre os vários profissionais e uma visão da medicina como qualquer
coisa de inacabado e de integrativo, mais do que especializado, estandardizado e
baseado em tecnologia. […] Talvez as metáforas devessem mudar do terreno da
guerra para o do ambiente. A medicina precisa de procurar ir para lá da
consideração de sintomas isolados e de soluções agressivas e posicionar-se ao
nível da ecologia de sistema alargado.\textsuperscript{(}\footnote{\fn17}\textsuperscript{)}

\end{quote}

Em última análise, o que a MN propõe é que venham a mudar-se as metáforas, e,
com elas, toda uma cultura que tem caracterizado o modo como a medicina é hoje
praticada. Não duvidamos de que se trata dum processo lento e com muitos
obstáculos, mas a boa notícia é que ele já foi iniciado. Acreditamos, ainda, que
os estudos literários, apesar de toda a sua história recente, caracterizada por
antagonismos e clivagens, movimentos de ação/reação sucessivos ou, se calhar,
precisamente por causa de tudo isso, atingiram uma maturidade que, no
indispensável diálogo interdisciplinar, lhes permite darem um contributo
decisivo, se souberem ser criteriosos no uso dos seus melhores instrumentos e
métodos. E, para além do mais, não nos esqueçamos de que, como alguém fez
questão de assinalar: “Mais do que qualquer outro assunto, a literatura
presta-se a um ensino poderoso”\textsuperscript{(}\footnote{\fn18}\textsuperscript{) }
(Tradução minha).

\section*{Referências}
\begin{itemize}

\item[1] Tchékhov A. Contos. Guerra N, Guerra F, tradutores. Lisboa: Relógio
d’Água Editores; 2006. v. 6, p. 261-73.

\item[2] Tchékhov A. A doctor’s visit. In: Coulehan J, editor. Tchékhov’s
doctors: a collection of Tchékhov’s medical tales. Kent: Kent State University
Press; 2003. p. 174-82.

\item[3] Del Lungo A. L’Incipit romanesque. Paris: Éditions du Seuil; 2003.
(Coll. Poétique)

\item[4] Shelley PB. A defence of poetry. In: Enright DJ, De Chickera E,
editores. English critical texts (1962). Delhi: Oxford University Press; 1979.
p. 225-55.

\item[5] Stanzel FK. A theory of narrative. Goedsche C, tradutor. Cambridge:
Cambridge University Press; 1984.

\item[6] Cohn D. Transparent minds: narrative modes for presenting consciousness
in fiction. Princeton: Princeton University Press; 1978.

\item[7] Ricoeur P. Soi même comme un autre. Paris: Seuil; 1990.

\item[8] Bakhtin MM. Art and answerability: early philosophical essays. Holquist
M, editor; Liapunov V, tradutor; Brostrom K, tradutor do suplemento. Austin:
University of Texas Press; 1990.

\item[9] Hippocrates. Epidemics 1st, 11 [acesso 2014 Abr 15]. Disponível em:
http://www.humanehealthcare.com/Article.asp?art\_{}id=806

\item[10] Antunes JL. A nova medicina. Lisboa: Fundação Francisco Manuel dos
Santos; 2012.

\item[11] Charon R. Narrative medicine: form, function, and ethics. Ann Intern
Med. 2001; 134: 83-7.

\item[12] Hurtwitz B. Narrative (in) Medicine. In:Spinozzi P, Hurtwitz B,
editores. Discourses and Narrations in the Biosciences. Göttingen: Vandenhoeck
\& Ruprecht Unipress; 2011. p. 73-87.

\item[13] Sackett DL, Rosenberg WM, Gray JA, Haynes RB, Richardson WS. Evidence
Based Medicine: what it is and what it isn’t. Br Med J. 1996; 312:71-2.

\item[14] Newman DH. Hippocrates’ shadow. New York: Scribner; 2008.

\item[15] Newman DH. Onde falham os médicos. Lucas AG, tradutor. Alfragide: Casa
das Letras; 2010.

\item[16] Parker M. False dichotomies: \textsc{ebm}, clinical freedom and the art of
medicine. Med Humanit [Internet]. 2005 [acesso 2012 Ago 8]; 31(1):23-30.
Disponível em: http://mh.bmj.com/content/31/1/23.full

\item[17] Charon R. Narrative medicine: honoring the stories of illness. Oxford:
Oxford University Press; 2006.

\item[18] Tauber AI. Science and the quest for meaning. Waco: Baylor University
Press; 2009.

\item[19] Danou G. Langue, récit, littérature dans l’éducation médicale.
Limoges: Éditions-Lucas; 2007.

\item[20] Fernandes I. Confronting the other: the interpersonal challenge in
narrative medicine. In: Fernandes I, editor. Creative dialogues: narrative and
medicine. Cambridge: Cambridge Scholars Publishing. Forthcoming.

\item[21] Aristóteles. Ética a Nicómaco [Internet]. Vallandro L, Bornheim G,
tradutores. São Paulo: Abril Cultural; 1984. Livro VI, cap. 7, p. 145-7 [acesso
2014 Ago 8]. Disponível em: http://sumateologica.files.wordpress.com/2009/07/ari
stoteles\_{}-\_{}metafisica\_{}etica\_{}a\_{}nicomaco\_{}politica.pdf

\item[22] Davis LJ, Morris DB. Biocultures manifesto. New Lit Hist. 2007;
38(3):411-8.

\item[23] Coulehan J. Metaphor and medicine: narrative in clinical practice.
Yale J Biol Med [Internet]. 2003 [acesso 2012 Ago 21]; 76:87-95. Disponível em:
http://www.ncbi.nlm.nih.gov/pmc/articles/\textsc{pmc}2582695/pdf/yjbm00203-0044.pdf

\item[24] Jackson M. Pain: the science and culture of why we hurt. London:
Bloomsbury; 2003.

\item[25] Harpham GG. Politics, professionalism, and the pleasure of reading.
Daedalus. 2005; 134(3):68-75.

\end{itemize}

\end{document}
