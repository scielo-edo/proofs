% Generated by jats2tex@0.11.1.0
\documentclass{article}
\usepackage[T1]{fontenc}
\usepackage[utf8]{inputenc} %% *
\usepackage[portuges,spanish,english,german,italian,russian]{babel} %% *
\usepackage{amstext}
\usepackage{authblk}
\usepackage{unicode-math}
\usepackage{multirow}
\usepackage{graphicx}
\usepackage{etoolbox}
\usepackage{xtab}
\usepackage{enumerate}
\usepackage{hyperref}
\usepackage{penalidades}
\usepackage[footnotesize,bf,hang]{caption}
\usepackage[nodayofweek,level]{datetime}
\usepackage[top=0.85in,left=2.75in,footskip=0.75in]{geometry}
\newlength\savedwidth
\newcommand\thickcline[1]{\noalign{\global
\savedwidth
\arrayrulewidth
\global\arrayrulewidth 2pt}
\cline{#1}
\noalign{\vskip\arrayrulewidth}
\noalign{\global\arrayrulewidth\savedwidth}}
\newcommand\thickhline{\noalign{\global
\savedwidth\arrayrulewidth
\global\arrayrulewidth 2pt}
\hline
\noalign{\global\arrayrulewidth\savedwidth}}
\usepackage{lastpage,fancyhdr}
\usepackage{epstopdf}
\pagestyle{myheadings}
\pagestyle{fancy}
\fancyhf{}
\setlength{\headheight}{27.023pt}
\lhead{\includegraphics[width=10mm]{logo.png}}
\rhead{\ifdef{\journaltitle}{\journaltitle}{}
\ifdef{\volume}{vol.\,\volume}{}
\ifdef{\issue}{(\issue)}{}
\ifdef{\fpage}{\fpage--\lpage\,pp.}}
\rfoot{\thepage/\pageref{LastPage}}
\renewcommand{\footrule}{\hrule height 2pt \vspace{2mm}}
\fancyheadoffset[L]{2.25in}
\fancyfootoffset[L]{2.25in}
\lfoot{\sf \ifdef{\articledoi}{\articledoi}{}}
\setmainfont{Linux Libertine O}
\renewcommand*{\thefootnote}{\alph{footnote}}
\makeatletter
\newcommand{\fn}{\afterassignment\fn@aux\count0=}
\newcommand{\fn@aux}{\csname fn\the\count0\endcsname}
\makeatother

\newcommand{\journalid}{Interface (Botucatu)}
\newcommand{\publisherid}{icse}
\newcommand{\journaltitle}{Interface - Comunicação, Saúde, Educação}
\newcommand{\abbrevjournaltitle}{Interface}
\newcommand{\issnppub}{1414-3283}
\newcommand{\issnepub}{1807-5762}
\newcommand{\publishername}{Laboratório de Educação e Comunicação em Saúde,
Departamento de Saúde Pública, Faculdade de Medicina de Botucatu e Instituto de
Biociências de Botucatu - \textsc{unesp}}
\newcommand\articleid{1807-57622014.0511}
\newcommand\articledoi{\textsc{doi} 10.1590/1807-57622014.0511}
\def\subject{Teses}\newcommand{\subtitlestyle}[1]{-- \emph{#1}\medskip}
\newcommand{\transtitlestyle}[1]{\par\medskip\Large #1}
\newcommand{\transsubtitlestyle}[1]{-- \Large\emph{ #1}}

\newcommand{\titlegroup}{
\ifdef{\subtitle}{\subtitlestyle{\subtitle}}{}
\ifdef{\transtitle}{\transtitlestyle{\transtitle}}{}
\ifdef{\transsubtitle}{\transsubtitlestyle{\transsubtitle}}{}}

\title{Promoção da gestão do regime terapêutico em pacientes com Doença Pulmonar
Obstrutiva Crónica (\textsc{dpoc}): um percurso de investigação-ação\titlegroup{}}
\newcommand{\transtitle}{Manejo del régimen terapéutico en pacientes con
Enfermedad Pulmonar Obstructiva Crónica (\textsc{epoc}): un curso de investigación
acción}
\author{Padilha, José Miguel}
\affil{Instituto de Ciências da Saúde do Porto}
\date{ 2015}
\def\volume{19}
\def\issue{52}
\def\fpage{201}
\def\lpage{202}
\def\permissions{This is an Open Access article distributed under the terms of
the Creative Commons Attribution Non-Commercial License, which permits
unrestricted non-commercial use, distribution, and reproduction in any medium,
provided the original work is properly cited.}
\newcommand{\kwdgroup}{Terapêutica, Doença Pulmonar Obstrutiva Crônica,
Autocuidado, Investigação-ação}

\begin{document}
\selectlanguage{portuges}
\section*{Metadados não aplicados}
\begin{itemize}
\item[\textbf{língua do artigo}]{Português}
\ifdef{\journalid}{\item[\textbf{journalid}] \journalid}{}
\ifdef{\journaltitle}{\item[\textbf{journaltitle}] \journaltitle}{}

\ifdef{\journalsubtitle}{\item[\textbf{journalsubtitle}] \journaltitle}{}
\ifdef{\transjournaltitle}{\item[\textbf{journaltitle}] \journaltitle}{}
\ifdef{\transjournalsubtitle}{\item[\textbf{journalsubtitle}] \journaltitle}{}

\ifdef{\abbrevjournaltitle}{\item[\textbf{abbrevjournaltitle}]
\abbrevjournaltitle}{}
\ifdef{\issnppub}{\item[\textbf{issnppub}] \issnppub}{}
\ifdef{\issnepub}{\item[\textbf{issnepub}] \issnepub}{}
\ifdef{\publishername}{\item[\textbf{publishername}] \publishername}{}
\ifdef{\publisherid}{\item[\textbf{publisherid}] \publisherid}{}
\ifdef{\subject}{\item[\textbf{subject}] \subject}{}
\ifdef{\transtitle}{\item[\textbf{transtitle}] \transtitle}{}
\ifdef{\authornotes}{\item[\textbf{authornotes}] \authornotes}{}
\ifdef{\articleid}{\item[\textbf{articleid}] \articleid}{}
\ifdef{\articledoi}{\item[\textbf{articledoi}] \articledoi}{}
\ifdef{\volume}{\item[\textbf{volume}] \volume}{}
\ifdef{\issue}{\item[\textbf{issue}] \issue}{}
\ifdef{\fpage}{\item[\textbf{fpage}] \fpage}{}
\ifdef{\lpage}{\item[\textbf{lpage}] \lpage}{}
\ifdef{\permissions}{\item[\textbf{permissions}] \permissions}{}
\end{itemize}
\maketitle

A \textsc{dpoc} é uma doença crónica que influencia negativamente o nível de energia
disponível para o autocuidado. Os pacientes portadores de \textsc{dpoc}, para controlarem
a progressão da doença, necessitam de desenvolver competências para gerir o
regime terapêutico (ex.: autocontrolar a dispneia; executar inaloterapia;
identificar as exacerbações da doença). A competência dos pacientes para gerirem
o regime terapêutico pode influenciar: a condição de saúde, a autonomia no
autocuidado e a qualidade de vida.

Com este estudo, pretendíamos contribuir para a melhoria contínua da qualidade
dos cuidados, por meio do desenvolvimento de uma abordagem terapêutica
progressivamente mais sistematizada, tomando por foco a promoção da gestão do
regime terapêutico, em pacientes com \textsc{dpoc}.

Utilizamos um paradigma de investigação construtivista e uma metodologia de
investigação-ação (IA) participativa. Para a recolha de dados, utilizamos,
simultaneamente, estratégias qualitativas e quantitativas, recorremos à análise
comparativa e iterativa dos dados. O estudo decorreu num serviço de internamento
e na consulta externa de uma instituição hospitalar portuguesa, e contou com a
participação de 52 enfermeiros. Utilizamos estratégias promotoras da
participação e do comprometimento interno dos enfermeiros para viabilizar a
mudança.

O ciclo de IA empreendido gerou mudanças no modelo de cuidados em uso, que
evoluiu de uma lógica de gestão de sinais e sintomas da doença, para uma visão,
progressivamente, mais centrada na gestão do regime terapêutico. Para isso,
foram construídas e implementadas linhas de orientação para a ação dos
enfermeiros, que melhoraram a continuidade de cuidados e a monitorização dos
resultados. Na organização dos cuidados, a mudança permitiu otimizar a partilha
de informação, tendo por horizonte a promoção da continuidade dos cuidados. A
mudança implementada permitiu a reorganização da consulta de enfermagem,
garantindo uma maior acessibilidade dos pacientes aos cuidados de enfermagem.

Na documentação dos cuidados de enfermagem, assistimos a um aumento
significativo da documentação de dados relativos ao autocuidado – gestão do
regime terapêutico. Uma vez estabilizada a mudança, foi-nos possível verificar
uma melhoria na continuidade dos cuidados e um aumento da documentação de
informação válida para a monitorização do impacte da ação terapêutica de
enfermagem sobre a condição de saúde dos pacientes.

Os resultados permitem verificar que os pacientes com níveis de
consciencialização mais adequados são aqueles que apresentam melhores resultados
em termos de ganhos em conhecimentos e capacidades para gerir o regime
terapêutico.

Este estudo revela que a dinamização dos processos organizacionais e logísticos
inerentes à mudança, a agregação da intenção e objetivos de todos os envolvidos,
e o desenvolvimento/ disponibilização de recursos de suporte à decisão, são
fatores-chave na mudança. A disponibilização e a estabilidade dos recursos são
funções-chave das organizações para viabilizar a mudança. O envolvimento e a
participação dos afetados na conceção, implementação e avaliação da mudança, por
meio da promoção de consensos e da intercolaboração, são aspetos centrais no
desenvolvimento de uma cultura propícia à melhoria contínua da qualidade dos
cuidados de enfermagem e à investigação.

A sistematização da ação terapêutica implementada foi um contributo relevante
para a melhoria contínua da qualidade dos cuidados de enfermagem a pacientes com
\textsc{dpoc}.

\end{document}
