% Generated by jats2tex@0.11.1.0
\documentclass{article}
\usepackage[T1]{fontenc}
\usepackage[utf8]{inputenc} %% *
\usepackage[portuges,spanish,english,german,italian,russian]{babel} %% *
\usepackage{amstext}
\usepackage{authblk}
\usepackage{unicode-math}
\usepackage{multirow}
\usepackage{graphicx}
\usepackage{etoolbox}
\usepackage{xtab}
\usepackage{enumerate}
\usepackage{hyperref}
\usepackage{penalidades}
\usepackage[footnotesize,bf,hang]{caption}
\usepackage[nodayofweek,level]{datetime}
\usepackage[top=0.85in,left=2.75in,footskip=0.75in]{geometry}
\newlength\savedwidth
\newcommand\thickcline[1]{\noalign{\global
\savedwidth
\arrayrulewidth
\global\arrayrulewidth 2pt}
\cline{#1}
\noalign{\vskip\arrayrulewidth}
\noalign{\global\arrayrulewidth\savedwidth}}
\newcommand\thickhline{\noalign{\global
\savedwidth\arrayrulewidth
\global\arrayrulewidth 2pt}
\hline
\noalign{\global\arrayrulewidth\savedwidth}}
\usepackage{lastpage,fancyhdr}
\usepackage{epstopdf}
\pagestyle{myheadings}
\pagestyle{fancy}
\fancyhf{}
\setlength{\headheight}{27.023pt}
\lhead{\includegraphics[width=10mm]{logo.png}}
\rhead{\ifdef{\journaltitle}{\journaltitle}{}
\ifdef{\volume}{vol.\,\volume}{}
\ifdef{\issue}{(\issue)}{}
\ifdef{\fpage}{\fpage--\lpage\,pp.}}
\rfoot{\thepage/\pageref{LastPage}}
\renewcommand{\footrule}{\hrule height 2pt \vspace{2mm}}
\fancyheadoffset[L]{2.25in}
\fancyfootoffset[L]{2.25in}
\lfoot{\sf \ifdef{\articledoi}{\articledoi}{}}
\setmainfont{Linux Libertine O}
\renewcommand*{\thefootnote}{\alph{footnote}}
\makeatletter
\newcommand{\fn}{\afterassignment\fn@aux\count0=}
\newcommand{\fn@aux}{\csname fn\the\count0\endcsname}
\makeatother

\newcommand{\journalid}{Bull World Health Organ}
\newcommand{\journaltitle}{Bulletin of the World Health Organization}
\newcommand{\abbrevjournaltitle}{Bull. World Health Organ.}
\newcommand{\issnppub}{0042-9686}
\newcommand{\publishername}{World Health Organization}
\newcommand\articleid{\textsc{blt}.13.011113}
\newcommand\articledoi{\textsc{doi} 10.2471/\textsc{blt}.13.011113}
\def\subject{News}\newcommand{\subtitlestyle}[1]{-- \emph{#1}\medskip}
\newcommand{\transtitlestyle}[1]{\par\medskip\Large #1}
\newcommand{\transsubtitlestyle}[1]{-- \Large\emph{ #1}}

\newcommand{\titlegroup}{
\ifdef{\subtitle}{\subtitlestyle{\subtitle}}{}
\ifdef{\transtitle}{\transtitlestyle{\transtitle}}{}
\ifdef{\transsubtitle}{\transsubtitlestyle{\transsubtitle}}{}}

\title{Public health round-up\titlegroup{}}
\date{ 11 2013}
\def\volume{91}
\def\issue{11}
\def\fpage{802}
\def\lpage{803}
\def\permissions{(c) World Health Organization (\textsc{who}) 2013. All rights
reserved.2013}

\begin{document}
\selectlanguage{english}
\section*{Metadados não aplicados}
\begin{itemize}
\item[\textbf{língua do artigo}]{Inglês}
\ifdef{\journalid}{\item[\textbf{journalid}] \journalid}{}
\ifdef{\journaltitle}{\item[\textbf{journaltitle}] \journaltitle}{}
\ifdef{\abbrevjournaltitle}{\item[\textbf{abbrevjournaltitle}]
\abbrevjournaltitle}{}
\ifdef{\issnppub}{\item[\textbf{issnppub}] \issnppub}{}
\ifdef{\issnepub}{\item[\textbf{issnepub}] \issnepub}{}
\ifdef{\publishername}{\item[\textbf{publishername}] \publishername}{}
\ifdef{\publisherid}{\item[\textbf{publisherid}] \publisherid}{}
\ifdef{\subject}{\item[\textbf{subject}] \subject}{}
\ifdef{\transtitle}{\item[\textbf{transtitle}] \transtitle}{}
\ifdef{\authornotes}{\item[\textbf{authornotes}] \authornotes}{}
\ifdef{\articleid}{\item[\textbf{articleid}] \articleid}{}
\ifdef{\articledoi}{\item[\textbf{articledoi}] \articledoi}{}
\ifdef{\volume}{\item[\textbf{volume}] \volume}{}
\ifdef{\issue}{\item[\textbf{issue}] \issue}{}
\ifdef{\fpage}{\item[\textbf{fpage}] \fpage}{}
\ifdef{\lpage}{\item[\textbf{lpage}] \lpage}{}
\ifdef{\permissions}{\item[\textbf{permissions}] \permissions}{}
\end{itemize}
\maketitle

\section{}

People with diabetes attend group classes at a hospital in the Russian city of
Stavropol on how
to manage their condition. World Diabetes Day, on 14 November, marks the end of
a four-year
International Diabetes Federation campaign to raise awareness about the disease
and how to prevent
it.

\section{Getting health into the 2015 climate change agreement}

World Health Organization (\textsc{who}) experts will join their colleagues from health
and environment
nongovernmental organizations at a Climate and Health Summit on 16 November to
prepare their case
for the strong inclusion of health in a global climate agreement.

The one-day summit in Warsaw, Poland, will take place in parallel with climate
change talks in
the city from 11–22 November, during which about 200 participating nations will
work towards
a new global agreement on greenhouse gas emissions to be signed in 2015.

“The UN Framework Convention on Climate Change (\textsc{unfccc}) process is taking more
account of
health and \textsc{who} is involved in operational mechanisms established to support
countries to mitigate
and adapt to the negative consequences of climate change,” said Dr Diarmid
Campbell-Lendrum, a scientist working with \textsc{who}'s Evidence and Policy in Emerging
Environmental Health Issues unit.

Health is included in two key articles of the \textsc{unfccc} and \textsc{who} is working with the
\textsc{unfccc}
secretariat to provide support to countries in designing the health component of
national plans for
adapting to climate change, he explained. In addition, \textsc{who} closely follows the
\textsc{unfccc} process to
identify opportunities for health – a key sector for building resilience to
climate change
effects.

“Countries are beginning to heed our message that well planned action to reduce
greenhouse
gas emissions can also bring big health gains, most obviously through reducing
air
pollution,” Campbell-Lendrum said.

The climate change talks will consider the latest findings by the International
Panel on Climate
Change (\textsc{ipcc}), which has described different effects of climate change on human
health in its
reports since the early 1990s.

The \textsc{ipcc}'s latest findings – the first of three parts of the Fifth Assessment
Report – released in September, added even more weight to already substantial
evidence that
climate change is happening and that people are causing it.

The second part of that report, to be released early next year, will consider
more fully both the
potentially negative effects of climate change on human health and the health
co-benefits of various
strategies for reducing carbon emissions, Campbell-Lendrum said.

\href{http://unfccc.int/2860.php}; \href{http://www.ipcc.ch/report/ar5/}

\section{\textsc{gavi} to consider Chinese vaccine}

The \textsc{gavi} Alliance board meeting in Cambodia this month will consider providing
financial support
for a Chinese vaccine against Japanese encephalitis that was prequalified by \textsc{who}
last month.

The vaccine, known as the Japanese encephalitis (live) vaccine, was added to
\textsc{who}'s list of
prequalified medicines last month.

Prequalification gives health products a \textsc{who} stamp of approval in terms of
safety and efficacy,
so that United Nations agencies can buy them in bulk. The Japanese encephalitis
vaccine is the first
vaccine produced in China to be prequalified by \textsc{who}.

If the \textsc{gavi} board meeting agrees to provide financial support, countries that
are eligible for
this support will be able to apply from 2014 and the United Nations Children's
Fund will lead
international procurement efforts for the vaccine.

Japanese encephalitis is an inflammation of the brain caused by infection with a
mosquito-borne
virus.

It is a major public health problem in parts of China, south-eastern region of
the Russian
Federation, as well as in south and south-east Asia. As there is no specific
treatment for Japanese
encephalitis, supportive care in a medical facility is important to reduce the
risk of death or
disability.

The disease can be prevented by vaccination. One dose of the vaccine is
sufficient to confer
protection and it can safely be administered to infants.

\href{http://www.who.int/mediacentre/news/releases/2013/japanese\_{}encephalitis
\_{}20131009}

A community health worker visits the home of a sick patient in a rural area in
Nepal, many
kilometres from the nearest health clinic. This month's cover photo shows the
importance of
providing access to health care no matter where people live. It illustrates the
theme of this
month's issue: human resources for universal health coverage.

\section{Plan to stop child TB deaths}

\textsc{who} and its partners launched a new plan last month to prevent an estimated 74
000 child
deaths from tuberculosis around the world.

The plan known as \textit{The roadmap for childhood tuberculosis: towards zero
deaths}

hinges on closer collaboration and joint planning between tuberculosis control
programmes, maternal
and child health services, and \textsc{hiv} services.

It estimates that US\$ 120 million per year would be needed to prevent these
deaths;
one third of this sum would provide \textsc{hiv} antiretroviral therapy and preventive
therapy (to prevent
active TB disease) to children co-infected with tuberculosis and \textsc{hiv}.

The funds would also contribute towards improving paediatric case detection and
developing better
medicines for children.

\textsc{who} estimates that up to 1 in 10 tuberculosis cases globally (6–10\% of all
cases)
are among children aged 15 years and less. But the real figure could be even
higher because
many children with tuberculosis are not detected due to difficulties in making
the diagnosis.

Getting more paediatric health professionals to actively screen for TB with
better and more rapid
diagnostics will help capture the full scope of the epidemic and reach more
children with
life-saving treatment sooner. Getting new drug formulations for children and
ultimately a new
vaccine would save thousands of lives.

\section{Implementation research and why we need it}

\textsc{who} released a new guidebook last month on implementation research, the study of
how best to
apply public health innovations to develop interventions in the field.

According to \textit{Implementation}
\textit{research in health: a practical guide}
many effective treatments, diagnostics,
vaccines and medical devices exist, but there is often little understanding of
how to deliver those
interventions in the real world.

The guide provides an introduction to basic concepts and language in
implementation research. It
briefly outlines what this field of study involves, and discusses the potential
benefits it holds
for public health practitioners.

It argues that implementation research should be an integral part of programme
planning and
execution and, as such, should be incorporated into programmes at the very
start.

\href{http://who.int/alliance-hpsr/alliancehpsr\_{}irpguide.pdf}

\section{\textsc{iarc} cancer monograph wins accolade}

Volume 100 of a series of monographs published by the International Agency for
Research on Cancer
(\textsc{iarc}) has been “highly commended” in the Public Health category of the 2013
British
Medical Association Medical Book Awards.

The shortlisted work, \textit{Review of human carcinogens}, is in itself a six-volume
summary of all the carcinogens included in the 99 preceding volumes of the
series, known as the \textsc{iarc}
Monographs on the Evaluation of Carcinogenic Risks to Humans, since 1971.

The \textsc{iarc} monographs identify environmental factors that can increase the risk of
human cancer,
including chemicals, occupational exposures, physical agents, biological agents
and lifestyle
factors. National health agencies can use this information as evidence for
action to prevent
exposure to potential carcinogens.

Commenting on Volume 100 of the \textsc{iarc} monographs, the judging panel said: “This
is an
important resource in that it defines the current state of evidence-based
thinking on cancer-causing
agents.”

\href{http://monographs.iarc.fr}

1 December – World \textsc{aids}
Day\href{http://www.who.int/campaigns/aids-day/2013/event}

3 December – International Day of Disabled Persons

\end{document}
