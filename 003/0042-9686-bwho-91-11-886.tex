% Generated by jats2tex@0.11.1.0
\documentclass{article}
\usepackage[T1]{fontenc}
\usepackage[utf8]{inputenc} %% *
\usepackage[portuges,spanish,english,german,italian,russian]{babel} %% *
\usepackage{amstext}
\usepackage{authblk}
\usepackage{unicode-math}
\usepackage{multirow}
\usepackage{graphicx}
\usepackage{etoolbox}
\usepackage{xtab}
\usepackage{enumerate}
\usepackage{hyperref}
\usepackage{penalidades}
\usepackage[footnotesize,bf,hang]{caption}
\usepackage[nodayofweek,level]{datetime}
\usepackage[top=0.85in,left=2.75in,footskip=0.75in]{geometry}
\newlength\savedwidth
\newcommand\thickcline[1]{\noalign{\global
\savedwidth
\arrayrulewidth
\global\arrayrulewidth 2pt}
\cline{#1}
\noalign{\vskip\arrayrulewidth}
\noalign{\global\arrayrulewidth\savedwidth}}
\newcommand\thickhline{\noalign{\global
\savedwidth\arrayrulewidth
\global\arrayrulewidth 2pt}
\hline
\noalign{\global\arrayrulewidth\savedwidth}}
\usepackage{lastpage,fancyhdr}
\usepackage{epstopdf}
\pagestyle{myheadings}
\pagestyle{fancy}
\fancyhf{}
\setlength{\headheight}{27.023pt}
\lhead{\includegraphics[width=10mm]{logo.png}}
\rhead{\ifdef{\journaltitle}{\journaltitle}{}
\ifdef{\volume}{vol.\,\volume}{}
\ifdef{\issue}{(\issue)}{}
\ifdef{\fpage}{\fpage--\lpage\,pp.}}
\rfoot{\thepage/\pageref{LastPage}}
\renewcommand{\footrule}{\hrule height 2pt \vspace{2mm}}
\fancyheadoffset[L]{2.25in}
\fancyfootoffset[L]{2.25in}
\lfoot{\sf \ifdef{\articledoi}{\articledoi}{}}
\setmainfont{Linux Libertine O}
\renewcommand*{\thefootnote}{\alph{footnote}}
\makeatletter
\newcommand{\fn}{\afterassignment\fn@aux\count0=}
\newcommand{\fn@aux}{\csname fn\the\count0\endcsname}
\makeatother

\newcommand{\journalid}{Bull World Health Organ}
\newcommand{\journaltitle}{Bulletin of the World Health Organization}
\newcommand{\abbrevjournaltitle}{Bull. World Health
Organ.}
\newcommand{\issnppub}{0042-9686}
\newcommand{\publishername}{World Health Organization}
\newcommand\articleid{\textsc{blt}.13.126656}
\newcommand\articledoi{\textsc{doi} 10.2471/\textsc{blt}.13.126656}
\def\subject{Round table}\newcommand{\subtitlestyle}[1]{-- \emph{#1}\medskip}
\newcommand{\transtitlestyle}[1]{\par\medskip\Large #1}
\newcommand{\transsubtitlestyle}[1]{-- \Large\emph{ #1}}

\newcommand{\titlegroup}{
\ifdef{\subtitle}{\subtitlestyle{\subtitle}}{}
\ifdef{\transtitle}{\transtitlestyle{\transtitle}}{}
\ifdef{\transsubtitle}{\transsubtitlestyle{\transsubtitle}}{}}

\title{Health workforce indicators: let's get real\titlegroup{}}
\author[{a}]{Boerma, Ties}
\author[{a}]{Siyam, Amani}
\affil[a]{World Health Organization}
\def\authornotes{Correspondence to Ties Boerma (\mbox{\mbox{\mbox{\mbox{\mbox{\mbox{\mbox{\mbox{\mbox{\mbox{\mbox{\mbox{\mbox{\mbox{\mbox{\mbox{e-mail}}}}}}}}}}}}}}}}: boermat@who.int).}
\date{ 11 2013}
\def\volume{91}
\def\issue{11}
\def\fpage{886}
\def\lpage{887}
\def\permissions{(c) World Health Organization (\textsc{who}) 2013. All rights
reserved.2013}
%%% Nota %%%%%%%%%%%%%%%%%%%%%%%%%%%%%%%%%%%%%%%%%%%%%%%%%%%%%%%%
\expandafter\newcommand\csname \endcsname{
None declared.}

\begin{document}
\selectlanguage{english}
\section*{Metadados não aplicados}
\begin{itemize}
\item[\textbf{língua do artigo}]{Inglês}
\ifdef{\journalid}{\item[\textbf{journalid}] \journalid}{}
\ifdef{\journaltitle}{\item[\textbf{journaltitle}] \journaltitle}{}
\ifdef{\abbrevjournaltitle}{\item[\textbf{abbrevjournaltitle}]
\abbrevjournaltitle}{}
\ifdef{\issnppub}{\item[\textbf{issnppub}] \issnppub}{}
\ifdef{\issnepub}{\item[\textbf{issnepub}] \issnepub}{}
\ifdef{\publishername}{\item[\textbf{publishername}] \publishername}{}
\ifdef{\publisherid}{\item[\textbf{publisherid}] \publisherid}{}
\ifdef{\subject}{\item[\textbf{subject}] \subject}{}
\ifdef{\transtitle}{\item[\textbf{transtitle}] \transtitle}{}
\ifdef{\authornotes}{\item[\textbf{authornotes}] \authornotes}{}
\ifdef{\articleid}{\item[\textbf{articleid}] \articleid}{}
\ifdef{\articledoi}{\item[\textbf{articledoi}] \articledoi}{}
\ifdef{\volume}{\item[\textbf{volume}] \volume}{}
\ifdef{\issue}{\item[\textbf{issue}] \issue}{}
\ifdef{\fpage}{\item[\textbf{fpage}] \fpage}{}
\ifdef{\lpage}{\item[\textbf{lpage}] \lpage}{}
\ifdef{\permissions}{\item[\textbf{permissions}] \permissions}{}
\end{itemize}
\maketitle

Health workforce indicators?\textsuperscript{[}\textsuperscript{1}\textsuperscript{]}
Those should be easy. We just need to count the numbers entering from training
institutions or through re-entry, the numbers working, and the numbers exiting.
If we
know where these people work, we have the distribution of health workers within
a
country, and if we also have information on their competencies, responsiveness
and
productivity, we can know about their performance.

Sound health workforce statistics enable countries to develop policies that
ensure the
equitable and effective distribution of the workforce. They can be used to
forecast
needs by making projections and to plan accordingly. They can also be the basis
for
implementing policies to improve performance and the regulation of the public
and
private sectors. These statistics would also allow for reliable global
monitoring of
progress, including progress towards achieving benchmark
targets,\textsuperscript{[}\textsuperscript{2}\textsuperscript{]}
and for monitoring the implementation of the \textsc{who} Global Code of Practice on the
International Recruitment of Health
Personnel.\textsuperscript{[}\textsuperscript{3}\textsuperscript{]}

And yet, health workforce statistics are fraught with measurement problems. This
is not
for lack of agreement on core indicators or because we do not know what needs to
be
monitored. And it is not because measuring indicators is complicated or costly,
as is
true in other areas of health. For some indicators, such as those that capture
productivity, more work is needed, but many indicators are well
established.\textsuperscript{[}\textsuperscript{4}\textsuperscript{]}\textsuperscript{,}\textsuperscript{[}\textsuperscript{5}\textsuperscript{]}

Health workforce information systems fail to deliver comprehensive, reliable and
timely
data in many countries. As a consequence, planning and policy-making are often
based on
very limited evidence and global monitoring in areas such as the implementation
of the
Global Code and the setting of benchmarks is conducted with inadequate country
statistics.

The challenges begin at the very basis: with the definition and classification
of health
workers. Indicators are intended for tracking progress over time, so
country-specific
definitions make it difficult to assess trends and conduct comparative analyses.
The
International Standard Classification of Occupations of the International Labour
Organization facilitates the mapping of country health labour data, but it does
little
to take the statistical dimension into account, as is done, for example, for the
International Classification of Diseases
(\textsc{icd}).\textsuperscript{[}\textsuperscript{6}\textsuperscript{]}
Some solvable issues are not well addressed, among them the classification of
non-physician clinicians and community health
workers.\textsuperscript{[}\textsuperscript{7}\textsuperscript{]}

Measuring the size and distribution of the health workforce involves drawing
data from
several sources, including sources outside the health
sector.\textsuperscript{[}\textsuperscript{4}\textsuperscript{]}
Currently too little is done to make use of these multiple, imperfect sources,
reconcile the numbers and develop a best estimate. Human resources for health
observatories aim to improve the information
base,\textsuperscript{[}\textsuperscript{8}\textsuperscript{]}
yet to date they have had little impact on the quality of health workforce data
and statistics.

It's time to get real. Reliable and comparable health workforce statistics are
essential and global partners and countries simply have not invested enough. It
is
necessary to invest in health workforce registries. Carefully designed, these
become
timely and consistent sources of data on the health workforce. Creating such
registries
will take time. In addition, a census of health facilities should be conducted
to update
a database of the public and private sector workforce and lay the groundwork for
a
continuous health workforce registry. Such a census could also be used to
collect
information on characteristics such as infrastructure, medicines, diagnostic
readiness
and the observance of universal precautions for the prevention of nosocomial
infections,
and could therefore provide a comprehensive picture of service availability and
readiness.\textsuperscript{[}\textsuperscript{9}\textsuperscript{]}
Finally, investments in strengthening country analytical capacity are crucial
for improving the quality of health workforce statistics.

\section*{References}
\begin{itemize}

\item[1] Cometto G, Witter S. Tackling health workforce challenges to
universal health coverage: setting targets and measuring progress. \textit{Bull
World Health Organ}
2013;91:881–5.

\item[2] \textit{The world health report 2006: working together for
health}. Geneva: World Health Organization; 2006. Available from:
http://www.who.int/whr/2006. [accessed 16 August 2013].

\item[3] \textit{\textsc{who} Global Code of Practice on the International Recruitment
of Health Personnel}. Geneva: World Health Organization; 2010.
Available from: http://www.who.int/hrh/migration/code/code\_{}en.pdf [accessed
16
August 2013].

\item[4] Dal Poz MR, Gupta N, Quain E, Soucat \textsc{alb}, editors. \textit{Handbook
on monitoring and evaluation of human resources for health with special
applications for low- and middle-income countries}. Geneva: World
Health Organization; 2009. Available from:
http://www.who.int/hrh/resources/handbook/en/index.html [accessed 16 August
2013].

\item[5] Pacqué-Margolis S, Ng C, Kauffman S. Human resources for health
(\textsc{hrh}) indicator compendium. Washington: United States Agency for International
Development; 2011. Available from:
http://capacityplus.org/files/resources/\textsc{hrh}\_{}Indicator\_{}Compendium.pdf
[accessed
16 August 2013].

\item[6] International classification of diseases and related health
problems, 10th revision. Geneva: World Health Organization; 2010. Available
from: http://apps.who.int/classifications/icd10/browse/2010/en [accessed 16
August 2013].

\item[7] Mullan F, Frehywoth S. Non-physician clinicians in 47 sub-Saharan
African countries. \textit{Lancet}
2007;370:2158–63. doi:
http://dx.doi.org/10.1016/S0140-6736(07)60785-5 \textsc{pmid}:17574662

\item[8] Gedik G, Dal Poz M. Human Resources for Health Observatories:
contributing to evidence-based policy decisions. \textit{Hum Res Health
Obs}
2012;10.

\item[9] O’Neill K, Takane M, Sheffel A, Abouzahr C, Boerma T. Monitoring
service delivery for universal health coverage: service availability and
readiness assessment (\textsc{sara}). \textit{Bull World Health Organ}
2013.
Forthcoming.

\end{itemize}

\end{document}
