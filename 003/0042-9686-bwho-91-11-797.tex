% Generated by jats2tex@0.11.1.0
\documentclass{article}
\usepackage[T1]{fontenc}
\usepackage[utf8]{inputenc} %% *
\usepackage[portuges,spanish,english,german,italian,russian]{babel} %% *
\usepackage{amstext}
\usepackage{authblk}
\usepackage{unicode-math}
\usepackage{multirow}
\usepackage{graphicx}
\usepackage{etoolbox}
\usepackage{xtab}
\usepackage{enumerate}
\usepackage{hyperref}
\usepackage{penalidades}
\usepackage[footnotesize,bf,hang]{caption}
\usepackage[nodayofweek,level]{datetime}
\usepackage[top=0.85in,left=2.75in,footskip=0.75in]{geometry}
\newlength\savedwidth
\newcommand\thickcline[1]{\noalign{\global
\savedwidth
\arrayrulewidth
\global\arrayrulewidth 2pt}
\cline{#1}
\noalign{\vskip\arrayrulewidth}
\noalign{\global\arrayrulewidth\savedwidth}}
\newcommand\thickhline{\noalign{\global
\savedwidth\arrayrulewidth
\global\arrayrulewidth 2pt}
\hline
\noalign{\global\arrayrulewidth\savedwidth}}
\usepackage{lastpage,fancyhdr}
\usepackage{epstopdf}
\pagestyle{myheadings}
\pagestyle{fancy}
\fancyhf{}
\setlength{\headheight}{27.023pt}
\lhead{\includegraphics[width=10mm]{logo.png}}
\rhead{\ifdef{\journaltitle}{\journaltitle}{}
\ifdef{\volume}{vol.\,\volume}{}
\ifdef{\issue}{(\issue)}{}
\ifdef{\fpage}{\fpage--\lpage\,pp.}}
\rfoot{\thepage/\pageref{LastPage}}
\renewcommand{\footrule}{\hrule height 2pt \vspace{2mm}}
\fancyheadoffset[L]{2.25in}
\fancyfootoffset[L]{2.25in}
\lfoot{\sf \ifdef{\articledoi}{\articledoi}{}}
\setmainfont{Linux Libertine O}
\renewcommand*{\thefootnote}{\alph{footnote}}
\makeatletter
\newcommand{\fn}{\afterassignment\fn@aux\count0=}
\newcommand{\fn@aux}{\csname fn\the\count0\endcsname}
\makeatother

\newcommand{\journalid}{Bull World Health Organ}
\newcommand{\journaltitle}{Bulletin of the World Health Organization}
\newcommand{\abbrevjournaltitle}{Bull. World Health Organ.}
\newcommand{\issnppub}{0042-9686}
\newcommand{\publishername}{World Health Organization}
\newcommand\articleid{\textsc{blt}.13.001113}
\newcommand\articledoi{\textsc{doi} 10.2471/\textsc{blt}.13.001113}
\def\subject{In this month's Bulletin}\newcommand{\subtitlestyle}[1]{--
\emph{#1}\medskip}
\newcommand{\transtitlestyle}[1]{\par\medskip\Large #1}
\newcommand{\transsubtitlestyle}[1]{-- \Large\emph{ #1}}

\newcommand{\titlegroup}{
\ifdef{\subtitle}{\subtitlestyle{\subtitle}}{}
\ifdef{\transtitle}{\transtitlestyle{\transtitle}}{}
\ifdef{\transsubtitle}{\transsubtitlestyle{\transsubtitle}}{}}

\title{In this month's \textit{Bulletin}
\titlegroup{}}
\date{ 11 2013}
\def\volume{91}
\def\issue{11}
\def\fpage{797}
\def\lpage{797}
\def\permissions{(c) World Health Organization (\textsc{who}) 2013. All rights
reserved.2013}

\begin{document}
\selectlanguage{english}
\section*{Metadados não aplicados}
\begin{itemize}
\item[\textbf{língua do artigo}]{Inglês}
\ifdef{\journalid}{\item[\textbf{journalid}] \journalid}{}
\ifdef{\journaltitle}{\item[\textbf{journaltitle}] \journaltitle}{}
\ifdef{\abbrevjournaltitle}{\item[\textbf{abbrevjournaltitle}]
\abbrevjournaltitle}{}
\ifdef{\issnppub}{\item[\textbf{issnppub}] \issnppub}{}
\ifdef{\issnepub}{\item[\textbf{issnepub}] \issnepub}{}
\ifdef{\publishername}{\item[\textbf{publishername}] \publishername}{}
\ifdef{\publisherid}{\item[\textbf{publisherid}] \publisherid}{}
\ifdef{\subject}{\item[\textbf{subject}] \subject}{}
\ifdef{\transtitle}{\item[\textbf{transtitle}] \transtitle}{}
\ifdef{\authornotes}{\item[\textbf{authornotes}] \authornotes}{}
\ifdef{\articleid}{\item[\textbf{articleid}] \articleid}{}
\ifdef{\articledoi}{\item[\textbf{articledoi}] \articledoi}{}
\ifdef{\volume}{\item[\textbf{volume}] \volume}{}
\ifdef{\issue}{\item[\textbf{issue}] \issue}{}
\ifdef{\fpage}{\item[\textbf{fpage}] \fpage}{}
\ifdef{\lpage}{\item[\textbf{lpage}] \lpage}{}
\ifdef{\permissions}{\item[\textbf{permissions}] \permissions}{}
\end{itemize}
\maketitle

This month the \textit{Bulletin}
explores the challenges of supplying enough human
resources to achieve universal health coverage. In the lead editorial, Mozart
Sales at al. (798)
introduce the issue and the Third Global Forum on Human Resources for Health at
which it will be
launched. Feng Zhao et al. (799) follow in explaining the necessity for change,
Alexandre Padilha et
al. (800) call on politicians worldwide to pay more attention to their health
systems and Viroj
Tangcharoensathien \& David B Evans (801) argue for better training on health
policy.

In the news section, Claudia Jurberg interviews Francisco Eduardo de Campos
(806–807)
about Brazil's effort to distribute doctors more evenly in rural areas. Priya
Shetty
(804–805) reports on efforts to train and employ more midwives in Ethiopia and
Somalia.
\section{Lao People's Democratic Republic \& South Africa}

\section{Providing health care in rural and remote areas}

James Buchan et al. (834–840) study implementation of the \textsc{who} guidelines on
rural
retention.

\section{Bangladesh, Brazil, Ethiopia, India, Islamic Republic of Iran, Pakistan
\&
Thailand}

\section{Overcoming obstacles}

Kate Tulenko et al. (847–852) examine factors for successful engagement of
community
health workers.

\section{Brazil, Ghana, Mexico \& Thailand}

\section{Available, accessible, acceptable and high-quality}

James Campbell et al. (853–863) look at the factors for effective coverage of
health
services.

\section{Cameroon}

\section{Addressing the shortfall}

S Kingue et al. (864–867) describe the effects of an emergency plan to supply
more health
workers.

\section{Sudan}

\section{Reaching all stakeholders}

Elsheikh Badr et al. (868–873) describe national coordination of a fragmented
workforce.

\section{Thailand}

\section{Improving matters quickly}

Viroj Tangcharoensathien et al. (874–880) describe how universal health coverage
was
achieved.

\section{Global}

\section{How much do doctors cost?}

P Hernandez-Peña et al. (808–815) quantify the wage bill for health
workers.

\section{Sticking to the code}

Amani Siyam et al. (816–823) report use of the \textit{\textsc{who} Global Code of
Practice on the
International Recruitment of Health Personnel.}

\section{Who delivers better care?}

Zohra S Lassi et al. (824–833) review the evidence on mid-level health workers.

\section{Analysing the market}

Barbara McPake et al. (841–846) examine factors influencing supply of – and
demand
for – health workers.

\section{Debating the measures}

Giorgio Cometto \& Sophie Witter (881–885) weigh the options for benchmarks and
monitoring frameworks. Ties Boerma \& Amani Siyam (886) argue for definitions,
registries and a
census of health workers. James Campbell (887–888) advocates for a workforce
that's
fit for purpose. Xenia Scheil-Adlung (888–889) points out that health workforce
benchmarks
should be compatible with sustainable development. Brook K Baker (889) explains
why targets
need to be patient-centred.

\section{4 million missing}

Robert Bollinger et al. (890–891) suggest that information technology can help
fill
workforce gaps.

\section{Covering all bases}

Angelica Sousa et al. (892–894) propose a health labour market framework.

\section{Rethinking the health system}

Sania Nishtar \& Johanna Ralston (895–896) suggest that the health workforce can
catalyse change.

\end{document}
