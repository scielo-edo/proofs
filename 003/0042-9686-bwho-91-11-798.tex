% Generated by jats2tex@0.11.1.0
\documentclass{article}
\usepackage[T1]{fontenc}
\usepackage[utf8]{inputenc} %% *
\usepackage[portuges,spanish,english,german,italian,russian]{babel} %% *
\usepackage{amstext}
\usepackage{authblk}
\usepackage{unicode-math}
\usepackage{multirow}
\usepackage{graphicx}
\usepackage{etoolbox}
\usepackage{xtab}
\usepackage{enumerate}
\usepackage{hyperref}
\usepackage{penalidades}
\usepackage[footnotesize,bf,hang]{caption}
\usepackage[nodayofweek,level]{datetime}
\usepackage[top=0.85in,left=2.75in,footskip=0.75in]{geometry}
\newlength\savedwidth
\newcommand\thickcline[1]{\noalign{\global
\savedwidth
\arrayrulewidth
\global\arrayrulewidth 2pt}
\cline{#1}
\noalign{\vskip\arrayrulewidth}
\noalign{\global\arrayrulewidth\savedwidth}}
\newcommand\thickhline{\noalign{\global
\savedwidth\arrayrulewidth
\global\arrayrulewidth 2pt}
\hline
\noalign{\global\arrayrulewidth\savedwidth}}
\usepackage{lastpage,fancyhdr}
\usepackage{epstopdf}
\pagestyle{myheadings}
\pagestyle{fancy}
\fancyhf{}
\setlength{\headheight}{27.023pt}
\lhead{\includegraphics[width=10mm]{logo.png}}
\rhead{\ifdef{\journaltitle}{\journaltitle}{}
\ifdef{\volume}{vol.\,\volume}{}
\ifdef{\issue}{(\issue)}{}
\ifdef{\fpage}{\fpage--\lpage\,pp.}}
\rfoot{\thepage/\pageref{LastPage}}
\renewcommand{\footrule}{\hrule height 2pt \vspace{2mm}}
\fancyheadoffset[L]{2.25in}
\fancyfootoffset[L]{2.25in}
\lfoot{\sf \ifdef{\articledoi}{\articledoi}{}}
\setmainfont{Linux Libertine O}
\renewcommand*{\thefootnote}{\alph{footnote}}
\makeatletter
\newcommand{\fn}{\afterassignment\fn@aux\count0=}
\newcommand{\fn@aux}{\csname fn\the\count0\endcsname}
\makeatother

\newcommand{\journalid}{Bull World Health Organ}
\newcommand{\journaltitle}{Bulletin of the World Health Organization}
\newcommand{\abbrevjournaltitle}{Bull. World Health Organ.}
\newcommand{\issnppub}{0042-9686}
\newcommand{\publishername}{World Health Organization}
\newcommand\articleid{\textsc{blt}.13.131110}
\newcommand\articledoi{\textsc{doi} 10.2471/\textsc{blt}.13.131110}
\def\subject{Editorials}\newcommand{\subtitlestyle}[1]{-- \emph{#1}\medskip}
\newcommand{\transtitlestyle}[1]{\par\medskip\Large #1}
\newcommand{\transsubtitlestyle}[1]{-- \Large\emph{ #1}}

\newcommand{\titlegroup}{
\ifdef{\subtitle}{\subtitlestyle{\subtitle}}{}
\ifdef{\transtitle}{\transtitlestyle{\transtitle}}{}
\ifdef{\transsubtitle}{\transsubtitlestyle{\transsubtitle}}{}}

\title{Human resources for universal health coverage: from evidence to policy
and
action\titlegroup{}}
\author[{a}]{Sales, Mozart}
\author[{b}]{Kieny, Marie-Paule}
\author[{c}]{Krech, Ruediger}
\author[{d}]{Etienne, Carissa}
\affil[a]{Ministry of Health}
\affil[b]{World Health Organization}
\affil[c]{World Health Organization}
\affil[d]{Pan American Health Organization}
\def\authornotes{Correspondence to Ruediger Krech (\mbox{\mbox{\mbox{\mbox{\mbox{\mbox{\mbox{\mbox{\mbox{\mbox{\mbox{\mbox{\mbox{\mbox{\mbox{\mbox{e-mail}}}}}}}}}}}}}}}}: krechr@who.int).}
\date{ 11 2013}
\def\volume{91}
\def\issue{11}
\def\fpage{798}
\def\lpage{798A}
\def\permissions{(c) World Health Organization (\textsc{who}) 2013. All rights
reserved.2013}

\begin{document}
\selectlanguage{english}
\section*{Metadados não aplicados}
\begin{itemize}
\item[\textbf{língua do artigo}]{Inglês}
\ifdef{\journalid}{\item[\textbf{journalid}] \journalid}{}
\ifdef{\journaltitle}{\item[\textbf{journaltitle}] \journaltitle}{}
\ifdef{\abbrevjournaltitle}{\item[\textbf{abbrevjournaltitle}]
\abbrevjournaltitle}{}
\ifdef{\issnppub}{\item[\textbf{issnppub}] \issnppub}{}
\ifdef{\issnepub}{\item[\textbf{issnepub}] \issnepub}{}
\ifdef{\publishername}{\item[\textbf{publishername}] \publishername}{}
\ifdef{\publisherid}{\item[\textbf{publisherid}] \publisherid}{}
\ifdef{\subject}{\item[\textbf{subject}] \subject}{}
\ifdef{\transtitle}{\item[\textbf{transtitle}] \transtitle}{}
\ifdef{\authornotes}{\item[\textbf{authornotes}] \authornotes}{}
\ifdef{\articleid}{\item[\textbf{articleid}] \articleid}{}
\ifdef{\articledoi}{\item[\textbf{articledoi}] \articledoi}{}
\ifdef{\volume}{\item[\textbf{volume}] \volume}{}
\ifdef{\issue}{\item[\textbf{issue}] \issue}{}
\ifdef{\fpage}{\item[\textbf{fpage}] \fpage}{}
\ifdef{\lpage}{\item[\textbf{lpage}] \lpage}{}
\ifdef{\permissions}{\item[\textbf{permissions}] \permissions}{}
\end{itemize}
\maketitle

The seminal role of human resources for health (\textsc{hrh}) in the attainment of
health-related goals
has long been recognized and was recently reaffirmed by the United Nations
General Assembly, which
identified the need for “an adequate, skilled, well-trained and motivated
workforce”
to accelerate progress towards universal health coverage
(\textsc{uhc}).\textsuperscript{[}\textsuperscript{1}\textsuperscript{]}
Yet, under existing affordability and sustainability constraints, countries at
all levels of socioeconomic development are confronted with challenges in trying
to match health
worker supply and demand. Against this backdrop, the Third Global Forum on Human
Resources for
Health, which will take place in Recife, Brazil, from 10 to 13 November 2013,
seeks to set out a
contemporary and forward-looking \textsc{hrh} agenda and to bolster political commitment
to support its
implementation.

Health workforce development is partly a technical process and, as such, it
requires expertise in
human resource planning, education and management. It is also, however, a
political process
requiring the will and the capacity to coordinate efforts on the part of
different sectors and
constituencies in society and different levels of government. This theme issue
on human resources
for \textsc{uhc} covers both of these aspects by providing examples of how countries have
aligned political
will and sound technical strategies and by presenting new analytical tools and
evidence surrounding
successful or promising innovative approaches.

Several success stories – from Brazil to
Sudan,\textsuperscript{[}\textsuperscript{2}\textsuperscript{]}
from Cameroon\textsuperscript{[}\textsuperscript{3}\textsuperscript{]}
to
Thailand,\textsuperscript{[}\textsuperscript{4}\textsuperscript{]}
from Ghana to Mexico\textsuperscript{[}\textsuperscript{5}\textsuperscript{]}
and Indonesia\textsuperscript{[}\textsuperscript{6}\textsuperscript{]}
– have sprung from efforts to improve the availability, accessibility,
acceptability and quality of the health workforce, with corresponding
improvements in health
outcomes. The pathways chosen have varied in accordance with the needs,
contextual factors and
opportunities specific to each setting. But as Padilha et al. point out, without
high-level
political commitment we will not be able to progress beyond piecemeal and
short-term approaches and
ensure the alignment and coordination of different sectors and constituencies in
support of
long-term human resource development
efforts.\textsuperscript{[}\textsuperscript{6}\textsuperscript{]}

Other articles in this theme issue contribute to strengthening the policy
frameworks and evidence
base surrounding \textsc{hrh} by: (i) helping us to understand the market forces
affecting \textsc{hrh};\textsuperscript{[}\textsuperscript{7}\textsuperscript{]}\textsuperscript{,}\textsuperscript{[}\textsuperscript{8}\textsuperscript{]}
(ii) highlighting best practices and lessons learnt in relation to the
retention of health workers in rural areas\textsuperscript{[}\textsuperscript{9}\textsuperscript{]}
and the
international migration of health
workers;\textsuperscript{[}\textsuperscript{10}\textsuperscript{]}

(iii) providing new evidence and recommendations on the effectiveness of
mid-level\textsuperscript{[}\textsuperscript{11}\textsuperscript{]}
and community-based\textsuperscript{[}\textsuperscript{12}\textsuperscript{]}
health workers and on the system support they require; and
(iv) identifying opportunities for innovation in \textsc{hrh} education and management
support through
the use of emerging technologies.\textsuperscript{[}\textsuperscript{13}\textsuperscript{]}

As countries aspire to achieve \textsc{uhc}, new demands will be placed on health
workers.\textsuperscript{[}\textsuperscript{14}\textsuperscript{]}
New competencies will be required of them as part of
a deeper transformation of professional
education,\textsuperscript{[}\textsuperscript{15}\textsuperscript{]}
which in future will have to contribute more broadly to building institutional
capacities.\textsuperscript{[}\textsuperscript{16}\textsuperscript{]}
Equipping trainees with clinical
skills will not suffice.

Implementing an \textsc{hrh} agenda conducive to the attainment of \textsc{uhc} will require both
more resources
and their more efficient use. Domestic spending on \textsc{hrh} is lower than is
typically assumed\textsuperscript{[}\textsuperscript{17}\textsuperscript{]}
and in many countries larger investments are both
necessary and possible. In settings where external support is still required,
the impact of
development assistance for \textsc{hrh} development can be maximized through more
strategic targeting.\textsuperscript{[}\textsuperscript{18}\textsuperscript{]}

Only systemic action can address deep-seated challenges in the area of \textsc{hrh}; only
sustained
political commitment can, in turn, provide a basis for such action. By linking
the evidence to the
policies and politics surrounding health workforce development, this theme issue
provides a
foundation for the Third Global Forum on Human Resources for Health and, more
generally, for a
health workforce discourse instrumental in the pursuit of
\textsc{uhc}.\textsuperscript{[}\textsuperscript{19}\textsuperscript{]}

We, the national and international partners convening the Third Global Forum on
Human Resources
for Health, encourage everyone to support an ambitious and transformative agenda
that places
citizens' right to health at the heart of development policies and that treats
progress in
the area of \textsc{hrh} as a key driver of broader health system development. We call
upon national leaders
to confirm their commitment to this agenda by creating a governance and policy
environment that is
conducive to the transformative development of \textsc{hrh} and by investing the
necessary resources in
health workforce development, deployment and management. We also call upon
health workforce planners
and managers to adopt and put in place effective, evidence-based policies.
Finally, we call upon the
international community to work together on the development of \textsc{hrh} as a shared
global priority and
to let all its actions be inspired by the principles of international
solidarity, multilateral
collaboration and mutual accountability.

\section*{References}
\begin{itemize}

\item[1] Resolution A/\textsc{res}.63/33. Global health and foreign policy. In: United
Nations
[Internet]. Sixty-third General Assembly of the United Nations, New York, 16
September 2008 to 14
September 2009. Resolutions. New York: United Nations; 2013. Available from:
http://www.un.org/en/ga/63/resolutions.shtml [accessed 3 October 2013].

\item[2] Badr E, Nazar MA, Afzal MM, Bile KM. Strengthening human resources for
health
through information, coordination and accountability mechanisms: the case of the
Sudan. \textit{Bull
World Health Organ}
2013;91:868–73.

\item[3] Kingue S, Rosskam E, Bela AC, Adjidja A, Codjia L. Strengthening human
resources for
health through multisectoral approaches and leadership: the case of Cameroon.
\textit{Bull World
Health Organ}
2013;91:864–7.

\item[4] Tangcharoensathien V, Limwattananon S, Suphanchaimat R, Patcharanarumol
W,
Sawaengdeea K, Putthasria W. Health workforce contributions to health system
development: a platform
for universal health coverage. Bull World Health Organ
2013;91:874–80.

\item[5] Campbell J, Buchan J, Cometto G, David B, Dussault G, Fogstad H et al.
Human
resources for health and universal health coverage: fostering equity and
effective coverage.
\textit{Bull World Health Organ}
2013;91:853–63.

\item[6] Padilha A, Kasonde J, Mukti G, Crisp N, Takemi K, Buch E. Human
resources for
universal health coverage: leadership needed. \textit{Bull World Health Organ}

2013;91:800–0A.

\item[7] Sousa A, Scheffler RM, Nyoni J, Boerma T. A comprehensive health labour
market
framework for universal health coverage. \textit{Bull World Health Organ}

2013;91:892–4.

\item[8] McPake B, Maeda A, Araujo EC, Lemiere C, El Maghraby A, Cometto G. Why
do health
labour market forces matter? \textit{Bull World Health Organ}

2013;91:841–6.

\item[9] Buchan J, Couper ID, Tangcharoensathien V, Thepannya K, Jaskiewicz W,
Perfilieva G
et al. Early implementation of \textsc{who} recommendations for the retention of health
workers in remote and
rural areas. \textit{Bull World Health Organ}
2013;91:834–40.

\item[10] Siyam A, Zurn P, Rø OC, Gedik G, Ronquillo K, Co CJ et al. Monitoring
the
implementation of the \textsc{who} Global Code of Practice on the International
Recruitment of Health
Personnel. \textit{Bull World Health Organ}
2013;91:816–23.

\item[11] Lassi ZS, Cometto G, Huicho L, Bhutta ZA. Quality of care provided by
mid-level
health workers: systematic review and meta-analysis. \textit{Bull World Health
Organ}

2013;91:824–33I.

\item[12] Tulenko K, Møgedal S, Afzal MM, Frymus D, Oshin A, Pate M et al.
Community health
workers for universal health-care coverage: from fragmentation to synergy.
\textit{Bull World Health
Organ}
2013;91:847–52.

\item[13] Bollinger R, Chang L, Jafari R, O’Callaghan T, Ngatia P, Settle D et
al. Leveraging
information technology to bridge the health workforce gap. \textit{Bull World
Health Organ}

2013;91:890–91.

\item[14] Nishtar S, Ralston J. Can human resources for health in the context of
noncommunicable disease control be a lever for health system changes?
\textit{Bull World Health
Organ}
2013;91:895–6.

\item[15] Frenk J, Chen L, Bhutta ZA, Cohen J, Crisp N, Evans T et al. Health
professionals
for a new century: transforming education to strengthen health systems in an
interdependent world.
\textit{Lancet}
2010;376:1923–58. doi: http://dx.doi.org/10.1016/S0140-6736(10)61854-5
\textsc{pmid}:21112623

\item[16] Tangcharoensathien V, Evans DB. Beyond clinical skills: key capacities
needed for
universal health coverage. \textit{Bull World Health Organ}
2013;91:801–1A.

\item[17] Hernandez-Peña P, Poullier JP, Van Mosseveld \textsc{cjm}, Van de Maele N,
Cherilova V,
Indikadahena C et al. Health worker remuneration in \textsc{who} Member States.
\textit{Bull World Health
Organ}
2013;91:808 –15.

\item[18] Zhao F, Squires N, Weakliam D, Van Lerberghe W, Soucat A, Toure K et
al. Investing
in human resources for health: the need for a paradigm shift. \textit{Bull World
Health
Organ}
2013;91:799–9A.

\item[19] Cometto G, Witter S. Tackling health workforce challenges to universal
health
coverage: setting targets and measuring progress. \textit{Bull World Health
Organ}

2013;91:881–5.

\end{itemize}

\end{document}
