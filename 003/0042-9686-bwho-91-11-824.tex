% Generated by jats2tex@0.11.1.0
\documentclass{article}
\usepackage[T1]{fontenc}
\usepackage[utf8]{inputenc} %% *
\usepackage[portuges,spanish,english,german,italian,russian]{babel} %% *
\usepackage{amstext}
\usepackage{authblk}
\usepackage{unicode-math}
\usepackage{multirow}
\usepackage{graphicx}
\usepackage{etoolbox}
\usepackage{xtab}
\usepackage{enumerate}
\usepackage{hyperref}
\usepackage{penalidades}
\usepackage[footnotesize,bf,hang]{caption}
\usepackage[nodayofweek,level]{datetime}
\usepackage[top=0.85in,left=2.75in,footskip=0.75in]{geometry}
\newlength\savedwidth
\newcommand\thickcline[1]{\noalign{\global
\savedwidth
\arrayrulewidth
\global\arrayrulewidth 2pt}
\cline{#1}
\noalign{\vskip\arrayrulewidth}
\noalign{\global\arrayrulewidth\savedwidth}}
\newcommand\thickhline{\noalign{\global
\savedwidth\arrayrulewidth
\global\arrayrulewidth 2pt}
\hline
\noalign{\global\arrayrulewidth\savedwidth}}
\usepackage{lastpage,fancyhdr}
\usepackage{epstopdf}
\pagestyle{myheadings}
\pagestyle{fancy}
\fancyhf{}
\setlength{\headheight}{27.023pt}
\lhead{\includegraphics[width=10mm]{logo.png}}
\rhead{\ifdef{\journaltitle}{\journaltitle}{}
\ifdef{\volume}{vol.\,\volume}{}
\ifdef{\issue}{(\issue)}{}
\ifdef{\fpage}{\fpage--\lpage\,pp.}}
\rfoot{\thepage/\pageref{LastPage}}
\renewcommand{\footrule}{\hrule height 2pt \vspace{2mm}}
\fancyheadoffset[L]{2.25in}
\fancyfootoffset[L]{2.25in}
\lfoot{\sf \ifdef{\articledoi}{\articledoi}{}}
\setmainfont{Linux Libertine O}
\renewcommand*{\thefootnote}{\alph{footnote}}
\makeatletter
\newcommand{\fn}{\afterassignment\fn@aux\count0=}
\newcommand{\fn@aux}{\csname fn\the\count0\endcsname}
\makeatother

\newcommand{\journalid}{Bull World Health Organ}
\newcommand{\journaltitle}{Bulletin of the World Health Organization}
\newcommand{\abbrevjournaltitle}{Bull. World Health Organ.}
\newcommand{\issnppub}{0042-9686}
\newcommand{\publishername}{World Health Organization}
\newcommand\articleid{\textsc{blt}.13.118786}
\newcommand\articledoi{\textsc{doi} 10.2471/\textsc{blt}.13.118786}
\def\subject{Systematic reviews}\newcommand{\subtitlestyle}[1]{--
\emph{#1}\medskip}
\newcommand{\transtitlestyle}[1]{\par\medskip\Large #1}
\newcommand{\transsubtitlestyle}[1]{-- \Large\emph{ #1}}

\newcommand{\titlegroup}{
\ifdef{\subtitle}{\subtitlestyle{\subtitle}}{}
\ifdef{\transtitle}{\transtitlestyle{\transtitle}}{}
\ifdef{\transsubtitle}{\transsubtitlestyle{\transsubtitle}}{}}

\title{Quality of care provided by mid-level health workers: systematic review
and
meta-analysis\titlegroup{}}
\newcommand{\transtitle}{Qualité des soins prodigués par les agents de santé
de niveau intermédiaire: revue systématique et méta-analyse}
\newcommand{\transtitle}{La calidad de la atención proporcionada por
trabajadores
sanitarios de nivel intermedio: examen sistemático y meta-análisis}
\newcommand{\transtitle}{جودةالرعايةالمقدمةبواسطةالعاملينالصحيينعلىالمستوىالمتوس
ط:
استعراضمنهجيوتحليلوصفي}
\newcommand{\transtitle}{中级卫生工作者提供的护理质量:系统回顾和元分析}
\newcommand{\transtitle}{Качество
медицинской
помощи,
предоставляемой
средним
медицинским
персоналом:
систематический
обзор и
метаанализ}
\author[{a}]{Lassi, Zohra S}
\author[{b}]{Cometto, Giorgio}
\author[{c}]{Huicho, Luis}
\author[{a}]{Bhutta, Zulfiqar A}
\affil[a]{Aga Khan University}
\affil[b]{World Health Organization}
\affil[c]{Universidad Peruana Cayetano Heredia}
\def\authornotes{Correspondence to Zulfiqar A Bhutta (\mbox{\mbox{\mbox{\mbox{\mbox{\mbox{\mbox{\mbox{\mbox{\mbox{\mbox{\mbox{\mbox{\mbox{\mbox{\mbox{e-mail}}}}}}}}}}}}}}}}:
zulfiqar.bhutta@aku.edu).}
\date{ 11 2013}
\def\volume{91}
\def\issue{11}
\def\fpage{824}
\def\lpage{833I}
\def\permissions{(c) World Health Organization (\textsc{who}) 2013. All rights
reserved.2013}
%%% Nota %%%%%%%%%%%%%%%%%%%%%%%%%%%%%%%%%%%%%%%%%%%%%%%%%%%%%%%%
\expandafter\newcommand\csname \endcsname{
The review was supported financially by the Global Health Workforce Alliance.}
%%% Nota %%%%%%%%%%%%%%%%%%%%%%%%%%%%%%%%%%%%%%%%%%%%%%%%%%%%%%%%
\expandafter\newcommand\csname \endcsname{
None declared.}

\begin{document}
\selectlanguage{english}
\section*{Metadados não aplicados}
\begin{itemize}
\item[\textbf{língua do artigo}]{Inglês}
\ifdef{\journalid}{\item[\textbf{journalid}] \journalid}{}
\ifdef{\journaltitle}{\item[\textbf{journaltitle}] \journaltitle}{}
\ifdef{\abbrevjournaltitle}{\item[\textbf{abbrevjournaltitle}]
\abbrevjournaltitle}{}
\ifdef{\issnppub}{\item[\textbf{issnppub}] \issnppub}{}
\ifdef{\issnepub}{\item[\textbf{issnepub}] \issnepub}{}
\ifdef{\publishername}{\item[\textbf{publishername}] \publishername}{}
\ifdef{\publisherid}{\item[\textbf{publisherid}] \publisherid}{}
\ifdef{\subject}{\item[\textbf{subject}] \subject}{}
\ifdef{\transtitle}{\item[\textbf{transtitle}] \transtitle}{}
\ifdef{\authornotes}{\item[\textbf{authornotes}] \authornotes}{}
\ifdef{\articleid}{\item[\textbf{articleid}] \articleid}{}
\ifdef{\articledoi}{\item[\textbf{articledoi}] \articledoi}{}
\ifdef{\volume}{\item[\textbf{volume}] \volume}{}
\ifdef{\issue}{\item[\textbf{issue}] \issue}{}
\ifdef{\fpage}{\item[\textbf{fpage}] \fpage}{}
\ifdef{\lpage}{\item[\textbf{lpage}] \lpage}{}
\ifdef{\permissions}{\item[\textbf{permissions}] \permissions}{}
\end{itemize}
\maketitle

Zohra S Lassi et al.Effectiveness of mid-level health workers
\begingroup

\begin{abstract}
\section{Objective}

To assess the effectiveness of care provided by mid-level health workers.

\section{Methods}

Experimental and observational studies comparing mid-level health workers and
higher level health
workers were identified by a systematic review of the scientific literature. The
quality of the
evidence was assessed using Grading of Recommendations Assessment, Development
and Evaluation
criteria and data were analysed using Review Manager.

\section{Findings}

Fifty-three studies, mostly from high-income countries and conducted at tertiary
care facilities,
were identified. In general, there was no difference between the effectiveness
of care provided by
mid-level health workers in the areas of maternal and child health and
communicable and
noncommunicable diseases and that provided by higher level health workers.
However, the rates of
episiotomy and analgesia use were significantly lower in women giving birth who
received care from
midwives alone than in those who received care from doctors working in teams
with midwives, and
women were significantly more satisfied with care from midwives. Overall, the
quality of the
evidence was low or very low. The search also identified six observational
studies, all from Africa,
that compared care from clinical officers, surgical technicians or non-physician
clinicians with
care from doctors. Outcomes were generally similar.

\section{Conclusion}

No difference between the effectiveness of care provided by mid-level health
workers and that
provided by higher level health workers was found. However, the quality of the
evidence was low.
There is a need for studies with a high methodological quality, particularly in
Africa – the
region with the greatest shortage of health workers.

\iflanguage{portuges}{\medskip\noindent\textbf{Palavras-chave:} \kwdgroup}{}
\iflanguage{english}{\medskip\noindent\textbf{Keywords:} \kwdgroupen}{}
\iflanguage{spanish}{\medskip\noindent\textbf{Palavras claves:} \kwdgroupes}{}
\iflanguage{french}{\medskip\noindent\textbf{Mots clés:} \kwdgroupfr}{}
\end{abstract}
\endgroup

\begingroup
\renewcommand{\section}[1]{\subsection*{#1}}
\begin{otherlanguage}{french}
\renewcommand{\abstractname}{Résumé}
\begin{abstract}
\section{Objectif}

Évaluer l'efficacité des soins fournis par les agents de santé de niveau
intermédiaire.

\section{Méthodes}

Des études expérimentales et observationnelles comparant des agents de santé
de niveaux intermédiaire et de niveau supérieur ont été
identifiées à l'aide d'une revue systématique de la documentation scientifique.
La qualité des éléments de preuve a été évaluée
à l'aide des critères \textsc{grade} (Grading of Recommendations Assissment, Development
and
Evaluation – Méthode d'évaluation des recommandations, de détermination,
d'élaboration et d'évaluation), et les données ont été
analysées à l'aide d'un gestionnaire d'examen.

\section{Résultats}

Cinquante-trois études ont été identifiées, la plupart provenant de
pays à revenu élevé, et menées dans des établissements de soins
tertiaires. En général, il n'y avait pas de différence entre
l'efficacité des soins prodigués par des agents de santé de niveau
intermédiaire dans les domaines de la santé maternelle et infantile et des
maladies
contagieuses et non contagieuses et ceux prodigués par des agents de santé de
niveau
supérieur. Cependant, les taux de recours à l'épisiotomie et aux
analgésiques étaient significativement moins élevés chez les femmes
accouchant avec la seule aide d'une sage-femme que chez les femmes prises en
charge par des docteurs
secondés par des sages-femmes, et les femmes étaient significativement plus
satisfaites des soins prodigués par les sages-femmes. Dans l'ensemble, la
qualité des
éléments de preuve était basse, voire très basse. La recherche a
également identifié six études observationnelles, provenant toutes d'Afrique,
qui comparaient les soins de praticiens cliniques, de techniciens chirurgicaux
ou de cliniciens
non-médecins avec les soins prodigués par des médecins. Les résultats
étaient généralement similaires.

\section{Conclusion}

Aucune différence n'a été constatée entre l'efficacité des
soins prodigués par des agents de santé de niveau intermédiaire et ceux fournis
par des agents de santé de niveau supérieur. Cependant, la qualité des
éléments de preuve était basse. Il est nécessaire d'effectuer des
études basées sur une méthodologie de haute qualité, en particulier en
Afrique, la région qui manque le plus d'agents de santé.

\ifdef{\kwdgroupfr}{\medskip\noindent\textbf{Mots clés:} \kwdgroupfr}{}
\end{abstract}
\end{otherlanguage}
\endgroup

\begingroup
\renewcommand{\section}[1]{\subsection*{#1}}
\begin{otherlanguage}{spanish}
\renewcommand{\abstractname}{Resumen}
\begin{abstract}
\section{Objetivo}

Evaluar la eficacia de la atención proporcionada por los trabajadores sanitarios
de nivel
intermedio.

\section{Métodos}

A través de un examen sistemático de la literatura científica se identificaron
diversos estudios experimentales y observacionales que comparaban a los
trabajadores sanitarios de
nivel intermedio con los de nivel superior. Se evaluó la calidad de las pruebas
científicas
con ayuda de los criterios \textsc{grade} y se empleó el programa Review Manager para el
análisis de los datos.

\section{Resultados}

Se identificaron 53 estudios, la mayoría de ellos de países de ingresos elevados
y que se
habían efectuado en centros de atención sanitaria terciaria. En general, no se
observaron
diferencias entre la eficacia de la atención prestada por los trabajadores de
salud de nivel
intermedio y la proporcionada por los trabajadores de salud de nivel superior en
las áreas de
salud materno-infantil y en relación a las enfermedades transmisibles y no
transmisibles. Sin
embargo, los índices de episiotomía y el uso de analgésicos fueron
significativamente
inferiores en las mujeres que dieron a luz únicamente con la ayuda de una
matrona en
comparación con aquellas cuya atención corrió a cargo de médicos que
trabajaron conjuntamente con matronas. Las mujeres estuvieron mucho más
satisfechas con el
trabajo de las matronas. En general, la calidad de las pruebas científicas fue
baja o muy baja. La
búsqueda también identificó seis estudios observacionales, todos ellos
realizados en África, que comparaban la atención de los encargados clínicos y la
de
los instrumentadores quirúrgicos o clínicos sin licencia para practicar medicina
con la de
los médicos. Los resultados fueron, en su mayoría, similares.

\section{Conclusión}

No se encontró diferencia alguna entre la eficacia de la atención proporcionada
por
trabajadores sanitarios de nivel intermedio o de nivel superior. No obstante, la
calidad de las
pruebas científicas era baja. Es necesario realizar estudios con una calidad
metodológica
alta, especialmente en África, la región con la mayor escasez de personal
sanitario.

\ifdef{\kwdgroupes}{\medskip\noindent\textbf{Palavras claves:} \kwdgroupes}{}
\end{abstract}
\end{otherlanguage}
\endgroup
ملخص\section{الغرض}

تقييم فعالية
الرعاية
المقدمة من
العاملين
الصحيين على
المستوى
المتوسط.

\section{الطريقة}

تم تحديد
الدراسات
التجريبية
والقائمة على
الملاحظة التي
تقارن
العاملين
الصحيين على
المستوى
المتوسط
والعاملين
الصحيين على
المستوى
الأعلى عن طريق
استعراض منهجي
للأبحاث
العلمية. وتم
تقييم جودة
البينّات
باستخدام
معايير تقدير
وتطوير وتقييم
التوصيات، وتم
تحليل
البيانات
باستخدام مدير
المراجعة.

\section{النتائج}

تم تحديد ثلاث
وخمسين دراسة،
معظمها من
البلدان
المرتفعة
الدخل وتم
إجراؤها في
منشآت الرعاية
المتخصصة.
وبشكل عام، لم
يتم العثور على
أي اختلاف بين
فعالية
الرعاية
المقدمة من
العاملين
الصحيين على
المستوى
المتوسط في
مجالات صحة
الأم والطفل
والأمراض
السارية وغير
السارية وتلك
المقدمة من
العاملين
الصحيين على
المستوى
الأعلى. ومع
ذلك، انخفضت
معدلات بضع
الفرج
واستخدام
المسكنات بشكل
كبير لدى
النساء اللاتي
يلدن وتلقين
الرعاية من
القابلات فقط،
عنها لدى
اللاتي تلقين
الرعاية من
الأطباء
العاملين في
فرق مع
القابلات،
وازداد مستوى
رضا النساء عن
الرعاية
المقدمة من
القابلات بشكل
كبير. وبشكل
عام، كانت جودة
البيانات
منخفضة أو
شديدة
الانخفاض. وحدد
البحث كذلك ست
دراسات قائمة
على الملاحظة،
جميعها من
أفريقيا،
قارنت الرعاية
المقدمة من
العاملين
السريريين أو
الاختصاصيين
الجراحيين أو
الخبراء
السريريين غير
الأطباء
بالرعاية
المقدمة من
الأطباء. وكانت
الحصائل
متشابهة بشكل
عام.

\section{الاستنتاج}

لم يتم العثور
على اختلاف بين
فعالية
الرعاية
المقدمة من
العاملين
الصحيين على
المستوى
المتوسط وتلك
المقدمة من
العاملين
الصحيين على
المستوى
الأعلى. ومع
ذلك، كانت جودة
البينّات
منخفضة. وثمة
حاجة لإجراء
دراسات ذات
جودة منهجية
عالية، لاسيما
في أفريقيا –
المنطقة التي
تعاني من أعلى
نقص في
العاملين
الصحيين.

摘要\section{目的}

评估中级卫生工作者所提供护理的效果。

\section{方法}

通过系统回顾科学文献,对比较中级卫生工作者和高级卫生工作者的实验和观察性研究进行确认。使用推荐等级的评估、制定与评价评估、制定和评价标准分级来评估证据的质量,并
使用Review
Manager分析数据。

\section{结果}

确认了53
项研究,这些研究大多数来自高收入国家,并且是在三级医院中执行的。一般而言,在孕产妇和儿童卫生以及传染病和非传染性疾病方面,中级卫生工作者和高级卫生工作者所提供护
理的效果没有差别。但是,较之由医生与助产士合作提供护理的产妇,其会阴侧切率和镇痛使用率显著低于只接受助产士护理的产妇,并且产妇对助产士的护理明显更加满意。整体而
言,证据的质量较低或非常低。此次研究还确定了六项观察性研究,这些研究都来自非洲,它们对临床人员、外科工作人员和非医师临床人员提供的护理与医生提供的护理进行比较。
结局大致相似。

\section{结论}

在中级卫生工作者和高级卫生工作者提供的护理效果之间没有发现区别。但是,证据的质量很低。需要进行方法质量较高的研究,尤其是在非洲——该地区卫生工作者最为短缺。

\begingroup
\renewcommand{\section}[1]{\subsection*{#1}}
\begin{otherlanguage}{russian}
\renewcommand{\abstractname}{Резюме}
\begin{abstract}
\section{Цель}

Оценить
качество
медицинской
помощи,
предоставляемой
средним
медицинским
персоналом.

\section{Методы}

На основе
систематического
обзора
научной
литературы
были отобраны
экспериментальные
и
обсервационные
исследования,
в которых
сравнивается
качество
услуг,
получаемых от
медицинского
персонала
среднего и
более
высокого
уровня.
Качество
собранных
данных
оценивалось
на основе
методологии \textsc{grade}
(система
градации и
оценки
качества
рекомендаций),
собранные
данные были
проанализированы
с помощью
программы Review Manager.

\section{Результаты}

Было отобрано
53 исследования,
главным
образом из
стран с
высокими
доходами,
проведенных в
учреждениях
специализированной
медицинской
помощи. В целом
не было
выявлено
разницы между
эффективностью
медицинской
помощи,
оказываемой
медперсоналом
среднего
уровня в
области
материнского
и детского
здоровья и
инфекционных
и
неинфекционных
заболеваний, и
помощью,
оказываемой
медицинскими
работниками
более
высокого
уровня. Однако
показатели
использования
эпизиотомии и
анальгезии
были
значительно
ниже при родах
женщин,
получавших
помощь только
от акушерок, по
сравнению с
теми родами,
которые вели
врачи,
работающие в
группах с
акушерками; и
женщины были
значительно
более
удовлетворены
уходом
акушерок. Но
качество этих
данных было
низким или
очень низким. В
процессе
поиска также
было выявлено
шесть
обсервационных
исследований,
все из Африки, в
которых
проводилось
сравнение
медицинского
ухода,
получаемого
от
сотрудников
клиник,
хирургических
техников и
медицинских
работников, не
являющихся
врачами.
Результаты в
целом были
сходными.

\section{Вывод}

Не обнаружено
никаких
отличий между
эффективностью
медицинской
помощи,
оказываемой
медперсоналом
среднего
уровня, и
помощью,
оказываемой
медицинскими
работниками
более
высокого
уровня. Однако
качество этих
доказательств
являлось
низким.
Существует
потребность в
изучении
данного
вопроса с
более высоким
методологическим
качеством,
особенно в
Африке,
регионе с
наиболее
острой
нехваткой
работников
здравоохранения.

\end{abstract}
\ifdef{\kwdgroupru}{\medskip\noindent\textbf{ключевые слова:} \kwdgroupru}{}
\end{otherlanguage}
\endgroup
\section{Introduction}

In 2000, 189 countries adopted the United Nation's Millennium Declaration and
its eight
Millennium Development Goals, including Goals 4, 5 and 6, which are directly
related to health.
However, progress towards achieving the associated health targets falls far
below expectations,
especially in developing countries. Recent reviews have clearly identified
interventions that can
have a positive effect on maternal and child health and neonatal survival but
implementing them
throughout the general population has been hampered by a lack of trained and
motivated health
workers.\textsuperscript{[}\textsuperscript{1}\textsuperscript{]}\textsuperscript{–}\textsuperscript{[}\textsuperscript{6}\textsuperscript{]}
Moreover, the poor performance of health systems in delivering
effective, evidence-based interventions for priority health conditions has been
linked to the poor
retention, inadequate performance and poor motivation of health workers, as well
as to shortages of
personnel and their maldistribution. As health systems around the world and the
international health
community increasingly embrace the goal of universal health coverage, which will
inevitably result
in greater demands on health systems and existing health workers, the need to
address these
shortcoming is becoming imperative.\textsuperscript{[}\textsuperscript{7}\textsuperscript{]}
In parallel,
there is growing recognition that skilled and semi-skilled mid-level health
workers, who are
sometimes referred to as “outreach and facility health workers”, can play a
major role
in community mobilization and in delivering a range of health-care services.

Although mid-level health workers have been defined in a variety of ways (Table
1), the definitions commonly agree that they will have received shorter
training than physicians but will perform some of the same
tasks.\textsuperscript{[}\textsuperscript{10}\textsuperscript{]}
Typically, these workers follow certified training courses and receive
accreditation for their work.\textsuperscript{[}\textsuperscript{10}\textsuperscript{]}
Many, such as
nurse auxiliaries and medical assistants, undergo shorter training than
physicians and the scope of
their practice is narrower, but this is not necessarily the case for all. For
example, sometimes
nurses and nurse practitioners spend more than 5 years in training and perform
some of the
same tasks as doctors. Similarly, non-physician clinicians may have, in total,
spent an equal amount
of time in training as medical doctors and may perform a comparable range of
tasks, including
surgery. Despite differences in the roles and training of mid-level health
workers and despite a
continuing struggle for their acceptance, today many countries rely ever more
heavily on these
workers to improve the coverage and equity of health
care.\textsuperscript{[}\textsuperscript{11}\textsuperscript{]}
Although mid-level health workers have played a vital role in many
countries' health-care systems for over 100 years, interest in them has been
renewed
only in the past 10 years, principally because of the serious shortage of health
workers in
many developing countries, the burden of diseases such as human immunodeficiency
virus (\textsc{hiv})
infection and the emerging importance of other conditions, such as
noncommunicable diseases. Many
African and Asian countries have successfully invested in these
workers.\textsuperscript{[}\textsuperscript{12}\textsuperscript{]}\textsuperscript{–}\textsuperscript{[}\textsuperscript{15}\textsuperscript{]}

\textsc{who}, World Health Organization.

Our aim was to test the hypothesis that mid-level health workers are as
effective as higher level
health workers at providing good quality care in priority areas of the health
service. We also hoped
to increase understanding of their effectiveness and of how they can best be
integrated into
national health-care systems.

\section{Methods}

We performed a systematic review of studies on the role of mid-level health
workers in delivering
to the general population health-care services that are associated with the
achievement of
Millennium Development Goals on health and nutrition or with the management of
noncommunicable
diseases. We included all randomized and nonrandomized controlled trials,
controlled
before-and-after trials and interrupted time-series studies. Less rigorously
designed studies, such
as observational (cohort and case–control) and descriptive studies, were also
examined to
understand the context within which mid-level health worker programmes are
implemented, the types of
health-care providers involved, the types of interventions delivered and the
outcomes obtained. We
aimed to compare the effectiveness of: (i) different kinds of mid-level health
workers;
(ii) mid-level health workers and doctors or community health workers; and
(iii) mid-level health workers working alone or in a team.

For the purpose of this study, a mid-level health worker was defined as a
health-care provider
who is not a medical doctor or physician but who provides clinical care in the
community or at a
primary care facility or hospital. He or she may be authorized and regulated to
work autonomously,
to diagnose, manage and treat illness, disease and impairments, or to engage in
preventive care and
health promotion at the primary- or secondary-health-care level. The definition
includes midwives,
nurses, auxiliary nurses, nurse assistants, non-physician clinicians and
surgical technicians (Table 2). Workers who specialize in health administration
or
who perform only administrative tasks and those who provide rehabilitative or
dentistry services
were excluded. However, no type of patients or recipient of health services was
excluded.

A systematic search of the Cochrane Library, Medline, Embase and Cinahl
databases, the Latin
America and the Caribbean database \textsc{lilacs} and the Social Sciences Citation Index
was performed,
without language restrictions. Articles in both peer-reviewed and grey
literature were included and
the authors of relevant papers were contacted to help identify additional
published or unpublished
works.

The main health-care outcomes we considered were morbidity, mortality, outcomes
associated with
care delivery, health status, quality of life, service utilization and the
patient's
satisfaction with care. Two review authors independently extracted all outcome
information. Data
were collected on all health workers and care recipients involved, on
health-care settings and on
each study's design and outcomes.

The statistical analysis was performed using Review Manager (Nordic Cochrane
Centre, Copenhagen,
Denmark). Risk ratios (RRs) and mean differences, with 95\% confidence intervals
(CIs), were
calculated for dichotomous and continuous variables, respectively. Study
heterogeneity was assessed
using \textit{I}\textsuperscript{2}
and \textit{χ}\textsuperscript{2}
statistics. Two
review authors independently assessed the risk of bias in each study using a
form describing
standard criteria, which was obtained from the Cochrane Effective Practice and
Organisation of Care
Group.\textsuperscript{[}\textsuperscript{18}\textsuperscript{]}
We analysed the quality of the evidence
supporting study findings using the approach developed by the Grading of
Recommendations Assessment,
Development and Evaluation (\textsc{grade}) working
group.\textsuperscript{[}\textsuperscript{19}\textsuperscript{]}\textsuperscript{,}\textsuperscript{[}\textsuperscript{20}\textsuperscript{]}
The quality of
the evidence for each outcome was rated high, moderate, low or very low.

\section{Results}

The search identified 24 246 database records, which led to the retrieval of
documentation
on 327 studies for a full text review (Fig.~\ref{fig:F1}
). Of the
327, 53 met the eligibility criteria and were included in the review (Table 3,
available at: \href{http://www.who.int/bulletin/volumes/91/11/13-118786}). Most
studies compared either care
provided by midwives with that provided by doctors working in a team along with
midwives or care
provided by nurses with that provided by doctors. Moreover, most were conducted
in high-income
countries and at tertiary care facilities. The studies were experimental in
design and their results
were pooled for the meta-analysis (Table 4). Since the
evidence in all studies was found to be of low or very low quality, as assessed
using \textsc{grade}
criteria, the findings of the meta-analysis should be interpreted with caution.

Database search for experimental studies of mid-level health workers'
effectiveness,
1973–2012

\textsc{art}, antiretroviral therapy; \textit{\textsc{dsm}-\textsc{iii}-R, Diagnostic and Statistical
Manual of Mental
Disorders}, 3rd edition revised; \textsc{hiv}, human immunodeficiency virus.

\textsc{art}, antiretroviral therapy; CI, confidence interval; NA, not applicable; RR,
risk ratio.

Thirteen of the 53 studies\textsuperscript{[}\textsuperscript{21}\textsuperscript{]}\textsuperscript{–}\textsuperscript{[}\textsuperscript{33}\textsuperscript{]}

compared the care provided by midwives with that provided by doctors working in
a team with
midwives. On meta-analysis, no significant difference in the antenatal
hospitalization rate was
found between care provided by midwives alone and that provided by doctors
working with midwives
(RR: 0.95; 95\% CI: 0.79–1.13). However, the absence of intrapartum analgesia
was more likely
with care from midwives alone (RR: 1.13; 95\% CI: 0.96–1.33), but not
significantly so, and
the use of opiate or regional anaesthesia was significantly less likely (Table
4). Episiotomy was also significantly less likely with care from
midwives alone (Fig.~\ref{fig:F2}
). However, there was no
significant difference in rates for the induction of labour, instrumental
delivery or caesarean
section (Table 4). The postpartum haemorrhage rate was
not significantly lower with care from midwives alone and there was no
significant difference
between the groups in the rate of fetal or neonatal death, preterm birth or
admission to the
neonatal intensive care unit (Table 4).

Forest plot showing the risk of episiotomy when pregnancy care is provided only
by midwives
versus when it is provided by obstetricians or other types of doctors as part of
a team including
midwives, 1993–2012

CI, confidence interval; RR, risk ratio.

Note: The values to the left of the 1 indicate a lower risk of episiotomy when
pregnancy care is
provided only by midwives and those to the right of 1 indicate a higher risk
when the care is
administered by obstetricians or other types of doctors as part of a team
including midwives.

In one study, women were significantly more satisfied with antenatal care
provided by midwives
alone but there was no significant difference between the groups in satisfaction
with intrapartum or
postpartum care.\textsuperscript{[}\textsuperscript{31}\textsuperscript{]}
Turnball et al.\textsuperscript{[}\textsuperscript{30}\textsuperscript{]}
also reported that women were more satisfied with
care from midwives alone than care from doctors working with midwives in a team.
Wolke et al.\textsuperscript{[}\textsuperscript{33}\textsuperscript{]}
compared the level of satisfaction with health
workers in general between groups of patients managed by midwives and those
managed by junior
paediatricians: the care provided by midwives was perceived as being
significantly better than that
provided by physicians (RR: 1.23; 95\% CI: 1.10–1.37).

Four of the 53 studies\textsuperscript{[}\textsuperscript{34}\textsuperscript{]}\textsuperscript{–}\textsuperscript{[}\textsuperscript{37}\textsuperscript{]}

compared auxiliary nurse midwives with doctors. There was no significant
difference in the
likelihood of an incomplete abortion between groups of patients managed by
auxiliary nurse midwives
and those managed by doctors (RR: 0.93; 95\% CI: 0.45–1.90). Nor was the
likelihood of a
complication during (RR: 3.07; 95\% CI: 0.16–59.1) – or an adverse event after
(RR:
1.36; 95\% CI: 0.54–3.40) – manual vacuum aspiration significantly greater with
auxiliary nurse midwives. Similarly, there was no difference between the groups
in postoperative
complications in women who underwent tubal ligation or in those who were
referred to a specialist
after insertion of an intrauterine device (Table 4).

One study\textsuperscript{[}\textsuperscript{38}\textsuperscript{]}
compared the effects of
antiretroviral therapy (\textsc{art}) in patients managed by nurses and those managed by
doctors. There was
no significant difference in the likelihood of \textsc{art} failure between groups of
patients managed by
nurses and those managed by doctors (RR: 1.08; 95\% CI: 0.39–2.14). Nor was
there any
difference in mortality, failure of viral suppression or immune recovery between
the groups.

The search also identified one study\textsuperscript{[}\textsuperscript{39}\textsuperscript{]}
that
compared nursing care of depression in the general population with standard
care. There was no
significant difference in measures of depression between patients managed by
nurses compared with
those managed by physicians (RR: 1.28; 95\% CI: 0.83–1.98).

Twenty-eight studies\textsuperscript{[}\textsuperscript{40}\textsuperscript{]}\textsuperscript{–}\textsuperscript{[}\textsuperscript{45}\textsuperscript{]}\textsuperscript{,}\textsuperscript{[}\textsuperscript{47}\textsuperscript{]}\textsuperscript{–}\textsuperscript{[}\textsuperscript{51}\textsuperscript{]}\textsuperscript{,}\textsuperscript{[}\textsuperscript{53}\textsuperscript{]}\textsuperscript{–}\textsuperscript{[}\textsuperscript{69}\textsuperscript{]}

compared the effectiveness of care provided by nurses and care provided by
doctors in patients with
chronic diseases, such as heart disease and diabetes. Most concerned secondary
and tertiary care in
developed countries. The meta-analysis showed that care provided by nurses was
as effective as care
provided by doctors: no significant difference between the groups was found in
the need for a repeat
consultation, improved physical functioning, attendance at follow-up visits or
attendance at an
emergency department after receiving care (Table 4).
However, dissatisfaction was significantly lower with care received from nurses
than with that
received from doctors (RR: 0.20; 95\% CI: 0.14–0.26). The likelihood of death at
12-month
follow-up was also lower with care from nurses and the likelihood of compliance
with drug treatment
was higher (Table 4). However, these last two findings
are based on the results of only one study.

All of the lower quality, prospective observational studies identified came from
Africa and
compared care delivered by clinical officers, surgical technicians or
non-physician clinicians with
that delivered by doctors.

Six observational studies compared the effectiveness of care provided by
clinical officers and
surgical technicians with that of care provided by
doctors.\textsuperscript{[}\textsuperscript{70}\textsuperscript{]}\textsuperscript{–}\textsuperscript{[}\textsuperscript{75}\textsuperscript{]}

Detailed descriptions of the interventions and types of mid-level health workers
involved in these
studies are provided in Table 5 (available at:
\href{http://www.who.int/bulletin/volumes/91/11/13-118786}). Since the studies
were not
experimental in design, data could not be pooled for analysis. Two studies from
Malawi compared the
outcomes of surgical procedures carried out by clinical officers and medical
officers (i.e.
doctors).\textsuperscript{[}\textsuperscript{70}\textsuperscript{]}\textsuperscript{,}\textsuperscript{[}\textsuperscript{71}\textsuperscript{]}
In the prospective cohort study from Malawi, there was no significant
difference in postoperative maternal health outcomes, such as fever, wound
infection, the need for
re-operation and maternal death, after emergency obstetric procedures performed
by clinical officers
or by medical officers (RR: 0.99; 95\% CI: 0.95–1.03). In particular, there was
no significant
difference in the likelihood of a stillbirth with procedures performed by
clinical officers (RR:
0.75; 95\% CI: 0.52–1.09) or in the likelihood of early neonatal death (RR:
1.40; 95\% CI:
0.51–3.87). Although 22 maternal deaths occurred in 1875 procedures performed by
clinical
officers compared with 1 in 256 procedures performed by medical officers, the
difference was not
significant. In a prospective cohort study from
Mozambique,\textsuperscript{[}\textsuperscript{72}\textsuperscript{]}
haematomas occurred significantly more often after surgery performed by a
surgical technician than after surgery performed by an obstetrician (odds ratio:
2.2; 95\% CI:
1.3–3.9). Finally, a retrospective cohort study from the United Republic of
Tanzania\textsuperscript{[}\textsuperscript{73}\textsuperscript{]}
found no difference in maternal mortality or
perinatal mortality between care provided by an assistant medical officer and
that provided by a
medical officer.

\textsc{art}, antiretroviral therapy.

\section{Discussion}

The meta-analysis showed that the outcomes of numerous interventions in the
areas of maternal and
child health and communicable and noncommunicable diseases were similar when the
interventions were
performed by mid-level health workers or higher level health workers. However,
this finding must be
interpreted with caution as the evidence obtained in the systematic review was
generally of low or
very low quality.

Mid-level health workers play an important role in maternal and child health
since midwives are
the primary health-care providers in many settings. The results of our
meta-analysis indicate that
antenatal care provided by midwives alone gave comparable results on most
outcome measures to care
provided by doctors working in a team with midwives. In addition, mothers were
more satisfied with
neonatal examinations performed by midwives alone. Midwives can provide
continuity of care after
childbirth and can advise mothers on other health-care issues concerning
neonates, such as
breastfeeding.

Mid-level health workers often care for patients with chronic conditions such as
diabetes
mellitus and hypertension. Our meta-analysis indicated that patients were
significantly more
satisfied with care received from nurses than from doctors, though the evidence
available was of low
quality. Moreover, care provided by nurses was as effective as that provided by
doctors. Another
consideration is that consultations with mid-level health workers are less
expensive for
patients.

If health-related Millennium Development Goals are to be achieved, health
systems will have to be
strengthened so that more countries can deliver a wider range of health services
on a much larger
scale. It has been claimed that better quality health services could be achieved
using the existing
workforce, but there is compelling evidence that the number of people with
access to health-care
services is directly correlated with the number of health service
providers.\textsuperscript{[}\textsuperscript{76}\textsuperscript{]}
Furthermore, there is also a correlation between the health of the
population and the density of qualified health-care
workers.\textsuperscript{[}\textsuperscript{77}\textsuperscript{]}
Thus, the number of health-care workers has a positive effect not only on
access to health care but also on health outcomes. Clearly, any strategy that
aims to increase the
scope or reach of the health-care services must consider long-, medium- and
short-term initiatives
for increasing the skills and retention of health-care workers.

Although the use of mid-level health workers instead of medical doctors has
proved successful in
various contexts, such as in performing surgery, providing health-care services,
health promotion
and education and providing \textsc{art}, the quality of care can be poor when mid-level
health workers are
not properly supervised or are inadequately
trained.\textsuperscript{[}\textsuperscript{78}\textsuperscript{]}
Moreover, these factors can also have a negative effect on staff retention.
Once it has been accepted that less-qualified health-care workers can provide as
good a service as
more qualified workers, attention should shift to optimizing the skills mix of
the workforce. This
would mitigate the effect of personnel shortages and help countries achieve the
Millennium
Development Goals.

This meta-analysis provides evidence supporting the concept of task-sharing,
which is defined as
the situation in which health-care tasks are shared, as part of a team-based
approach to the
delivery of care, with either existing or new health workers who have been
trained for only a
limited period or within only a narrow field. Task-sharing can help achieve the
new paradigm of
universal health coverage as well as health-related Millennium Development
Goals. In addition,
mid-level health workers are less costly to train and employ than doctors and
they are easier to
retain in rural areas. However, it must be remembered that task-sharing alone
cannot produce
large-scale changes where there is a shortage of personnel. Any task-sharing
strategy should be
implemented alongside other strategies designed to increase the total number of
health-care
workers.\textsuperscript{[}\textsuperscript{79}\textsuperscript{]}\textsuperscript{–}\textsuperscript{[}\textsuperscript{82}\textsuperscript{]}

The main obstacle to ensuring that mid-level health workers can help improve
health outcomes is
that they are often ignored by government policies, health workforce strategies
and health system
support measures, despite their widespread use. Until these workers are more
comprehensively taken
into account and supported, their potential contribution will not be fully
realized.

This review has several limitations. First, most studies reviewed did not fully
describe the
characteristics of the mid-level health workers involved; in particular, the
level and amount of
training and supervision provided were not reported. Second, the meta-analysis
included few studies
of the role of mid-level health workers in \textsc{hiv} prevention and care, mental
health or nutrition.
Third, the quality of the evidence in the studies we identified was low or very
low and, in
particular, the majority of studies from Africa on non-physician clinicians and
clinical officers
were not experimental. Therefore, the results of these studies could not be
pooled to generate
evidence on the effectiveness of mid-level health workers.

There is a need for more studies of a high methodological quality, particularly
experimental
studies in primary health care and developing countries. In addition, further
research is required
on the effectiveness of mid-level health workers in low- and middle- income
settings, where the
challenge of accessing essential health services is greatest. There is also a
remarkable dearth of
information on the cost-effectiveness of programmes involving these health
workers and on whether
these programmes help ensure that care can be accessed on an equitable basis.
Finally, there is a
need for a systematic review to identify factors that determine whether
interventions involving
mid-level health workers are sustainable when scaled-up.

In conclusion, we found no difference between the effectiveness of care provided
by mid-level
health workers and that provided by higher level health workers. However, the
quality of the
evidence was low or very low. Better quality trials with longer follow-ups are
needed, particularly
in Africa. Countries in danger of missing health-related Millennium Development
Goals should
continue to scale up health-care interventions involving community health
workers and mid-level
health workers. Both national and subnational policies are needed to reduce the
shortfall in human
resources for health: the skills required by mid-level health workers and their
roles should be
clearly defined with reference to the level of demand from the local community
and changing disease
patterns in the country.

\section*{References}
\begin{itemize}

\item[1] Bhutta ZA, Ahmed T, Black RE, Cousens S, Dewey K, Giugliani E et al.;
Maternal and
Child Undernutrition Study Group. What works? Interventions for maternal and
child undernutrition
and survival. \textit{Lancet}
2008;371:417–40. doi:
http://dx.doi.org/10.1016/S0140-6736(07)61693-6 \textsc{pmid}:18206226

\item[2] Bhutta ZA, Ali S, Cousens S, Ali TM, Haider BA, Rizvi A et al.
Interventions to
address maternal, newborn, and child survival: what difference can integrated
primary health care
strategies make? \textit{Lancet}
2008;372:972–89. doi:
http://dx.doi.org/10.1016/S0140-6736(08)61407-5 \textsc{pmid}:18790320

\item[3] Bhutta ZA, Darmstadt GL, Haws RA, Yakoob MY, Lawn JE. Delivering
interventions to
reduce the global burden of stillbirths: improving service supply and community
demand. \textit{\textsc{bmc}
Pregnancy Childbirth}
2009;9(Suppl 1):S7. doi: http://dx.doi.org/10.1186/1471-2393-9-S1-S7
\textsc{pmid}:19426470

\item[4] Campbell \textsc{omr}, Graham WJ. Lancet Maternal Survival Series steering
group. Strategies
for reducing maternal mortality: getting on with what works. \textit{Lancet}

2006;368:1284–99. doi: http://dx.doi.org/10.1016/S0140-6736(06)69381-1
\textsc{pmid}:17027735

\item[5] Kerber KJ, de Graft-Johnson JE, Bhutta ZA, Okong P, Starrs A, Lawn JE.
Continuum of
care for maternal, newborn, and child health: from slogan to service delivery.
\textit{Lancet}
2007;370:1358–69. doi: http://dx.doi.org/10.1016/S0140-6736(07)61578-5
\textsc{pmid}:17933651

\item[6] Haws RA, Thomas AL, Bhutta ZA, Darmstadt GL. Impact of packaged
interventions on
neonatal health: a review of the evidence. \textit{Health Policy Plan}
2007;22:193–215. doi:
http://dx.doi.org/10.1093/heapol/czm009 \textsc{pmid}:17526641

\item[7] Resolution A/\textsc{res}.33/63. Global health and foreign policy. In: General
Assembly of
the United Nations [Internet]. Resolutions. New York: \textsc{who}; 2013 (A/\textsc{res}/33/63).
Available from:
http://www.who.int/trade/foreignpolicy/en/ [accessed 26 August 2013].

\item[8] \textit{Mid-level and nurse practitioners in the Pacific: models and
issues}. Manila: World Health Organization, Western Pacific Regional Office; 2001.
Available
from: http://whqlibdoc.who.int/wpro/2001/a76187.pdf [accessed 26 August 2013].

\item[9] Dovlo D. Using mid-level cadres as substitutes for internationally
mobile health
professionals in Africa: a desk review. \textit{Hum Resour Health}

2004;2:7.

\item[10] Lehman U. \textit{Mid-level health workers. The state of the evidence
on programmes,
activities, costs and impact on health outcomes: a literature review}. Geneva: World Health
Organization; 2008. Available from:
http://www.who.int/hrh/\textsc{mlhw}\_{}review\_{}2008.pdf [accessed 26 August
2013].

\item[11] \textit{Task shifting to tackle health worker shortages}. Geneva: World
Health Organization; 2007. Available from:
www.who.int/healthsystems/task\_{}shifting\_{}booklet.pdf
[accessed 26 August 2013].

\item[12] Zulu l. \textit{Clinical officers (CO) and health care delivery in
Zambia: a
response to physician shortage}. Lusaka: \textsc{cdc} Global \textsc{aids} Program. Available from:
http://csis.org/files/media/csis/events/080324\_{}zulu.pdf [accessed 4 July
2013].

\item[13] Hounton SH, Newlands D, Meda N, De Brouwere V. A cost-effectiveness
study of
caesarean-section deliveries by clinical officers, general practitioners and
obstetricians in
Burkina Faso. \textit{Hum Resour Health}
2009;7:34. doi:
http://dx.doi.org/10.1186/1478-4491-7-34 \textsc{pmid}:19371433

\item[14] Kruk ME, Pereira C, Vaz F, Bergström S, Galea S. Economic evaluation
of surgically
trained assistant medical officers in performing major obstetric surgery in
Mozambique.
\textit{\textsc{bjog}}
2007;114:1253–60. doi: http://dx.doi.org/10.1111/j.1471-0528.2007.01443.x
\textsc{pmid}:17877677

\item[15] Pereira C, Cumbi A, Malalane R, Vaz F, McCord C, Bacci A et al.
Meeting the need for
emergency obstetric care in Mozambique: work performance and histories of
medical doctors and
assistant medical officers trained for surgery. \textit{\textsc{bjog}}
2007;114:1530–3. doi:
http://dx.doi.org/10.1111/j.1471-0528.2007.01489.x \textsc{pmid}:17877775

\item[16] \textit{Optimizing the delivery of key interventions to attain \textsc{mdg}s 4
and 5:
background document for the first expert ‘scoping' meeting to develop \textsc{who}
recommendations to optimize health workers' roles to improve maternal and
newborn
health}. Geneva: World Health Organization; 2010.

\item[17] Mullan F, Frehywot S. Non-physician clinicians in 47 sub-Saharan
African countries.
\textit{Lancet}
2007;370:2158–63.

\item[18] Ottawa Hospital Research Institute [Internet]. \textsc{epoc} resources.
Suggested risk of
bias criteria for \textsc{epoc} reviews. Ottawa: \textsc{ohri}; 2013. Available from:
http://epoc.cochrane.org/search/google-appliance/Suggested\%20risk\%20of\%20bias
\%20criteria [accessed
26 August 2013].

\item[19] Guyatt GH, Oxman AD, Kunz R, Vist GE, Falck-Ytter Y, Schünemann HJ;
\textsc{grade} Working
Group. What is “quality of evidence” and why is it important to clinicians?
\textit{\textsc{bmj}}

2008;336:995–8. doi: http://dx.doi.org/10.1136/bmj.39490.551019.BE \textsc{pmid}:18456631

\item[20] Higgins \textsc{jpt}, Green S, editors. \textit{Cochrane handbook for
systematic reviews of
interventions version 5.1.0}
[updated March 2011]. Oxford: The Cochrane Collaboration; 2011.
Available from: www.cochrane-handbook.org [accessed 26 August 2013].

\item[21] Begley C, Devane D, Clarke M, McCann C, Hughes P, Reilly M et al.
Comparison of
midwife-led and consultant-led care of healthy women at low risk of childbirth
complications in the
Republic of Ireland: a randomised trial. \textit{\textsc{bmc} Pregnancy Childbirth}
2011;11:85. doi:
http://dx.doi.org/10.1186/1471-2393-11-85 \textsc{pmid}:22035427

\item[22] Harvey S, Jarrell J, Brant R, Stainton C, Rach DA. A randomized,
controlled trial of
nurse-midwifery care. \textit{Birth}
1996;23:128–35. doi:
http://dx.doi.org/10.1111/j.1523-536X.1996.tb00473.x \textsc{pmid}:8924098

\item[23] Hundley VA, Cruickshank FM, Lang GD, Glazener \textsc{cma}, Milne JM, Turner M
et al. Midwife
managed delivery unit: a randomised controlled comparison with consultant led
care.
\textit{\textsc{bmj}}
1994;309:1400–4. doi: http://dx.doi.org/10.1136/bmj.309.6966.1400
\textsc{pmid}:7819846

\item[24] MacVicar J, Dobbie G, Owen-Johnstone L, Jagger C, Hopkins M, Kennedy
J. Simulated
home delivery in hospital: a randomised controlled trial. \textit{Br J Obstet
Gynaecol}

1993;100:316–23. doi: http://dx.doi.org/10.1111/j.1471-0528.1993.tb12972.x
\textsc{pmid}:8494832

\item[25] Marks MN, Siddle K, Warwick C. Can we prevent postnatal depression? A
randomized
controlled trial to assess the effect of continuity of midwifery care on rates
of postnatal
depression in high-risk women. \textit{J Matern Fetal Neonatal Med}
2003;13:119–27. doi:
http://dx.doi.org/10.1080/jmf.13.2.119.127 \textsc{pmid}:12735413

\item[26] McLachlan HL, Forster DA, Davey M-A, Lumley J, Farrell T, Oats J et
al. \textsc{cosmos}:
Comparing standard maternity care with one-to-one midwifery support: a
randomised controlled trial.
\textit{\textsc{bmc} Pregnancy Childbirth}
2008;8:35. doi: http://dx.doi.org/10.1186/1471-2393-8-35
\textsc{pmid}:18680606

\item[27] Di Napoli A, Di Lallo D, Fortes C, Franceschelli C, Armeni E,
Guasticchi G. Home
breastfeeding support by health professionals: findings of a randomized
controlled trial in a
population of Italian women. \textit{Acta Paediatr}
2004;93:1108–14. doi:
http://dx.doi.org/10.1111/j.1651-2227.2004.tb02725.x \textsc{pmid}:15456204

\item[28] Rowley MJ, Hensley MJ, Brinsmead MW, Wlodarczyk JH. Continuity of care
by a midwife
team versus routine care during pregnancy and birth: a randomised trial.
\textit{Med J Aust}

1995;163:289–93. \textsc{pmid}:7565233

\item[29] Small R, Lumley J, Donohue L, Potter A, Waldenström U. Randomised
controlled trial
of midwife led debriefing to reduce maternal depression after operative
childbirth.
\textit{\textsc{bmj}}
2000;321:1043–7. doi: http://dx.doi.org/10.1136/bmj.321.7268.1043
\textsc{pmid}:11053173

\item[30] Turnbull D, Holmes A, Shields N, Cheyne H, Twaddle S, Gilmour WH et
al. Randomised,
controlled trial of efficacy of midwife-managed care. \textit{Lancet}
1996;348:213–8. doi:
http://dx.doi.org/10.1016/S0140-6736(95)11207-3 \textsc{pmid}:8684197

\item[31] Waldenström U, McLachlan H, Forster D, Brennecke S, Brown S. Team
midwife care:
maternal and infant outcomes. \textit{Aust N Z J Obstet Gynaecol}
2001;41:257–64. doi:
http://dx.doi.org/10.1111/j.1479-828X.2001.tb01225.x \textsc{pmid}:11592538

\item[32] Law \textsc{yyh}, Lam KY. A randomized controlled trial comparing
midwife-managed care and
obstetrician-managed care for women assessed to be at low risk in the initial
intrapartum period.
\textit{J Obstet Gynaecol Res}
1999;25:107–12. doi:
http://dx.doi.org/10.1111/j.1447-0756.1999.tb01131.x \textsc{pmid}:10379125

\item[33] Wolke D, Dave S, Hayes J, Townsend J, Tomlin M. Routine examination of
the newborn
and maternal satisfaction: a randomised controlled trial. \textit{Arch Dis Child
Fetal Neonatal
Ed}
2002;86:F155–60. doi: http://dx.doi.org/10.1136/fn.86.3.F155
\textsc{pmid}:11978744

\item[34] Eren N, Ramos R, Gray RH. Physicians vs. auxiliary nurse-midwives as
providers of
\textsc{iud} services: a study in Turkey and the Philippines. \textit{Stud Fam Plann}
1983;14:43–7.
doi: http://dx.doi.org/10.2307/1965401 \textsc{pmid}:6836665

\item[35] Dusitsin N, Chalapati S, Varakamin S, Boonsiri B, Ningsanon P, Gray
RH. Post-partum
tubal ligation by nurse-midwives and doctors in Thailand. \textit{Lancet}
1980;1:638–9. doi:
http://dx.doi.org/10.1016/S0140-6736(80)91129-0 \textsc{pmid}:6102637

\item[36] Warriner IK, Meirik O, Hoffman M, Morroni C, Harries J, My Huong NT et
al. Rates of
complication in first-trimester manual vacuum aspiration abortion done by
doctors and mid-level
providers in South Africa and Vietnam: a randomised controlled equivalence
trial.
\textit{Lancet}
2006;368:1965–72. doi: http://dx.doi.org/10.1016/S0140-6736(06)69742-0
\textsc{pmid}:17141703

\item[37] Warriner IK, Wang D, Huong NT, Thapa K, Tamang A, Shah I et al. Can
midlevel
health-care providers administer early medical abortion as safely and
effectively as doctors? A
randomised controlled equivalence trial in Nepal. \textit{Lancet}
2011;377:1155–61. doi:
http://dx.doi.org/10.1016/S0140-6736(10)62229-5 \textsc{pmid}:21458058

\item[38] Sanne I, Orrell C, Fox M, Conradie F, Ive P, Zeinecker J et al. Nurse
versus doctor
management of \textsc{hiv}-infected patients receiving antiretroviral therapy (\textsc{cipra}-SA):
a randomised
non-inferiority trial. \textit{Lancet}
2010;376:33–40. doi:
http://dx.doi.org/10.1016/S0140-6736(10)60894-X \textsc{pmid}:20557927

\item[39] Mann AH, Blizard R, Murray J, Smith JA, Botega N, MacDonald E et al.
An evaluation
of practice nurses working with general practitioners to treat people with
depression. \textit{Br J
Gen Pract}
1998;48:875–9. \textsc{pmid}:9604408

\item[40] Du Moulin MF, Hamers \textsc{jph}, Paulus A, Berendsen CL, Halfens R. Effects
of introducing
a specialized nurse in the care of community-dwelling women suffering from
urinary incontinence: a
randomized controlled trial. \textit{J Wound Ostomy Continence Nurs}
2007;34:631–40. doi:
http://dx.doi.org/10.1097/01.\textsc{won}.0000299814.98230.13 \textsc{pmid}:18030102

\item[41] Gordon DW. Health maintenance service: ambulatory patient care in the
general
medical clinic. \textit{Med Care}
1974;12:648–58. doi:
http://dx.doi.org/10.1097/00005650-197408000-00003 \textsc{pmid}:4852229

\item[42] Hemani A, Rastegar DA, Hill C, al-Ibrahim MS. A comparison of resource
utilization
in nurse practitioners and physicians. \textit{Eff Clin Pract}
1999;2:258–65.
\textsc{pmid}:10788023

\item[43] Katz DA, Brown RB, Muehlenbruch DR, Fiore MC, Baker TB; \textsc{ahrq} Smoking
Cessation
Guideline Study Group. Implementing guidelines for smoking cessation: comparing
the efforts of
nurses and medical assistants. \textit{Am J Prev Med}
2004;27:411–6.
\textsc{pmid}:15556742

\item[44] Kinnersley P, Anderson E, Parry K, Clement J, Archard L, Turton P et
al. Randomised
controlled trial of nurse practitioner versus general practitioner care for
patients requesting
“same day” consultations in primary care. \textit{\textsc{bmj}}
2000;320:1043–8. doi:
http://dx.doi.org/10.1136/bmj.320.7241.1043 \textsc{pmid}:10764366

\item[45] Strömberg A, Mårtensson J, Fridlund B, Levin LÃ, Karlsson J-E,
Dahlström U. Nurse-led heart failure clinics improve survival and self-care
behaviour in
patients with heart failure: results from a prospective, randomised trial.
\textit{Eur Heart
J}
2003;24:1014–23. doi: http://dx.doi.org/10.1016/S0195-668X(03)00112-X
\textsc{pmid}:12788301

\item[46] Mårtensson J, Strömberg A, Dahlström U, Karlsson JE, Fridlund
B. Patients with heart failure in primary health care: effects of a nurse-led
intervention on
health-related quality of life and depression. \textit{Eur J Heart Fail}
2005;7:393–403.
doi: http://dx.doi.org/10.1016/j.ejheart.2004.01.016 \textsc{pmid}:15718180

\item[47] Moher M, Yudkin P, Wright L, Turner R, Fuller A, Schofield T et al.
Cluster
randomised controlled trial to compare three methods of promoting secondary
prevention of coronary
heart disease in primary care. \textit{\textsc{bmj}}
2001;322:1338. doi:
http://dx.doi.org/10.1136/bmj.322.7298.1338 \textsc{pmid}:11387182

\item[48] Sakr M, Angus J, Perrin J, Nixon C, Nicholl J, Wardrope J. Care of
minor injuries by
emergency nurse practitioners or junior doctors: a randomised controlled trial.
\textit{Lancet}
1999;354:1321–6. doi: http://dx.doi.org/10.1016/S0140-6736(99)02447-2
\textsc{pmid}:10533859

\item[49] Smith JR, Mildenhall S, Noble MJ, Shepstone L, Koutantji M, Mugford M
et al. The
Coping with Asthma Study: a randomised controlled trial of a home based, nurse
led psychoeducational
intervention for adults at risk of adverse asthma outcomes. \textit{Thorax}
2005;60:1003–11.
doi: http://dx.doi.org/10.1136/thx.2005.043877 \textsc{pmid}:16055616

\item[50] Stein GH. The use of a nurse practitioner in the management of
patients with
diabetes mellitus. \textit{Med Care}
1974;12:885–90. doi:
http://dx.doi.org/10.1097/00005650-197410000-00008 \textsc{pmid}:4437220

\item[51] Chambers LW, Bruce-Lockhart P, Black DP, Sampson E, Burke M. A
controlled trial of
the impact of the family practice nurse on volume, quality, and cost of rural
health services.
\textit{Med Care}
1977;15:971–81. doi: http://dx.doi.org/10.1097/00005650-197712000-00001
\textsc{pmid}:592915

\item[52] Black DP, Riddle RJ, Sampson E. Pilot project: the family practice
nurse in a
Newfoundland rural area. \textit{Can Med Assoc J}
1976;114:945–7.
\textsc{pmid}:1268779

\item[53] Caine N, Sharples LD, Hollingworth W, French J, Keogan M, Exley A et
al. A
randomised controlled crossover trial of nurse practitioner versus doctor-led
outpatient care in a
bronchiectasis clinic. \textit{Health Technol Assess}
2002;6:1–71.
\textsc{pmid}:12433318

\item[54] Chambers LW, West AE. The St John’s randomized trial of the family
practice nurse:
health outcomes of patients. \textit{Int J Epidemiol}
1978;7:153–61. doi:
http://dx.doi.org/10.1093/ije/7.2.153 \textsc{pmid}:681061

\item[55] Chinn DJ, Poyner T, Sibley G. Randomized controlled trial of a single
dermatology
nurse consultation in primary care on the quality of life of children with
atopic eczema. \textit{Br
J Dermatol}
2002;146:432–9. doi: http://dx.doi.org/10.1046/j.1365-2133.2002.04603.x
\textsc{pmid}:11952543

\item[56] Cox C, Jones M. An evaluation of the management of patients with sore
throats by
practice nurses and GPs. \textit{Br J Gen Pract}
2000;50:872–6.
\textsc{pmid}:11141872

\item[57] D'Eramo-Melkus G, Spollett G, Jefferson V, Chyun D, Tuohy B, Robinson
T et
al. A culturally competent intervention of education and care for black women
with type 2 diabetes.
\textit{Appl Nurs Res}
2004;17:10–20. doi: http://dx.doi.org/10.1016/j.apnr.2003.10.009
\textsc{pmid}:14991551

\item[58] Dierick-van Daele \textsc{atm}, Metsemakers \textsc{jfm}, Derckx \textsc{ewcc}, Spreeuwenberg C,
Vrijhoef \textsc{hjm}.
Nurse practitioners substituting for general practitioners: randomized
controlled trial. \textit{J
Adv Nurs}
2009;65:391–401. doi: http://dx.doi.org/10.1111/j.1365-2648.2008.04888.x
\textsc{pmid}:19191937

\item[59] Federman DG, Krishnamurthy R, Kancir S, Goulet J, Justice A.
Relationship between
provider type and the attainment of treatment goals in primary care. \textit{Am
J Manag
Care}
2005;11:561–6. \textsc{pmid}:16159046

\item[60] Houweling ST, Kleefstra N, van Hateren KJ, Kooy A, Groenier KH, Ten
Vergert E et
al.; Langerhans Medical Research Group. Diabetes specialist nurse as main care
provider for patients
with type 2 diabetes. \textit{Neth J Med}
2009;67:279–84. \textsc{pmid}:19687522

\item[61] Mundinger MO, Kane RL, Lenz ER, Totten AM, Tsai W-Y, Cleary PD et al.
Primary care
outcomes in patients treated by nurse practitioners or physicians: a randomized
trial.
\textit{\textsc{jama}}
2000;283:59–68. doi: http://dx.doi.org/10.1001/jama.283.1.59
\textsc{pmid}:10632281

\item[62] Myers PC, Lenci B, Sheldon MG. A nurse practitioner as the first point
of contact
for urgent medical problems in a general practice setting. \textit{Fam Pract}
1997;14:492–7.
doi: http://dx.doi.org/10.1093/fampra/14.6.492 \textsc{pmid}:9476082

\item[63] Rushforth H, Burge D, Mullee M, Jones S, McDonald H, Glasper EA.
Nurse-led
paediatric preoperative assessment: an equivalence study. \textit{Paediatr Nurs}

2006;18:23–9. \textsc{pmid}:16634381

\item[64] Sharples LD, Edmunds J, Bilton D, Hollingworth W, Caine N, Keogan M et
al. A
randomised controlled crossover trial of nurse practitioner versus doctor led
outpatient care in a
bronchiectasis clinic. \textit{Thorax}
2002;57:661–6. doi:
http://dx.doi.org/10.1136/thorax.57.8.661 \textsc{pmid}:12149523

\item[65] Shum C, Humphreys A, Wheeler D, Cochrane M-A, Skoda S, Clement S.
Nurse management
of patients with minor illnesses in general practice: multicentre, randomised
controlled trial.
\textit{\textsc{bmj}}
2000;320:1038–43. doi: http://dx.doi.org/10.1136/bmj.320.7241.1038
\textsc{pmid}:10764365

\item[66] Venning P, Durie A, Roland M, Roberts C, Leese B. Randomised
controlled trial
comparing cost effectiveness of general practitioners and nurse practitioners in
primary care.
\textit{\textsc{bmj}}
2000;320:1048–53. doi: http://dx.doi.org/10.1136/bmj.320.7241.1048
\textsc{pmid}:10764367

\item[67] Sackett DL, Spitzer WO, Gent M, Roberts RS. The Burlington randomized
trial of the
nurse practitioner: health outcomes of patients. \textit{Ann Intern Med}

1974;80:137–42.

\item[68] Babor TE, Higgins-Biddle J, Dauser D, Higgins P, Burleson JA. Alcohol
screening and
brief intervention in primary care settings: implementation models and
predictors. \textit{J Stud
Alcohol}
2005;66:361–8. \textsc{pmid}:16047525

\item[69] McIntosh MC, Leigh G, Baldwin NJ, Marmulak J. Reducing alcohol
consumption.
Comparing three brief methods in family practice. \textit{Can Fam Physician}

1997;43:1959–62. \textsc{pmid}:9386883

\item[70] Chilopora G, Pereira C, Kamwendo F, Chimbiri A, Malunga E, Bergström
S.
Postoperative outcome of caesarean sections and other major emergency obstetric
surgery by clinical
officers and medical officers in Malawi. \textit{Hum Resour Health}
2007;5:17. doi:
http://dx.doi.org/10.1186/1478-4491-5-17 \textsc{pmid}:17570847

\item[71] McGuire M, Goossens S, Kukasha W, Ahoua L, Pujades M, Le Paih M, et
al. Nurses and
medical assistants taking charge: task-shifting \textsc{hiv} care and \textsc{haart} initiation in
resource-constrained and rural Malawi. In: \textsc{xvii} International \textsc{aids} Conference,
3–8 August 2008
Mexico City, Mexico [Internet]. Available from:
http://www.doctorswithoutborders.org/events/symposiums/2008-aids-iac/assets/file
s/Nurses-and-medical-assistants-taking-charge.pdf
[accessed 26 August 2013].

\item[72] Pereira C, Bugalho A, Bergström S, Vaz F, Cotiro M. A comparative
study of caesarean
deliveries by assistant medical officers and obstetricians in Mozambique.
\textit{Br J Obstet
Gynaecol}
1996;103:508–12. doi: http://dx.doi.org/10.1111/j.1471-0528.1996.tb09797.x
\textsc{pmid}:8645640

\item[73] McCord C, Mbaruku G, Pereira C, Nzabuhakwa C, Bergstrom S. The quality
of emergency
obstetrical surgery by assistant medical officers in Tanzanian district
hospitals. \textit{Health
Aff (Millwood)}
2009;28:w876–85. doi: http://dx.doi.org/10.1377/hlthaff.28.5.w876
\textsc{pmid}:19661113

\item[74] Gimbel-Sherr K, Augusto O, Micek M, Gimbel-Sherr S, Tomo MI, Pfeiffer
J, et al. Task
shifting to mid-level clinical health providers: an evaluation of quality of \textsc{art}
provided by
tecnicos de medicina and physicians in Mozambique. In: \textit{\textsc{xvii}
International \textsc{aids} Conference,
3–8 August 2008, Mexico City, Mexico}
[Internet]. Available from:
http://www.aids2008.org/pag/PSession.aspx?s=321 [accessed 26 August 2013].

\item[75] Labhardt ND, Balo JR, Ndam M, Grimm JJ, Manga E. Task shifting to
non-physician
clinicians for integrated management of hypertension and diabetes in rural
Cameroon: a programme
assessment at two years. \textit{\textsc{bmc} Health Serv Res}
2010;10:339. doi:
http://dx.doi.org/10.1186/1472-6963-10-339 \textsc{pmid}:21144064

\item[76] Anand S, Bärnighausen T. Health workers and vaccination coverage in
developing
countries: an econometric analysis. \textit{Lancet}
2007;369:1277–85. doi:
http://dx.doi.org/10.1016/S0140-6736(07)60599-6 \textsc{pmid}:17434403

\item[77] Anand S, Bärnighausen T. Human resources and health outcomes:
cross-country
econometric study. \textit{Lancet}
2004;364:1603–9. doi:
http://dx.doi.org/10.1016/S0140-6736(04)17313-3 \textsc{pmid}:15519630

\item[78] Munga MA, Kilima SP, Mutalemwa PP, Kisoka WJ, Malecela MN.
Experiences,
opportunities and challenges of implementing task shifting in underserved remote
settings: the case
of Kongwa district, central Tanzania. \textit{\textsc{bmc} Int Health Hum Rights}
2012;12:27. doi:
http://dx.doi.org/10.1186/1472-698X-12-27 \textsc{pmid}:23122296

\item[79] \textit{Task shifting: rational redistribution of tasks among health
workforce
teams; global recommendations and guidelines}. Geneva: World Health Organization; 2008.
Available from:
http://chwcentral.org/sites/default/files/Task\%20Shifting-\%20Global\%20Recomme
ndations\%20and\%20Guidelines.pdf
[accessed 26 August 2013].

\item[80] Huicho L, Miranda JJ, Diez-Canseco F, Lema C, Lescano AG, Lagarde M et
al. Job
preferences of nurses and midwives for taking up a rural job in Peru: a discrete
choice experiment.
\textit{PLoS \textsc{one}}
2012;7:e50315. doi: http://dx.doi.org/10.1371/journal.pone.0050315
\textsc{pmid}:23284636

\item[81] Miranda JJ, Diez-Canseco F, Lema C, Lescano AG, Lagarde M, Blaauw D et
al. Stated
preferences of doctors for choosing a job in rural areas of Peru: a discrete
choice experiment.
\textit{PLoS \textsc{one}}
2012;7:e50567. doi: http://dx.doi.org/10.1371/journal.pone.0050567
\textsc{pmid}:23272065

\item[82] Lagarde M, Pagaiya N, Tangcharoensathian V, Blaauw D. One size does
not fit all:
investigating doctors’ stated preference heterogeneity for job incentives to
inform policy in
Thailand. \textit{Health Econ}
2013:Epub 24 January 2013. doi:
http://dx.doi.org/10.1002/hec.2897 \textsc{pmid}:23349119

\end{itemize}

\end{document}
