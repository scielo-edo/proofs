% Generated by jats2tex@0.11.1.0
\documentclass{article}
\usepackage[T1]{fontenc}
\usepackage[utf8]{inputenc} %% *
\usepackage[portuges,spanish,english,german,italian,russian]{babel} %% *
\usepackage{amstext}
\usepackage{authblk}
\usepackage{unicode-math}
\usepackage{multirow}
\usepackage{graphicx}
\usepackage{etoolbox}
\usepackage{xtab}
\usepackage{enumerate}
\usepackage{hyperref}
\usepackage{penalidades}
\usepackage[footnotesize,bf,hang]{caption}
\usepackage[nodayofweek,level]{datetime}
\usepackage[top=0.85in,left=2.75in,footskip=0.75in]{geometry}
\newlength\savedwidth
\newcommand\thickcline[1]{\noalign{\global
\savedwidth
\arrayrulewidth
\global\arrayrulewidth 2pt}
\cline{#1}
\noalign{\vskip\arrayrulewidth}
\noalign{\global\arrayrulewidth\savedwidth}}
\newcommand\thickhline{\noalign{\global
\savedwidth\arrayrulewidth
\global\arrayrulewidth 2pt}
\hline
\noalign{\global\arrayrulewidth\savedwidth}}
\usepackage{lastpage,fancyhdr}
\usepackage{epstopdf}
\pagestyle{myheadings}
\pagestyle{fancy}
\fancyhf{}
\setlength{\headheight}{27.023pt}
\lhead{\includegraphics[width=10mm]{logo.png}}
\rhead{\ifdef{\journaltitle}{\journaltitle}{}
\ifdef{\volume}{vol.\,\volume}{}
\ifdef{\issue}{(\issue)}{}
\ifdef{\fpage}{\fpage--\lpage\,pp.}}
\rfoot{\thepage/\pageref{LastPage}}
\renewcommand{\footrule}{\hrule height 2pt \vspace{2mm}}
\fancyheadoffset[L]{2.25in}
\fancyfootoffset[L]{2.25in}
\lfoot{\sf \ifdef{\articledoi}{\articledoi}{}}
\setmainfont{Linux Libertine O}
\renewcommand*{\thefootnote}{\alph{footnote}}
\makeatletter
\newcommand{\fn}{\afterassignment\fn@aux\count0=}
\newcommand{\fn@aux}{\csname fn\the\count0\endcsname}
\makeatother

\newcommand{\journalid}{Cad Saude Publica}
\newcommand{\publisherid}{csp}
\newcommand{\journaltitle}{Cadernos de Saúde Pública}
\newcommand{\abbrevjournaltitle}{Cad. Saúde Pública}
\newcommand{\issnppub}{0102-311X}
\newcommand{\issnepub}{1678-4464}
\newcommand{\publishername}{Escola Nacional de Saúde Pública Sergio Arouca,
Fundação Oswaldo Cruz}
\newcommand\articledoi{\textsc{doi} 10.1590/0102-311X00110016}
\def\subject{\textsc{essay}}\newcommand{\subtitlestyle}[1]{-- \emph{#1}\medskip}
\newcommand{\transtitlestyle}[1]{\par\medskip\Large #1}
\newcommand{\transsubtitlestyle}[1]{-- \Large\emph{ #1}}

\newcommand{\titlegroup}{
\ifdef{\subtitle}{\subtitlestyle{\subtitle}}{}
\ifdef{\transtitle}{\transtitlestyle{\transtitle}}{}
\ifdef{\transsubtitle}{\transsubtitlestyle{\transsubtitle}}{}}

\title{Shadows of doubt: the uneasy incorporation of identification science into
legal determination of paternity in Brazil\titlegroup{}}
\newcommand{\transtitle}{Sombras da dúvida: a difícil incorporação da ciência de
identificação na determinação legal da paternidade no Brasil}
\newcommand{\transtitle}{Sombras de duda: la ardua incorporación de las pruebas
científicas de identificación en la determinación legal de la paternidad en
Brasil}
\author[{1}]{Caulfield, Sueann}
\author[{1}]{Stern, Alexandra Minna}
\affil[1]{University of Michigan}
\def\authornotes{* Correspondence A. M. Stern University of Michigan. 3700 Haven
Hall, Ann Arbor / MI - 48103, U.S.A. amstern@med.umich.edu
S. Caulfield and A. M. Stern contributed equally to the research and analysis
for this article.}
\date{08 05 2017}
\def\volume{33}
\def\issue{Suppl 1}
\def\permissions{This is an open-access article distributed under the terms of
the Creative Commons Attribution License}
\newcommand{\kwdgroupen}{Paternity, Forensic Genetics, Forensic Medicine}
\newcommand{\kwdgroup}{Paternidade, Genética Forense, Medicina Legal}
\newcommand{\kwdgroupes}{Paternidad, Genética Forense, Medicina Legal}

\begin{document}
\selectlanguage{english}
\section*{Metadados não aplicados}
\begin{itemize}
\item[\textbf{língua do artigo}]{Inglês}
\ifdef{\journalid}{\item[\textbf{journalid}] \journalid}{}
\ifdef{\journaltitle}{\item[\textbf{journaltitle}] \journaltitle}{}
\ifdef{\abbrevjournaltitle}{\item[\textbf{abbrevjournaltitle}]
\abbrevjournaltitle}{}
\ifdef{\issnppub}{\item[\textbf{issnppub}] \issnppub}{}
\ifdef{\issnepub}{\item[\textbf{issnepub}] \issnepub}{}
\ifdef{\publishername}{\item[\textbf{publishername}] \publishername}{}
\ifdef{\publisherid}{\item[\textbf{publisherid}] \publisherid}{}
\ifdef{\subject}{\item[\textbf{subject}] \subject}{}
\ifdef{\transtitle}{\item[\textbf{transtitle}] \transtitle}{}
\ifdef{\authornotes}{\item[\textbf{authornotes}] \authornotes}{}
\ifdef{\articleid}{\item[\textbf{articleid}] \articleid}{}
\ifdef{\articledoi}{\item[\textbf{articledoi}] \articledoi}{}
\ifdef{\volume}{\item[\textbf{volume}] \volume}{}
\ifdef{\issue}{\item[\textbf{issue}] \issue}{}
\ifdef{\fpage}{\item[\textbf{fpage}] \fpage}{}
\ifdef{\lpage}{\item[\textbf{lpage}] \lpage}{}
\ifdef{\permissions}{\item[\textbf{permissions}] \permissions}{}
\end{itemize}
\maketitle

e00110016
\begingroup

\begin{abstract}

The arrival of \textsc{dna} paternity testing in the 1980s was met with great enthusiasm
in the Brazilian courts. Yet, over the past two decades, Brazilian legal
doctrine and jurisprudence have increasingly rejected \textsc{dna} proof as the sine qua
non for paternity cases. Instead, \textsc{dna} paternity testing has generated mountains
of litigation, as biological proof has been challenged by the argument that
paternity is primarily “socio-affective”. Leading family law specialists
describe this new conception of paternity as an outcome of the “revolutionary”
provisions of the 1988 Constitution, which recognizes the “pluralism” of family
forms in modern society and guarantees equal family rights for all children.
Without denying the significance of the constitution’s dignitary framework, we
show that new legal understandings of paternity represent less a paradigm shift
than a continuation of longstanding historical tensions between biological and
socio-cultural understandings of family and identity. In this article, we
explore the development of biological and eventually genetic typing in Brazil,
both of which had ties to the fields of criminology and race science. Our review
suggests that techniques of biological identification, no matter how
sophisticated or precise, were ineffective means for establishing identity,
whether of individual personhood, as in the case of paternity, or national
make-up. Instead, they became incorporated as supplemental methods into complex
legal, social, and cultural decision-making around families.

\iflanguage{portuges}{\medskip\noindent\textbf{Palavras-chave:} \kwdgroup}{}
\iflanguage{english}{\medskip\noindent\textbf{Keywords:} \kwdgroupen}{}
\iflanguage{spanish}{\medskip\noindent\textbf{Palavras claves:} \kwdgroupes}{}
\iflanguage{french}{\medskip\noindent\textbf{Mots clés:} \kwdgroupfr}{}
\end{abstract}
\endgroup

O surgimento dos testes de \textsc{dna} para determinação de paternidade, nos anos 1980,
foi recebida com grande entusiasmo nos tribunais brasileiros. No entanto, ao
longo das últimas duas décadas, a doutrina jurídica e a jurisprudência
brasileiras têm rejeitado cada vez mais a prova de \textsc{dna} como condição
\textit{sine qua non}
para os casos de paternidade. Testes de paternidade de \textsc{dna} geraram inúmeros
litígios que contestaram a prova biológica com o argumento de que a paternidade
é principalmente “socioafetiva”. Os principais especialistas em direito de
família descrevem essa nova concepção de paternidade como resultado das
disposições “revolucionárias” da Constituição Federal de 1988, que reconhece a
“pluralidade” das formas familiares na sociedade moderna e garante direitos
iguais para todas as crianças. Sem negar a importância de novos princípios
constitucionais, mostramos que os novos conceitos jurídicos da paternidade
representam menos uma mudança de paradigma do que a continuação de antigas
tensões históricas entre concepções biológicas e socioculturais da família e da
identidade. Neste artigo, exploramos o desenvolvimento da tipologia biológica e,
posteriormente, genética no Brasil, ambas ligadas aos campos da criminologia e
da ciência racial. Nossa análise sugere que as técnicas de identificação
biológica, por mais sofisticadas ou precisas que fossem, eram meios ineficazes
para estabelecer a identidade, seja da personalidade individual, como no caso da
paternidade, ou da composição nacional. Em vez disso, elas foram incorporadas
como métodos suplementares para as decisões legais, sociais e culturais
complexas em torno das famílias.

\begingroup
\renewcommand{\section}[1]{\subsection*{#1}}
\begin{otherlanguage}{spanish}

\begin{abstract}

La inclusión de las pruebas de \textsc{adn} para la determinación de la paternidad en los
años ochenta fue recibida con gran entusiasmo en los tribunales brasileños. A
pesar de ello hoy en día, tras haber trascurrido dos décadas, la doctrina legal
y la jurisprudencia brasileña han rechazado cada vez más las pruebas de \textsc{adn} como
pruebas determinantes de los casos de paternidad. Es más, las pruebas de \textsc{adn}
para la paternidad han generado muchísimos litigios, puesto que las pruebas
biológicas han sido rebatidas por argumentos basados en que la paternidad es
primordialmente una cuestión “socio-afectiva”. Los letrados especialistas en
familia consideran esta nueva concepción de la paternidad como una revolución de
la Constitución de 1988, la cual reconoce la existencia del pluralismo de
familias y equipara los derechos de todos los niños. Sin menoscabar la
interpretación del marco constitucional indicado, entendemos que las nuevas
interpretaciones jurídicas de paternidad representan, cuanto menos, un cambio
generado por las continuas tensiones entre las interpretaciones biológicas y
socioculturales de los conceptos de familia y de identidad. En este artículo,
analizamos el desarrollo de esta cuestión, desde el punto de vista biológico y
genético en Brasil, los cuales se relacionan con los campos de criminología y
los estudios raciales. Nuestro análisis sugiere que las técnicas de
identificación biológicas, sin importar lo precisas y sofisticadas que sean, son
ineficientes en el sentido de establecer una identidad, sea individual como
persona, como en el caso de la paternidad, o sea colectiva, como en el seno de
una nación. En su lugar, han sido incorporados como métodos complementarios, en
el ámbito de toma de decisiones legal, social y cultural, sobre los estudios
acerca de las familias.

\ifdef{\kwdgroupes}{\medskip\noindent\textbf{Palavras claves:} \kwdgroupes}{}
\end{abstract}
\end{otherlanguage}
\endgroup
\section{Introduction}

Brazil has a long history of the utilization of genetic research and tests in
the fields of population genetics and in clinical and forensic medicine. It is
not surprising that it was the first Latin American nation to develop the
capacity for \textsc{dna} paternity testing in the 1980s. In the early 1990s, both the
scientific community and popular media created the expectation that \textsc{dna} testing
would easily edge out other methods of paternity determination. For some, this
heralded a social transformation that would make insecurities about fatherhood -
which Machado de Assis famously depicted as a central feature of Brazilian
culture in his 1899 novel, \textit{Dom Casmurro}
- a relic of the patriarchal past\textsuperscript{[}\textsuperscript{1}\textsuperscript{]}. Yet around the world, studies have shown that the uptake of genetic
technologies has been messy. Examples include popular and clinical uses of
ancestry tests or tests for genes associated with breast and ovarian cancer.
Although heredity increasingly has been understood as playing an important role
in human identity and health, “geneticization” - or the ascendency of genetic
explanations - has not been clear-cut. Instead, genomics and genetic
technologies become caught up in dynamic social processes that cannot escape
history, culture and emotion\textsuperscript{[}\textsuperscript{2}\textsuperscript{]}\textsuperscript{,}\textsuperscript{[}\textsuperscript{3}\textsuperscript{]}. This seems to be particularly true regarding \textsc{dna} paternity testing.

Across Latin America, advances in genetic identification have played a prominent
role in human rights cases, including the Abuelas de Plaza de Mayo’s search for
biological grandchildren following the dictatorship in Argentina (1976-1983),
the forensic identification of victims of state and paramilitary violence in
Guatemala, Colombia, and Brazil, and, in the 1990s, the identification of
Brazilian children whose parents were committed forcibly to leper’s colonies
from the 1940s to the 1980s\textsuperscript{[}\textsuperscript{4}\textsuperscript{]}. The implicit relationship between \textsc{dna} testing and human rights also emerged in
international law in 1989, when the United Nations Convention of the Rights of
the Child, influenced by the Argentinean Abuelas, included the child’s right to
know his or her parents (Art. 7) and to preserve his or her identity (Art. 8).
Given these precedents, it is not surprising that the arrival of \textsc{dna} paternity
testing initially generated tremendous enthusiasm in the Brazilian scientific
community, popular culture, and courts. Yet although the explosion of this
testing in the 1990s provided good business for medical laboratories and lowbrow
television entertainment, it did not resolve long-standing moral and legal
debates over the meaning and responsibilities of paternity. Instead, reliance on
\textsc{dna} testing has generated mountains of litigation, as biological proof was
challenged by the argument that paternity is primarily “socio-affective”.

\textsc{dna} paternity testing has provoked similar responses elsewhere in Latin America
and around the world\textsuperscript{[}\textsuperscript{5}\textsuperscript{]}\textsuperscript{,}\textsuperscript{[}\textsuperscript{6}\textsuperscript{]}\textsuperscript{,}\textsuperscript{[}\textsuperscript{7}\textsuperscript{]}. We demonstrate, however, that the intersecting histories of family law,
identity science, and population genetics contributed to the particular
resonance of these debates in Brazil, where legal doctrine in support of
“socio-affective paternity” and children’s rights is especially strong. Brazil’s
leading family law specialists see the law’s embrace of socio-affective
paternity as emblematic of a paradigm shift underway since the 1980s, which, in
fits and starts, brought radically new conceptions of citizenship and dissolved
the patriarchal underpinnings of family law. Evoking the human rights framework
that emerged as an antidote to authoritarianism across Latin America, they have
described new conceptions of paternity as an outcome of the “revolutionary”
provisions of Brazil’s 1988 Federal Constitution, which recognizes “pluralistic”
family forms and guarantees equal family rights as essential to human dignity\textsuperscript{[}\textsuperscript{8}\textsuperscript{]}\textsuperscript{,}\textsuperscript{[}\textsuperscript{9}\textsuperscript{]}\textsuperscript{,}\textsuperscript{[}\textsuperscript{10}\textsuperscript{]}\textsuperscript{,}\textsuperscript{[}\textsuperscript{11}\textsuperscript{]}\textsuperscript{,}\textsuperscript{[}\textsuperscript{12}\textsuperscript{]}. The Superior Court of Justice (\textsc{stj}, the highest federal court of appeal) and
Supreme Court (\textsc{stf}) have largely adopted these interpretations, although they
have not always been consistently implemented at lower instances of law and
public policy\textsuperscript{[}\textsuperscript{13}\textsuperscript{]}\textsuperscript{,}\textsuperscript{[}\textsuperscript{14}\textsuperscript{]}.

Without diminishing the significance of the constitution’s dignitary framework,
we show that responses to the advent of \textsc{dna} paternity testing also represent a
continuation of longstanding tensions between biological and socio-cultural
understandings of identity and family. These tensions surfaced both in the realm
of family law and among international teams of biologists and geneticists who
saw Brazil as an ideal laboratory for observation of racial mixture.

\section{Illegitimacy and paternal recognition in Brazilian history}

Concern with verification of paternity has deep roots in Brazilian law. During
the colonial period (1500-1822), illegitimacy rates were notoriously high,
particularly among the enslaved and free poor population, while Catholic
marriage and legitimate birth were critical marks of honor for the white elite.
Yet illegitimate children were not uniformly bereft of family honor or access to
patrimony. Regardless of the child’s birth status, both parents were required to
provide nurturance and education, and for those who could afford it, a royal
dispensation could permit a child’s “legitimization”. Even without such
dispensation, a father’s recognition of paternity could significantly reduce
social stigma, and if a child was of “natural” birth - that is, not adulterous,
incestuous, or sacrilegious - parental recognition gave the child the same
hereditary rights as legitimate children. A child could sue for recognition, and
while maternal recognition required no documentation and was seldom disputed,
investigations of paternity were commonly carried out during settlement of a
deceased father’s estate, as is true today. Evidence generally consisted of
witness testimony confirming that the alleged father had behaved publicly as
such, particularly by giving the child his surname and affection and providing
for the child’s care and education. The law excluded slaves, but many masters
freed and then recognized the children they fathered with slave women\textsuperscript{[}\textsuperscript{15}\textsuperscript{]}.

After political independence in 1822, liberal reforms expanded access to the
justice system, making it easier for the rapidly growing population of freed and
free people of color to sue for paternal recognition. Many liberals also called
for the expansion of illegitimate children’s rights. Their arguments were
defeated by a series of laws that first reiterated colonial restrictions, then,
in 1847, rescinded natural children’s right to demand paternal recognition in
court. This radical change was consistent with a trend throughout
post-independence Latin America, due in part to the strong influence of the 1804
Napoleonic Code\textsuperscript{[}\textsuperscript{5}\textsuperscript{]}\textsuperscript{,}\textsuperscript{[}\textsuperscript{16}\textsuperscript{]}\textsuperscript{,}\textsuperscript{[}\textsuperscript{17}\textsuperscript{]}. Yet whereas French revolutionaries had framed their arguments against
paternity suits around enhancing individual men’s freedom, the Brazilian debate
took place in the context of regional unrest and slave revolts, and arguments
that traditional family honor required protection from individual men’s folly
and disreputable outsiders carried heavy racial overtones\textsuperscript{[}\textsuperscript{7}\textsuperscript{]}\textsuperscript{,}\textsuperscript{[}\textsuperscript{16}\textsuperscript{]}.

The debate over the rights of illegitimate children erupted again at the start
of the first republic (1890-1930), reaching its height during the lengthy
legislative review that preceded the approval of Brazil’s first Civil Code in
1916. This time, liberals held sway, and the code restored natural children’s
right to sue for paternal recognition if, during the time of conception, the
father had sexual relations, abducted, or “lived in concubinage” with the mother
(Art. 363)\textsuperscript{[}\textsuperscript{18}\textsuperscript{]}. Citing precedents in the modern laws of “civilized nations”, liberal jurists
and legislators lobbied to include “possession of status of filiation” as an
additional basis for paternity investigation\textsuperscript{[}\textsuperscript{19}\textsuperscript{]}\textsuperscript{,}\textsuperscript{[}\textsuperscript{20}\textsuperscript{]}\textsuperscript{,}\textsuperscript{[}\textsuperscript{21}\textsuperscript{]}. This Roman Law concept was intended to extend rights to children raised
jointly by unmarried parents, which its supporters recognized was a longstanding
norm in Brazilian law and popular culture. Conservative legislators, however,
struck “possession of status” from the final draft of the code. They also
defeated liberal attempts to permit parental recognition of adulterous and
incestuous children (such children could still demand child support) (Art. 358
and 405)\textsuperscript{[}\textsuperscript{18}\textsuperscript{]}\textsuperscript{,}\textsuperscript{[}\textsuperscript{20}\textsuperscript{]}. Nonetheless, Brazilian liberals boasted that the code’s provisions regarding
illegitimate children were among the world’s most liberal\textsuperscript{[}\textsuperscript{22}\textsuperscript{]}\textsuperscript{,}\textsuperscript{[}\textsuperscript{23}\textsuperscript{]}. The French law that reintroduced paternity investigations in 1912, for
instance, was much more restrictive, as were similar Portuguese and Spanish laws
of 1867 and 1889, respectively\textsuperscript{[}\textsuperscript{7}\textsuperscript{]}\textsuperscript{,}\textsuperscript{[}\textsuperscript{21}\textsuperscript{]}.

Once the Civil Code was implemented, paternity investigations quickly became one
of the most common types of suits heard in the family courts. Jurists debated
how to define “sexual relations” and “concubinage”, but agreed that the mother’s
reputation as “honest”, that is, modest and respectable, was indispensable\textsuperscript{[}\textsuperscript{24}\textsuperscript{]}. Surprisingly however, given the narrow wording of the law, evidence of
concubinage or sexual relations at the time of conception was not usually the
sole deciding issue. Instead, by the 1930s, jurists largely agreed that
“possession of status” was not only admissible, but among the most valuable
evidence of paternity\textsuperscript{[}\textsuperscript{24}\textsuperscript{]}\textsuperscript{,}\textsuperscript{[}\textsuperscript{25}\textsuperscript{]}. In subsequent decades, judges frequently decided in favor of children whose
father had provided a name, financial support, education, or affection,
particularly if this were public knowledge. Although these elements were
understood as evidence of a biological relationship, in practice, judgments
reinforced a social and emotional conception of fatherhood\textsuperscript{[}\textsuperscript{13}\textsuperscript{]}\textsuperscript{,}\textsuperscript{[}\textsuperscript{24}\textsuperscript{]}.

Jurisprudence also grappled with the validity of forensic evidence in paternity
cases from the 1920s-1950s, as identification sciences wedded to theories of
heredity were taking root, first through the development of race science in
legal medicine, and eventually through the loosely organized field of
biotypology. Nina Rodrigues (1862-1906), chair of Legal Medicine at the Medical
Faculty in Bahia at the turn of the century, is widely credited with
transforming Brazilian legal medicine into an interdisciplinary applied science
focused on identifying the specific propensities and anomalies of different
“types” within Brazil’s multiracial population. Inspired by Cesare Lombroso’s
positivist school of criminal anthropology, Rodrigues introduced phrenology and
anthropometry as well as evaluation of psychological and cognitive development,
which led him to racist conclusions regarding the hereditary degeneracy and
general inferiority of his subjects. The field of legal medicine was thus
“nationalized” through the introduction of race science\textsuperscript{[}\textsuperscript{26}\textsuperscript{]}.

Ironically, given the centrality of race in the development of identification
science, Rodrigues’ disciples found that in practice, determining paternity
through comparison of “racial” traits was particularly challenging in Brazil due
to its level of racial complexity\textsuperscript{[}\textsuperscript{27}\textsuperscript{]}\textsuperscript{,}\textsuperscript{[}\textsuperscript{28}\textsuperscript{]}. More than Rodrigues’ racism, it was his commitment to local empirical
research, and its direct application to Brazilian law and society, that inspired
his students, many of whom continued his struggle to institutionalize legal
medicine over the first half of the twentieth century\textsuperscript{[}\textsuperscript{26}\textsuperscript{]}. This interest in applied research was evident among Brazilian biotypologists,
who emerged in the 1930s principally in anthropology and medicine, and were
disenchanted if not hostile to scientific proclamations about the supposed
biological degeneration of mixed-race societies\textsuperscript{[}\textsuperscript{29}\textsuperscript{]}. Biotypology was a dimension of the Lamarckian variant of eugenics popular in
Latin America and “Latin” European countries such as Italy, Spain, and France.
It thrived in countries that tended to reject the hard-line racist Mendelian
eugenics found in the United States and Germany, even as its adherents sought to
map human differences\textsuperscript{[}\textsuperscript{30}\textsuperscript{]}. Biotypology was central to the anthropological search to identify the
quintessential Brazilian “normotype” or normal man, the subject of dozens of
specialized publications, and the field secured a solid foothold in Rio de
Janeiro’s School of Medicine\textsuperscript{[}\textsuperscript{29}\textsuperscript{]}.

Biotypology facilitated the entry of \textsc{abo} blood typing to Brazil. Kurt
Landsteiner, an immunologist, elaborated this system in Germany in the early
twentieth century. In 1901 he discovered that different types of blood would
agglutinate when mixed together, and based on those clumping patterns, divided
blood types into A, B, O, and AB. This led to advances in transfusion medicine
and a “\textit{proliferation of studies on the blood-group frequencies of
different racial and national populations}
” around the world\textsuperscript{[}\textsuperscript{31}\textsuperscript{]}
(p. 75). Blood typing did not simply reveal racial and ethnic differences, it
also constructed social and legal categories. This was most evident in Nazi
Germany, where blood typing tests to exclude paternity were used to determine
“racial purity”\textsuperscript{[}\textsuperscript{32}\textsuperscript{]}. Elsewhere, these tests were used haphazardly in legal disputes, as they were
integrated into the battery of so-called classical genetic markers that included
enzyme and protein testing. Fitfully these techniques displaced morphologically
based analyses such as anthropometry and phrenology, favored by Rodrigues, and
the photographic composites used by some legal experts\textsuperscript{[}\textsuperscript{33}\textsuperscript{]}.

The first blood test to verify paternity in all of the Americas was performed in
1927 at the Oscar Freire Institute of Legal Medicine in São Paulo, Brazil, one
of the nation’s premier forensic laboratories and research centers\textsuperscript{[}\textsuperscript{34}\textsuperscript{]}. The Institute’s forensic team, led by Flamínio Fávero, conducted \textsc{abo} blood
type analysis to exonerate a well-to-do fellow physician accused of fathering
the child of his former domestic servant\textsuperscript{[}\textsuperscript{35}\textsuperscript{]}\textsuperscript{,}\textsuperscript{[}\textsuperscript{36}\textsuperscript{]}. Their report and subsequent publications on blood type and paternity
investigations circulated widely, contributing to the evolving forensic
literature on individual identification as well as to the development of
biological methods for studying Brazil’s racial and ethnic types. Nevertheless,
the nation’s top experts warned that although combined observations of various
heritable traits might serve as supplemental evidence, none could confirm a
genetic relationship, and they insisted that forensic reports consider physical
evidence alongside contextual information gathered by interviewing the alleged
father and child\textsuperscript{[}\textsuperscript{24}\textsuperscript{]}\textsuperscript{,}\textsuperscript{[}\textsuperscript{25}\textsuperscript{]}. Against their admonitions, countless private examiners, police legal-medical
services, and even some experts at elite institutes commonly offered medical
assessments of the “degree of probability” of paternity from the late 1930s to
the 1950s, using techniques such as superimposition of photographs of alleged
parents and offspring to measure facial structures; cranial or dental formation;
sensitivity to the taste of phenylthiourea; eyelash length; earlobe attachment;
and fingerprinting. Some judges and even appeals courts ruled on the basis of
the medical examination, although this contradicted dominant legal doctrine.
Sabrina Finamori describes several paternity cases that contained medical
examinations of this sort in the 1930s, although none of them were decided
primarily on the physical evidence\textsuperscript{[}\textsuperscript{36}\textsuperscript{]}. In the juridical literature, specialists routinely criticized exams performed
by private physicians and the reliance on unproven anthropometric methods, which
some characterized as completely worthless\textsuperscript{[}\textsuperscript{34}\textsuperscript{]}\textsuperscript{,}\textsuperscript{[}\textsuperscript{36}\textsuperscript{]}\textsuperscript{,}\textsuperscript{[}\textsuperscript{37}\textsuperscript{]}\textsuperscript{,}\textsuperscript{[}\textsuperscript{38}\textsuperscript{]}\textsuperscript{,}\textsuperscript{[}\textsuperscript{39}\textsuperscript{]}\textsuperscript{,}\textsuperscript{[}\textsuperscript{40}\textsuperscript{]}. Nonetheless, well into the 1980s, legal-medical examiners still completed
standardized forms that required various morphological observations, followed by
the blood test results\textsuperscript{[}\textsuperscript{39}\textsuperscript{]}.

Despite medical examiners’ enthusiasm about the potential of blood typing for
verification of paternity in the 1930s, very few tests were performed in
Brazilian legal-medical labs in the first half of the twentieth century. In his
1941 dissertation, Antônio Almeida Júnior counted only 73 in all of Brazil - 19
at the Oscar Freire Institute and 54 in the state of Pernambuco, in addition to
various exams performed in Rio de Janeiro. Only two of the 54 Pernambucan tests
were conclusive, excluding paternity; Almeida did not report on the outcome of
the others\textsuperscript{[}\textsuperscript{36}\textsuperscript{]}\textsuperscript{,}\textsuperscript{[}\textsuperscript{38}\textsuperscript{]}. In the 1950s, blood tests became increasingly accessible, and the degree of
accuracy in paternity exclusions improved steadily - rising from less than 20\%
in the 1920s, to 65\% in the 1950s, to over 99\% after the introduction of \textsc{hla}
technology, introduced in Brazil in 1976\textsuperscript{[}\textsuperscript{40}\textsuperscript{]}. In 1984, Dr. Ayush Morad Amar reported that he had performed 10,000 blood
tests over twenty years as medical examiner at the Oscar Freire Institute,
achieving a 26\% exclusion rate with 1,600 tests done after 1978\textsuperscript{[}\textsuperscript{40}\textsuperscript{]}. As the technology became more reliable, the courts began to weigh medical
evidence more heavily, particularly when it excluded paternity\textsuperscript{[}\textsuperscript{41}\textsuperscript{]}. After decades of debate regarding its precariousness, however, the Procedural
Code of 1973 specified “\textit{the judge is not bound by the medical
examination}
” (Art. 436). Even in the 1980s, when blood tests offered high degrees of
accuracy, many judges insisted that medical proof was unreliable, and paternity
must be decided on the basis of “\textit{indirect proof}
” and “\textit{moral certainty}
”\textsuperscript{[}\textsuperscript{42}\textsuperscript{]}
(p. 355).

Efforts to extend blood typing in other areas met with mixed success. Fávero
advocated widespread use of blood type, suggesting it be recorded on school
registration and state identity cards, a goal only attained at the University of
São Paulo in 1935\textsuperscript{[}\textsuperscript{36}\textsuperscript{]}. Foreshadowing larger-scale genetic studies, in the 1930s Fávero attempted to
map \textsc{abo} blood type among different ethnic groups - European, Brazilian, and
Asian, determined by the birthplace of the subjects’ grandparents - but his
research was limited to 265 University of São Paulo students, hardly a
representative group\textsuperscript{[}\textsuperscript{43}\textsuperscript{]}. By 1953, his laboratory’s data was more robust: his colleague Arnaldo Ferreira
published a study that year that mapped blood types of 3,000 white, mulatto, and
black Brazilians and found racial variations similar to those in the
international literature\textsuperscript{[}\textsuperscript{34}\textsuperscript{]}.

As blood typing gained some traction in legal medicine, it became a key
instrument in several large-scale population-level genetics studies launched in
Brazil after World War II. These studies shifted hereditary frameworks from
questions of individual identification to puzzles of racial ancestry and
national demography and helped to cement genetics as a dynamic scientific field
with international and national relevance. Starting in the 1950s, Brazil became
a premier site for anthropological and genetic research into human types, a kind
of “living laboratory” for exploring racial ancestry, population migrations, and
regional distinctions\textsuperscript{[}\textsuperscript{44}\textsuperscript{]}. With significant financial backing from the Rockefeller Foundation, this
research moved into two seemingly different directions, both of which expressed
profound concern with the national and evolutionary aspects of ethnic and racial
diversity in Brazil.

Much of this research focused on identifying the characteristics of “racial
isolates”, ostensibly primitive, or what one scholar has called “ultraprimitive”
groups that exhibited pre-modern social organization and embodied unsullied pure
human biology\textsuperscript{[}\textsuperscript{33}\textsuperscript{]}. The foremost example of this strand were studies conducted by geneticists
James Neel, of the University of Michigan, and Francisco Salzano, of the Federal
University of Rio Grande do Sul. This duo launched their collaboration in the
1950s by tracking far into the Amazon to acquire samples from the Xavante
Indians living in Mato Grosso, eventually incorporating Yanomami living between
Brazil and Venezuela, as well as indigenous groups in Central America\textsuperscript{[}\textsuperscript{44}\textsuperscript{]}\textsuperscript{,}\textsuperscript{[}\textsuperscript{45}\textsuperscript{]}\textsuperscript{,}\textsuperscript{[}\textsuperscript{46}\textsuperscript{]}.

At the same time, geneticists assessed the degree of racial admixture in
Brazil’s population, confirming that it was high and variable, and a biological
testament to centuries of tri-hybrid miscegenation among Portuguese,
Amerindians, and Africans. The \textsc{abo} system was a core technique for many of these
studies. Several prominent scientists dedicated the bulk of their careers to
studying racial crossing in Brazil. Newton Morton traveled from the University
of Hawai’i to conduct a genetics study of migration from the more racially
diverse Northeast to the Southeast, setting up his station in the São Paulo
Immigrant Hotel. This living laboratory allowed him to develop
widely-influential theories of genetic linkage analysis in humans, notably the
\textsc{lod} (logarithm of the odds) score which compared the “\textit{likelihood that
traits were actually being inherited together to the likelihood that they had
appeared simply by chance in the observed pattern}
”\textsuperscript{[}\textsuperscript{46}\textsuperscript{]}
(p. 727). Fritz Ottensooser, a Jewish refugee whose family had fled to São
Paulo, developed mathematical formulas to measure degrees of racial mixture
utilizing serological analysis. He was joined by the prolific Pedro Henrique
Saldanha, trained at the Universidade do Brasil in Rio de Janeiro, who published
a major study of “\textit{gene flow from the white population to the black
population}
”\textsuperscript{[}\textsuperscript{46}\textsuperscript{]}
(p. 102) in the flagship genetics journal \textit{American Journal of Human
Genetics}
in 1957.

All of these studies relied on classical genetic markers to “\textit{understand
the formation and evolution of the composition of the Brazilian population from
a genetic perspective}
”\textsuperscript{[}\textsuperscript{33}\textsuperscript{]}
(p. 45). The underlying message was that Brazil was a melting pot of biological
amalgamation, most ideally, a genetic portrait of racial synthesis and
democracy. This research helped to consolidate human genetics and genomics in
Brazil. With the rise of molecular technologies in the 1990s, tools became more
sophisticated, including mt\textsc{dna} (mitochondrial \textsc{dna}), Y chromosome, and \textsc{aim}
(ancestry-informative markers). This quest for biogeographic archetypes
continues today in studies that seek to demonstrate the degree of African
ancestry among various regional populations and to underscore the tri, if not
poly, hybrid genome of the entire nation\textsuperscript{[}\textsuperscript{47}\textsuperscript{]}\textsuperscript{,}\textsuperscript{[}\textsuperscript{48}\textsuperscript{]}. Yet, like modalities of paternity testing, these macro-level approaches to
understanding the biological and molecular composition of Brazil’s population
have not provided categorical answers to complex questions of national and/or
racial identity. Many scientists, including the “father” of \textsc{dna} paternity
testing, Sérgio Danilo Pena, a geneticist at the Federal University of Minas
Gerais, have rejected race as a biological fact, explicitly viewing it as a
cultural and social construct\textsuperscript{[}\textsuperscript{49}\textsuperscript{]}.

In the mid 1980s, Pena (2015, personal communication) learned about the work of
Alec Jeffreys at the University of Leicester, who was developing novel
techniques of “genetic fingerprinting” using \textsc{pcr} (polymerase chain reaction). In
1985 Jeffreys described the development of multilocus \textsc{dna} fingerprints and
surmised that this new technique could play an important role in legal cases.
Within one year, this application had been used in an immigration dispute,
murder case, and paternity disputes\textsuperscript{[}\textsuperscript{50}\textsuperscript{]}.

Pena, who had studied in the UK, the United States and Canada, was eager to
bring this technique to Brazil. At the time he was running a private lab, \textsc{gene},
in Belo Horizonte, and wanted to calibrate genetic fingerprinting to work in his
country’s particular legal, scientific, and social environment. Aware that the
judicial system remained very patriarchal, tending to hold women to a double
sexual standard, he wanted to bring \textsc{dna} paternity testing to Brazil. Like many
of his contemporaries, Pena had experience with \textsc{hla} and blood typing for
paternity exclusions, but wanted the greater accuracy afforded by emergent \textsc{dna}
sequencing, particularly with the visualization of banding techniques. In the
late 1980s, Pena performed the first \textsc{dna} paternity test in Latin America. Due to
his entrepreneurship, the city of Belo Horizonte became an early adapter of
genetic sequencing methods, four years before the country’s second lab, in São
Paulo, began doing \textsc{dna} paternity tests (Sergio Pena, 2015, personal
communication).

In 1993, Pena’s group demonstrated the accuracy of \textsc{dna} paternity testing in an
article on the techniques of F10 multilocus fingerprinting in 200 paternity
cases (156 exclusions and 44 inclusions) evaluated at \textsc{gene}: “\textit{it was
capable of distinguishing fathers from non-fathers in every case}
”\textsuperscript{[}\textsuperscript{51}\textsuperscript{]}
(p. 237). Pena and colleagues soon expanded beyond multi- and single-locus
probes to include two additional techniques (Amp-\textsc{flp}s and microsatellites) which
he encouraged all laboratories to utilize, thus enabling them to
“\textit{resolve all paternity disputes without any probable, possible shadow of
doubt}
”\textsuperscript{[}\textsuperscript{51}\textsuperscript{]}
(p. 209).

The 1990s witnessed an explosion of \textsc{dna} paternity testing, with labs appearing
throughout Brazil, the pursuit of thousands of cases of contested paternity
through the court system, and increased use of testing by private clients. The
São Paulo Institute of Medicine performed 600 exams each month for legal cases.
An estimate by \textit{Folha de S. Paulo}
in 1997 reported that 6,500 exams were processed each year by private labs, with
an expected increase in these numbers in coming years\textsuperscript{[}\textsuperscript{36}\textsuperscript{]}\textsuperscript{,}\textsuperscript{[}\textsuperscript{52}\textsuperscript{]}.

As geneticists and the media publicized the advent of \textsc{dna} testing in the
1980s-1990s, many assumed that contentious investigations of paternity would
soon become obsolete. Indeed, as the procedure became increasingly accessible
over the next twenty years, countless families, particularly among the middle
and upper classes, resolved disputes privately. According to Pena\textsuperscript{1}, the \textsc{dna} test revolutionized family law in favor of single mothers, putting an
end to their humiliation in trials in which the defense invariably attacked the
woman’s sexual honor. This perspective was supported by advocates for single
mothers in the 1990s, who demanded \textsc{dna} tests in order to hold fathers
accountable, pointing out that up to 25\% of Brazilian children were not legally
recognized by their fathers\textsuperscript{[}\textsuperscript{13}\textsuperscript{]}\textsuperscript{,}\textsuperscript{[}\textsuperscript{53}\textsuperscript{]}. Whereas many observers had decried the effect of previous blood tests that
could only exclude, but not affirm paternity as potentially benefitting only the
alleged father, \textsc{dna} tests were thus promoted by Pena and others as vindication
for women\textsuperscript{[}\textsuperscript{54}\textsuperscript{]}.

More importantly, \textsc{dna} testing was hailed as a tool that supported children’s
rights. Brazil had been a leader in the elaboration of children’s rights
legislation since passage of the Minors’ Code in 1927 (\textit{Decree n.
17,943A}
), though its implementation remained deficient. In the early 1980s, the global
debt crisis swelled the ranks of destitute “street children”, who came to
symbolize the inadequacy of social policies enacted by the 26-year military
regime (1964-1985)\textsuperscript{[}\textsuperscript{55}\textsuperscript{]}. In response, the 1988 Constitution defined children’s rights as an “absolute
priority”, guaranteeing all children, whether biological or adoptive, the right
to nurturance within a family and eliminating legal distinctions among them
(Art. 227). In 1990, the \textit{Statute of the Child and Adolescent}
(\textit{Law n. 8,069/90}
), in consonance with the 1990 UN Convention on the Rights of the Child, defined
the right to “\textit{recognition of the status of filiation}
” as an “\textit{inalienable, essential individual right}
” (Art. 27). In 2009, the statute was amended to include adopted children’s
right “\textit{to know their biological origin}
”, although this knowledge does not establish legal parentage (Art. 48)
(children born through gamete donation do not share this right)\textsuperscript{[}\textsuperscript{56}\textsuperscript{]}. Judicial doctrine linked the right to knowledge of genetic heritage to the
right to personhood (personalidade), a core element of humanity, dignity, and
citizenship\textsuperscript{[}\textsuperscript{12}\textsuperscript{]}.

State initiatives that aimed to secure children’s right to parental recognition
supported the growing faith in \textsc{dna} testing. A 1992 law (\textit{Law n. 8,560/92}, Art. 2) mandated a paternity investigation whenever a child’s birth was
registered with only the mother’s name. The law was seldom implemented, but
various state agencies embarked on “responsible paternity” campaigns,
identifying fathers who had not registered their children and facilitating
voluntary recognition\textsuperscript{[}\textsuperscript{13}\textsuperscript{]}. When an alleged father denied his paternity, officials arranged a \textsc{dna} test if
financial resources were available; if not, they sometimes helped women file
suit. In 2003, the Supreme Court confirmed that state prosecutors could
intervene in judicial proceedings on behalf of mothers and children\textsuperscript{[}\textsuperscript{57}\textsuperscript{]}. Until recently, the cost of \textsc{dna} tests was prohibitively high for most
Brazilians, leading some states, followed by the federal government, to mandate
its provision free of charge in legal disputes (\textit{Law n. 10,317/2001}
).

Two controversial \textsc{stf} decisions regarding \textsc{dna} testing, both of which supported
prevailing trends in the jurisprudence of the \textsc{stj}, also rested on children’s
constitutional rights to paternal recognition and personhood. In a split 1994
habeas corpus decision, the Court determined that the state could not force
someone to submit to a \textsc{dna} test, but indicated that paternity would be presumed
if an alleged father refused. Emphasizing the broad significance of the
decision, Minister Francisco Rezek pointed out that “\textit{with the new exam,
for the first time, legal truth, generally based on presumption, has come to
correspond to scientific truth}
”, allowing the courts “\textit{to replace legal fiction}
[a verdade ficta, literally, fictive truth] \textit{with the real truth}
[a verdade real]”\textsuperscript{[}\textsuperscript{58}\textsuperscript{]}.

In another controversial split decision, the Court ruled in 2011 that paternity
investigations that had been litigated when the parties did not have access to
\textsc{dna} testing could be re-tried after the lapse of the two-year limit set by law
(Art. 975, \textit{Law n. 13,105/2015}
)\textsuperscript{[}\textsuperscript{59}\textsuperscript{]}. Once again, the effects of \textsc{dna} paternity testing were far-reaching: the
decision opened an exception, for the first time under the 1988 Constitution, to
the laws governing res judicata, the principle that precludes further litigation
on an “already judged” case. While acknowledging that res judicata is necessary
for judicial security, and thus democracy, the Court argued that in matters of
fundamental rights, “the real truth” constitutes a higher principle. Writing for
the majority, Minister Dias Toffoli quoted appeals decisions that argued that
“\textit{the substitution of legal fiction with real truth}
” represented “\textit{the advancement of juridical science}
” in the service of justice, which “\textit{must be held above security, for
without Justice there is no freedom}
”\textsuperscript{[}\textsuperscript{59}\textsuperscript{]}
(p. 26).

Both of these Supreme Court decisions hinged on the rights of children
(including adult children) to obtain information regarding their ancestry and
demand recognition by their fathers. The exception to res judicata, however,
could also benefit men who wished to disestablish paternity, a procedure that
was greatly facilitated by \textsc{dna} testing. As is true around the world, \textsc{dna} testing
provoked intense debates over whether to lift traditional restrictions on
paternity disestablishment. Among the most contentious of these debates focuses
on the “marital presumption”, a nearly universal norm derived from Roman Law
that attributes paternity to husbands and limits the time frame and grounds for
rescission. Recent reviews of laws and jurisprudence in the United States and
Europe show continued (though uneven) support for maintaining the marital
presumption even when \textsc{dna} testing reveals the absence of biological fatherhood\textsuperscript{[}\textsuperscript{6}\textsuperscript{]}. The Brazilian courts and legislature were more receptive to the argument that
the “real truth”, revealed through \textsc{dna} testing, could render this legal fiction
obsolete, and in 2002, the new civil code eliminated previous restrictions on
husbands who wished to disestablish paternity (Arts. 178 and 340)\textsuperscript{[}\textsuperscript{18}\textsuperscript{]}
(Arts. 1,597, 1,601, and 1,604)\textsuperscript{[}\textsuperscript{60}\textsuperscript{]}.

Despite their enthusiasm over the potential of \textsc{dna} testing to reveal “the real
truth”, Brazil’s jurists and legislators did not abandon the long-standing
principle that medical evidence should supplement, not supplant, the corpus of
legal evidence. When the Supreme Court indicated in 1994 that paternity would be
presumed when the defendant refused a \textsc{dna} test, it explained that “\textit{the
refusal must be resolved}
(...) \textit{through legal instruments}
”\textsuperscript{[}\textsuperscript{58}\textsuperscript{]}
(p. 420), allowing a judge to analyze and weigh all of the evidence\textsuperscript{[}\textsuperscript{58}\textsuperscript{]}. Controversy over the relative weight of a refusal continued over the next two
decades, finally resulting in a 2009 law that placed the presumption “\textit{in
the context of other evidence}
” (\textit{Law n. 12,004/2009}
)\textsuperscript{[}\textsuperscript{61}\textsuperscript{]}. Moreover, refusal by a child did not carry a corresponding presumption of the
absence of paternity\textsuperscript{[}\textsuperscript{62}\textsuperscript{]}. Most importantly, although \textsc{dna} evidence alone was generally sufficient to
establish paternity in the interest of the child, by 2011, jurisprudence had
firmly established that it was not sufficient for disestablishment of paternity
by a legal father. In its exception to res judicata, for example, the Supreme
Court indicated that the precedent would apply only to cases in which
“\textit{there is not a dispute between paternity of a biological type and
paternity of an affective type}
”\textsuperscript{[}\textsuperscript{59}\textsuperscript{]}
(p. 2).

These decisions reflected the intense doctrinal and jurisprudential debate that
accompanied the incorporation of \textsc{dna} testing into family law in Brazil and
around the world. The most pressing concern was (and is) that legal support for
\textsc{dna} testing, and massive publicity surrounding it, encouraged fathers to
“resolve their doubts” without consideration for the best interests of the
child, an issue that was also raised by social scientists\textsuperscript{[}\textsuperscript{14}\textsuperscript{]}\textsuperscript{,}\textsuperscript{[}\textsuperscript{63}\textsuperscript{]}\textsuperscript{,}\textsuperscript{[}\textsuperscript{64}\textsuperscript{]}\textsuperscript{,}\textsuperscript{[}\textsuperscript{65}\textsuperscript{]}. As Finamori\textsuperscript{[}\textsuperscript{36}\textsuperscript{]}
observes, this dilemma was not new: it had arisen when legal-medical specialist
Flamínio Fávero first introduced \textsc{abo} blood testing in Brazil in 1927. Fávero
himself believed that since the process could prove devastating to a child,
paternity exams should be legally permitted only as part of a judicial process
in which the judge carefully considered the circumstances of the case\textsuperscript{[}\textsuperscript{36}\textsuperscript{]}. The advent of \textsc{dna} testing heightened this concern. In France and Germany, the
two countries whose civil law had most influenced Brazil’s since the nineteenth
century, legislators prohibited private paternity testing, allowing it only with
a court order.

In Brazil, where the law did not limit access to \textsc{dna} paternity testing, many
critics noted that by the early 2000s, the role of family court judges had been
reduced to “merely verifying the test results” in paternity cases. The nation’s
leading family law specialists derided this trend as the “biologization of
paternity” or “sanctification of \textsc{dna}”\textsuperscript{[}\textsuperscript{66}\textsuperscript{]}\textsuperscript{,}\textsuperscript{[}\textsuperscript{67}\textsuperscript{]}. A common early complaint was that \textsc{dna} testing unnecessarily strained the
budgets of public legal services when less costly \textsc{hla} blood testing was often
equally effective. Moreover, like earlier generations of jurists, many were
concerned about the reliability of biological evidence in the absence of
consistent state regulation and oversight of the exams at public and private
laboratories. Complaints of poor quality control and even fraud were common, and
many tests had to be resubmitted multiple times\textsuperscript{[}\textsuperscript{8}\textsuperscript{]}\textsuperscript{,}\textsuperscript{[}\textsuperscript{36}\textsuperscript{]}\textsuperscript{,}\textsuperscript{[}\textsuperscript{67}\textsuperscript{]}.

The most significant criticism rested on moral and philosophical grounds. Just
as it became possible to prove biological paternity with near certainty, the
longstanding emphasis on social and emotional attributes of paternity gained new
significance. The trend was not unique to Brazil, but according to legal scholar
Paulo Lôbo, Brazilian doctrine has moved further in this direction than that of
any other nation\textsuperscript{[}\textsuperscript{68}\textsuperscript{]}. Lôbo does not support this claim with extensive comparative research, but his
account identifies the particular ways the concept developed in Brazilian legal
doctrine. A major catalyst was the 1988 Constitution, which led legal scholars
to reconceptualize the family through the principles of equality,
non-discrimination, and human dignity. This facilitated the adaptation of family
law to social reality, at a time when the wave of social mobilization that had
shaped the constitution’s human rights framework inspired new demands for
children’s rights and equality for increasingly varied family forms\textsuperscript{[}\textsuperscript{55}\textsuperscript{]}\textsuperscript{,}\textsuperscript{[}\textsuperscript{69}\textsuperscript{]}.

In the late 1990s, jurists affiliated with the Brazilian Institute of Family Law
(\textsc{ibdfam}), a progressive legal association, adopted the term “socio-affective” (a
term also used in Francophone nations\textsuperscript{[}\textsuperscript{70}\textsuperscript{]}
) as a way to describe and advocate for protection of these and other “plural
family forms”\textsuperscript{[}\textsuperscript{8}\textsuperscript{]}\textsuperscript{,}\textsuperscript{[}\textsuperscript{9}\textsuperscript{]}\textsuperscript{,}\textsuperscript{[}\textsuperscript{68}\textsuperscript{]}\textsuperscript{,}\textsuperscript{[}\textsuperscript{71}\textsuperscript{]}. Lôbo notes that the concept of the socio-affective family, borrowed from the
social sciences, is especially appropriate in Brazil, where scholars have long
observed the strong tradition in popular culture of non-consanguineous family
formation through consensual unions, informal adoption, and other practices\textsuperscript{[}\textsuperscript{55}\textsuperscript{]}\textsuperscript{,}\textsuperscript{[}\textsuperscript{68}\textsuperscript{]}\textsuperscript{,}\textsuperscript{[}\textsuperscript{72}\textsuperscript{]}\textsuperscript{,}\textsuperscript{[}\textsuperscript{73}\textsuperscript{]}
(though the same could be said regarding Latin America as a whole\textsuperscript{[}\textsuperscript{74}\textsuperscript{]}\textsuperscript{,}\textsuperscript{[}\textsuperscript{75}\textsuperscript{]}
). Brazilian scholars and feminist activists have also long observed the
especially deep roots of the patriarchal and “patrimonial” model of family in
law and society. Their struggle to dismantle it did not begin with the 1988
Constitution, but had accompanied the elaboration of family law from its
inception in the nineteenth century\textsuperscript{[}\textsuperscript{69}\textsuperscript{]}.

Like their predecessors of earlier decades, jurists affiliated with \textsc{ibdfam}
argued that social and emotional bonds should constitute legal grounds for
paternity investigations, but no longer merely as evidence of a biological
relationship. Their influence was felt in jurisprudence and legislative reform
efforts such as a 2005 bill to prohibit disestablishment of paternity
“\textit{in cases where possession of status of filiation has been established}
”. The proposal explicitly challenged the conflation of “real truth” with the
results of \textsc{dna} testing while asserting that this error was being corrected by
“\textit{doctrine and jurisprudence}
[that] \textit{has increasingly emphasized that the real paternal-filial
relationship does not derive from biological truth, but rather socio-affective
truth}
”\textsuperscript{[}\textsuperscript{76}\textsuperscript{]}
(p. 9788).

Although the 2005 bill did not become law, this norm was solidified in national
jurisprudence through scores of cases of contested paternity heard by federal
appeals courts over the decade that followed. It was applied most consistently
in cases in which the mother’s companion had assumed paternity of her child by
another man and “legalized” the relationship by signing the child’s birth
registry, a practice so common that it is known as “Brazilian-style adoption”\textsuperscript{[}\textsuperscript{55}\textsuperscript{]}. According to research by anthropologist Claudia Fonseca in the State of Rio
Grande do Sul, in 2002-2003, many such fathers procured a \textsc{dna} test in order to
disestablish paternity after separating from the mother, and judges routinely
approved their petitions\textsuperscript{[}\textsuperscript{63}\textsuperscript{]}. Yet cases of this sort were frequently overturned on appeal, and by 2006,
several state appeals courts had “\textit{enshrined the understanding that
so-called ‘Brazilian-style adoption’ ... is irrevocable}
”\textsuperscript{[}\textsuperscript{77}\textsuperscript{]}
(p. 2). The federal appeals court upheld this position, viewing “Brazilian-style
adoption” as a form of “real adoption”, based on the father’s informed consent
and the child’s possession of status\textsuperscript{[}\textsuperscript{78}\textsuperscript{]}\textsuperscript{,}\textsuperscript{[}\textsuperscript{79}\textsuperscript{]}. Yet the \textsc{stj} also established that a child who was adopted “Brazilian style”
does has the right to disestablish legal paternity on the basis of a \textsc{dna} test,
arguing that this is consistent with the prioritization of the best interests of
the child and the fundamental right to personhood\textsuperscript{[}\textsuperscript{80}\textsuperscript{]}.

In decisions regarding cases in which a man claims to have been deceived when he
established legal paternity (a situation known as “paternity fraud” in the
United States), appeals courts have been less likely to discount the biological
“truth”, and there has been conflicting jurisprudence. Influential decisions in
2006 and 2007 established that a in such cases, a legal father, whether or not
married to the child’s mother, has the right to rescind paternity on the basis
of a \textsc{dna} test, and the child’s best interest lies in the “real truth”\textsuperscript{[}\textsuperscript{77}\textsuperscript{]}\textsuperscript{,}\textsuperscript{[}\textsuperscript{81}\textsuperscript{]}. In 2013, however, the \textsc{stj} published a verdict that exemplifies its
jurisprudence: “\textit{the \textsc{stj} has enshrined the understanding that ... to
disestablish paternity requires demonstration, at the same time, of the absence
of biological origins and also that possession of status of filiation, marked by
socio-affective relations, has not been constituted}
”\textsuperscript{[}\textsuperscript{62}\textsuperscript{]}
(p. 10).

This does not mean that the tension between biological and socio-affective
paternity has been definitively resolved. Disputes continue to generate
conflicting decisions; the doctrine influenced by progressive legal scholarship
does not consistently prevail in family court, public policy, or popular
culture; and new situations and arguments constantly emerge. Examples include
demands for legal recognition of “parental multiplicity”, or more than two
parents, and “parallel families”, traditionally characterized as a man’s
simultaneous relationships with more than one woman and their respective
children, as well as issues surrounding assisted reproduction, such as gamete
donors’ right to anonymity\textsuperscript{[}\textsuperscript{82}\textsuperscript{]}\textsuperscript{,}\textsuperscript{[}\textsuperscript{83}\textsuperscript{]}. As Lôbo indicated in 2004, the criteria for determining when socio-affective
paternity should take precedence over biology, and what is in the best interest
of the child, continues to require careful attention to specific circumstances,
and to socio-cultural and technological changes, on a case-by-case basis\textsuperscript{[}\textsuperscript{12}\textsuperscript{]}.

\section{Conclusion}

Exploring the history of legal contests over paternity in Brazil highlights the
complexities and tensions that have accompanied the development of
identification sciences, whether applied to individual families or population
groups. Over the past century, for Brazilian scientists and their international
collaborators, Brazil was an ideal laboratory for the elaboration and use of
biotypology and a field site for large-scale population genetics. In their quest
to map Brazil’s composition, these scientists confirmed that the country is
characterized not by fixed racial or ethnic types, but by complex ancestry and
significant regional variation. At the same time, Brazilian jurists struggled to
apply identification sciences in the courts, with limited success. Ironically,
the arrival of \textsc{dna} testing in the 1980s, despite its high level of accuracy, did
not enshrine genetic science as the ultimate arbiter of either racial identity
or family relationships. On both the macro level of population and nation, and
the micro level of individual and family, genetic identification techniques
resulted in most cases not in precision and certitude but in a range of “shadows
of doubt” and in legal doctrine that distinguishes socio-affective from genetic
paternity. By the turn of the twenty-first century, jurists and scientists alike
came to embrace social and cultural criteria not merely as proxy for biological
proof of racial or family identity, but as its fundamental constitutive element.
The holy grail of truth sought by identification and juridical science
ultimately was not found in \textsc{dna}, but rather in the shifting relationship between
biological and cultural dimensions of individual and collective identity.

Acknowledgments
We thank the University of Michigan Brazil Initiative for support for travel and
research, special issue co-editors Dora Chor and Ricardo Ventura Santos for
their incisive feedback, Leandro Carvalho for his expert editorial assistance,
and two anonymous reviewers for their constructive criticism on early drafts.

\section*{References}
\begin{itemize}

\item[1] Pena SD. A revolução dos testes de \textsc{dna}. Ciênc Hoje 2010; 9 jul.

\item[2] Nelson A. The social life of \textsc{dna}: reparations and reconciliation
after the genome. Boston: Beacon Press; 2016.

\item[3] Wade P, Santos RV, Restrepo E, Lopez Beltran C. Mestizo genomics:
race mixture, nation, and science in Latin America. Durham: Duke University
Press; 2014.

\item[4] Penchaszadeh V, editor. Genética y derechos humanos: encuentros y
desencuentros. Buenos Aires: Paidós; 2012.

\item[5] Milanich N. Certain mothers, uncertain fathers: placing reproductive
technologies in historical perspective. In: Michel S, Ergas Y, editors. Bodies
and borders: negotiating motherhood in the 21st century. New York: Columbia
University Press; in press.

\item[6] Browne-Barbour V. "Mama's baby, papa's maybe": disestablishment of
paternity. Akron Law Rev 2015; 48:264-313.

\item[7] Fuchs RG. Contested paternity: constructing families in modern
France. Baltimore: Johns Hopkins University Press; 2008.

\item[8] Fachin LE. Da paternidade: relação biológica e afetiva. Belo
Horizonte: Del Rey; 1996.

\item[9] Welter BP. Igualdade entre as filiações biológica e socioafetiva.
São Paulo: Editora Revista dos Tribunais; 2003.

\item[10] Dias MB. Investigando a parentalidade. Revista \textsc{cej} 2004; 8:64-8.

\item[11] Lôbo \textsc{pln}. O exame de \textsc{dna} e o princípio da dignidade da pessoa
humana. Revista Brasileira de Direito de Família 1999; 1:67-78.

\item[12] Lôbo \textsc{pln}. Direito ao estado de filiação e direito à origem
genética: uma distinção necessária. Revista \textsc{cej} 2004; 8:47-56.

\item[13] Caulfield S. The right to a father's name: a historical
perspective on state efforts to combat the stigma of illegitimate birth in
Brazil. Law and History Review 2012; 30:1-36.

\item[14] Fonseca C. Parentesco, tecnologia e lei na era do \textsc{dna}. Rio de
Janeiro: \textsc{eduerj}; 2014.

\item[15] Lewin L. Surprise heirs: illegitimacy, patrimonial rights, and
legal nationalism in Luso-Brazilian inheritance, 1750-1821. v. 1. Stanford:
Stanford University Press; 2003.

\item[16] Lewin L. Surprise heirs: illegitimacy, inheritance rights, and
public power in the formation of imperial Brazil. v. 2. Stanford: Stanford
University Press; 2003.

\item[17] Milanich N. To make all children equal is a change in the power
structures of society: the politics of family law in twentieth century Chile and
Latin America. Law and History Review 2015; 33:767-802.

\item[18] Brasil. Lei nº 3.071, de 1º de janeiro de 1916. Código Civil dos
Estados Unidos do Brasil. Diário Oficial da União 1916; 5 jan.

\item[19] Brasil. Código Civil brasileiro: trabalhos relativos á sua
elaboração. v. 1. Rio de Janeiro: Imprensa Nacional; 1917.

\item[20] Soares de Faria S. Investigação da paternidade illegitima. São
Paulo: Saraiva; 1926.

\item[21] Souza A. Posição juridica dos filhos naturais em face do nosso
direito. São Paulo: Hennies Irmãos; 1916.

\item[22] Ferreira AA. Determinação médico-legal da paternidade (legislação,
doutrina e perícia). São Paulo: Companhia Melhoramentos de São Paulo; 1939.

\item[23] Azevedo N. Da prova na investigação da paternidade. São Paulo:
Empresa Gráfica da Revista dos Tribunais; 1928.

\item[24] Zicarelli Filho F. Investigação da paternidade natural. Curitiba:
Guaíra; 1941.

\item[25] Góes Filho JF. A Investigação da paternidade illegitima no codigo
civil brasileiro [Associate Professorship Dissertation]. Salvador: Faculdade de
Direito da Bahia; 1930.

\item[26] Corrêa M. As ilusões da liberdade: a escola Nina Rodrigues e a
antropologia no Brasil. Rio de Janeiro: Editora Fiocruz; 2013.

\item[27] Peixoto A. Novos rumos da medicina legal. Rio de Janiero:
Companhia Nacional; 1938.

\item[28] Silva EL. Manual de medicina legal. 3ª Ed. São Paulo: Sugestões
Literárias; 1964.

\item[29] Vimieiro Gomes AC. The emergence of biotypology in Brazilian
medicine: the Italian model, textbooks, and discipline building, 1930-1940. In:
Simões A, Diogo MP, Gavroglu K, editors. History of European universities, 19th
and 20th centuries: challenges and transformations. New York: Springer; 2015. p.
361-80.

\item[30] Stern AM. From mestizophilia to biotypology: racialization and
science in Mexico, 1920-1960. In: MacPherson AS, Appelbaum NP, Rosemblatt KA,
editors. Race and nation in modern Latin America. Chapel Hill: University of
North Carolina Press; 2003. p. 187-209.

\item[31] Bangham J. Blood groups and human groups: collecting and
calibrating genetic data after World War Two. Stud Hist Philos Biol Biomed Sci
2014; 47 Part A:74-86.

\item[32] Mouton M. From nurturing the nation to purifying the volk: Weimar
and Nazi family policy, 1918-1945. Cambridge: Cambridge University Press; 2007.

\item[33] Santos RV, Kent M, Gaspar Neto VV. From degeneration to meeting
point: historical views on race, mixture, and biological diversity of the
Brazilian population. In: Wade P, López Beltrán C, Restrepo E, Santos RV,
editors. Mestizo genomics: race mixture, nation, and science in Latin America.
Durham: Duke University Press; 2014. p. 33-54.

\item[34] Ferreira AA. Investigação médico-legal da paternidade. Revista
Médica 1953; 37:1-25. Reprinted 2006; 4:142-56.

\item[35] Sociedade de Medicina Legal e Criminologia. Correio Paulistano
1927; 9 sep.

\item[36] Finamori SD. Os sentidos da paternidade: dos "pais desconhecidos"
ao exame de \textsc{dna} [Doctoral Dissertation]. Campinas: Universidade Estadual de
Campinas; 2012.

\item[37] Lins e Silva A. Estudos de medicina legal. Rio de Janeiro: A.
Coelho Branco; 1938.

\item[38] Almeida Júnior AF. As provas genéticas da filiação. São Paulo:
Revista dos Tribunais; 1941.

\item[39] Fida O, Albuquerque \textsc{jbt}. Investigação de paternidade: teoria,
formulários, jurisprudência, legislação. 4ª Ed. Campinas: Julex Livros; 1987.

\item[40] Amar AM. Revisão da experiência de 23 anos em investigações de
paternidade e maternidade. Arquivos da Polícia Civil de São Paulo 1984;
42:67-76.

\item[41] Barra WE. Investigação de paternidade. Revista do Advogado 1988;
25:66-9.

\item[42] Gomes O, Teodoro Jr. H. Direito de família. 14ª Ed. Rio de
Janeiro: Forense; 2001.

\item[43] Fávero F, Novah E. Os typos sanguineos no meio universitário de
São Paulo. Archivos da Sociedade Medicina Legal e Criminologia de São Paulo
1938; 28-30.

\item[44] Lindee S, Santos RV. The biological anthropology of living human
populations: world histories, national styles, and international networks. Curr
Anthropol 2012; 53 Suppl 5:S3-16.

\item[45] Dent R. Including the indigenous: Xavante genes in Brazilian human
population genetics since 1950. In: \textsc{xxxiii} International Congress of the Latin
American Studies Association. San Juan: Latin American Studies Association;
2015.

\item[46] de Souza VS, Santos RV. The emergence of human population genetics
and narratives about the formation of the Brazilian nation (1950-1960). Stud
Hist Philos Biol Biomed Sci 2014; 47 Pt A:97-107.

\item[47] Kent M, Santos RV. "The charrua are alive": the genetic
resurrection of an extinct indigenous population in southern Brazil. In: Wade P,
López Beltrán C, Restrepo E, Santos RV, editors. Mestizo genomics: race mixture,
nation, and science in Latin America. Durham: Duke University Press; 2014. p.
109-34.

\item[48] Callegari-Jacques SM, Grattapaglia D, Salzano FM, Salamoni SP,
Crossetti SG, Ferreira ME, et al. Historical genetics: spatiotemporal analysis
of the formation of the Brazilian population. Am J Hum Biol 2003; 15:824-34.

\item[49] Pena \textsc{sdj}. Razões para banir o conceito de raça da medicina
brasileira. Hist Ciênc Saúde-Manguinhos 2005; 12:321-46.

\item[50] Jeffreys AJ, Pena \textsc{sdj}. Brief introduction to human \textsc{dna}
fingerprinting. In: Pena SD, editor. \textsc{dna} fingerprinting: state of the science.
Boston: Springer; 1993. p. 1-20.

\item[51] Pena SD, Chakraborty R. Paternity testing in the \textsc{dna} era. Trends
Genet 1994; 10:204-9.

\item[52] Magalhães M. Mercado já movimenta R\$ 6,5 mi. Folha S. Paulo 1997;
3 nov.

\item[53] Thurler AL. Outros horizontes para a paternidade brasileira no
século \textsc{xxi}? Sociedade e Estado 2006; 21:681-707.

\item[54] Pena \textsc{sdj}. O \textsc{dna} como (única) testemunha em determinação de
paternidade. Rev Bioét 1997; 5:231-41.

\item[55] Fonseca C. Inequality near and far: adoption as seen from the
Brazilian favelas. Law Soc Rev 2002; 36:397-432.

\item[56] Conselho Federal de Medicina. Resolução \textsc{cfm} nº 2.121/2015. Diário
Oficial da União 2015; 24 set.

\item[57] Supremo Tribunal Federal. RE 248869-1/SP 2003. Diário da Justiça
2004; 12 mar. \href{http://redir.stf.jus.br/paginadorpub/paginador.jsp?docTP=AC\
&docID=257829}.

\item[58] Supremo Tribunal Federal. HC 71373-4/RG 1994. Diário da Justiça
1996; 22 nov.
\href{http://redir.stf.jus.br/paginadorpub/paginador.jsp?docTP=AC\&docID=73066}.

\item[59] Supremo Tribunal Federal. RE 363889/DF 2011. Diário da Justiça
Eletrônico 2011; 15 dez. \href{http://redir.stf.jus.br/paginadorpub/paginador.js
p?docTP=TP\&docID=1638003}.

\item[60] Brasil. Lei nº 10.406, de 10 de janeiro de 2002. Institui o Código
Civil. Diário Oficial da União 2002; 11 jan.

\item[61] Lôbo \textsc{pln}. Paternidade socioafetiva e o retrocesso da Súmula no 301
do \textsc{stj}. Revista Jus Navigandi 2006; 11(1036).

\item[62] Superior Tribunal de Justiça. REsp 1.115.428-SP. Informativo de
Jurisprudência 2013; 20 nov. \href{http://www.stj.jus.br/\textsc{scon}/Search\textsc{brs}?b=\textsc{infj}\&
tipo=informativo\&livre=@\textsc{cod}=\%270530\%27}.

\item[63] Fonseca C. Paternidade brasileira na era do \textsc{dna}: a certeza que
pariu a dúvida. Cuadernos de Antropologia Social 2005; (22):27-51.

\item[64] Milanich N. Blood will sell.
\href{https://medium.com/@NaraMilanich/blood-will-sell-add5cd8179c5\#.g91tldn34}
(accessed on 15/Oct/2016).

\item[65] Draper H, Ives J. Paternity testing: a poor test of fatherhood. J
Soc Welf Fam Law 2009; 31:407-18.

\item[66] Madaleno R. A sacralização da presunção na investigação de
paternidade. \href{http://www.rolfmadaleno.com.br/novosite/conteudo.php?id=30}
(accessed on 21/Jun/2016).

\item[67] Rodrigues CS. O exame de \textsc{dna} e sua influência nas ações de
investigação de paternidade. Direito \& Justiça 2005; 31(2).
\href{http://revistaseletronicas.pucrs.br/ojs/index.php/fadir/article/view/570}.

\item[68] Lôbo \textsc{pln}. Socioafetividade em família e a orientação do \textsc{stj}.
Revista Jus Navigandi 2013; 18(3760).

\item[69] Caulfield S. From liberalism to human dignity: the transformation
of marriage and family rights in Brazil. In: Moses J, editor. Ties that bind:
global histories of marriage and modernity. London: Bloomsbury; in press.

\item[70] Gallus N. Le droit de la filiation rôle de la vérité
socio-affective et de la volonté en droit belge. Brussels: Larcier; 2009.

\item[71] Boeira \textsc{jbr}. Investigação de paternidade. Posse de estado de filho.
Paternidade socioafetiva. Porto Alegre: Livraria do Advogado; 1999.

\item[72] Cardoso \textsc{rcl}. Creating kinship: the fostering of children in favela
families in Brazil. In: Smith RT, editor. Kinship ideology and practice in Latin
America. Chapel Hill: University of North Carolina Press; 1984. p. 196-203.

\item[73] Moreno AZ. Vivendo em lares alheios: filhos de criação e adoção em
São Paulo colonial e em Portugal, 1765-1822. São Paulo: Annablume; 2013.

\item[74] Leinaweaver J. Introduction: cultural and political economies of
adoption in Latin America. J Lat Am Caribb Anthropol 2009; 14:1-19.

\item[75] Milanich N. Latin American childhoods and the concept of
modernity. In: Fass P, editor. The Routledge history of childhood in the Western
world. New York: Routledge; 2013. p. 491-508.

\item[76] Projeto de Lei nº 4.946. Diário da Câmara dos Deputados 2005; 2
abr.

\item[77] Superior Tribunal de Justiça. REsp 884.560/MS 2006. Diário da
Justiça Eletrônico 2008; 12 set. \href{https://ww2.stj.jus.br/processo/revista/d
ocumento/mediado/?componente=\textsc{mon}\&sequencial=4205009\&num\_{}registro=2006019356
91\&data=20080912}.

\item[78] Superior Tribunal de Justiça. REsp 1088157/PB 2008. Diário da
Justiça Eletrônico 2009; 4 ago. \href{https://ww2.stj.jus.br/processo/revista/do
cumento/mediado/?componente=\textsc{ita}\&sequencial=896969\&num\_{}registro=200801995643
\&data=20090804\&formato=\textsc{pdf}}.

\item[79] Superior Tribunal de Justiça. REsp 1098036/GO 2008. Diário da
Justiça Eletrônico 2012; 1 mar. \href{https://ww2.stj.jus.br/processo/revista/do
cumento/mediado/?componente=\textsc{ita}\&sequencial=1083493\&num\_{}registro=20080239670
2\&data=20120301\&formato=\textsc{pdf}}.

\item[80] Superior Tribunal de Justiça. REsp 1167993/RS. Informativo de
Jurisprudência 2013; (512), 20 fev. \href{http://www.stj.jus.br/\textsc{scon}/Search\textsc{brs}?b
=\textsc{infj}\&tipo=informativo\&livre=@\textsc{cod}=\%270512\%27}.

\item[81] Superior Tribunal de Justiça. REsp 878.954/RS, 2007. Diário da
Justiça Eletrônico 2007; 28 mai. \href{https://ww2.stj.jus.br/processo/revista/d
ocumento/mediado/?componente=\textsc{ita}\&sequencial=689273\&num\_{}registro=20060182349
0\&data=20070528\&formato=\textsc{pdf}}.

\item[82] Couto C. Famílias paralelas e poliafetivas. Revista Jus Navigandi
2015; 20(4409).

\item[83] Gozzo D. A controversa norma do \textsc{cnj} sobre registro em caso de
reprodução assistida. O Estado de S. Paulo 2016; 12 out.

\end{itemize}

\end{document}
