% Generated by jats2tex@0.11.1.0
\documentclass{article}
\usepackage[T1]{fontenc}
\usepackage[utf8]{inputenc} %% *
\usepackage[portuges,spanish,english,german,italian,russian]{babel} %% *
\usepackage{amstext}
\usepackage{authblk}
\usepackage{unicode-math}
\usepackage{multirow}
\usepackage{graphicx}
\usepackage{etoolbox}
\usepackage{xtab}
\usepackage{enumerate}
\usepackage{hyperref}
\usepackage{penalidades}
\usepackage[footnotesize,bf,hang]{caption}
\usepackage[nodayofweek,level]{datetime}
\usepackage[top=0.85in,left=2.75in,footskip=0.75in]{geometry}
\newlength\savedwidth
\newcommand\thickcline[1]{\noalign{\global
\savedwidth
\arrayrulewidth
\global\arrayrulewidth 2pt}
\cline{#1}
\noalign{\vskip\arrayrulewidth}
\noalign{\global\arrayrulewidth\savedwidth}}
\newcommand\thickhline{\noalign{\global
\savedwidth\arrayrulewidth
\global\arrayrulewidth 2pt}
\hline
\noalign{\global\arrayrulewidth\savedwidth}}
\usepackage{lastpage,fancyhdr}
\usepackage{epstopdf}
\pagestyle{myheadings}
\pagestyle{fancy}
\fancyhf{}
\setlength{\headheight}{27.023pt}
\lhead{\includegraphics[width=10mm]{logo.png}}
\rhead{\ifdef{\journaltitle}{\journaltitle}{}
\ifdef{\volume}{vol.\,\volume}{}
\ifdef{\issue}{(\issue)}{}
\ifdef{\fpage}{\fpage--\lpage\,pp.}}
\rfoot{\thepage/\pageref{LastPage}}
\renewcommand{\footrule}{\hrule height 2pt \vspace{2mm}}
\fancyheadoffset[L]{2.25in}
\fancyfootoffset[L]{2.25in}
\lfoot{\sf \ifdef{\articledoi}{\articledoi}{}}
\setmainfont{Linux Libertine O}
\renewcommand*{\thefootnote}{\alph{footnote}}
\makeatletter
\newcommand{\fn}{\afterassignment\fn@aux\count0=}
\newcommand{\fn@aux}{\csname fn\the\count0\endcsname}
\makeatother

\newcommand{\publisherid}{rpsp}
\newcommand{\journaltitle}{Revista Panamericana de Salud Pública}
\newcommand{\abbrevjournaltitle}{Rev Panam Salud Publica}
\newcommand{\issnepub}{1680-5348}
\newcommand{\issnppub}{1020-4989}
\newcommand{\publishername}{Organización Panamericana de la Salud}
\newcommand\articleid{40-029}
\def\subject{Informe especial}\newcommand{\subtitlestyle}[1]{--
\emph{#1}\medskip}
\newcommand{\transtitlestyle}[1]{\par\medskip\Large #1}
\newcommand{\transsubtitlestyle}[1]{-- \Large\emph{ #1}}

\newcommand{\titlegroup}{
\ifdef{\subtitle}{\subtitlestyle{\subtitle}}{}
\ifdef{\transtitle}{\transtitlestyle{\transtitle}}{}
\ifdef{\transsubtitle}{\transsubtitlestyle{\transsubtitle}}{}}

\title{La eco-epidemiología retrospectiva como herramienta aplicada a la
vigilancia de la leishmaniasis en Misiones, Argentina, 1920-2014\titlegroup{}}
\newcommand{\transtitle}{Retrospective eco-epidemiology as a tool for the
surveillance of leishmaniasis in Misiones, Argentina, 1920–2014}
\author[{1}]{Salomón, Oscar Daniel}
\author[{2}]{Mastrángelo, Andrea Verónica}
\author[{3}]{Santini, María Soledad}
\author[{1}]{Liotta, Domingo Javier}
\author[{4}]{Yadón, Zaida Estela}
\affil[1]{Consejo Nacional de Investigaciones Científicas y Técnicas (\textsc{conicet})}
\affil[2]{Universidad Nacional de San Martín (\textsc{unsam})}
\affil[3]{Consejo Nacional de Investigaciones Científicas y Técnicas (\textsc{conicet})}
\affil[4]{Organização Pan-Americana da Saúde}
\def\authornotes{La correspondencia se debe dirigir a odanielsalomon@gmail.com}
\date{ 07 2016}
\def\volume{40}
\def\issue{1}
\def\fpage{29}
\def\lpage{39}
\def\permissions{Este es un artículo publicado en acceso abierto bajo una
licencia Creative Commons}
\newcommand{\kwdgroupes}{Palabras claveLeishmaniasis cutánea, leishmaniasis
visceral, fenómenos ecológicos y ambientales, Argentina}
\newcommand{\kwdgroupen}{Key wordsLeishmaniasis, cutaneous, leishmaniasis,
visceral, ecological and environmental phenomena, Argentina}

\begin{document}
\selectlanguage{spanish}
\section*{Metadados não aplicados}
\begin{itemize}
\item[\textbf{língua do artigo}]{Espanhol}
\ifdef{\journalid}{\item[\textbf{journalid}] \journalid}{}
\ifdef{\journaltitle}{\item[\textbf{journaltitle}] \journaltitle}{}
\ifdef{\abbrevjournaltitle}{\item[\textbf{abbrevjournaltitle}]
\abbrevjournaltitle}{}
\ifdef{\issnppub}{\item[\textbf{issnppub}] \issnppub}{}
\ifdef{\issnepub}{\item[\textbf{issnepub}] \issnepub}{}
\ifdef{\publishername}{\item[\textbf{publishername}] \publishername}{}
\ifdef{\publisherid}{\item[\textbf{publisherid}] \publisherid}{}
\ifdef{\subject}{\item[\textbf{subject}] \subject}{}
\ifdef{\transtitle}{\item[\textbf{transtitle}] \transtitle}{}
\ifdef{\authornotes}{\item[\textbf{authornotes}] \authornotes}{}
\ifdef{\articleid}{\item[\textbf{articleid}] \articleid}{}
\ifdef{\articledoi}{\item[\textbf{articledoi}] \articledoi}{}
\ifdef{\volume}{\item[\textbf{volume}] \volume}{}
\ifdef{\issue}{\item[\textbf{issue}] \issue}{}
\ifdef{\fpage}{\item[\textbf{fpage}] \fpage}{}
\ifdef{\lpage}{\item[\textbf{lpage}] \lpage}{}
\ifdef{\permissions}{\item[\textbf{permissions}] \permissions}{}
\end{itemize}
\maketitle

\begingroup
\renewcommand{\abstractname}{\textsc{resumen}}
\begin{abstract}

Se presenta una metodología analítica retrospectiva, basada en el marco teórico
de la eco-epidemiología, anclada en una escala espacial subnacional. Esta
metodología, aplicada aquí a la caracterización de escenarios de transmisión de
la leishmaniasis en la provincia argentina de Misiones —fronteriza con Brasil y
Paraguay— permitió fundamentar recomendaciones de vigilancia y control
apropiadas a dicha escala. Se realizó una búsqueda exhaustiva de la literatura
sobre leishmaniasis en esa provincia y se determinaron tres escenarios de
transmisión de leishmaniasis cutánea (LC) y visceral (LV), correspondientes a
tres períodos: 1920-1997, en el que se constató la transmisión de LC, dispersa
en el tiempo y el espacio; 1998-2005, en el que hubo brotes focales de LC; y
2006-2014 en el que, además, se registraron brotes y se documentó la dispersión
geográfica de la LV. Para caracterizar los escenarios de riesgo y los procesos
antrópicos que los producen, los resultados se sintetizaron e integraron en el
contexto socio-histórico y bio-ecológico de cada período analizado. Se
fundamentan recomendaciones de vigilancia y control en el territorio estudiado,
entre ellas, establecer una vigilancia activa para monitorear posibles
tendencias al incremento de la circulación parasitaria y vectorial y, ante la
aparición de un foco, realizar estudios para verificar la transmisión autóctona
y la intensidad del evento. Además, se debe establecer la obligación legal de
tomar medidas adicionales de control por los responsables de los proyectos que
impliquen modificación ambiental, como la realización de estudios de evaluación
del riesgo de transmisión, y acciones de mitigación del riesgo, detección
temprana y tratamiento oportuno de los casos.

\iflanguage{portuges}{\medskip\noindent\textbf{Palavras-chave:} \kwdgroup}{}
\iflanguage{english}{\medskip\noindent\textbf{Keywords:} \kwdgroupen}{}
\iflanguage{spanish}{\medskip\noindent\textbf{Palavras claves:} \kwdgroupes}{}
\iflanguage{french}{\medskip\noindent\textbf{Mots clés:} \kwdgroupfr}{}
\end{abstract}
\endgroup

\begingroup
\renewcommand{\section}[1]{\subsection*{#1}}
\begin{otherlanguage}{english}
\renewcommand{\abstractname}{\textsc{abstract}}
\begin{abstract}

A retrospective analytical method is presented, based on theoretical
eco-epidemiology, applied on a subnational spatial scale. This method was used
here to describe scenarios for the transmission of leishmaniasis in the
Argentine province of Misiones— bordering Brazil and Paraguay—and formed the
basis for recommendations for surveillance and control appropriate to the
subnational scale. An exhaustive search of the literature on leishmaniasis in
the province was carried out. Three scenarios for the transmission of cutaneous
leishmaniasis (CL) and visceral leishmaniasis (VL) were found, corresponding to
three periods: from 1920 to 1997, during which the transmission of CL
distributed over time and space was confirmed; 1998 to 2005, during which there
were focal outbreaks of CL; and 2006 to 2014, during which outbreaks were also
reported and the geographical dispersion of VL was documented. To describe the
risk scenarios and the anthropic processes that produce them, the results were
summarized and integrated into the social, historical, and bio-ecological
context of each period. Surveillance and control recommendations are based on
the territory studied. They include establishing active surveillance to monitor
possible rising trends in parasitic and vector circulation, conducting studies
of any focal outbreak in order to confirm indigenous transmission and severity.
Also, it should be a legal requirement for persons responsible for projects that
alter the environment to adopt additional control measures, such as studies
assessing transmission risk, risk mitigation, early detection, and timely case
management.

\ifdef{\kwdgroupen}{\medskip\noindent\textbf{Keywords:} \kwdgroupen}{}
\end{abstract}
\end{otherlanguage}
\endgroup

La leishmaniasis es la tercera de las enfermedades de transmisión vectorial a
humanos en importancia por el número de casos y la población en riesgo, y la
novena con mayor carga de enfermedad (\textsuperscript{[}1\textsuperscript{]},\textsuperscript{[}2\textsuperscript{]}
). Se estima que el número anual de casos de leishmaniasis cutánea (LC) en el
mundo es de 0,7-1,2 millones, mientras que de leishmaniasis visceral (LV) se
registran 0,2-0,4 millones de casos (\textsuperscript{[}2\textsuperscript{]}
). En las Américas, entre 2001 y 2011, la notificación de LC osciló entre 47 000
y 68 000 casos al año, mientras que de LV se registraron 39 000 casos al año,
con una letalidad promedio de 8,4\%; en los países del Cono Sur se observó una
tendencia al incremento de casos en áreas urbanas
(\textsuperscript{[}3\textsuperscript{]}
).

En Argentina, la LC ha estado presente posiblemente desde tiempos precolombinos
(\textsuperscript{[}4\textsuperscript{]}
), aunque los primeros casos esporádicos registrados formalmente datan de las
primeras décadas del siglo XX. Esos casos se localizaron en 10 provincias del
norte y se asociaron con la entrada de personas a áreas silvestres para la
deforestación extractiva y la pesca (\textsuperscript{[}5\textsuperscript{]},\textsuperscript{[}6\textsuperscript{]}
). En la década de 1980 se registró el primer brote de LC en el noroeste del
país y desde entonces se registran focos epidémicos en el área de transmisión,
que comprende 500 000 km\textsuperscript{2}
(\textsuperscript{[}7\textsuperscript{]}
). Por su parte, la LV, de la que hasta 2006 se habían registrado 14 casos
dispersos en zonas rurales (\textsuperscript{[}8\textsuperscript{]}
), comenzó a dispersarse a partir de un foco registrado en la ciudad de Posadas,
capital de la provincia de Misiones. A inicios de 2015 se habían registrado 140
casos en humanos en cuatro provincias, y la presencia del vector ya se había
detectado en seis provincias (\textsuperscript{[}9\textsuperscript{]},\textsuperscript{[}10\textsuperscript{]}
).

El análisis de la dinámica de la leishmaniasis a escala global, continental o
nacional contribuye a elaborar políticas de salud. Sin embargo, para el diseño
de acciones programáticas adecuadas a las escalas espaciales inferiores a la
nacional se deben contemplar también estudios en el marco teórico de la
eco-epidemiología, que integran el conocimiento multidisciplinario y la
causalidad multinivel (\textsuperscript{[}11\textsuperscript{]}
). De esta manera, el uso de una perspectiva socio-histórica de esta enfermedad
en escalas intermedias —entre la global de larga duración y el estudio focal o
de casos que analiza usualmente períodos de días a meses— puede ayudar a
identificar procesos sociales asociados con brotes epidémicos, tales como obras
de desarrollo (gasoductos, carreteras, represas), deforestación, migración y
urbanización no planificada, entre otros
(\textsuperscript{[}12\textsuperscript{]}
-\textsuperscript{[}14\textsuperscript{]}
).

No obstante, pocas veces se ha utilizado la información del contexto
socio-histórico y bio-ecológico en una escala intermedia (provincia o
departamento) para fundamentar estrategias de vigilancia. Con vistas a explorar
esta perspectiva, se seleccionó la provincia de Misiones, por su localización
subtropical y por las transformaciones socio-ambientales que se sucedieron en
ella durante el siglo XX. En relación con ambas leishmaniasis, estas
transformaciones han generado dos tipos de “fronteras”: una al interior de su
territorio por los frentes de deforestación y otra en el límite externo de su
territorio con otras regiones endemo-epidémicas con las que tiene un intenso
tránsito vecinal fronterizo.

El objetivo de este trabajo fue ejemplificar la aplicación de una metodología
analítica retrospectiva, basada en el marco teórico de la eco-epidemiología,
anclada en una escala espacial subnacional. Esta metodología, aplicada aquí a la
caracterización de escenarios de transmisión de la leishmaniasis en la provincia
argentina de Misiones, permitió fundamentar recomendaciones de vigilancia y
control apropiadas a dicha escala.
\section{\textsc{fuentes} DE \textsc{información} Y \textsc{procedimientos}}

La provincia de Misiones, en el noreste de Argentina, posee un área de 29 801
km² y en 2012 contaba aproximadamente con 1 100 000 habitantes. Comparte 90\% de
su frontera con Paraguay y Brasil (1 080 km), mientras al suroeste limita con la
provincia argentina de Corrientes (110 km). Su clima es subtropical templado sin
estación seca; la temperatura promedio invernal es de 19 °C y la estival es de
29 °C. La flora y la fauna son típicas de la zona subtropical de dominio
amazónico, conservada en mosaicos selváticos protegidos. No obstante, 65\% de su
territorio se encuentra intensamente modificado, principalmente por la
agroindustria forestal y cultivos como yerba mate, té y tabaco. La capital
estatal, Posadas (27º 22’ S, 55º 53’ W), es la ciudad más poblada, mientras que
Puerto Iguazú (25º 36’ S, 54º 34’ W), a 304 km al norte de Posadas y fronteriza
con Brasil y Paraguay, es el sitio de mayor flujo turístico nacional e
internacional (> 1 000 000 personas/año), con numerosos ambientes protegidos
(\textsuperscript{[}15\textsuperscript{]}
).

Se realizó una búsqueda exhaustiva de la literatura sobre leishmaniasis, que
permitió describir y analizar los eventos y procesos eco-epidemiológicos
asociados con la leishmaniasis, así como los cambios en el uso de la tierra, en
la provincia de Misiones, Argentina, en el período entre 1920 y 2014. Estos
resultados, provenientes de diferentes enfoques disciplinarios, se integraron
luego para caracterizar los escenarios eco-epidemiológicos de transmisión de la
LC y la LV.

Para analizar retrospectivamente la información restringida a la provincia de
Misiones, se utilizó el método narrativo cronológico ajustado a tres etapas:
1920-1997, 1998-2005 y 2006-2014. Esta división se fundamenta en el patrón de
incidencia histórico a partir del registro del primer brote de LC en 1998 y el
registro del caso índice de LV en 2006 (\textsuperscript{[}16\textsuperscript{]},\textsuperscript{[}17\textsuperscript{]}
).

Para la determinación de los tipos sociales expuestos a riesgo y la indagación
del proceso salud-enfermedad-atención (\textsuperscript{[}18\textsuperscript{]}
) se aplicaron sucesivamente entrevistas en profundidad y semiestructuradas a
informantes clave: agricultores a escala familiar de las áreas de muestreo
entomológico, agentes del sistema asistencial y pacientes con leishmaniasis
(para reconstruir sus itinerarios terapéuticos). Estos actores sociales se
consideraron informantes clave por ser constitutivos del proceso
salud-enfermedad-atención-cuidado.

En las publicaciones entomológicas analizadas, para la captura de Phlebotominae
se utilizaron minitrampas de luz del tipo \textsc{cdc}, activadas durante toda la noche,
mediante trampeos transversales y longitudinales, en paisajes clasificados según
la cobertura vegetal y el uso de la tierra.

En las publicaciones consultadas sobre reservorios de leishmaniasis se
utilizaron dos enfoques: para reservorios silvestres se emplearon muestreos con
trampas Sherman y trampas jaula en sitios asociados con las capturas de
vectores; mientras que para reservorios domésticos, se realizaron estudios de
foco en sitios asociados con casos de LV o se realizaron muestreos
transversales. Los vectores y los reservorios se identificaron hasta el nivel de
especie, y se procesaron muestras para estudios parasitológicos y de biología
molecular.

Para la detección de la infección y la identificación del parásito en los
vectores, los posibles reservorios y las muestras clínicas de pacientes, en los
trabajos publicados se utilizó la técnica de la reacción en cadena de la
polimerasa combinada con el análisis de polimorfismo de fragmentos de \textsc{adn}
obtenidos por enzimas de restricción (\textsc{pcr}-\textsc{rflp}, por sus siglas en inglés), con
la subsiguiente secuenciación.

Dada la diversidad de métodos y objetivos encontrados en la literatura revisada,
para una descripción detallada de los procedimientos utilizados en cada caso se
deben consultar los artículos originales citados en esta y las siguientes
secciones.

\section{\textsc{situación} \textsc{epidemiológica} DE LA \textsc{leishmaniasis}}

En 1930, el primer censo sobre LC en Argentina no registró casos en el
territorio de la provincia de Misiones (\textsuperscript{[}5\textsuperscript{]}
). Sin embargo, Bertoni describió en 1927 la LC como una de las enfermedades que
afectaba a las comunidades nativas de la frontera argentino-paraguaya
(\textsuperscript{[}19\textsuperscript{]}
). En 1964 se reglamentó la notificación obligatoria de la leishmaniasis
(\textsuperscript{[}20\textsuperscript{]}
). En el período 1954-1974, en Misiones se informaron como promedio 7,2 casos al
año, mientras que entre 1975 y 1995 se notificaron 0,25 casos/año y de 1996 a
1997 se registraron 5,5 casos/año, en promedio
(\textsuperscript{[}16\textsuperscript{]},\textsuperscript{[}21\textsuperscript{]}
) (figura~\ref{fig:f1}
). Entre 1959 y 1970, se observó un incremento no significativo de casos de LC
en algunos años, sin que se pudiera encontrar alguna otra información en los
registros sanitarios o la prensa escrita. Los agentes sanitarios entrevistados
refirieron casos aislados en la zona norte, específicamente en los departamentos
Iguazú y General Manuel Belgrano (GM Belgrano), sin mencionar brotes.

Por otra parte, de los 14 casos de leishmaniasis con manifestaciones clínicas
viscerales notificados en Argentina hasta 1997 —aunque sin identificación del
agente etiológico—, ninguno ocurrió en Misiones
(\textsuperscript{[}8\textsuperscript{]}
).

Hasta 1950, la información sobre los vectores de \textit{Leishmania}
spp. proviene de campañas descriptivas de la fauna, en las que se registró una
gran riqueza de especies de Phlebotominae en zonas silvestres, incluidos
vectores de LC (\textsuperscript{[}6\textsuperscript{]},\textsuperscript{[}22\textsuperscript{]}
) (figura~\ref{fig:f2}
). Los muestreos recomenzaron más de 40 años después con capturas mensuales a
orillas del río Paraná entre 1993 y 1998, con el objetivo de evaluar el impacto
de la represa hidroeléctrica de Yacyretá y, en el mismo año 1998, en localidades
con casos recientes de LC. Los resultados mostraron el predominio de
\textit{Nyssomyia neivai}
en los ambientes modificados (98,7\%) y de \textit{Ny. whitmani}
en ambientes con menos intervención antrópica pero con antecedentes de casos
esporádicos de LC; ambas especies se consideran vectores de \textit{L.
braziliensis}
(\textsuperscript{[}23\textsuperscript{]}
).

En cuanto a \textit{Lutzomyia longipalpis}
—vector de \textit{L. infantum}, agente de la LV—, en 1951 se encontró un ejemplar hembra en el ambiente
selvático de Candelaria, Misiones; en el año 2000 se capturaron seis ejemplares
en Corpus, también en paisajes poco modificados
(\textsuperscript{[}8\textsuperscript{]}
) (figura~\ref{fig:f2}
).

El patrón epidémico de LC, que empezó a extenderse desde el noroeste de
Argentina en la década de 1980, alcanzó Misiones en 1998 (figuras~\ref{fig:f2}
y~\ref{fig:f3}
) (\textsuperscript{[}7\textsuperscript{]}
). El primer brote registrado ocurrió en la localidad de Puerto Esperanza
(departamento de Iguazú), donde más de 80\% de los casos (64\% en hombres; edad
de los casos entre 1 y 87 años) se concentró en un barrio periférico de 30
viviendas que no tenía continuidad urbana con el centro de la ciudad. La
distribución y la abundancia relativa de los vectores (\textit{Ny. neivai:}
79,7\%; \textit{Ny. whitmani:}
10,9\%) sugieren el riesgo de transmisión peridoméstico, especialmente en las
viviendas localizadas en el margen limítrofe con un parche de selva residual y
en los límites de un barrio situado en el borde de deforestación; en ambos
sitios se registraron los primeros casos en humanos
(\textsuperscript{[}16\textsuperscript{]}
).

En 2004 se registró otro brote de LC en personas residentes en localidades
próximas a la laguna artificial Urugua-í. El estudio se basó en 20 casos, todos
hombres adultos, de los cuales 18 trabajaban en la reforestación de coníferas en
zonas contiguas a esa laguna artificial; no se observaron síntomas clínicos en
sus convivientes. La especie más abundante en los remanentes de vegetación
primaria intercalados con monocultivos de coníferas fue \textit{Ny. whitmani}
; no se encontraron vectores en los domicilios de los casos detectados
(localidades de Libertad y Wanda) ni en los rodales de pinos asociados en tiempo
y espacio con la infección (\textsuperscript{[}16\textsuperscript{]}
). Los trabajadores ingresaban a los remanentes de vegetación en momentos de
descanso o permanecían allí al comienzo y al final de la jornada mientras
esperaban el traslado hacia sus viviendas
(\textsuperscript{[}24\textsuperscript{]}
).

En 2004-2005, en el hospital de Puerto Iguazú, aproximadamente a 50 km al norte
del brote anterior, se notificaron 36 casos de LC, 75\% de los cuales eran
hombres. De ellos, 86\% refirió haber estado en una zona conocida como Dos Mil
Hectáreas, que tiene remanentes de selva paranaense secundaria y está habitada
por ocupantes recientes de tierras para uso agrícola de subsistencia o
producción familiar. En el muestreo realizado entre 2005 y 2006 en esta misma
área, se encontró que \textit{Ny. whitmani}
(87,4\%) era la especie más abundante en los sitios de deforestación reciente
(roza, tumba y quema) para parcelas agrícolas, con presencia de corrales para
cerdos y gallinas, próximos al borde de deforestación
(\textsuperscript{[}25\textsuperscript{]},\textsuperscript{[}26\textsuperscript{]}
). Los casos clínicos y los ejemplares colectados de \textit{Ny. whitmani}
resultaron positivos a la presencia de \textsc{adn} de \textit{Leishmania}. Se encontraron lesiones compatibles con la infestación de este parásito en
roedores \textit{Akodon}
spp. y \textit{Rattus rattus}
y en el marsupial \textit{Didelphis albiventris}, aunque resultaron negativas a la \textsc{pcr}-\textsc{rflp} para especies de \textit{Leishmania}
; asimismo, se constató una asociación significativa entre la distribución
espacio-temporal de este parásito, la presencia de micromamíferos y la
abundancia de vectores (\textsuperscript{[}25\textsuperscript{]}
).

En 2006, se notificó en Posadas el primer caso humano autóctono urbano de LV en
Argentina (figura~\ref{fig:f3}
). En los registros hospitalarios no se encontraron casos previos con
sintomatología compatible y sin diagnóstico etiológico
(\textsuperscript{[}17\textsuperscript{]}
). Se identificó la especie \textit{L. infantum}
mediante \textsc{pcr}-\textsc{rflp} con secuenciación de \textsc{adn}, a partir de muestras de canes
sintomáticos y de \textit{Lu. longipalpis}
(\textsuperscript{[}27\textsuperscript{]}
) (figura~\ref{fig:f2}
). De los canes de una muestra obtenida por conveniencia en Posadas, 57,3\% dio
resultados positivos a \textit{L. infantum}
mediante pruebas serológicas o de \textsc{adn} (\textsuperscript{[}28\textsuperscript{]}
); los genotipos circulantes eran LiA1 y LiA2
(\textsuperscript{[}29\textsuperscript{]}
).

La feromona sexual de las poblaciones de \textit{Lu. longipalpis}
encontradas en Posadas es idéntica a la de las de Brasil, asociada a poblaciones
con tendencia a la dispersión de la LV en zonas urbanas, aunque el gen
\textit{per}
muestra diferenciación con las poblaciones del noreste y sudeste de Brasil
(\textsuperscript{[}30\textsuperscript{]}
). La distribución espacial de \textit{Lu. longipalpis}
en la ciudad presentó “islas” de alta abundancia, con una autocorrelación
espacial de 590-700 m, coherente con una dinámica de población fuente
metapoblacional, en la que estas islas de alta abundancia tienen la posibilidad
de colonizar por migración poblaciones locales extinguibles en áreas urbanas con
abundancia media o baja. En 2007, en el momento en que se habían registrado 4
casos de LV en humanos, se realizó un muestreo de vectores según una grilla que
dividía la ciudad de Posadas en áreas de 400 x 400 m, y se encontró el vector en
4\% de los sitios muestreados. En 2009, en el momento en que ya se había
registrado un acumulado de 32 casos de LV, se encontró mediante la misma
metodología de muestreo el vector en 31\% de los sitios, con un promedio por
trampa tres veces mayor que en el estudio anterior
(\textsuperscript{[}31\textsuperscript{]}
). La abundancia de \textit{Lu. longipalpis}
se asoció positivamente con las superficies cubiertas por árboles y arbustos en
un radio de 50 m y con el número de macetas con plantas. El vector se concentró
preferentemente en lugares con densidad urbana intermedia y espacios verdes
intercalados, y menos en áreas con densidades extremas, como la zona periférica
ruralizada y el centro de la ciudad (\textsuperscript{[}32\textsuperscript{]}
). Las hembras, aunque presentes desde las primeras horas de la tarde, registran
su mayor actividad en el período entre 7:00 pm y 9:00 pm, y continúan luego
ingresando a la trampa, con menor frecuencia, hasta el período comprendido entre
1:00 am y 3.00 am (verano y otoño). Entre perros y humanos, el vector solo optó
por alimentarse de humanos cuando estos estaban a una distancia de 1,5 m o menos
del perro. Así, los espacios domésticos de esparcimiento fuera de la vivienda y
la proximidad física entre los perros y las personas constituyen los sitios y
las prácticas de mayor riesgo para la transmisión de la LV en las zonas urbanas
(\textsuperscript{[}33\textsuperscript{]}
).

A partir de Posadas, ubicada en el límite sur de Misiones, los casos de LV
comenzaron a dispersarse progresivamente por el resto de la provincia
(\textsuperscript{[}9\textsuperscript{]}
) hasta llegar a Puerto Iguazú, en el extremo norte (cuadro 1), donde en 2010 se
encontró \textit{Lu. longipalpis}
y se detectaron canes infectados en el perímetro urbano. Entre 2006 y 2008, las
capturas en las áreas de transición rural-silvestre de Puerto Iguazú mostraron
el predominio de \textit{Ny. whitmani}
y la ausencia de \textit{Lu. longipalpis}. Pero ya en 2011, luego de la aparición de perros infectados, se colectó
\textit{Lu. longipalpis}
(26,7\%) en 31\% de las unidades domésticas, exclusivamente en zonas urbanas,
mientras que \textit{Ny. whitmani}
(67,5\%) se encontró en 21\% de los sitios ubicados principalmente en zonas
rurales y periurbanas (\textsuperscript{[}34\textsuperscript{]}
); esto implicó una segregación espacio-ambiental del riesgo de transmisión de
la LC y la LV. Sin embargo, en el área próxima a la frontera con Brasil y
Paraguay, las zonas de riesgo de LV y LC tienden a superponerse, tanto por el
tránsito de personas y canes entre las áreas rurales y urbanas, como por el
gradiente continuo de paisajes. Así, la colonización del vector en el sentido
del área rural a la urbana, \textit{Ny. whitmani}
comenzó a estar presente en unidades domésticas urbanas en la periferia de
Puerto Iguazú; mientras, en el sentido contrario (urbano a rural), \textit{Lu.
longipalpis}
comenzó a aparecer gradualmente (0,4\% de 38 277 casos de Phlebotominae
estudiados) en viviendas rurales con al menos 10 años de intervención humana, en
las que nunca se colectó el vector antes del año 2013 a pesar de los muestreos
regulares (\textsuperscript{[}35\textsuperscript{]}
).

\textit{Fuente:}
Elaboración propia a partir de datos suministrados por J. Estévez y J.
Gutiérrez, (comunicación personal, julio de 2015).

Las celdas en blanco indican que no hubo notificaciones.

Los departamentos se han ordenado de norte a sur.

En 2013, se encontró que 7,17\% de los perros de Puerto Iguazú resultó positivo
al diagnóstico serológico de LV y se tipificó el agente como \textit{L.
infantum}
(\textsuperscript{[}36\textsuperscript{]}
); por su parte, a comienzos de 2014 se registraron en Iguazú dos casos de LV en
personas adultas con enfermedades previas asociadas con deficiencias
inmunológicas; además, se detectó \textsc{adn} de \textit{L. infantum}
en individuos de \textit{Ny. whitmani}
y \textit{Migonemyia migonei}
capturados en esa localidad (\textsuperscript{[}37\textsuperscript{]}
).

Los casos detectados de LC, aunque luego de los brotes de 2004 disminuyeron con
canales endémicos más elevados que la media histórica, se concentraron en los
departamentos del norte de la provincia (cuadro 1). En relación con los
vectores, las capturas sistemáticas mostraron que en la región norte también
había una mayor diversidad de especies de Phlebotominae (cuadro 2).

\textit{Fuente:}
Elaboración propia a partir de capturas (\textsuperscript{[}8\textsuperscript{]},\textsuperscript{[}9\textsuperscript{]},\textsuperscript{[}21\textsuperscript{]},\textsuperscript{[}22\textsuperscript{]},\textsuperscript{[}24\textsuperscript{]},\textsuperscript{[}31\textsuperscript{]}
).

Las celdas en blanco indican que no hubo notificaciones.

Los departamentos se han ordenado de norte a sur.

\textit{Ec: Evandromyia cortelezzii/sallesi; Ps: Psathyromyia shannoni; Pmi:
Pintomyia misionensis; Mq: Micropygomyia quinquefer}
; \textit{Pf: Pintomyia fischeri; Ma: Martinsmyia alphabetica; Pmo: Pintomyia
montícola; Ppe: Pintomyia pessoai; Ppa: Psathyromyia pascalei}
; \textit{Ee: Evandromyia evandroi; Pl: Psathyromyia lanei}
; \textit{Ba: Brumptomyia avellari; Bg: Brumptomyia guimaraesi}
; \textit{Xx: Micropygomyia oswaldoi, Pintomyia bianchigalatiae, Pintomyia
damascenoi, Psathyromyia punctigeniculata Expapillata firmatoi, Micropygomyia
capixaba, Evandromyia correalimai, Evandromyia edwardsi, Evandromyia bourrouli,
Brumptomyia brumpti}
y \textit{Brumptomyia pintoi.}

En Misiones, los estudios realizados muestran diferentes escenarios de
transmisión y riesgo de leishmaniasis durante las décadas analizadas. Para
generar recomendaciones en la escala de análisis utilizada mediante la
perspectiva eco-epidemiológica, es necesario reinterpretar los resultados en el
contexto de los procesos socioeconómicos de construcción del territorio.

Hasta la década de 1930, la selva de Misiones era un espacio ocupado por
poblaciones dispersas y nómadas de nativos, fugitivos, contrabandistas y
trabajadores que se internaban temporalmente en ella. Desde la percepción
social, era una frontera a conquistar para la extracción forestal y la
recolección de yerba mate silvestre (\textit{Hylex}
sp.) González y Arce Queirolo, al analizar un brote de LC registrado en 1934,
durante la Guerra del Chaco, describen la enfermedad como endémica del oriente
de Paraguay (\textsuperscript{[}38\textsuperscript{]}
), donde ya era conocida desde comienzos del siglo XX
(\textsuperscript{[}39\textsuperscript{]}
). Sin embargo, aunque se sabe que la LC estaba presente
(\textsuperscript{[}19\textsuperscript{]}
), los registros —de baja calidad— no aportan evidencias suficientes para
conocer su incidencia.

Entre 1930 y 1960, la Selva Misionera se percibía como una fuente de materia
prima a muy bajo costo y se caracterizaba por su colonización por migrantes, la
deforestación por concesiones y la instalación de aserraderos de bajo nivel de
capitalización. En esta etapa, previa al brote de LC en Puerto Esperanza, el
registro de casos incidentes de esta enfermedad mantuvo variaciones (figura~\ref{fig:f1}
), aunque esto pudo deberse a cambios en la capacidad diagnóstica. Los colonos
tendían a concentrarse en centros urbanos de nueva creación, con estancias
esporádicas en el borde de deforestación, lo que provocó brotes epidémicos
aislados de fuente común, tanto en hombres ocupados en la deforestación como en
grupos familiares dedicados al cultivo de la yerba mate. De esta forma, estos
bordes de deforestación, que no se originaron necesariamente para la creación de
espacios domésticos, pudieron generar también un incremento microfocal de la
abundancia de vectores y el consecuente riesgo de transmisión
(\textsuperscript{[}40\textsuperscript{]}
).

El desarrollo y el avance de tres nuevas fronteras relacionadas con la
deforestación y los cambios asociados al uso y la ocupación de la tierra pueden
explicar la secuencia y la localización de los brotes epidémicos de LC
registrados en la provincia de Misiones a partir de 1970.

En la década de 1980, después de haber disminuido por la deforestación masiva,
la incidencia de LC reemergió y comenzó su expansión desde el centro-sur de
Brasil y desde el noroeste en Argentina. Esta “onda epidémica” se asoció con la
adaptación de vectores al ámbito doméstico en áreas de asentamientos con décadas
de intervención antrópica (\textsuperscript{[}41\textsuperscript{]},\textsuperscript{[}42\textsuperscript{]}
), como se constató durante el brote con \textit{Ny. neivai}
en Puerto Esperanza en 1998 (\textsuperscript{[}16\textsuperscript{]}
) y en ciudades fronterizas con Misiones desde el año 2000
(\textsuperscript{[}43\textsuperscript{]}
). Sin embargo, en Misiones estos vectores aún no han colonizado con éxito el
ambiente modificado o al menos la tasa de infección permanece baja, pues los
brotes siguen siendo de incidencia baja y localización limitada, situación
diferente a la encontrada en la zona hiperendémica del noroeste argentino
(\textsuperscript{[}7\textsuperscript{]}
).

El agotamiento de tierras plausibles de ser deforestadas y el comienzo de la
explotación agroforestal industrial, con tercerización del trabajo, creó nuevos
y extensos bordes no lineales, pero permanentes, de exposición laboral, puesto
de manifiesto en el brote registrado en las inmediaciones de la represa
Urugua-í. A su vez, tras la crisis financiera de 2001 en Argentina, la migración
interna (\textsuperscript{[}44\textsuperscript{]}
), junto con la expulsión del mercado de pequeños agricultores, generó la
búsqueda de tierras en el norte de Misiones para la explotación o la radicación
familiar. Cercados por territorios ya urbanizados —propiedad de las empresas
agroforestales— o áreas conservadas intangibles, los nuevos pobladores no
tuvieron otra opción que desplazarse hacia los bordes y ocupar tierras fiscales
en un marco de conflicto social. Este fue el escenario del brote registrado en
la zona denominada Dos Mil Hectáreas.

Las prácticas empleadas por las granjas familiares producen un patrón de
deforestación irregular en el espacio y dinámico en el tiempo, donde la
proximidad del borde de deforestación y las viviendas generan un área de riesgo
permanente de exposición a los vectores
(\textsuperscript{[}40\textsuperscript{]}
). Por ello, los casos de LC se concentran en Iguazú y GM Belgrano, dos
departamentos del norte de Misiones (cuadro 1), donde se encuentran las áreas de
reservas naturales. Así, la determinación social de riesgo que implica
desplazamiento a zonas de borde de bajo valor de mercado o tierras fiscales
(brote de Dos Mil Hectáreas) o condiciones laborales precarizadas (brote de
Urugua-í), en acción sinérgica con el riesgo biológico generado por la relación
del borde de deforestación y el ambiente doméstico o la actividad laboral,
aumentó el riesgo de transmisión asociado a las nuevas fronteras internas de la
deforestación.

Diversos fenómenos confluyentes pueden haber contribuido al aumento de la
circulación parasitaria. La deforestación extensiva
(\textsuperscript{[}45\textsuperscript{]}
) y la construcción de la represa de Itaipú —en la triple frontera entre
Argentina, Paraguay y Brasil— generaron un impacto demográfico, ambiental y
climático extraordinario (\textsuperscript{[}46\textsuperscript{]},\textsuperscript{[}47\textsuperscript{]}
). En el estado de Paraná, Brasil, en el período 1980-2007 se registraron 13 976
casos de LC, y el riesgo de transmisión se asoció con la proximidad de grupos de
población a los bordes de selva residual
(\textsuperscript{[}48\textsuperscript{]}
).

La superficie cubierta de bosques de la región oriental de Paraguay se redujo de
55\% en 1945 a 5\% en el año 2000, debido al uso de sus árboles para leña y el
avance de la agricultura (\textsuperscript{[}49\textsuperscript{]}
). En los últimos años, el cambio del uso de la tierra se debió a las
plantaciones de soja a gran escala (\textsuperscript{[}45\textsuperscript{]}
), lo que ha provocado la migración de agricultores minifundistas hacia centros
urbanos (\textsuperscript{[}50\textsuperscript{]}
). El departamento de Canindeyú, uno de los más afectados por este desarrollo
agroindustrial, presentó la mayor incidencia de LC, con una tasa de 90,73 por
100 000 habitantes en el período 2007–2009, mientras el departamento de Alto
Paraná presentó una tasa de 8,4 por 100 000 habitantes en ese mismo período
(\textsuperscript{[}51\textsuperscript{]}
).

De esta manera, en el territorio de Misiones, a la presión selectiva de la
deforestación sobre los vectores para su adaptación al ambiente modificado
(frontera biológica) y a la creación de bordes dinámicos de deforestación que
aumentó el contacto de los vectores con la población expuesta (frontera
interna), se agregó la relocalización y la migración masiva de la población en
los países vecinos (frontera externa). Estas transformaciones externas se
caracterizaron por cambios en el uso de la tierra, que elevaron el riesgo de
transmisión de la LC en el espacio de la unidad doméstica —domicilio y
peridomicilio— y el incremento de la circulación parasitaria en un escenario de
tránsito vecinal fronterizo intenso (\textsuperscript{[}52\textsuperscript{]}
). La persistencia de estos procesos de fronteras podría indicar que la
situación reciente de baja incidencia debe considerarse como un período
interepidémico y no una tendencia a la reducción de la transmisión.

Este escenario de riesgo se puede extrapolar a diversos sitios de América Latina
y explicar los brotes de LC ocurridos por la mayor proximidad de las personas a
los bordes de deforestación, la migración de personas susceptibles, la
colonización rural y la incorporación de nuevas áreas para el cultivo
(\textsuperscript{[}53\textsuperscript{]},\textsuperscript{[}54\textsuperscript{]}
). Estos escenarios se han registrado en Bolivia
(\textsuperscript{[}55\textsuperscript{]}
), Brasil (\textsuperscript{[}56\textsuperscript{]}
), Ecuador (\textsuperscript{[}57\textsuperscript{]}
), Guayana Francesa (\textsuperscript{[}58\textsuperscript{]}
), Perú (\textsuperscript{[}59\textsuperscript{]}
) y Surinam (\textsuperscript{[}13\textsuperscript{]}
).

El incremento de la migración, el turismo y el flujo comercial a partir de la
década de 1970 favorecieron la aparición de los eventos epidémicos de LV en el
sur de Brasil y Paraguay. A partir de 2006, la dispersión por contigüidad en el
sur de Misiones y la aparición discontinua en el norte (cuadro 1) sugieren la
presencia de mecanismos de progresión espacial por las rutas más transitadas y
por la migración generada por la demanda de servicios para el turismo en Iguazú.
Asimismo, se deben considerar los cambios en la interacción entre las personas y
los perros y, en consecuencia, los conflictos multisectoriales asociados con las
acciones programáticas, que favorecieron la dispersión de la enfermedad
(\textsuperscript{[}60\textsuperscript{]}
).

En los focos registrados, la urbanización no planificada está caracterizada por
la intercalación de áreas verdes y de uso doméstico, que aumenta las fuentes de
sangre para los parásitos, la cantidad de material orgánico y el pobre
escurrimiento hídrico que favorece la colonización y la persistencia de
\textit{Lu. longipalpis}
(\textsuperscript{[}61\textsuperscript{]}
). Estas características ambientales se asocian por lo general, tal como ocurre
con la LC, con la ocupación de tierras fiscales y de bajo valor de mercado —con
deficientes servicios sanitarios y dificultades de acceso al sistema de salud—,
lo que a su vez resulta en la determinación social del riesgo de la LV.

La LV urbana en Argentina muestra una tendencia a la reducción de la incidencia
de casos clínicos en los primeros focos y su dispersión hacia nuevos sitios; sin
embargo, esa tendencia se debe interpretar con precaución, dado el corto período
considerado y la posibilidad de reactivación de focos.

Como fenómeno regional, la LV en Brasil también se asoció con la urbanización no
planificada y las migraciones de personas y perros acompañantes infectados. En
los estados de São Paulo y Mato Grosso do Sul, desde donde habría ingresado la
enfermedad a Paraguay y luego a Argentina, esta dispersión se asoció —en tiempo
y espacio— con la construcción de autopistas federales y el gasoducto entre
Bolivia y Brasil (\textsuperscript{[}12\textsuperscript{]},\textsuperscript{[}14\textsuperscript{]}
). Como en el caso de la LC, la deforestación, unida a la migración masiva, se
relaciona con procesos socioeconómicos que se pueden regular y monitorear a
diferentes escalas.

\section{\textsc{conclusiones}}

El enfoque retrospectivo aplicado a la eco-epidemiología de la LC y la LV
permitió integrar los contextos sociales y biológicos, que explican la
modulación del riesgo de transmisión de la leishmaniasis en Misiones, Argentina,
entre 1920 y 2014. Esta metodología, aplicada en la escala de primera
jurisdicción subnacional a fin de caracterizar los escenarios de riesgo y los
procesos antrópicos que los producen, generó evidencias que permiten fundamentar
recomendaciones de vigilancia y control en dicha escala. Se resumen a
continuación las recomendaciones derivadas de este estudio de caso.

\section{\textsc{recomendaciones}}

Establecer una vigilancia activa para monitorear posibles tendencias al
incremento de la circulación parasitaria (riesgo de foco epidémico).

Establecer la vigilancia vectorial para monitorear la colonización doméstica
(riesgo de incremento de casos) debido al cambio de comportamiento vectorial o
la aparición de nuevas especies de vectores (riesgo de foco epidémico).

Sensibilizar al personal de salud para lograr la detección temprana, el correcto
diagnóstico y el tratamiento oportuno de los casos, así como establecer el
monitoreo vectorial para evaluar y prever la aparición de casos (riesgo de brote
de fuente común).

Establecer la obligación legal de los responsables de los proyectos que
impliquen la modificación ambiental o la generación de bordes de deforestación
—como la tala de árboles, el anegamiento de terrenos y el crecimiento periférico
urbano, entre otros— de tomar medidas adicionales de control. Entre esas medidas
están la realización de estudios de evaluación del riesgo de transmisión de LC
—tanto antes como después de la intervención—, así como acciones de mitigación
del riesgo, detección temprana y tratamiento oportuno de los casos. Estas
medidas y acciones adicionales pueden ejecutarse directamente en el marco de los
proyectos o mediante la financiación de las investigaciones correspondientes,
según sea más factible.

Establecer legalmente la responsabilidad de las empresas en el establecimiento
del monitoreo del riesgo, la cobertura médica apropiada y la protección de los
trabajadores (riesgo laboral), así como en la mitigación del impacto ambiental.

Emprender el estudio del foco para verificar la transmisión autóctona y la
intensidad del evento (estratificación del riesgo) y aplicar las medidas y
acciones de intervención programática que correspondan.

Establecer la vigilancia vectorial y hacer las estimaciones pertinentes de la
prevalencia en perros de las localidades contiguas a las que presentan
transmisión autóctona; para ello, se debe tener en cuenta la conectividad y la
intensidad del tránsito de personas y el intercambio de bienes y servicios
(estratificación del riesgo), y aplicar las medidas y las acciones de
intervención programática que correspondan.

Establecer la obligación legal de los responsables de los proyectos o los
desarrollos que impliquen la migración masiva de personas en áreas con riesgo de
transmisión o con transmisión endémica de LV, de tomar medidas y acciones
específicas para evitar o reducir la transmisión de esta enfermedad. Entre esas
medidas y acciones se pueden mencionar la realización de evaluaciones
epidemiológicas y el mantenimiento del monitoreo del riesgo de transmisión de LV
mediante estudios socio-ambientales, vectoriales y de reservorios, tanto antes
como después de la intervención y según los escenarios de tránsito fronterizo.
Además, deben actuar para mitigar el riesgo y lograr la detección temprana de
los casos y su tratamiento oportuno.

Generar marcos legales y órganos de aplicación para la regulación sanitaria del
tránsito de perros, tanto entre países como entre zonas con y sin transmisión.

Promover el ordenamiento territorial en áreas de recepción de migrantes y el
manejo ambiental responsable de las áreas públicas y privadas con población
residente, tanto a nivel provincial como comunal, barrial e individual.

Establecer regulaciones consensuadas entre los diferentes actores y sectores
involucrados sobre la tenencia y la reproducción responsables de perros y gatos,
requisitos sanitarios para el tránsito de animales de estimación y el manejo de
poblaciones de perros callejeros.

Promover activamente —tanto a nivel nacional y provincial como comunal, barrial
e individual— una gestión intersectorial que incluya la leishmaniasis entre las
enfermedades de transmisión vectorial asociadas al ambiente y los animales
domésticos. Esta gestión, dirigida a apoyar la reducción sostenible de la
transmisión de la leishmaniasis, deberá considerar la determinación social de
riesgo, por lo que deberá contemplar entre sus objetivos una mayor equidad en el
acceso a los servicios de salud de calidad, y a viviendas y espacios laborales
saludables.

Agradecimientos
A Julio Estévez, director de Epidemiología, y Jorge Gutiérrez, Jefe del
Departamento de Vigilancia Epidemiológica, ambos del Ministerio de Salud Pública
de la Provincia de Misiones, Argentina, por la información sobre los casos de
leishmaniasis por departamentos. Al Centro Internacional de Investigaciones para
el Desarrollo (\textsc{idrc}) y la Organización Panamericana de la Salud por contribuir
con un proyecto de investigación en la zona de la frontera entre Argentina,
Brasil y Paraguay, que permitirá integrar los resultados del presente análisis.
\section{Conflicto de intereses}

Ninguno.

\section{Declaración}

Las opiniones expresadas en este manuscrito son responsabilidad del autor y no
reflejan necesariamente los criterios ni la política de la \textsc{rpsp}/\textsc{pajph} y/o de la
\textsc{ops}.

\section*{\textsc{referencias}}
\begin{itemize}

\item[1] Maudlin I, Eisler MC, Welburn SC. Neglected and endemic zoonoses.
Philos Trans R Soc Lond B Biol Sci. 2009;364(1530):2777–87.

\item[2] Alvar J, Vélez ID, Bern C, Herrero M, Desjeux P, Cano J, et al.
Leishmaniasis worldwide and global estimates of its incidence. PLoS One.
2012;7(5):e35671.

\item[3] Organización Panamericana de la Salud. Leishmaniasis, informe
epidemiológico de las Américas. Informe Leishmaniasis N.° 1. Washington, D.C.:
\textsc{ops}; 2013.

\item[4] Costa MA, Llagostera A. Leishmaniasis en Coyo Oriente: migrantes
trasandinos en San Pedro de Atacama. Estud Atacameños (Atacama). 2014;47:5–18.

\item[5] Bernasconi VE. Consideraciones sobre el censo de leishmaniosis. Rev
Soc Patol Reg Norte. 1930;5:590–602.

\item[6] Bejarano \textsc{jrf}, Duret JP. Contribución al conocimiento de los
flebótomos argentinos (Diptera: Psychodidae). Rev Sanid Mil Arg. 1950;49:327–36.

\item[7] Quintana MG, Fernández MS, Salomón OD. Distribution and abundance of
Phlebotominae, vectors of leishmaniasis, in Argentina: spatial and temporal
analysis at different scales. J Trop Med. 2012;2012. doi: 10.1155/2012/652803.

\item[8] Salomón OD, Rossi G, Sosa Estani S, Spinelli G. Presencia de
Lutzomyia longipalpis y situación de la leishmaniosis visceral en Argentina.
Medicina (Buenos Aires). 2001;61:174–8.

\item[9] Gould TI, Perner MS, Santini MS, Saavedra SB, Bezzi G, Maglianese MI,
et al. Leishmaniasis visceral en la Argentina: notificación y situación
vectorial (2006-2012). Medicina (Buenos Aires). 2013;73: 104–10.

\item[10] Argentina, Ministerio de Salud de la Nación. Boletín Integrado de
Vigilancia. \textsc{snvs}. Buenos Aires: Ministerio de Salud; 2013-2014. Hallado en:
\href{http://www.msal.gov.ar/index.php/home/boletin-integrado-de-vigilancia}.
Acceso el 15 de abril de 2016.

\item[11] Susser M, Susser E. Choosing a future for epidemiology: II. From
black box to Chinese boxes and eco-epidemiology. Am J Public Health.
1996;86:674–7.

\item[12] Cardim MF, Rodas LA, Dibo MR, Guirado MM, Oliveira AM,
Chiaravalloti-Neto F. Introduction and expansion of human American visceral
leishmaniasis in the state of Sao Paulo, Brazil, 1999-2011. Rev Saude Publica.
2013;47:691–700.

\item[13] Kent AD, Dos Santos TV, Gangadin A, Samjhawan A, Mans DR, Schallig
HD. Studies on the sand fly fauna (Diptera: Psychodidae) in high-transmission
areas of cutaneous leishmaniasis in the Republic of Suriname. Parasit Vectors.
2013;6:318.

\item[14] Correa Antonialli SA, Torres TG, Paranhos Filho AC, Tolezano JE.
Spatial analysis of American visceral leishmaniasis in Mato Grosso do Sul state,
Central Brazil. J Infect. 2007;54:509–14.

\item[15] Argentina, Instituto Provincial de Estadística y Censos. Gran atlas
de Misiones. Posadas: \textsc{ipec}; 2012.

\item[16] Salomón OD, Orellano PW, Quintana MG, Pérez S, Sosa Estani S,
Acardi S, et al. Transmisión de la leishmaniasis tegumentaria en Argentina.
Medicina (Buenos Aires). 2006;66:211–9.

\item[17] Salomón OD, Sinagra A, Nevot MC, Barberian G, Paulin P, Estevez JO,
et al. First visceral leishmaniasis focus in Argentina. Mem Inst Oswaldo Cruz.
2008;103:109–11.

\item[18] Menéndez E. Modelos de atención de los padecimientos: de
exclusiones teóricas y articulaciones prácticas. Cienc Saude Colet.
2003;8(1):185–207.

\item[19] Bertoni MS. La civilización guaraní: descripción física, económica
y social del Paraguay. Parte 3: Etnografía. Alto Paraná: Editorial Ex Sylvis;
1927.

\item[20] Poder Ejecutivo Nacional. Ley N° 15.465 sobre notificación
obligatoria en todo el país de los casos de enfermedades transmisibles, su
reglamentación. Boletín Oficial de la República Argentina, mayo 23 de 1964.
Disponible en: \href{https://www.boletinoficial.gob.ar/}. Acceso el 30 de mayo
de 2016.

\item[21] Cedillos RA, Walton BC. Leishmaniosis: special situations in other
areas of the Americas. En: International Development Research Centre. Research
on control strategies for the leishmaniosis. Ottawa: \textsc{idrc}; 1988. Pp. 156–61.
(Manuscript Report 184e).

\item[22] Castro M. Diptera: Psychodidae–Flebotominae. En: Bejarano \textsc{jrf}, del
Ponte E, Orfila RN, eds. Primeras Jornadas Entoepidemiológicas Argentinas. Prens
Med Argent. 1959. Pp. 545–6.

\item[23] Salomón OD, Rossi GC, Spinelli GR. Ecological aspects of
Phelobotomine (Diptera: Psychodidae) in an endemic area of tegumentary
leishmaniasis in the Northeastern Argentina, 1993-1998. Mem Inst Oswaldo Cruz.
2002;97:163–8.

\item[24] Mastrángelo AV, Salomón OD. Trabajo forestal y leishmaniasis
cutánea: un análisis social centrado en el riesgo para el N de Misiones
(Argentina). Talleres \textsc{ula}-Inst Experimental JW Torrealba. 2009;12:60–8.

\item[25] Salomón OD, Acardi SA, Liotta DJ, Fernández MS, Lestani E, López D,
et al. Epidemiological aspects of cutaneous leishmaniasis in the Iguazú falls
area of Argentina. Acta Trop. 2009;109:5–11.

\item[26] Mastrángelo AV, Salomón OD. Contribución de la antropología a la
comprensión ecoepidemiológica de la leishmaniasis tegumentaria americana en las
“2000 hectáreas”, Puerto Iguazú, Misiones, Argentina. Rev Arg Salud Publica.
2010;1:6–13.

\item[27] Acardi SA, Liotta DJ, Santini MS, Romagosa CM, Salomón OD.
Detection of Leishmania infantum in naturally infected Lutzomyia longipalpis
(Diptera: Psychodidae: Phlebotominae) and Canis familiaris in Misiones,
Argentina: The first report of \textsc{pcr}-\textsc{rflp} and sequencing-based confirmation assay.
Mem Inst Oswaldo Cruz. 2010;105:796–9.

\item[28] Cruz I, Acosta L, Gutiérrez MN, Nieto J, Cañavate C, Deschutter J,
et al. A canine leishmaniasis pilot survey in an emerging focus of visceral
leishmaniasis: Posadas (Misiones, Argentina). \textsc{bmc} Infect Dis. 2010;10:342.

\item[29] Barroso PA, Nevot MC, Hoyos CL, Locatelli FM, Lauthier JJ, Ruybal
P, et al. Genetic and clinical characterization of canine leishmaniasis caused
by Leishmania (Leishmania infantum) in northeastern Argentina. Acta Trop.
2015;150:218–23.

\item[30] Salomón OD, Araki AS, Hamilton \textsc{jgc}, Acardi AS, Peixoto AA. Sex
pheromone and period gene characterization of Lutzomyia longipalpis (Lutz \&
Neiva) (Diptera: Psychodidae) from Posadas, Argentina. Mem Inst Oswaldo Cruz.
2010;105:928–30.

\item[31] Fernández MS, Salomón OD, Cavia R, Pérez AA, Guccione JD. Lutzomyia
longipalpis spatial distribution and association with environmental variables in
an urban focus of visceral leishmaniasis, Misiones, Argentina. Acta Trop.
2010;114:81–7.

\item[32] Santini MS, Fernández MS, Pérez AA, Sandoval EA, Salomón OD.
Lutzomyia longipalpis abundance in the city of Posadas, northeastern Argentina:
variation at different spatial scales. Mem Inst Oswaldo Cruz. 2012;107:767–71.

\item[33] Santini MS, Salomón OD, Acardi SA, Sandoval EA, Tartaglino LC.
Lutzomyia longipalpis behavior at an urban visceral leishmaniasis focus in
Argentina. Rev Inst Med Trop Sao Paulo. 2010;52:187–92.

\item[34] Santini MS, Gould IT, Manteca Acosta M, Acardi SA, Fernández MS,
Gómez A, et al. Phlebotominae of sanitary interest in the
Argentina-Brazil-Paraguay border area. Rev Inst Med Trop Sao Paulo.
2013;55:239–43.

\item[35] Manteca M, Molina J, Utgés ME, Mastrangelo AV, Pérez AA, Santini
MS, et al. Efficacy of impregnated bednets and species composition in
experimental henhouses. Bol Soc Entomol Argentina. 2015;26(1S):28.

\item[36] Acosta L, Díaz R, Torres P, Silva G, Ramos M, Fattore G, et al.
Identification of Leishmania infantum in Puerto Iguazú, Misiones, Argentina. Rev
Inst Med Trop Sao Paulo. 2015;57:175–6.

\item[37] Moya SL, Giuliani MG, Acosta MM, Salomón OD, Liotta DJ. First
description of Migonemyia migonei (França) and Nyssomyia whitmani (Antunes \&
Coutinho) (Psychodidae: Phlebotominae) naturally infected by Leishmania infantum
in Argentina. Acta Trop. 2015;152:181–84.

\item[38] González G, Arce Queirolo A. Leishmaniosis II. Leishmaniosis
cutáneomucosa y guerra en el bosque. Rev Med Paraguay. 1955;1:69–74.

\item[39] Migone LE. La buba du Paraguay, leishmaniose americaine. Bull Soc
Pathol Exot. 1913;6:210–8.

\item[40] Quintana MG, Salomón OD, Lizarralde de Grosso MS. Distribution of
Phlebotominae in primary forest-crop interface, Salta, Argentina. J Med Entomol.
2010;47:1003–10.

\item[41] Shaw J. The leishmaniasis-survival and expansion in a changing
world. Mem Inst Oswaldo Cruz. 2007;102:541–7.

\item[42] Rangel EF, Lainson R. Ecologia das leishmanioses. En: Rangel EF,
Lainson R, eds. Flebotomíneos do Brasil. Rio de Janeiro: Editora Fiocruz; 2003.
Pp. 291–310.

\item[43] Cruz CF, Cruz MF, Galati EA. Sandflies (Diptera: Psychodidae) in
rural and urban environments in an endemic area of cutaneous leishmaniasis in
southern Brazil. Mem Inst Oswaldo Cruz. 2013;108:303–11.

\item[44] Zeballos JL. Argentina: efectos sociosanitarios de la crisis
2001-2003. Buenos Aires: Organización Panamericana de la Salud; 2003.

\item[45] Di Bitetti MS, Placci G, Dietz LA. Una visión de biodiversidad para
la ecorregión del bosque atlántico del Alto Paraná: Diseño de un paisaje para la
conservación de la biodiversidad y prioridades para las acciones de
conservación. Washington, D.C.: World Wildlife Fund; 2003.

\item[46] Limberger L, Cecchin J. Percepção climática de moradores lindeiros
ao reservatorio da usina hidrelétrica de Itaipu. Acta Geogr (Boa Vista).
2012:11–29.

\item[47] Conte CH. Do milagre econômico á construção de Itaipu: configurando
a cidade de Foz do Iguaçu/PR. Econ Desenvolvimento. 2013;12:166–92.

\item[48] Monteiro WM, Neitzke-Abreu HC, Ferreira ME, Melo GC, Barbosa MD,
Lonardoni MV, et al. Mobilidade populacional e produção da leishmaniose
tegumentar americana no estado do Paraná, sul do Brasil. Rev Soc Bras Med Trop.
2009;42:509–14.

\item[49] Organización Panamericana de la Salud. Salud en las Américas.
Panorama regional y perfiles de país. Washington, D.C.: \textsc{ops}; 2012. (Publicación
Científica y Técnica N.° 636).

\item[50] Albuquerque \textsc{jlc}. Campesinos paraguayos y “brasiguayos” en la
frontera este del Paraguay. En: Fogel R, Riquelme M, eds. Enclave sojero: merma
de soberanía y pobreza. Asunción: Centro de Estudios Rurales Interdisciplinares;
2005. Pp. 157–90.

\item[51] Canese A, Oddone R, Maciel JD. Manual de diagnóstico y tratamiento
de las leishmaniosis. Ministerio de Salud Pública y Bienestar Social, Programa
Nacional de Control de Leishmaniosis. Asunción: Organización Panamericana de la
Salud/Organización Mundial de la Salud; 2011.

\item[52] Fogel R. La región de la triple frontera: territorios de
integración y desintegración. Sociologias (Porto Alegre). 2008;10: 270–90.

\item[53] Weil C. Health problems associated with agricultural colonization
in Latin America. Soc Sci Med D. 1981;15:449–61.

\item[54] Davies CR, Reithinger R, Campbell-Lendrum D, Feliciangeli D, Borges
R, Rodriguez N. The epidemiology and control of leishmaniasis in Andean
countries. Cad Saude Publica. 2000;16: 925–50.

\item[55] Alcais A, Abel L, David C, Torrez ME, Flandre P, Dedet JP. Risk
factors for onset of cutaneous and mucocutaneous leishmaniasis in Bolivia. Am J
Trop Med Hyg. 1997;57:79–84.

\item[56] Dourado MI, Noronha CV, Alcantara N, Ichihara MY, Loureiro S.
Epidemiologia da leishmaniose tegumentar americana e suas relações com a lavoura
e o garimpo, em localidade do estado da Bahia (Brasil). Rev Saude Publica.
1989;23:2–8.

\item[57] Barrera C, Herrera M, Martinez F, León R, Richard A, Guderian RH,
et al. Leishmaniose en Equateur: 1. Incidence de la leishmaniose tégumentaire
sur la façade pacifique. Ann Soc Belg Med Trop. 1994;74:1–12.

\item[58] Rotureau B, Joubert M, Clyti E, Djossou F, Carme B. Leishmaniasis
among gold miners, French Guiana. Emerg Infect Dis. 2006;12:1169–70.

\item[59] Guthmann JP, Calmet J, Rosales E, Cruz M, Chang J, Dedet JP.
Patients’ associations and the control of leishmaniasis in Peru. Bull World
Health Organ. 1997;75:39–44.

\item[60] Salomón OD, Mastrángelo AV, Santini MS, Ruvinsky S, Orduna T,
Sinagra A, et al. Leishmaniasis visceral: senderos que confluyen, se bifurcan.
Salud Colectiva (Buenos Aires). 2012;8(S1):49–63.

\item[61] Fernández MS, Santini MS, Cavia R, Sandoval AE, Pérez AA, Acardi S,
et al. Spatial and temporal changes in Lutzomyia longipalpis abundance, a
Leishmania infantum vector in an urban area in northeastern Argentina. Mem Inst
Oswaldo Cruz. 2013;108:817–24.

\end{itemize}

\end{document}
