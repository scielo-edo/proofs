% Generated by jats2tex@0.11.1.0
\documentclass{article}
\usepackage[T1]{fontenc}
\usepackage[utf8]{inputenc} %% *
\usepackage[portuges,spanish,english,german,italian,russian]{babel} %% *
\usepackage{amstext}
\usepackage{authblk}
\usepackage{unicode-math}
\usepackage{multirow}
\usepackage{graphicx}
\usepackage{etoolbox}
\usepackage{xtab}
\usepackage{enumerate}
\usepackage{hyperref}
\usepackage{penalidades}
\usepackage[footnotesize,bf,hang]{caption}
\usepackage[nodayofweek,level]{datetime}
\usepackage[top=0.85in,left=2.75in,footskip=0.75in]{geometry}
\newlength\savedwidth
\newcommand\thickcline[1]{\noalign{\global
\savedwidth
\arrayrulewidth
\global\arrayrulewidth 2pt}
\cline{#1}
\noalign{\vskip\arrayrulewidth}
\noalign{\global\arrayrulewidth\savedwidth}}
\newcommand\thickhline{\noalign{\global
\savedwidth\arrayrulewidth
\global\arrayrulewidth 2pt}
\hline
\noalign{\global\arrayrulewidth\savedwidth}}
\usepackage{lastpage,fancyhdr}
\usepackage{epstopdf}
\pagestyle{myheadings}
\pagestyle{fancy}
\fancyhf{}
\setlength{\headheight}{27.023pt}
\lhead{\includegraphics[width=10mm]{logo.png}}
\rhead{\ifdef{\journaltitle}{\journaltitle}{}
\ifdef{\volume}{vol.\,\volume}{}
\ifdef{\issue}{(\issue)}{}
\ifdef{\fpage}{\fpage--\lpage\,pp.}}
\rfoot{\thepage/\pageref{LastPage}}
\renewcommand{\footrule}{\hrule height 2pt \vspace{2mm}}
\fancyheadoffset[L]{2.25in}
\fancyfootoffset[L]{2.25in}
\lfoot{\sf \ifdef{\articledoi}{\articledoi}{}}
\setmainfont{Linux Libertine O}
\renewcommand*{\thefootnote}{\alph{footnote}}
\makeatletter
\newcommand{\fn}{\afterassignment\fn@aux\count0=}
\newcommand{\fn@aux}{\csname fn\the\count0\endcsname}
\makeatother

\newcommand{\publisherid}{rpsp}
\newcommand{\journaltitle}{Revista Panamericana de Salud Pública}
\newcommand{\abbrevjournaltitle}{Rev Panam Salud Publica}
\newcommand{\issnepub}{1680-5348}
\newcommand{\issnppub}{1020-4989}
\newcommand{\publishername}{Organización Panamericana de la Salud}
\newcommand\articleid{40-070}
\def\subject{Comunicación breve}\newcommand{\subtitlestyle}[1]{--
\emph{#1}\medskip}
\newcommand{\transtitlestyle}[1]{\par\medskip\Large #1}
\newcommand{\transsubtitlestyle}[1]{-- \Large\emph{ #1}}

\newcommand{\titlegroup}{
\ifdef{\subtitle}{\subtitlestyle{\subtitle}}{}
\ifdef{\transtitle}{\transtitlestyle{\transtitle}}{}
\ifdef{\transsubtitle}{\transsubtitlestyle{\transsubtitle}}{}}

\title{Emergencia de β-lactamasa AmpC plasmídica del grupo \textsc{cmy}-2 en
\textit{Shigella sonnei}
y \textit{Salmonella}
spp. en Costa Rica, 2003-2015\titlegroup{}}
\newcommand{\transtitle}{Emergence of \textsc{cmy}-2-type plasmid-mediated AmpC
\textit{β}
-lactamase in Shigella sonnei and Salmonella spp. in Costa Rica, 2003–2015}
\author[{1}]{Ayala, Anamariela Tijerino}
\author[{1}]{Acuña, Hilda María Bolaños}
\author[{2}]{Calvo, María Teresa Acuña}
\author[{1}]{Morales, José Luis Vargas}
\author[{1}]{Chacón, Elena Campos}
\affil[1]{Centro Nacional de Referencia de Bacteriología}
\affil[2]{Centro Nacional de Referencia en Inocuidad Microbiológica de los
Alimentos}
\def\authornotes{La correspondencia se debe dirigir a Anamariela Tijerino Ayala.
Correo electrónico: atijerino@inciensa.sa.cr}
\date{ 07 2016}
\def\volume{40}
\def\issue{1}
\def\fpage{70}
\def\lpage{75}
\def\permissions{Este es un artículo publicado en acceso abierto bajo una
licencia Creative Commons}
\newcommand{\kwdgroupes}{Palabras claveSalmonella, Shigella, farmacorresistencia
microbiana, Costa Rica}
\newcommand{\kwdgroupen}{Key wordsSalmonella, Shigella, drug resistance,
microbial, Costa Rica}

\begin{document}
\selectlanguage{spanish}
\section*{Metadados não aplicados}
\begin{itemize}
\item[\textbf{língua do artigo}]{Espanhol}
\ifdef{\journalid}{\item[\textbf{journalid}] \journalid}{}
\ifdef{\journaltitle}{\item[\textbf{journaltitle}] \journaltitle}{}
\ifdef{\abbrevjournaltitle}{\item[\textbf{abbrevjournaltitle}]
\abbrevjournaltitle}{}
\ifdef{\issnppub}{\item[\textbf{issnppub}] \issnppub}{}
\ifdef{\issnepub}{\item[\textbf{issnepub}] \issnepub}{}
\ifdef{\publishername}{\item[\textbf{publishername}] \publishername}{}
\ifdef{\publisherid}{\item[\textbf{publisherid}] \publisherid}{}
\ifdef{\subject}{\item[\textbf{subject}] \subject}{}
\ifdef{\transtitle}{\item[\textbf{transtitle}] \transtitle}{}
\ifdef{\authornotes}{\item[\textbf{authornotes}] \authornotes}{}
\ifdef{\articleid}{\item[\textbf{articleid}] \articleid}{}
\ifdef{\articledoi}{\item[\textbf{articledoi}] \articledoi}{}
\ifdef{\volume}{\item[\textbf{volume}] \volume}{}
\ifdef{\issue}{\item[\textbf{issue}] \issue}{}
\ifdef{\fpage}{\item[\textbf{fpage}] \fpage}{}
\ifdef{\lpage}{\item[\textbf{lpage}] \lpage}{}
\ifdef{\permissions}{\item[\textbf{permissions}] \permissions}{}
\end{itemize}
\maketitle

\begingroup
\renewcommand{\abstractname}{\textsc{resumen}}
\begin{abstract}

Las AmpC plasmídicas son enzimas del grupo de las β-lactamasas, codificadas por
genes bla\textit{AmpC}. Entre ellas, las del tipo \textsc{cmy}-2 son las que se reportan con mayor frecuencia a
nivel mundial. La detección de enterobacterias productoras de AmpC plasmídicas
\textsc{cmy}-2 es de importancia clínica, ya que pueden conducir a fracasos terapéuticos
al emplear antibióticos β-lactámicos. Además, tienen importancia para la salud
pública por su capacidad de transferirse por conjugación a otras
enterobacterias, tanto en la comunidad como en ambiente nosocomial, por lo que
se considera que tienen un claro potencial epidémico.

Con el fin de conocer la circulación de este mecanismo de resistencia entre
aislamientos de \textit{Salmonella y Shigella}
en Costa Rica, se realizó un análisis retrospectivo de la información contenida
en las bases de datos de vigilancia de laboratorio del Centro Nacional de
Referencia de Bacteriología (\textsc{cnrb}) del Instituto Costarricense de Investigación
y Enseñanza en Nutrición y Salud (Inciensa), entre enero de 2003 y mayo de 2015.

En dicho período se analizaron 4 363 aislamientos de \textit{Shigella}
y 1 785 aislamientos de \textit{Salmonella}. Entre ellos se detectaron 15 aislamientos de \textit{Shigella sonnei}
y nueve de \textit{Salmonella}
(cuatro de origen clínico humano y cinco de origen aviar) con fenotipo
sospechoso de portar AmpC plasmídica, todos los cuales se confirmaron
pertenecientes al tipo \textsc{cmy}-2 mediante reacción en cadena de la polimerasa.

Considerando estos resultados, se recomienda a los laboratorios de microbiología
de la Red Nacional mantener la vigilancia y realizar la confirmación
correspondiente de cualquier aislamiento sospechoso por métodos fenotípicos y
moleculares. Lo anterior es especialmente importante en bacterias aisladas de
infecciones extraintestinales, para evitar fallas en el tratamiento.

\iflanguage{portuges}{\medskip\noindent\textbf{Palavras-chave:} \kwdgroup}{}
\iflanguage{english}{\medskip\noindent\textbf{Keywords:} \kwdgroupen}{}
\iflanguage{spanish}{\medskip\noindent\textbf{Palavras claves:} \kwdgroupes}{}
\iflanguage{french}{\medskip\noindent\textbf{Mots clés:} \kwdgroupfr}{}
\end{abstract}
\endgroup

\begingroup
\renewcommand{\section}[1]{\subsection*{#1}}
\begin{otherlanguage}{english}
\renewcommand{\abstractname}{\textsc{abstract}}
\begin{abstract}

Plasmid-mediated AmpC are enzymes belonging to the group of \textit{β}
-lactamases and encoded by \textit{bla}
AmpC genes. Of these enzymes, those known as type \textsc{cmy}-2 are the most frequently
reported worldwide. Detection of enterobacteria that produce \textsc{cmy}-2-type
plasmid-mediated AmpC is clinically important since the use of \textit{β}
-lactam antibiotics can result in treatment failure. It is also important from a
public health standpoint owing to the capacity for conjugative plasmid transfer
to other enterobacteria, both within the community and in nosocomial
environments. Thus, bacteria of this kind are considered to have clear epidemic
potential.

To investigate the circulation of this resistance mechanism among
\textit{Salmonella}
and \textit{Shigella}
isolates in Costa Rica, from January 2003 to May 2015 we carried out a
retrospective review of the data contained in the laboratory surveillance
databases of the National Reference Bacteriology Center (\textsc{cnrb}) of the Costa
Rican Nutrition and Health Research Institute (Inciensa).

Over this period, 4363 \textit{Shigella}
isolates and 1785 \textit{Salmonella}
isolates were examined. Among them, 15 \textit{Shigella sonnei}
isolates and nine \textit{Salmonella}
isolates (four from human clinical specimens and five of avian origin) displayed
a phenotype suspected of carrying plasmid-mediated AmpC. Polymerase chain
reaction confirmed that all these isolates belong to type \textsc{cmy}-2.

In light of these results, we recommend that the microbiology laboratories in
the national network continue to conduct surveillance and confirm any suspicious
isolates using phenotypic and molecular methods. This is particularly relevant
when dealing with bacterial isolates from extraintestinal infections so as to
prevent treatment failure.

\ifdef{\kwdgroupen}{\medskip\noindent\textbf{Keywords:} \kwdgroupen}{}
\end{abstract}
\end{otherlanguage}
\endgroup

Las AmpC plasmídicas son enzimas del grupo de las β-lactamasas, asociadas a
integrones o transposones localizados en plásmidos conjugativos. Tienen su
origen en las β-lactamasas Amp-C cromosómicas propias de \textit{Enterobacter
cloacae}, \textit{Citrobacter freundii}, \textit{Morganella morganii}
y \textit{Hafnia alvei}
(\textsuperscript{[}1\textsuperscript{]}
). Estas enzimas se han descrito en enterobacterias de importancia clínica como
\textit{Salmonella}
spp., \textit{Shigella}
sp., \textit{Klebsiella}
sp., \textit{Proteus mirabilis}
y \textit{Esherichia coli}, entre otras. Dentro de las AmpC plasmídicas, las del tipo \textsc{cmy}-2 son las que se
reportan con mayor frecuencia a nivel mundial. Los primeros aislamientos
clínicos portadores de este mecanismo de resistencia se documentaron en Estados
Unidos en los años 90 y posteriormente en diferentes países europeos y en
Argentina (\textsuperscript{[}2\textsuperscript{]}
-\textsuperscript{[}4\textsuperscript{]}
). En Japón y China también se aislaron enterobacterias, como
\textit{Salmonella}, portadoras de este mecanismo de resistencia en muestras de vacas y cerdos
(\textsuperscript{[}5\textsuperscript{]}
).

La detección de enterobacterias productoras de AmpC plasmídicas \textsc{cmy}-2 reviste de
importancia clínica, ya que pueden conducir a fracasos terapéuticos al emplear
antibióticos β-lactámicos, incluyendo carboxipenicilinas, acilureido
penicilinas, cefalosporinas de tercera generación (C3G) como cefotaxima,
ceftazidima y ceftriaxona y las cefalosporinas de cuarta generación como
cefepima (\textsuperscript{[}6\textsuperscript{]}
). Este mecanismo de resistencia es de relevancia para la salud pública, ya que
el gen \textit{bla}
\textsc{cmy}-2 que codifica para la AmpC plasmídica \textsc{cmy}-2, también tiene la capacidad de
transferirse a otras enterobacterias, tanto en la comunidad como en el ambiente
nosocomial, por lo que se considera que tienen un claro potencial epidémico
(\textsuperscript{[}7\textsuperscript{]}
).

Este hallazgo, además, es de gran importancia en el caso de las infecciones
extraintestinales por \textit{Salmonella}
y \textit{Shigella}, donde las C3G son antibióticos considerados de primera elección, cuando la
bacteria presenta resistencia o sensibilidad disminuida a las fluoroquinolonas
(\textsuperscript{[}2\textsuperscript{]},\textsuperscript{[}8\textsuperscript{]}
-\textsuperscript{[}10\textsuperscript{]}
).

Por lo tanto, con el fin de conocer la circulación de AmpC plasmídico entre
aislamientos de Salmonella y Shigella en Costa Rica, se realizó un análisis
retrospectivo de la información contenida en las bases de datos de vigilancia de
laboratorio del \textsc{cnrb}-Inciensa, entre enero 2003 y mayo 2015. Finalmente, la
confirmación oportuna de cepas sospechosas de portar este mecanismo de
resistencia es fundamental para alertar al personal médico y a las autoridades
de salud, no solo para la elección del tratamiento del paciente, sino también
para establecer medidas de control tendientes a evitar la diseminación
horizontal de este tipo de resistencia, ya que cepas con estas características
han provocado brotes en diferentes países
(\textsuperscript{[}6\textsuperscript{]}
).
\section{\textsc{materiales} Y \textsc{métodos}}

Con el fin de conocer la circulación en Costa Rica de β-lactamasas AmpC
plasmídicas entre aislamientos de \textit{Salmonella}
y \textit{Shigella}
referidos para la vigilancia basada en laboratorio entre enero del 2003 y mayo
del 2015, se realizó un análisis retrospectivo de la información contenida en
las bases de datos Excel del Centro Nacional de Referencia de Bacteriología
(\textsc{cnrb}), en busca de fenotipos sugestivos: resistentes a ampicilina (\textsc{amp}),
ceftazidima (\textsc{caz}), cefotaxima (\textsc{ctx}) y cefoxitin (\textsc{fox}), de acuerdo a las
recomendaciones de la Red Latinoamericana de Vigilancia de la Resistencia a los
Antibióticos (Re\textsc{lavra}-\textsc{ops}) y Jacoby en el año 2009
(\textsuperscript{[}6\textsuperscript{]},\textsuperscript{[}11\textsuperscript{]}
).

Las bacterias incluidas en esta base de datos fueron aisladas por laboratorios
clínicos y de la industria de los alimentos, que forman parte de la Red Nacional
de Laboratorios de Bacteriología (\textsc{rnlb}) de Costa Rica y referidas al \textsc{cnrb} del
Instituto Costarricense de Investigación y Enseñanza en Nutrición y Salud
(Inciensa) para su confirmación/tipificación y vigilancia de la resistencia a
los antibióticos. Todos los aislamientos de Shigella y Salmonella se confirmaron
y serotipificaron por métodos bioquímicos y serológicos convencionales
(\textsuperscript{[}12\textsuperscript{]}
-\textsuperscript{[}13\textsuperscript{]}
).

La prueba de sensibilidad a los antibióticos (\textsc{psa}) de todos los aislamientos de
\textit{Shigella}
y \textit{Salmonella}
incluidos se realizó por el método Kirby Bauer
(\textsuperscript{[}14\textsuperscript{]}
), empleando los puntos de corte y cepas control establecidas por la Clinical
Laboratory Standards Institute (\textsc{clsi}) del año correspondiente
(\textsuperscript{[}15\textsuperscript{]}
), para antibióticos de uso clínico y otros empleados para la vigilancia,
incluyendo: ampicilina\textsuperscript{[}10\textsuperscript{]}
μg (\textsc{amp}), cefotaxima 30 μg (\textsc{ctx}), cefotaxima-ácido clavulánico 30/10 μg
(\textsc{ctx}-\textsc{cla}), ceftazidima 30 μg (\textsc{caz}), ceftazidima-ácido clavulánico 30/10 μg
(\textsc{caz}-\textsc{cla}), amoxicilina ácido clavulánico 30 μg (\textsc{amc}), cefoxitin 30 μg (\textsc{fox}),
imipenem 10 μg (\textsc{imp}), meropenem 10 μg (\textsc{mem}), ertapenem 10 μg (\textsc{etp}),
trimetoprima-sulfametoxazol 25 μg (\textsc{sxt}), ciprofloxacina 5 μg (\textsc{cip}) y ácido
nalidíxico 30 μg (\textsc{nal}).

En todos los aislamientos con perfil de resistencia a los antibióticos
sugestivos de AmpC plasmídicos, en el \textsc{cnrb} se descartó la presencia de
β-lactamasa de espectro extendido (\textsc{blee}), mediante la técnica de Kirby Bauer
(\textsuperscript{[}14\textsuperscript{]}
). Para esto se incubó la bacteria en presencia de discos de \textsc{caz} y \textsc{ctx} con y sin
\textsc{amc} (inhibidor de \textsc{blee}) (\textsuperscript{[}15\textsuperscript{]}
).

La confirmación fenotípica de \textsc{cmy}-2 se realizó mediante la observación de
sinergia entre \textsc{fox} y las C3G con el ácido 3-amino-fenil-borónico (\textsc{apb}) (300 μg,
Laboratorios Britania S.A., Argentina), que actúa como inhibidor de las
β-lactamasas tipo AmpC, independientemente si es cromosómica o plasmídica
(Re\textsc{lavra}-\textsc{ops} y Jacoby 2009) (\textsuperscript{[}2\textsuperscript{]},\textsuperscript{[}6\textsuperscript{]},\textsuperscript{[}11\textsuperscript{]},\textsuperscript{[}16\textsuperscript{]}
).

Se realizó a todos los aislamientos con perfil de resistencia a los antibióticos
sugestivos de AmpC plasmídicos. Para esto se empleó \textsc{adn} obtenido mediante lisis
celular (por ebullición durante 10 minutos) de una suspensión 0,5 McFarland de
la cepa en estudio. Seguidamente se centrifugó a 11 300 revoluciones por minuto
(rpm) y se extrajo el sobrenadante, con el cual se realizó la reacción en cadena
de la polimerasa (\textsc{pcr}), según el protocolo descrito por Pérez-Pérez y Hanson
2002 (\textsuperscript{[}17\textsuperscript{]}
). Para el análisis molecular se incluyeron como controles: \textit{Proteus
mirabilis}
Re\textsc{lavra}-\textsc{ops} 146 (positivo) y \textit{Escherichia coli}
\textsc{atcc} 25922 (negativo). Además del marcador de peso molecular de 100-1000 pb,
marca Thermo\textsuperscript{®}. Se utilizó el equipo Termociclador Veriti (Applied Biosystems, modelo
9902\textsuperscript{®}. La visualización y análisis de los productos del \textsc{pcr} se realizó por medio de
electroforesis capilar, empleando el equipo \textsc{qia}xcel Advanced
System\textsuperscript{®}, para identificar el amplicón de 462 pb correspondiente a \textsc{cmy}-2.

\section{\textsc{resultados}}

En la revisión retrospectiva de las bases de datos del período comprendido entre
enero de 2003 hasta mayo de 2015, que incluía 4 363 aislamientos de diferentes
especies de \textit{Shigella}
y 1 785 de \textit{Salmonella}
spp., se detectaron un total de 24 aislamientos con un perfil fenotípico
sugestivo de β-lactamasa AmpC plasmídica. De estos, 15 correspondieron a
\textit{S. sonnei}
todos de origen clínico humano y nueve a \textit{Salmonella}
spp. (cuatro de origen clínico humano y cinco no humano).

Los 24 aislamientos antes mencionados presentaron sensibilidad reducida a las
C3G (\textsc{caz}: 11-16 mm y \textsc{ctx}: 12-20 mm), con zonas de inhibición consideradas dentro
del rango de sospecha de \textsc{blee} (según normas de \textsc{clsi} M100-S25). Sin embargo, en
todos ellos se descartó la presencia de \textsc{blee}, ya que ninguno presentó
agrandamiento igual o mayor que 5 mm para las combinaciones \textsc{ctx}/\textsc{ctx}-\textsc{amc} y
\textsc{caz}/\textsc{caz}-\textsc{amc}, ni la clásica inhibición (efecto “huevo”) entre \textsc{amc} y las C3G.
Todos los aislamientos presentaron zonas de inhibición para los carbapenemes
(\textsc{imp}, \textsc{mem} y \textsc{ert}) dentro de los rangos esperados de sensibilidad.

En los 24 aislamientos de \textit{S. sonnei}
y \textit{Salmonella}
spp. con perfil fenotípico sugestivo de β-lactamasa AmpC plasmídica se observó
sinergia entre \textsc{fox} y las C3G con el \textsc{apb}, confirmándose el mecanismo (figura~\ref{fig:f1}
).

Por métodos moleculares se confirmó la presencia del gen plasmídico bla\textsc{cmy}-2 en
los 24 aislamientos antes mencionados.

En el caso de S. \textit{sonnei}, aislamientos portadores del gen \textit{bla}
\textsc{cmy}-2 se confirmaron a partir del año 2012 (4/261 = 1,5 \%), mientras que entre
enero y mayo de 2015 se confirmaron 11/255 (4,3 \%) (cuadro 1). Todos los
aislamientos provenían de pacientes con diarrea, la mayoría adultos jóvenes, de
las provincias de San José, Limón y Alajuela, incluyendo tres casos relacionados
a un brote de diarrea en una comunidad fronteriza en el 2012. Los 15
aislamientos de \textit{S. sonnei}
fueron resistentes a \textsc{sxt}, catorce resultaron sensibles a \textsc{cip} y uno presentó
sensibilidad disminuida a este antibiótico.

Incluye solo de enero a mayo de 2015.

\textit{Fuente:}
Centro Nacional de Referencia en Bacteriología, Red Nacional de Laboratorios de
Bacteriología, Costa Rica.

Con relación a \textit{Salmonella}
de origen clínico portadores del gen \textit{bla}
\textsc{cmy}-2, los cuatro aislamientos que se confirmaron a partir del 2010
correspondían a diferentes serovariedades (cuadro 2). Tres de estos aislamientos
corresponden a pacientes menores de dos años de edad, incluyendo el obtenido de
sangre. A excepción de \textit{S}. \textit{cholerasuis}
var. Kunzendorf que presentó sensibilidad disminuida a \textsc{cip}, todos los
aislamientos clínicos fueron sensibles a \textsc{sxt} y \textsc{cip}.

NA, no aplica; ND: dato no disponible.

Número de aislamientos de \textit{Salmonella}
de origen clínico humano \textsc{cmy}-2 positivo/total de aislamientos
\textit{Salmonella}
de origen clínico humano a los que se realizó la prueba de sensibilidad a los
antibióticos durante el año.

\textit{Fuente:}
Centro Nacional de Referencia en Bacteriología, Red Nacional de Laboratorios de
Bacteriología, Costa Rica.

Con respecto a los aislamientos de \textit{Salmonella}
de origen no humano portadores del gen \textit{bla}
\textsc{cmy}-2, todos se recuperaron de muestras de aves de granjas de diferentes
provincias, tres de ellos en el 2010 y dos en el 2011. De estos aislamientos,
tres corresponden a \textit{S}. Heidelberg y dos a \textit{S}. Kentucky (cuadro 3). Los cinco aislamientos resultaron sensibles a \textsc{sxt} y dos
de ellos (\textit{S}. Heidelberg) presentaron sensibilidad disminuida a \textsc{cip}.

NA, no aplica.

Número de aislamientos de \textit{Salmonella}
de origen no humano \textsc{cmy}-2 positivo/total de aislamientos \textit{Salmonella}
de origen no humano a los que se realizó la prueba de sensibilidad a los
antibióticos durante el año.

Corresponde a un alimento importado.

\textit{Fuente:}
Centro Nacional de Referencia en Bacteriología, Red Nacional de Laboratorios de
Bacteriología, Costa Rica.

\section{\textsc{discusión}}

Las enzimas AmpC plasmídicas se han descrito en todos los continentes, con una
prevalencia variable según el microorganismo, del tipo de enzima y del área
geográfica. En general, la prevalencia de estas enzimas suele ser baja (menor al
2\% en las enterobacterias), aunque se ha reportado una tendencia al incremento
(\textsuperscript{[}18\textsuperscript{]}
).

En la literatura se encuentran reportes de aislamientos de \textit{S. flexneri}
y \textit{S. sonnei}
portadoras de \textsc{cmy}-2 en Irán, Japón, China y Argentina, entre otros
(\textsuperscript{[}2\textsuperscript{]},\textsuperscript{[}19\textsuperscript{]}
). También se describen aislamientos de diferentes serovariedades de
\textit{Salmonella}
(ej: \textit{S}. Typhimurium, \textit{S}. Heidelberg, \textit{S}. Kentucky) portadoras de este mecanismo en Estados Unidos, Japón y
recientemente en América del Sur (\textsuperscript{[}20\textsuperscript{]},\textsuperscript{[}21\textsuperscript{]}
). Sin embargo, existen muy pocas publicaciones científicas en relación al tema
en Centroamérica. En el caso particular de Costa Rica, hasta donde se tiene
conocimiento, este es el primer reporte de confirmación de cepas de
\textit{Shigella}
y \textit{Salmonella}
portadoras de \textsc{cmy}-2.

Con relación a los aislamientos de \textit{Shigella}
portadores de \textsc{cmy}-2, cabe resaltar que estos se confirmaron solo en la especie
\textit{S. sonnei}, a partir del 2012 (1,5\% de los aislamientos), observándose una tendencia al
incremento en el 2015 (4,3 \% de los aislamientos). Este hallazgo contrasta con
lo observado en Argentina, donde desde el 2006 se reportó la emergencia clonal
de \textit{S. flexneri}
portadora de \textsc{cmy}-2 (\textsuperscript{[}2\textsuperscript{]}
). Al respecto vale la pena destacar que a diferencia de otros países de la
región y similar a lo observado en países como Tailandia en el período 2002-2003
(\textsuperscript{[}22\textsuperscript{]}
), desde el 2005 S. \textit{sonnei}
es la especie más prevalente en Costa Rica como causa de shigelosis, seguida por
\textit{S. flexneri}. Por otra parte, la emergencia de \textsc{cmy}-2 en Costa Rica a partir del 2012
concuerda con lo reportado en México donde previo al 2012 no se habían
encontraron aislamientos de \textit{Shigella}
portadores de este mecanismo de resistencia
(\textsuperscript{[}23\textsuperscript{]}
).

Con relación a \textit{Salmonella}
de origen humano, en Costa Rica se documentaron únicamente cuatro aislamientos
portadores de \textsc{cmy}-2, todos entre 2010 y 2015, pertenecientes a serovariedades
poco frecuentes en el país (a excepción de \textit{S}. Typhimurium). Lo anterior contrasta con la elevada prevalencia observada en
\textit{S}. Typhimurium (10-27\%) en México para el período 2005-2011
(\textsuperscript{[}23\textsuperscript{]}
).

Cabe mencionar que en los aislamientos de Costa Rica se descartó la presencia de
\textsc{blee} en conjunto con \textsc{cmy}-2, lo que contrasta con los hallazgos reportados por
Liebana y col. (2004), que describen el aislamiento de \textit{S}. Infantis multirresistente portadora de \textsc{cmy}-2 y de \textsc{blee} tipo \textsc{ctx}-M15
(\textsuperscript{[}5\textsuperscript{]}
). De igual manera, en Venezuela durante el período 2010-2011 se notificaron
aislamientos de \textit{Salmonella}, que producían simultáneamente \textsc{cmy} plasmídico y \textsc{blee} de diferentes tipos
(\textsc{ctxm}-1, \textsc{ctmx}-2, \textsc{tem}-1) (datos no publicados)
(\textsuperscript{[}10\textsuperscript{]}
).

Estudios sugieren que los genes \textit{bla}
\textsc{cmy}-2 pueden ser transferidos entre diferentes géneros o especies bacterianas
asociadas a animales, alimentos para animales y para consumo humano y a
infecciones en los seres humanos (\textsuperscript{[}24\textsuperscript{]}
). En este estudio se demuestra la circulación de cepas de \textit{S}. Heidelberg y \textit{S}. Kentucky de origen aviar portadoras de \textsc{cmy}-2 en Costa Rica. Sin embargo, cabe
destacar que estos serovares se aíslan con muy poca frecuencia de infecciones de
humanos en Costa Rica, y a la fecha no se ha encontrado ninguno portador de
\textsc{cmy}-2. Al respecto, en Canadá se notificó por primera vez en el 2003 \textit{S}. Heidelberg \textsc{cmy}-2 positivo en intestino de jabalí
(\textsuperscript{[}16\textsuperscript{]}
). Además, en estudios realizados en México entre 2000-2005, se describen
aislamientos \textit{S}. Typhimurium portadores de \textsc{cmy}-2 de origen aviar (con una prevalencia de 3,9
\%), en carne de cerdo (3,5 \%) y en intestino porcino (7,4 \%)
(\textsuperscript{[}4\textsuperscript{]}
).

La detección de AmpC plasmídicas es especialmente importante en aquellos géneros
y especies que carecen naturalmente de estas enzimas a nivel cromosómico, como
es el caso de \textit{Salmonella}
spp. \textit{Proteus mirabilis}, \textit{Klebsiella}
sp., y en aquellas que la producen en pequeñas cantidades, como
\textit{Shigella}
sp. y \textit{Escherichia coli}. En estos microorganismos que no poseen una βeta-lactamasa tipo AmpC en su
cromosoma, la detección de AmpC plasmídicas implica la adquisición de este
elemento móvil.

A la fecha no existe consenso sobre el reporte de las C3G en cepas productoras
de AmpC plasmídico. Por otra parte, es escasa la bibliografía internacional
sobre la utilidad clínica de C3G en infecciones provocadas por enterobacterias
con AmpC plasmídica. Sin embargo, es prudente evitar el uso de las C3G en
infecciones severas. Lo anterior dado que, según Re\textsc{lavra}-\textsc{ops} y Jacoby (2009), la
resistencia enzimática, como la debida a AmpC plasmídico, eleva las
concentraciones inhibitorias mínimas (\textsc{cim}) de las C3G que pueden verse afectadas
por inóculos bacterianos de alta densidad
(\textsuperscript{[}2\textsuperscript{]},\textsuperscript{[}6\textsuperscript{]},\textsuperscript{[}13\textsuperscript{]}
).

Por lo tanto, se recomienda a los laboratorios de microbiología estar vigilantes
y enviar al Laboratorio Nacional de Referencia los aislamientos de
\textit{Salmonella}
y \textit{Shigella}
para realizar la confirmación correspondiente por métodos fenotípicos y
moleculares. Lo anterior es especialmente importante en bacterias aisladas de
infecciones extraintestinales.

Agradecimientos
Se reconoce la contribución de los microbiólogos de los laboratorios de la Red
Nacional de Laboratorios de Costa Rica, quienes suministraron los aislamientos,
y la información clínico-epidemiológica, incluida en este informe.
\section{Conflicto de intereses}

Ninguno declarado por los autores.

\section{Declaración}

Las opiniones expresadas en este manuscrito son responsabilidad del autor y no
reflejan necesariamente los criterios ni la política de la \textsc{rpsp}/\textsc{pajph} y/o de la
\textsc{ops}.

\section*{\textsc{referencias}}
\begin{itemize}

\item[1] Bush K, Jacoby G. Updated functional classification of β-lactamases.
Antimicrob Agents Chemother. 2010;54(3):969-76.

\item[2] Rapoport M, Monzani V, Pasteran F, Morvay L, Faccone D, Petroni A,
\textit{et al.}
\textsc{cmy}-2-type plasmid-mediated AmpC β-lactamase finally emerging in Argentina. Int
J Antimicrob Agents. 2008;31(4):385-87.

\item[3] Seral C, Gude MJ, Castillo FJ. Emergencia de β-lactamasas Amp-C
plasmídicas (pAmpC o cefamicinasas): origen, importancia, detección y
alternativas terapéuticas. Rev Esp Quimioter. 2012;25(2):89-99.

\item[4] Zaidi MB, Leon V, Canche C, Perez C, Zhao S, Hubert SH, \textit{et
al.}
Rapid and widespread dissemination of multidrug-resistant \textit{bla}
\textsc{cmy}-2 \textit{Salmonella typhimurium}
in Mexico. J Antimicrob Chemother. 2007;60(2):398-401.

\item[5] Lee KE, Lim SI, Choi HW, Lim SK, Song JY, An DJ. Plasmid-mediated
AmpC β-lactamase (\textsc{cmy}-2) gene in \textit{Salmonella}
Typhimurium isolated from diarrheic pigs in South Korea. \textsc{bmc} Res Notes.
2014;7:329. doi:10.1186/1756-0500-7-329.

\item[6] Red Latinoamericana de Vigilancia de la Resistencia a los
Antimicrobianos (Re\textsc{lavra})-Organización Panamericana de la Salud. Proyecto de
Prevención y Control de la Resistencia a los Antimicrobianos en las Américas.
Programa Latinoamericano de Control de Calidad en Bacteriología y Resistencia a
los Antimicrobianos (2008) Argentina, Encuesta- Informe N°15.

\item[7] Boyle F, Morris D, O’Connor J, DeLappe N, Ward J, Cormican M. First
report of extended-spectrum-β-lactamase-producing \textit{Salmonella enterica}
serovar Kentucky isolated from poultry in Ireland. Antimicrob Agents Chemother.
2010; 54(1):551-3.

\item[8] Folster JP, Pecic G, Bolcen S, Theobald L, Hise K, Carattoli A,
\textit{et al.}
Characterization of extended-spectrum cephalosporin resistant \textit{Salmonella
enterica}
serovar Heidelberg isolated from humans in the United States. Foodborne Pathog
Dis. 2010;7(2):181-7.

\item[9] Liebana E, Batchelor M, Torres C, Briñas L, Lagos LA, Abdalhamid B,
\textit{et al.}
Pediatric infection due to multiresistant \textit{Salmonella enterica}
serotype Infantis in Honduras. J Clin Microbiol. 2004;42(10):4885-8.

\item[10] Pulsenet America Latina y Caribe. Reunión Pulsenet America Latina y
Caribe. 9na Reunión. Santiago de Chile - Chile, 2011.

\item[11] Jacoby G. AmpC β-lactamases. Clin Microbiol Rev. 2009;22(1):61-182.

\item[12] Edwards PR, Ewing WH. Identification of Enterobacteriaceae, 4th
edition. New York: Elsevier; 1986.

\item[13] Grimont \textsc{pad}, Weill FX. Antigenic formulae of the
\textit{Salmonella}
serovars, 9th ed., \textsc{who} Collaborating Centre for Reference and Research on
\textit{Salmonella}, Institut Pasteur, Paris, France, 2007.

\item[14] \textsc{clsi} M02-A11 Performance Standards for Antimicrobial Disk
Susceptibility Tests; approved standard- Eleventh edition. Clinical and
Laboratory Standards Institute, Wayne, Pennsylvania, \textsc{usa}; 2012;32(1).

\item[15] \textsc{clsi} M100-S25 Performance Standards for Antimicrobial
Susceptibility Testing; twenty-fifth informational supplement. Clinical and
Laboratory Standards Institute, Wayne, Pennsylvania, \textsc{usa}; 2015;35(3).

\item[16] Aarestrup F, Hasman H, Olsen I, Sørensen G. International Spread of
\textit{bla}
\textsc{cmy}-2 mediated cephalosporin resistance in a multiresistant \textit{Salmonella}
enterica serovar Heidelberg isolate stemming from the importation of a boar by
Denmark from Canada. Antimicrob Agents Chemother. 2004;48(5):1916-7.

\item[17] Pérez-Pérez F, Hanson ND. Detection of plasmid-mediated AmpC
β-lactamase genes in clinical isolates by using multiplex \textsc{pcr}. J Clin Microbiol.
2002;40(6):2153-62.

\item[18] Calvo J, Cantón R, Fernández F, Mirelis B, Navarro F.
Procedimientos en microbiología clínica. Recomendaciones de la Sociedad Española
de Enfermedades Infecciosas y Microbiología Clínica. Detección fenotípica de
mecanismos de resistencia en Gram negativos. Segunda edición, 2011.

\item[19] Taneja N, Mewara A, Kumar A, Verma G, Sharma MJ.
Cephalosporin-resistant \textit{Shigella flexneri}
over 9 years (2001-2009) in India. Antimicrob Agents Chemother.
2012;67(6):1347-53.

\item[20] Cejas D, Vignoli R, Quinteros M, Marino R, Callejo R, Betancor L,
\textit{et al.}
First detection of \textsc{cmy}-2 plasmid mediated β-lactamase in \textit{Salmonella}
Heidelberg in South America. Rev Argent Microbiol. 2014;46(1):30-3.

\item[21] Lee K, Kusumoto M, Sekizuka T, Kuroda M, Uchida I, Iwata T,
\textit{et al.}
Extensive amplification of GI-\textsc{vii}-6, a multidrug resistance genomic island of
\textit{Salmonella enterica}
serovar Typhimurium, increases resistance to extended-spectrum cephalosporins.
Front Microbiol. 2015;6:1-10 doi:10.3389/fmicb.2015.00078

\item[22] Von Seidlein L, Kim DR, Ali M, Lee H, Wang XY, Thiem VD, \textit{et
al.}
A multicentre study of \textit{Shigella}
diarrhoea in six Asian countries: disease burden, clinical manifestations, and
microbiology. PLoS Med. 2006;3(9):1556-9.

\item[23] Zaidi MB, Estrada-García T, Campos FD, Chim R, Arjona F, Leon M,
\textit{et al.}
Incidence, clinical presentation, and antimicrobial resistance trends in
\textit{Salmonella}
and \textit{Shigella}
infections from children in Yucatan, Mexico. Front Microbiol. 2013;1(4):288.
doi:10.3389/fmicb.2013.00288

\item[24] Winokur PL, Vonstein DL, Hoffman LJ, Uhlenhopp EK, Doern GV.
Evidence for transfer of \textsc{cmy}-2 AmpC β-lactamase plasmids between
\textit{Escherichia coli}
and \textit{Salmonella}
isolates from food animals and humans. Antimicrob Agents Chemother. 2001;45(10):
2716-22.

\end{itemize}

\end{document}
