% Generated by jats2tex@0.11.1.0
\documentclass{article}
\usepackage[T1]{fontenc}
\usepackage[utf8]{inputenc} %% *
\usepackage[portuges,spanish,english,german,italian,russian]{babel} %% *
\usepackage{amstext}
\usepackage{authblk}
\usepackage{unicode-math}
\usepackage{multirow}
\usepackage{graphicx}
\usepackage{etoolbox}
\usepackage{xtab}
\usepackage{enumerate}
\usepackage{hyperref}
\usepackage{penalidades}
\usepackage[footnotesize,bf,hang]{caption}
\usepackage[nodayofweek,level]{datetime}
\usepackage[top=0.85in,left=2.75in,footskip=0.75in]{geometry}
\newlength\savedwidth
\newcommand\thickcline[1]{\noalign{\global
\savedwidth
\arrayrulewidth
\global\arrayrulewidth 2pt}
\cline{#1}
\noalign{\vskip\arrayrulewidth}
\noalign{\global\arrayrulewidth\savedwidth}}
\newcommand\thickhline{\noalign{\global
\savedwidth\arrayrulewidth
\global\arrayrulewidth 2pt}
\hline
\noalign{\global\arrayrulewidth\savedwidth}}
\usepackage{lastpage,fancyhdr}
\usepackage{epstopdf}
\pagestyle{myheadings}
\pagestyle{fancy}
\fancyhf{}
\setlength{\headheight}{27.023pt}
\lhead{\includegraphics[width=10mm]{logo.png}}
\rhead{\ifdef{\journaltitle}{\journaltitle}{}
\ifdef{\volume}{vol.\,\volume}{}
\ifdef{\issue}{(\issue)}{}
\ifdef{\fpage}{\fpage--\lpage\,pp.}}
\rfoot{\thepage/\pageref{LastPage}}
\renewcommand{\footrule}{\hrule height 2pt \vspace{2mm}}
\fancyheadoffset[L]{2.25in}
\fancyfootoffset[L]{2.25in}
\lfoot{\sf \ifdef{\articledoi}{\articledoi}{}}
\setmainfont{Linux Libertine O}
\renewcommand*{\thefootnote}{\alph{footnote}}
\makeatletter
\newcommand{\fn}{\afterassignment\fn@aux\count0=}
\newcommand{\fn@aux}{\csname fn\the\count0\endcsname}
\makeatother

\newcommand{\journalid}{Rev Bras Epidemiol}
\newcommand{\publisherid}{rbepid}
\newcommand{\journaltitle}{Revista Brasileira de Epidemiologia}
\newcommand{\abbrevjournaltitle}{Rev. bras. epidemiol.}
\newcommand{\issnppub}{1415-790X}
\newcommand{\issnepub}{1980-5497}
\newcommand{\publishername}{Associação Brasileira de Saúde Coletiva}
\newcommand\articledoi{\textsc{doi} 10.1590/1980-5497201500010008}
\def\subject{Artigos Originais}\newcommand{\subtitlestyle}[1]{--
\emph{#1}\medskip}
\newcommand{\transtitlestyle}[1]{\par\medskip\Large #1}
\newcommand{\transsubtitlestyle}[1]{-- \Large\emph{ #1}}

\newcommand{\titlegroup}{
\ifdef{\subtitle}{\subtitlestyle{\subtitle}}{}
\ifdef{\transtitle}{\transtitlestyle{\transtitle}}{}
\ifdef{\transsubtitle}{\transsubtitlestyle{\transsubtitle}}{}}

\title{Preenchimento da notificação compulsória em serviços de saúde que atendem
mulheres que sofrem violência sexual\titlegroup{}}
\author[{I}]{Sousa, Maria Helena de}
\author[{I}\textsuperscript{II}]{Bento, Silvana Ferreira}
\author[{I}\textsuperscript{\textsc{iii}}]{Osis, Maria José Duarte}
\author[{I}]{Ribeiro, Meire de Paula}
\author[{I}]{Faúndes, Anibal}
\affil[i]{Centro de Pesquisas em Saúde Reprodutiva de Campinas}
\affil[ii]{Universidade Estadual de Campinas}
\affil[iii]{Universidade Estadual de Campinas}
\def\authornotes{Autor correspondente: Maria Helena de Sousa. Departamento de
Pesquisa Social do Centro de Pesquisas em Saúde Reprodutiva de Campinas. Cidade
Universitária Zeferino Vaz, 6.181, \textsc{cep}: 13084-971, Campinas, SP, Brasil. E-mail:
mhestat@yahoo.com.br
Conflito de interesses: nada a declarar}
\date{ 2015}
\def\volume{18}
\def\issue{1}
\def\fpage{94}
\def\lpage{107}
\def\permissions{This is an Open Access article distributed under the terms of
the Creative Commons Attribution Non-Commercial License, which permits
unrestricted non-commercial use, distribution, and reproduction in any medium,
provided the original work is properly cited.}
\newcommand{\kwdgroup}{Violência sexual, Mulheres, Serviços de saúde,
Atendimento de emergência, Notificação de doenças, Violência contra a mulher}
\newcommand{\kwdgroupen}{Sexual Violence, Women, Health services, Emergency
care, Disease notification, Violence against women}
%%% Nota %%%%%%%%%%%%%%%%%%%%%%%%%%%%%%%%%%%%%%%%%%%%%%%%%%%%%%%%
\expandafter\newcommand\csname \endcsname{
Fonte de financiamento: Fundação de Amparo à Pesquisa do Estado de São Paulo
(\textsc{fapesp}) (processo n\textsuperscript{o}
2010/50345-6).}
%%% Nota fn01 %%%%%%%%%%%%%%%%%%%%%%%%%%%%%%%%%%%%%%%%%%%%%%%%%%%%%%%%
\expandafter\newcommand\csname fn01\endcsname{
Estabelece a notificação compulsória, no território nacional, de caso de
violência contra a mulher que for atendida em serviços de saúde públicos ou
privados\textsuperscript{21}.}
%%% Nota fn02 %%%%%%%%%%%%%%%%%%%%%%%%%%%%%%%%%%%%%%%%%%%%%%%%%%%%%%%%
\expandafter\newcommand\csname fn02\endcsname{
Regulamenta a Lei no. 10.778, de 24 de novembro de 2003, e institui os serviços
de referência sentinela\textsuperscript{22}.}
%%% Nota fn03 %%%%%%%%%%%%%%%%%%%%%%%%%%%%%%%%%%%%%%%%%%%%%%%%%%%%%%%%
\expandafter\newcommand\csname fn03\endcsname{
Define as terminologias adotadas em legislação nacional, conforme o disposto no
Regulamento Sanitário Internacional 2005 (\textsc{rsi} 2005), a relação de doenças,
agravos e eventos em saúde pública de notificação compulsória em todo o
território nacional e estabelece fluxo, critérios, responsabilidades e
atribuições aos profissionais e serviços de saúde\textsuperscript{23}.}

\begin{document}
\selectlanguage{portuges}
\section*{Metadados não aplicados}
\begin{itemize}
\item[\textbf{língua do artigo}]{Português}
\ifdef{\journalid}{\item[\textbf{journalid}] \journalid}{}
\ifdef{\journaltitle}{\item[\textbf{journaltitle}] \journaltitle}{}

\ifdef{\journalsubtitle}{\item[\textbf{journalsubtitle}] \journaltitle}{}
\ifdef{\transjournaltitle}{\item[\textbf{journaltitle}] \journaltitle}{}
\ifdef{\transjournalsubtitle}{\item[\textbf{journalsubtitle}] \journaltitle}{}

\ifdef{\abbrevjournaltitle}{\item[\textbf{abbrevjournaltitle}]
\abbrevjournaltitle}{}
\ifdef{\issnppub}{\item[\textbf{issnppub}] \issnppub}{}
\ifdef{\issnepub}{\item[\textbf{issnepub}] \issnepub}{}
\ifdef{\publishername}{\item[\textbf{publishername}] \publishername}{}
\ifdef{\publisherid}{\item[\textbf{publisherid}] \publisherid}{}
\ifdef{\subject}{\item[\textbf{subject}] \subject}{}
\ifdef{\transtitle}{\item[\textbf{transtitle}] \transtitle}{}
\ifdef{\authornotes}{\item[\textbf{authornotes}] \authornotes}{}
\ifdef{\articleid}{\item[\textbf{articleid}] \articleid}{}
\ifdef{\articledoi}{\item[\textbf{articledoi}] \articledoi}{}
\ifdef{\volume}{\item[\textbf{volume}] \volume}{}
\ifdef{\issue}{\item[\textbf{issue}] \issue}{}
\ifdef{\fpage}{\item[\textbf{fpage}] \fpage}{}
\ifdef{\lpage}{\item[\textbf{lpage}] \lpage}{}
\ifdef{\permissions}{\item[\textbf{permissions}] \permissions}{}
\end{itemize}
\maketitle

\begingroup

\begin{abstract}
\section{\textsc{objetivo}:}

Avaliar a proporção de serviços de saúde (SSs) que preenchem a notificação
compulsória e quais os principais obstáculos para o preenchimento de tal
documento.

\section{\textsc{métodos}:}

Realizou-se um estudo com abordagem mista. Para a abordagem quantitativa, foi
realizado um estudo de corte transversal, com coleta de dados por telefone.
Foram identificados 291 SSs no Estado de São Paulo que referiam dar atendimento
a mulheres que sofrem violência sexual. A amostra foi composta por 172 serviços
que referiram prestar atendimento de emergência às mulheres. Na abordagem
qualitativa realizaram-se estudos de casos, com amostra intencional e de
conveniência. Foram escolhidos seis municípios, nos quais foram entrevistados
profissionais de dois SSs. Realizaram-se 45 entrevistas semiestruturadas. Para
os dados quantitativos, realizou-se uma análise descritiva simples. Para os
dados qualitativos, realizou-se análise temática do conteúdo das entrevistas.

\section{\textsc{resultados}:}

A proporção de serviços que referiram sempre preencher a ficha de notificação em
casos de violência sexual foi de 79,1\%. Mais da metade (53,5\%) referiu
dificuldades relativas às mulheres atendidas, um terço referiu motivos
referentes à ficha e 29,7\% reportaram dificuldades relacionadas aos
profissionais. Na etapa qualitativa, as dificuldades mais referidas foram o
tamanho da ficha, os problemas para se obter as informações da mulher e a
dificuldade do profissional em obter essas informações.

\section{\textsc{conclusão}:}

Embora a maior parte dos SSs tenha referido preencher a ficha de notificação
compulsória, mencionou também várias dificuldades para fazê-lo, especialmente
relacionadas à sobrecarga de trabalho dos profissionais e à incompreensão acerca
da importância dessa notificação no contexto da atenção integral às mulheres que
sofrem violência sexual.

\iflanguage{portuges}{\medskip\noindent\textbf{Palavras-chave:} \kwdgroup}{}
\iflanguage{english}{\medskip\noindent\textbf{Keywords:} \kwdgroupen}{}
\iflanguage{spanish}{\medskip\noindent\textbf{Palavras claves:} \kwdgroupes}{}
\iflanguage{french}{\medskip\noindent\textbf{Mots clés:} \kwdgroupfr}{}
\end{abstract}
\endgroup

\begingroup
\renewcommand{\section}[1]{\subsection*{#1}}
\begin{otherlanguage}{english}

\begin{abstract}
\section{\textsc{objective}:}

To evaluate the proportion of health services that fill out the compulsory
notification and what the main difficulties to fill it out are.

\section{\textsc{methods}:}

A study was carried out with two different approaches. For the quantitative
approach, a cross sectional study was performed, with telephone data collection.
In the state of São Paulo, 291 health services that had reported providing care
to women who suffer from sexual violence were identified. The sample was
composed of 172 services that reported providing emergency care to women. In the
qualitative approach, case studies were conducted. Six cities were chosen by
intention and convenience. For each of them, professionals from two health
services were invited to participate. Forty-five semi-structured interviews were
conducted. For quantitative data, a descriptive analysis was carried out. For
qualitative data, a thematic analysis of content was performed.

\section{\textsc{results}:}

The proportion of health services which reported always filling out the
notification in cases of sexual violence was 79.1\%. More than half (53.5\%)
reported difficulties concerning the assisted women, one third reported reasons
related to the form, and 29.7\%, to the professionals. In the qualitative
approach, the main difficulties were the size of the form, the problems to
obtain the information about the woman and the difficulty for the professional
to obtain this information.

\section{\textsc{conclusion}:}

Although most health services claimed to fill out the compulsory notification,
they also mentioned several difficulties to do so, especially with regard to the
workload of professionals and the misunderstanding about the importance of the
notification in the context of comprehensive care to women who suffer from
sexual violence.

\ifdef{\kwdgroupen}{\medskip\noindent\textbf{Keywords:} \kwdgroupen}{}
\end{abstract}
\end{otherlanguage}
\endgroup
Fundação de Amparo à Pesquisa do Estado de São
Paulo2010/50345-6\section{\textsc{introdução}}

A violência contra as mulheres é atualmente um dos temas mais relevantes no
âmbito da Saúde Pública e dos direitos humanos, pois provoca sérios prejuízos à
saúde e ao desenvolvimento psicossocial dessas
mulheres\textsuperscript{[}1\textsuperscript{]}\textsuperscript{-}\textsuperscript{[}3\textsuperscript{]}. Dentre os tipos de violência contra as mulheres, a violência sexual, embora
não seja a mais prevalente, em geral é vista como aquela que pode causar
consequências tão ou mais impactantes do que os demais
tipos\textsuperscript{[}4\textsuperscript{]}.

A violência sexual contra as mulheres é praticada tanto por parceiros íntimos
como por outros homens, conhecidos ou não das
mulheres\textsuperscript{[}5\textsuperscript{]}\textsuperscript{,}\textsuperscript{[}6\textsuperscript{]}.

No Brasil, nos últimos anos, vários estudos têm apontado a grande frequência com
que as mulheres são agredidas, especialmente no ambiente doméstico e no âmbito
das relações afetivas\textsuperscript{[}7\textsuperscript{]}\textsuperscript{-}\textsuperscript{[}9\textsuperscript{]}. Em recente revisão sistemática sobre a prevalência de violência sexual
perpetrada por "não parceiros", chegou-se a uma estimativa mundial, para 2010,
de 7,2\%, e, para o Brasil, de 7,6\%\textsuperscript{[}10\textsuperscript{]}. Ao mesmo tempo, também se evidenciam muitas dificuldades para prover atenção
integral às mulheres que sofrem violência, principalmente a violência
sexual\textsuperscript{[}11\textsuperscript{]}\textsuperscript{-}\textsuperscript{[}15\textsuperscript{]}.

No âmbito das ações em saúde, têm-se proposto medidas para disponibilizar
atendimento efetivo às mulheres que sofrem violência
sexual\textsuperscript{[}16\textsuperscript{]}\textsuperscript{-}\textsuperscript{[}19\textsuperscript{]}. Desde 1999, o Ministério da Saúde estabeleceu a Norma Técnica "Prevenção e
tratamento dos agravos resultantes da violência sexual contra mulheres e
adolescentes", que define um protocolo de atendimento para as mulheres que
sofrem violência sexual. Apesar disso, não são todos os serviços de emergência
que atendem a esses casos de acordo com a norma. Pesquisa realizada em uma
amostra representativa de municípios brasileiros identificou 1.395 serviços de
saúde (SSs) que referiram atender a mulheres e crianças que sofrem violência
sexual, porém apenas 8\% tinham um protocolo de atendimento em conformidade com
a referida norma do Ministério da Saúde\textsuperscript{[}20\textsuperscript{]}.

Outra preocupação expressa pelas políticas públicas em relação à violência
contra as mulheres é a falta de registro dos casos por parte dos SSs, o que
poderia auxiliar o planejamento e a execução de medidas para prevenir as
ocorrências de violência, assim como para agilizar o atendimento às vítimas. No
sentido de suprir essa falta e permitir traçar estratégias de prevenção, o
Governo Federal estabeleceu a exigência de notificação compulsória em casos de
violência contra a mulher. De acordo com a Lei n\textsuperscript{o}
10.778, de 24 de novembro de 2003\textsuperscript{a}, todos os casos de violência contra a mulher - violência física, sexual e
psicológica - atendidos em SSs públicos ou privados necessitam ser notificados
por meio do preenchimento de uma ficha. Essa lei foi regulamentada pelo
Ministério da Saúde em 03 de junho de 2004, por meio do Decreto
n\textsuperscript{o}
5.099\textsuperscript{b}. Em 2011, uma portaria do Ministério da Saúde incluiu as violências doméstica,
sexual e/ou outras como parte da Lista de Notificação Compulsória
(\textsc{lnc})\textsuperscript{c}.

Até o presente, porém, especificamente em relação à violência sexual, não se
sabe em que medida a notificação compulsória da violência é realizada, nem como
esse procedimento é visto pelos profissionais de saúde. Em vista do exposto, o
presente estudo teve como objetivo verificar que proporção dos SSs que dão
atendimento de emergência a mulheres que sofrem violência sexual preenche a
notificação compulsória e quais são os principais obstáculos para o
preenchimento de tal documento.

\section{\textsc{métodos}}

Realizou-se um estudo com abordagem mista, quantitativa e qualitativa com SSs do
Estado de São Paulo que referiram prestar atendimento de emergência a mulheres
que sofrem violência sexual. Para a abordagem quantitativa, foi desenvolvido um
estudo descritivo, de corte transversal; para a qualitativa, foram realizados
estudos de casos\textsuperscript{[}24\textsuperscript{]}\textsuperscript{,}\textsuperscript{[}25\textsuperscript{]}.

Para a etapa quantitativa, considerou-se uma proporção de SSs que preenchem a
notificação compulsória estimada em 50\%, com uma probabilidade de erro tipo I
de 5\% e uma diferença absoluta aceitável entre a proporção amostral e a
populacional de 7,6\%, estabelecendo-se um número mínimo de 166
SSs\textsuperscript{[}26\textsuperscript{]}.

Foram incluídos na etapa quantitativa 172 SSs que referiram prestar atendimento
de emergência em caso de violência sexual. Esses serviços foram identificados e
selecionados da seguinte maneira: de todos os 240 SSs do Estado de São Paulo que
fizeram parte da pesquisa "Perfil do atendimento à violência sexual no
Brasil"\textsuperscript{[}13\textsuperscript{]}, foi convidado(a) a participar do estudo o(a) coordenador(a) ou a pessoa
responsável pelo atendimento a mulheres que sofrem violência sexual. Além
desses, foram contatados outros serviços indicados pelos primeiros; em casos de
indicação de mais de um serviço, sorteou-se apenas um dentre os indicados. Um
total de 291 SSs foi inicialmente abordado; 40 recusaram de imediato em
participar do estudo e 5 foram considerados "perdidos" devido a muitas
tentativas fracassadas de contato. Outros 25 serviços responderam não prestar
atendimento de emergência a mulheres que sofrem violência em geral, 16 afirmaram
não dar atendimento de emergência a mulheres que sofrem violência sexual, 32 não
notificavam casos de violência em geral (qualquer tipo), e um serviço não soube
informar.

Aplicou-se, por telefone, um questionário estruturado e pré-testado ao(à)
coordenador(a) ou à pessoa responsável pelo atendimento a mulheres que sofrem
violência sexual em cada serviço. As entrevistas foram gravadas mediante
autorização prévia.

A ficha de notificação compulsória faz parte do Sistema de Informação de Agravos
de Notificação (\textsc{sinan}) e possui diversos campos a serem preenchidos; entre eles,
os dados pessoais do paciente, dados complementares sobre a pessoa atendida,
dados da ocorrência, tipologia da violência e alguns campos específicos caso a
violência seja sexual. Também há dados do provável autor da agressão, e
informações sobre evolução e encaminhamento.

As variáveis independentes consideradas neste estudo foram: tipo do SS
(hospital; Centro de Referência em Saúde da Mulher; pronto-socorro/Unidade de
Pronto Atendimento (\textsc{upa}); Programa de Saúde da Família (\textsc{psf}); Unidade Básica de
Saúde (\textsc{ubs}); ambulatórios/Serviço de Atendimento Especializado (\textsc{sae})); e função
do respondente (assistente social; enfermeiro(a); médico(a); psicólogo(a); outra
função).

As variáveis dependentes foram:

1) Frequência de preenchimento da ficha de notificação em casos de violência
sexual:

\begin{enumerate}[a)]
\item
Sempre;

\item
Na maioria das vezes;

\item
Às vezes;

\item
Nunca;

\end{enumerate}

Sempre;

Na maioria das vezes;

Às vezes;

Nunca;

2) Conhecimento sobre a notificação ser compulsória:

\begin{enumerate}[a)]
\item
Sim;

\item
Não;

\item
Não sabe;

\end{enumerate}

Sim;

Não;

Não sabe;

3) Órgão ao qual é feita a notificação de casos de violência:

\begin{enumerate}[a)]
\item
Polícia (Delegacia de Polícia/Delegacia da Mulher/Polícia Militar);

\item
Conselhos específicos (Conselho Tutelar/Vara da Infância e Juventude/Conselho da
Mulher e Idoso);

\item
Órgãos gestores da Saúde (Ministério da Saúde; Secretaria Municipal de Saúde;
Secretaria Estadual de Saúde);

\item
Sistemas de Vigilância/Informação (Vigilância Epidemiológica/Sistemas de
Informações do Ministério da Saúde);

\item
Instituições que prestam atendimento a mulheres e adolescentes no município;

\item
Outro;

\item
Não sabe;

\end{enumerate}

Polícia (Delegacia de Polícia/Delegacia da Mulher/Polícia Militar);

Conselhos específicos (Conselho Tutelar/Vara da Infância e Juventude/Conselho da
Mulher e Idoso);

Órgãos gestores da Saúde (Ministério da Saúde; Secretaria Municipal de Saúde;
Secretaria Estadual de Saúde);

Sistemas de Vigilância/Informação (Vigilância Epidemiológica/Sistemas de
Informações do Ministério da Saúde);

Instituições que prestam atendimento a mulheres e adolescentes no município;

Outro;

Não sabe;

4) Razões referidas pelos SSs para o preenchimento ou não da ficha de
notificação compulsória em casos de violência sexual:

a. Razões para preencher:

\begin{enumerate}[i.]
\item
Obrigatoriedade (Porque é obrigatório/Ministério da Saúde e/ou município
exigem/Protocolo ou rotina da unidade);

\item
Informação ou estatísticas (Ajuda a identificar o número de casos/obter dados
estatísticos);

\item
Saúde da mulher que sofre a violência (Para a mulher seguir o protocolo/fazer
exames/tomar medicamentos);

\item
Programas de Saúde Pública (Ajuda a traçar estratégias para implantar programas
de prevenção da violência sexual/políticas públicas);

\item
Para encaminhar a outra instituição;

\end{enumerate}

Obrigatoriedade (Porque é obrigatório/Ministério da Saúde e/ou município
exigem/Protocolo ou rotina da unidade);

Informação ou estatísticas (Ajuda a identificar o número de casos/obter dados
estatísticos);

Saúde da mulher que sofre a violência (Para a mulher seguir o protocolo/fazer
exames/tomar medicamentos);

Programas de Saúde Pública (Ajuda a traçar estratégias para implantar programas
de prevenção da violência sexual/políticas públicas);

Para encaminhar a outra instituição;

b. Razões para não preencher:

\begin{enumerate}[i.]
\item
Razões que revelam desinformação do respondente (Nunca receberam informação/Nem
sempre a mulher quer denunciar/Tem que partir dela fazer o boletim de
ocorrência/Não é encaminhada para o setor interno que faz a notificação);

\item
Razões referentes ao despreparo dos profissionais do serviço (Nem todos os
profissionais sabem preencher a ficha/Alguns acham que não é necessário/Falta
sensibilização);

\item
Razões do funcionamento do serviço (Nem sempre conseguem todos os dados/Falta de
tempo/Falta de profissional);

\item
Outras razões;

\item
Não sabe/Não informou.

\end{enumerate}

Razões que revelam desinformação do respondente (Nunca receberam informação/Nem
sempre a mulher quer denunciar/Tem que partir dela fazer o boletim de
ocorrência/Não é encaminhada para o setor interno que faz a notificação);

Razões referentes ao despreparo dos profissionais do serviço (Nem todos os
profissionais sabem preencher a ficha/Alguns acham que não é necessário/Falta
sensibilização);

Razões do funcionamento do serviço (Nem sempre conseguem todos os dados/Falta de
tempo/Falta de profissional);

Outras razões;

Não sabe/Não informou.

Quanto às características dos SSs que preenchiam a ficha de notificação, foram
avaliadas as seguintes variáveis dependentes:

1) Profissional que preenche a notificação: médico(a); enfermeiro(a); assistente
social; psicólogo(a); técnico(a) ou auxiliar de enfermagem; outro;

2) Principais dificuldades enfrentadas durante o preenchimento da ficha de
notificação compulsória:

\begin{enumerate}[a)]
\item
Razões referentes às mulheres (Dificuldade em obter informação com a
paciente/Resistência e/ou medo das mulheres);

\item
Razões referentes às características da ficha (A ficha é difícil/complicada de
preencher; A ficha é muito extensa; Demora no preenchimento; Fluxo para fazer a
notificação está falho/falta sistematização);

\item
Razões dos profissionais (Falta de tempo/Esquecimento dos profissionais; Falta
de vontade/má vontade dos profissionais);

\item
Razões das características do local (Falta de um lugar mais reservado para fazer
o atendimento e as perguntas);

\item
Outra razão;

\end{enumerate}

Razões referentes às mulheres (Dificuldade em obter informação com a
paciente/Resistência e/ou medo das mulheres);

Razões referentes às características da ficha (A ficha é difícil/complicada de
preencher; A ficha é muito extensa; Demora no preenchimento; Fluxo para fazer a
notificação está falho/falta sistematização);

Razões dos profissionais (Falta de tempo/Esquecimento dos profissionais; Falta
de vontade/má vontade dos profissionais);

Razões das características do local (Falta de um lugar mais reservado para fazer
o atendimento e as perguntas);

Outra razão;

3) Opinião sobre a ficha de notificação trazer algum benefício para as mulheres:

\begin{enumerate}[a)]
\item
Sim;

\item
Não;

\item
Não sabe;

\end{enumerate}

Sim;

Não;

Não sabe;

4) Opinião sobre quais benefícios pode trazer às mulheres:

\begin{enumerate}[a)]
\item
Informação ou estatísticas (Ajuda a identificar o número de casos de violência
sexual/obter dados estatísticos);

\item
Programas de Saúde Pública (Ajuda a traçar estratégias para implantar programas
de prevenção da violência);

\item
Saúde da mulher que sofre violência (Ajuda para posterior
acompanhamento/tratamento adequado);

\item
Mulher pode falar sobre o assunto;

\item
Processo policial/judicial (Documentos para tomar providências/dar o devido
encaminhamento; Forma de conter esse tipo de violência/amedrontar o agressor);

\item
Outra opinião;

\item
Não sabe informar;

\end{enumerate}

Informação ou estatísticas (Ajuda a identificar o número de casos de violência
sexual/obter dados estatísticos);

Programas de Saúde Pública (Ajuda a traçar estratégias para implantar programas
de prevenção da violência);

Saúde da mulher que sofre violência (Ajuda para posterior
acompanhamento/tratamento adequado);

Mulher pode falar sobre o assunto;

Processo policial/judicial (Documentos para tomar providências/dar o devido
encaminhamento; Forma de conter esse tipo de violência/amedrontar o agressor);

Outra opinião;

Não sabe informar;

5) Medidas tomadas para assegurar o sigilo das informações preenchidas na ficha
de notificação:

\begin{enumerate}[a)]
\item
Cuidados no arquivamento das informações (Ficha arquivada em local
específico/sala de arquivo; Ficha arquivada no prontuário da paciente; Fica em
um arquivo/armário fechado; Arquivada em sala fechada/trancada);

\item
Limitação de quem tem acesso (Funcionários do serviço é que têm acesso à
ficha/Apenas um funcionário tem acesso à ficha/Acesso apenas a pessoas com
autorização por escrito do diretor, ou judicial);

\item
Encaminhamento a outras instituições/órgãos (Encaminhada para a Vigilância
Epidemiológica; Encaminhada para outras instituições/órgãos; Encaminhada para
outro serviço/órgão em envelope lacrado);

\item
Ficha arquivada no prontuário da paciente;

\item
O serviço não fica com cópia da ficha;

\item
Arquivada na recepção/no serviço.

\end{enumerate}

Cuidados no arquivamento das informações (Ficha arquivada em local
específico/sala de arquivo; Ficha arquivada no prontuário da paciente; Fica em
um arquivo/armário fechado; Arquivada em sala fechada/trancada);

Limitação de quem tem acesso (Funcionários do serviço é que têm acesso à
ficha/Apenas um funcionário tem acesso à ficha/Acesso apenas a pessoas com
autorização por escrito do diretor, ou judicial);

Encaminhamento a outras instituições/órgãos (Encaminhada para a Vigilância
Epidemiológica; Encaminhada para outras instituições/órgãos; Encaminhada para
outro serviço/órgão em envelope lacrado);

Ficha arquivada no prontuário da paciente;

O serviço não fica com cópia da ficha;

Arquivada na recepção/no serviço.

Realizou-se análise descritiva simples\textsuperscript{[}27\textsuperscript{]}
com o auxílio do pacote estatístico \textit{\textsc{spss} for Windows}.

Para o componente qualitativo, a partir dos dados obtidos na etapa quantitativa,
determinou-se uma amostra intencional\textsuperscript{[}25\textsuperscript{]}
de seis municípios cuja seleção foi feita de acordo com: a distribuição
geográfica no Estado de São Paulo dos municípios participantes na primeira
etapa; o tamanho dos mesmos (< 100.000 habitantes e ≥ 100.000 habitantes); a
existência de, ao menos, um SS que preenchia sempre a ficha de notificação
compulsória em casos de violência sexual e, pelo menos, um SS que preenchia com
outra frequência (na maioria das vezes, às vezes ou nunca). Selecionaram-se
quatro municípios grandes (≥ 100.000 habitantes) e dois pequenos, com dois SSs
em cada um.

Duas pesquisadoras visitaram cada SS selecionado para realizar entrevistas
semiestruturadas com o coordenador ou com a pessoa responsável pelo atendimento
a mulheres que sofrem violência sexual e outros profissionais envolvidos nesse
atendimento. Ao todo, foram realizadas 45 entrevistas. A maioria das pessoas
entrevistadas era do sexo feminino e trabalhava havia muitos anos nos serviços
visitados; 31 eram profissionais que atendiam diretamente às mulheres (16 da
área de enfermagem, 8 médicos, quatro assistentes sociais e 3 psicólogas); 14
eram gestores.

Para conduzir as entrevistas, utilizou-se um roteiro elaborado a partir dos
resultados da primeira etapa do estudo, visando a aprofundar alguns aspectos,
especialmente os relacionados às dificuldades para preencher a ficha de
notificação. Esse roteiro foi composto de oito perguntas sobre os seguintes
tópicos: como ocorria o atendimento de emergência às mulheres que sofrem
violência sexual, conhecimento sobre a ficha de notificação compulsória para os
casos de violência sexual, como era feito o preenchimento da ficha, as
dificuldades enfrentadas durante o preenchimento e qual encaminhamento era dado
à ficha preenchida. No caso de serviços que referiam não preencher a ficha ou
preencher somente algumas vezes, era perguntado por que isso ocorria.

As entrevistas duraram em média 20 minutos, foram gravadas e depois transcritas.
A análise temática de conteúdo foi realizada com o auxílio do programa NVivo. Os
pesquisadores fizeram uma leitura flutuante\textsuperscript{[}28\textsuperscript{]}
dos textos transcritos para identificar as unidades de significado relacionadas
aos objetivos do estudo e aos resultados quantitativos. Essas unidades foram
agrupadas e classificadas em categorias de análise que, ao serem articuladas,
podiam dar resposta aos objetivos
propostos\textsuperscript{[}29\textsuperscript{]}. Neste artigo são apresentados resultados referentes às seguintes categorias de
análise: conhecimento sobre a ficha de notificação compulsória para os casos de
violência sexual, seu preenchimento, as dificuldades encontradas para
preenchê-la, qual o encaminhamento dado às fichas preenchidas e sugestões para
melhorar as condições de preenchimento.

O protocolo desta pesquisa foi aprovado sem restrições pelo Comitê de Ética em
Pesquisa da Faculdade de Ciências Médicas da Universidade Estadual de Campinas,
sob o protocolo nº 144/2009. As diretrizes da Resolução nº 196/96 do Conselho
Nacional de Saúde (sobre pesquisas envolvendo seres
humanos\textsuperscript{[}30\textsuperscript{]}
) então vigentes foram respeitadas neste estudo. A participação dos sujeitos foi
voluntária e suas identidades foram mantidas em sigilo. Para o componente
quantitativo, um primeiro termo de consentimento livre e esclarecido (\textsc{tcle}) foi
lido para cada um dos possíveis entrevistados e eles foram questionados sobre a
decisão de participar ou não do estudo; quando a pessoa concordou em participar,
tal anuência foi gravada por telefone. Para a etapa qualitativa, antes da
entrevista foi fornecido outro \textsc{tcle} para que o profissional lesse e assinasse,
caso concordasse em participar do estudo.

\begin{enumerate}[a)]
\item
Sempre;

\item
Na maioria das vezes;

\item
Às vezes;

\item
Nunca;

\end{enumerate}
\begin{enumerate}[a)]
\item
Sim;

\item
Não;

\item
Não sabe;

\end{enumerate}
\begin{enumerate}[a)]
\item
Polícia (Delegacia de Polícia/Delegacia da Mulher/Polícia Militar);

\item
Conselhos específicos (Conselho Tutelar/Vara da Infância e Juventude/Conselho da
Mulher e Idoso);

\item
Órgãos gestores da Saúde (Ministério da Saúde; Secretaria Municipal de Saúde;
Secretaria Estadual de Saúde);

\item
Sistemas de Vigilância/Informação (Vigilância Epidemiológica/Sistemas de
Informações do Ministério da Saúde);

\item
Instituições que prestam atendimento a mulheres e adolescentes no município;

\item
Outro;

\item
Não sabe;

\end{enumerate}
\begin{enumerate}[i.]
\item
Obrigatoriedade (Porque é obrigatório/Ministério da Saúde e/ou município
exigem/Protocolo ou rotina da unidade);

\item
Informação ou estatísticas (Ajuda a identificar o número de casos/obter dados
estatísticos);

\item
Saúde da mulher que sofre a violência (Para a mulher seguir o protocolo/fazer
exames/tomar medicamentos);

\item
Programas de Saúde Pública (Ajuda a traçar estratégias para implantar programas
de prevenção da violência sexual/políticas públicas);

\item
Para encaminhar a outra instituição;

\end{enumerate}
\begin{enumerate}[i.]
\item
Razões que revelam desinformação do respondente (Nunca receberam informação/Nem
sempre a mulher quer denunciar/Tem que partir dela fazer o boletim de
ocorrência/Não é encaminhada para o setor interno que faz a notificação);

\item
Razões referentes ao despreparo dos profissionais do serviço (Nem todos os
profissionais sabem preencher a ficha/Alguns acham que não é necessário/Falta
sensibilização);

\item
Razões do funcionamento do serviço (Nem sempre conseguem todos os dados/Falta de
tempo/Falta de profissional);

\item
Outras razões;

\item
Não sabe/Não informou.

\end{enumerate}
\begin{enumerate}[a)]
\item
Razões referentes às mulheres (Dificuldade em obter informação com a
paciente/Resistência e/ou medo das mulheres);

\item
Razões referentes às características da ficha (A ficha é difícil/complicada de
preencher; A ficha é muito extensa; Demora no preenchimento; Fluxo para fazer a
notificação está falho/falta sistematização);

\item
Razões dos profissionais (Falta de tempo/Esquecimento dos profissionais; Falta
de vontade/má vontade dos profissionais);

\item
Razões das características do local (Falta de um lugar mais reservado para fazer
o atendimento e as perguntas);

\item
Outra razão;

\end{enumerate}
\begin{enumerate}[a)]
\item
Sim;

\item
Não;

\item
Não sabe;

\end{enumerate}
\begin{enumerate}[a)]
\item
Informação ou estatísticas (Ajuda a identificar o número de casos de violência
sexual/obter dados estatísticos);

\item
Programas de Saúde Pública (Ajuda a traçar estratégias para implantar programas
de prevenção da violência);

\item
Saúde da mulher que sofre violência (Ajuda para posterior
acompanhamento/tratamento adequado);

\item
Mulher pode falar sobre o assunto;

\item
Processo policial/judicial (Documentos para tomar providências/dar o devido
encaminhamento; Forma de conter esse tipo de violência/amedrontar o agressor);

\item
Outra opinião;

\item
Não sabe informar;

\end{enumerate}
\begin{enumerate}[a)]
\item
Cuidados no arquivamento das informações (Ficha arquivada em local
específico/sala de arquivo; Ficha arquivada no prontuário da paciente; Fica em
um arquivo/armário fechado; Arquivada em sala fechada/trancada);

\item
Limitação de quem tem acesso (Funcionários do serviço é que têm acesso à
ficha/Apenas um funcionário tem acesso à ficha/Acesso apenas a pessoas com
autorização por escrito do diretor, ou judicial);

\item
Encaminhamento a outras instituições/órgãos (Encaminhada para a Vigilância
Epidemiológica; Encaminhada para outras instituições/órgãos; Encaminhada para
outro serviço/órgão em envelope lacrado);

\item
Ficha arquivada no prontuário da paciente;

\item
O serviço não fica com cópia da ficha;

\item
Arquivada na recepção/no serviço.

\end{enumerate}

\section{\textsc{resultados}}

Dos 172 SSs que referiram prestar atendimento de emergência a mulheres que
sofreram violência sexual, 46,5\% eram hospitais, 23,3\%, prontos-socorros ou
\textsc{upa}s, 16,9\%, \textsc{ubs}s, 10,5\%, ambulatórios/\textsc{sae}s, 1,7\%, \textsc{psf}s, e 1,2\%, Centros de
Referência em Saúde da Mulher. Mais da metade (58,1\%) dos participantes
responsáveis pelo atendimento às mulheres que sofrem violência sexual era
composta de enfermeiros(as), 19,2\% eram médicos(as), 10,5\%, assistentes
sociais, 4,7\%, psicólogos(as), e 7,6\% referiram outra função. A maioria
(79,1\%) desses 172 serviços afirmou preencher sempre a ficha de notificação,
12,8\% relataram que preenchiam na maioria das vezes, 5,2\% referiram preencher
às vezes, 1,2\% afirmaram nunca preencher, e 1,7\% não souberam informar.

Dos 167 SSs que notificavam casos de violência sexual, 95,2\% referiram saber
que a notificação é compulsória, excetuando-se um SS que não respondeu.

A maior parte desses 167 serviços que preenchiam com alguma frequência a ficha
de notificação compulsória (excluindo-se um SS que não respondeu) referiu que a
notificação de violência sexual era feita aos sistemas de vigilância ou de
informações do Ministério da Saúde (65,7\%), e 20,5\% referiram órgãos gestores
da Saúde.

As razões mais referidas, espontaneamente, para o preenchimento da ficha de
notificação compulsória em casos de violência sexual foram a obrigatoriedade
(60,9\%) e a obtenção de informações ou estatísticas (27,8\%). Dentre as razões
para não se preencher a ficha, as mais referidas foram as que revelam
desinformação do respondente (8,3\%) e as referentes ao despreparo dos
profissionais (6,5\%).

Nos SSs que preenchiam com alguma frequência a ficha de notificação em casos de
violência sexual (n = 167), os profissionais mais referidos como sendo os que
faziam tal trabalho foram os enfermeiros(as) (78,4\%), seguidos dos médicos(as)
(46,1\%) e dos assistentes sociais (24,0\%) (Tabela 1). O preenchimento
exclusivamente por enfermeiros(as) foi referido por 51 SSs (30,5\%), enquanto
tal atitude por parte de médicos(as) foi registrada em 14 SSs (8,4\%). Em
relação às dificuldades enfrentadas durante o preenchimento da ficha, 57 SSs
(34,1\%) responderam que não tinham nenhuma (dados não apresentados em tabela);
dentre os 101 que indicaram alguma dificuldade, 53,5\% disseram que a
dificuldade era referente às mulheres; 33,7\% referiram razões das
características da ficha, e 29,7\% citaram razões dos profissionais que deveriam
preencher (Tabela 1). A grande maioria (86,2\%) dos(as) respondentes opinou que
a ficha de notificação pode trazer algum benefício às mulheres (dado não
apresentado em tabela); desses, quase a metade (48,6\%) referiu que a ficha pode
trazer informações ou estatísticas e contribuir para a implementação de
programas de Saúde Pública (38,2\%) (Tabela 1).

Tabela 1.Características dos serviços de saúde que preenchiam a ficha de
notificação em casos de violência sexual sempre, na maioria das vezes ou às
vezes.
\begin{table}
\begin{xtabular}{ l | l | l }
\hline
Característica & n & \%\\ \hline
\multicolumn{3}{l}{Profissional que preenche a notificação (n = 167)}
\\ \hline

Enfermeiro(a)
& 131
& 78,4
\\ \hline

Médico(a)
& 77
& 46,1
\\ \hline

Assistente social
& 40
& 24,0
\\ \hline

Técnico(a)/auxiliar de enfermagem
& 24
& 14,4
\\ \hline

Psicólogo(a)
& 17
& 10,2
\\ \hline

Outro
& 7
& 4,2
\\ \hline

\multicolumn{3}{l}{Principais dificuldades enfrentadas para o preenchimento da
ficha de notificação compulsória (n = 101)}
\\ \hline

Razões referentes às mulheres
& 54
& 53,5
\\ \hline

Razões referentes às características da ficha
& 34
& 33,7
\\ \hline

Razões dos profissionais
& 30
& 29,7
\\ \hline

Razões das características do local
& 3
& 3,0
\\ \hline

Outras
& 6
& 5,9
\\ \hline

\multicolumn{3}{l}{Opinião sobre quais benefícios pode trazer às mulheres (n =
144)}
\\ \hline

Informação/estatísticas
& 70
& 48,6
\\ \hline

Programas de Saúde Pública
& 55
& 38,2
\\ \hline

Saúde da mulher que sofre violência
& 36
& 25,0
\\ \hline

Mulher pode falar sobre o assunto
& 18
& 12,5
\\ \hline

Processo policial/judicial
& 17
& 11,8
\\ \hline

Outra
& 12
& 8,3
\\ \hline

Não soube informar
& 3
& 2,1
\\ \hline

\end{xtabular}
\end{table}

Como medidas mais frequentemente utilizadas para assegurar o sigilo das
informações da ficha de notificação dentro dos SSs foram citadas: cuidados no
arquivamento das informações (67,1\%); limitação em quem tem acesso à ficha
(65,2\%); encaminhamento a outras instituições/órgãos (60,9\%).

Na etapa qualitativa, em todos os municípios (quatro deles grandes e dois
pequenos) pelo menos um dos SSs referiu preencher sempre a ficha de notificação
compulsória. Em um SS situado em município pequeno houve o relato de que, quando
aparecia um caso de violência sexual, a Vigilância Epidemiológica do município
(que ficava dentro do outro serviço estudado) era acionada para ir até o SS
preencher a ficha e tomar as providências necessárias ao cuidado daquele caso.

A grande maioria dos profissionais que faziam o preenchimento era composta de
enfermeiros(as). Alguns médicos(as) que aceitaram participar da pesquisa não
sabiam da existência da ficha de notificação compulsória, nem quais
profissionais eram responsáveis pelo seu preenchimento.

Em relação às dificuldades encontradas para o preenchimento da ficha de
notificação compulsória, a mais referida foi sobre o tamanho da ficha ("longa";
"extensa"; "muitos detalhes"; "muita informação"). Uma segunda dificuldade foi o
problema de se obter informações da mulher ("fragilizada"; "vulnerável"; "às
vezes a mulher não fala que sofreu esse tipo de violência"; "receio"; "medo";
"vergonha"). Também houve o relato da dificuldade do profissional em obter as
informações ("constrangimento"; "difícil conversar com a mulher nesse momento
delicado"; entre outros). As duas últimas dificuldades foram algumas vezes
relacionadas também ao fato de não existir um local ou sala no SS onde a mulher
e o profissional do serviço pudessem conversar reservadamente.

Quase todos os SSs que preencheram a ficha afirmaram manter uma cópia da
notificação no próprio serviço, porém poucos funcionários além da pessoa que
preenchia tinham conhecimento sobre isso. Em relação ao encaminhamento dado à
ficha preenchida, não houve consenso entre os participantes. Diversos referiram
o encaminhamento à Vigilância Epidemiológica, mas alguns relataram o envio a
outras instituições, entre elas delegacias; muitos não sabiam o destino dado à
ficha de notificação.

As pessoas entrevistadas deram sugestões para melhorar as condições de
preenchimento da ficha de notificação nos serviços, entre elas: treinamento dos
profissionais, conscientização sobre a importância do preenchimento,
envolvimento dos profissionais, diminuição da quantidade de informações na
ficha, melhorar o atendimento à mulher, para que ela não tenha que repetir
informações. Porém, a sugestão mais frequente foi a de se fazer treinamentos
periódicos para os profissionais de cada SS, em especial devido à rotatividade
dos mesmos.

\section{\textsc{discussão}}

Os resultados apresentados indicam que o preenchimento da ficha de notificação
compulsória da violência sexual ainda não está totalmente incorporado às rotinas
dos SSs que atendem a esses casos no Estado de São Paulo. Isso se torna mais
preocupante quando consideramos que o Estado de São Paulo foi pioneiro no Brasil
no estabelecimento de serviços de atendimento a mulheres que sofrem violência
sexual. Além disso, também deve-se considerar que a grande maioria dos serviços
que participaram do estudo era composta de hospitais ou prontos-socorros, em
relação aos quais se poderia supor que tivessem maior familiaridade com a
questão da notificação.

Embora a grande maioria das pessoas entrevistadas tenha manifestado a opinião de
que o preenchimento da ficha pode trazer benefícios às mulheres, apenas pouco
mais de um quarto dos serviços referiu como motivo para o preenchimento a
possibilidade de obter dados estatísticos/informações sobre as ocorrências. Isso
reforça a hipótese de que ainda não está clara para todos os profissionais a
importância da notificação como instrumento de gestão para fornecer subsídios a
políticas públicas. Por intermédio das notificações é possível mapear as
ocorrências e as características da violência, o que possibilitaria traçar
intervenções mais efetivas para prevenir e combater tal
agressão\textsuperscript{[}31\textsuperscript{]}.

Essa situação pode estar relacionada tanto à formação dos profissionais de saúde
quanto à sua capacitação para desempenhar suas funções nos serviços públicos de
saúde. Estudo realizado com professores de faculdades de medicina e de
enfermagem\textsuperscript{[}32\textsuperscript{]}
apontou que, embora os docentes reconhecessem a violência como um problema de
saúde, eles tinham dificuldades para inserir o tema em sua prática docente. Ao
analisar os currículos dessas faculdades, verificou-se que o tema da violência
foi explicitado ou apareceu na forma de outro termo correlato em 23\% das
disciplinas dos cursos de medicina e em 16,3\% das disciplinas de enfermagem.
Por outro lado, alguns estudos com profissionais de serviços públicos de saúde
têm apontado que eles não se sentem capacitados a dar um atendimento integral às
mulheres em situação de violência\textsuperscript{[}15\textsuperscript{]}\textsuperscript{,}\textsuperscript{[}33\textsuperscript{]}\textsuperscript{,}\textsuperscript{[}34\textsuperscript{]}, o que, por certo, também pode comprometer seu entendimento acerca da
necessidade da notificação, inclusive da violência contra crianças e
adolescentes\textsuperscript{[}35\textsuperscript{]}. Vale lembrar que a necessidade de capacitação periódica para esse tipo de
atendimento é enfatizada pelo próprio Ministério da
Saúde\textsuperscript{[}36\textsuperscript{]}, e neste estudo apareceu como uma forte sugestão dos participantes. Contudo,
também é preciso considerar que nessa referida dificuldade de preencher a
notificação dos casos de violência pode estar implícita certa resistência dos
profissionais de tratar essa questão como problema de Saúde Pública, e não como
um problema de polícia.

Também é preciso considerar que, muitas vezes, a organização do trabalho nos SSs
apresenta dificuldades de sobrecarga de alguns profissionais, e a exigência de
preencher mais um formulário pode ser considerada excessiva, sobretudo quando
não se percebe a razão e a importância dessa exigência. Nesta pesquisa, se
observou nas duas etapas que, na maioria dos casos, os profissionais de
enfermagem eram os encarregados do preenchimento da notificação. Isso é coerente
com o modelo de trabalho que costumeiramente se observa nos serviços, no qual os
profissionais de enfermagem costumam ser responsáveis pelo preenchimento de
diversos formulários, tarefa que acumulam com a assistência propriamente dita.
Sobretudo nos serviços de emergência, isso pode representar uma grande
dificuldade a ser superada e uma atitude que pode, inclusive, resultar no não
preenchimento ou no preenchimento incompleto da ficha de notificação. Além de
ter sido claramente apontado na abordagem quantitativa, tal comportamento se
evidenciou mais explicitamente na abordagem qualitativa, quando as pessoas se
referiram à extensão da ficha e ao tempo necessário para o seu preenchimento.

Outro aspecto que se evidenciou neste estudo, nas duas etapas, foi a referência
dos entrevistados à dificuldade de obter as informações necessárias para a
notificação de mulheres que sofreram violência sexual. Isso é um fato já
verificado em outros estudos não apenas com relação à violência sexual mas
também a outros tipos de violência sofrida pelas
mulheres\textsuperscript{[}15\textsuperscript{]}\textsuperscript{,}\textsuperscript{[}37\textsuperscript{]}\textsuperscript{,}\textsuperscript{[}38\textsuperscript{]}. É preciso considerar que a mulher que sofreu violência sexual chega ao SS
debaixo de forte impacto e, com frequência, desejando esconder-se e falar o
menos possível. É necessário que os profissionais de saúde saibam como acolher a
mulher que foi agredida, dar espaço às suas queixas e obter com precisão todas
as informações necessárias, para evitar a necessidade de a mulher ficar
repetindo narrativas, por certo, tão
dolorosas\textsuperscript{[}19\textsuperscript{]}. Destaca-se que a ficha de notificação não necessitaria ser preenchida de
imediato, logo após a chegada da mulher ao serviço, justamente no momento em que
muitas vezes ela está emocionalmente fragilizada. Se as informações de
prontuário forem bem preenchidas por todos os profissionais de saúde que atendem
às mulheres, ele será importante fonte complementar para o preenchimento
posterior da ficha.

Não se pode deixar de mencionar também que o acolhimento das mulheres que sofrem
violência, por parte dos profissionais dos serviços de emergência, depende
também da perspectiva desses profissionais sobre o significado da violência
contra a mulher e do impacto que essa violência também pode exercer sobre os
próprios profissionais, que, por vezes, podem estar vivenciando ou já ter
vivenciado situações desse tipo\textsuperscript{[}11\textsuperscript{]}\textsuperscript{,}\textsuperscript{[}39\textsuperscript{]}.

Este estudo apresentou como principais limitações o fato de a amostra não ser
probabilística, baseada inicialmente em estudo prévio feito entre 2005 - 2006 em
SSs que referiam prestar atendimento de emergência a mulheres que sofriam
violência sexual\textsuperscript{[}13\textsuperscript{]}, incluindo outros serviços indicados pelos primeiros. Dessa forma, os
resultados não podem ser generalizados para o Estado de São Paulo. Além disso,
outra limitação se refere ao fato de que não foi possível verificar a qualidade
do preenchimento da ficha de notificação pelos serviços. Assim, não é possível
afirmar se os SSs que referiram preencher sempre a ficha de notificação o faziam
de maneira completa e adequada. Vale lembrar que na etapa qualitativa as pessoas
entrevistadas enfatizaram que a extensão da ficha e a necessidade de incluir
muitos detalhes sobre a violência sofrida pela mulher se constituíam em
dificuldades para o preenchimento.

Em que pesem as limitações acima mencionadas, os resultados tornam evidente que
o preenchimento da notificação compulsória dos casos de violência sexual contra
as mulheres ainda necessita ser mais bem trabalhado com os SSs, tanto para que
os profissionais saibam como fazê-lo quanto para que compreendam o seu valor
como parte do esforço que toda a sociedade deve seguir fazendo para diminuir a
violência contra a mulher. Entretanto, é preciso lembrar que, para que essa
compreensão ocorra, os profissionais necessitam perceber a utilidade das
informações coletadas em termos de sua prática cotidiana; caso contrário, a
ficha de notificação tende a ser vista progressivamente como uma exigência
inútil no que se refere a melhorar a atenção dada às mulheres que sofrem
violência sexual.

\section{\textsc{conclusão}}

Embora a maior parte dos SSs tenha referido preencher a ficha de notificação
compulsória, mencionou também várias dificuldades para fazê-lo, especialmente
relacionadas à sobrecarga de trabalho dos profissionais e à incompreensão acerca
da importância dessa notificação no contexto da atenção integral às mulheres que
sofrem violência sexual. Percebe-se a necessidade de estratégias que melhorem a
formação continuada dos profissionais, para que incorporem a ficha de
notificação como parte do atendimento às mulheres, bem como a devida valorização
desse atendimento dentre as várias atribuições dos SS.

\textsc{agradecimentos}
Os autores agradecem à Fundação de Amparo à Pesquisa do Estado de São Paulo
(\textsc{fapesp}), que financiou totalmente esta pesquisa (processo n\textsuperscript{o}
2010/50345-6).

\section*{\textsc{referências}}
\begin{itemize}

\item[1] Heise L. Gender-based abuse: The global epidemic. Cad Saúde Pública
1994; 10 (Suppl 1): 135-45.

\item[2] Heise L, Pitanguy J, German A. Violence against women: the hidden
health burden. Washington: The International Bank for Reconstruction and
Development/The World Bank; 1994.

\item[3] Garcia-Moreno C, Jansen HA, Ellsberg M, Heise L, Watts CH; \textsc{who}
Multi-country Study on Women's Health and Domestic Violence against Women Study
Team. Prevalence of intimate partner violence: findings from the \textsc{who}
multi-country study on women's health and domestic violence. Lancet 2006;
368(9543): 1260-9.

\item[4] Basile KC, Smith SG. Sexual violence victimization of women:
prevalence, characteristics, and the role of public health. A J Lifestyle Med
2011; 5: 407-17.

\item[5] Guedes A, García-Moreno C, Bott S. Violencia contra las mujeres en
Latinoamérica y el Caribe. Foreign Affairs Latinoamérica 2014; 14 (1): 41-8.
Disponível em: www.fal.itam.mx3. (Acessado em 27 de maio de 2014).

\item[6] Facuri CO, Fernandes \textsc{ams}, Oliveira KD, Andrade TS, Azevedo \textsc{rcs}.
Violência sexual: estudo descritivo sobre as vítimas e o atendimento em um
serviço universitário de referência no Estado de São Paulo, Brasil. Cad Saúde
Pública 2013; 29(5): 889-98.

\item[7] Faúndes A, Hardy E, Osis \textsc{mjd}, Duarte GA. Risco para queixas
ginecológicas e disfunções sexuais segundo história de violência sexual. Rev
Bras Ginecol Obstet 2000; 22(3): 153-7.

\item[8] Reichenheim ME, Moraes CL, Szklo A, Hasselmann MH, Souza ER, Lozana
JA, et al. The magnitude of intimate partner violence in Brazil: portraits from
15 capital cities and the Federal District. Cad Saúde Pública 2006; 22(2):
425-37.

\item[9] Schraiber LB, d'Oliveira \textsc{afpl}, França-Júnior I, Diniz S, Portella
AP, Ludermir AB, et al. Prevalência da violência contra a mulher por parceiro
íntimo em regiões do Brasil. Rev Saúde Pública 2007; 41(5): 797-807.

\item[10] Abrahams N, Devries K, Watts C, Pallitto C, Petzold M, Shamu S, et
al. Worldwide prevalence of non-partner sexual violence: a systematic review.
Lancet 2014; 383(9929): 1648-54.

\item[11] Oliveira EM, Barbosa RM, Moura \textsc{aavm}, Kossel K, Morelli K, Botelho
\textsc{lff}, et al. Atendimento às mulheres vítimas de violência sexual: um estudo
qualitativo. Rev Saúde Pública 2005; 39(3): 376-82.

\item[12] Bruschi A, Paula CS, Bordin \textsc{ias}. Prevalência e procura de ajuda na
violência conjugal física ao longo da vida. Rev Saúde Pública 2006; 40(2):
256-64.

\item[13] Faúndes A, Hardy E, Osis \textsc{mjd}, Makuch MY, Duarte GA, Andalaft Neto
J. Perfil do atendimento à violência sexual no Brasil. Relatório técnico
apresentado ao Ministério da Saúde. Campinas: \textsc{cemicamp}; 2006.

\item[14] Villela W, Lago T. Conquistas e desafios no atendimento às
mulheres que sofreram violência sexual. Cad Saúde Pública 2007; 23(2): 471-5.

\item[15] Osis \textsc{mjd}, Duarte GA, Faúndes A. Violência entre usuárias de
unidades de saúde: prevalência, perspectiva e conduta de gestores e
profissionais. Rev Saúde Pública 2012; 46(2): 351-8.

\item[16] Ministério da Saúde. Guia Prático do Programa da Saúde da Família.
Brasília: Ministério da Saúde; 2001.

\item[17] Ministério da Saúde. Secretaria de Atenção à Saúde. Departamento
de Ações Programáticas Estratégicas. Área Técnica de saúde da Mulher. Política
Nacional de Atenção Integral à Saúde da Mulher. Brasília: Ministério da Saúde;
2004.

\item[18] Ministério da Saúde. Secretaria de Vigilância em Saúde.
Departamento de análise de Situação de Saúde. Instrutivo Notificação de
violência doméstica, sexual e outras violências. Série G - Estatística e
Informação em Saúde. Brasília: Ministério da Saúde. Disponível em: http://portal.saude.gov.br/portal/arquivos/pdf/viva\_{}instrutivo\_{}not\_{}viol\_{}domestica
\_{}sexual\_{}e\_{}out.pdf. (Acessado em 24 de janeiro de 2013).

\item[19] Ministério da Saúde. Secretaria de Atenção à Saúde. Departamento
de Ações Programáticas Estratégicas. Área Técnica de Saúde da Mulher. Atenção
integral para mulheres e adolescentes em situação de violência doméstica e
sexual - Matriz pedagógica para formação de redes. Série B - Textos Básicos de
Saúde. Brasília: Ministério da Saúde; 2011.

\item[20] Andalaft Neto J, Faúndes A, Osis \textsc{mjd}, Pádua KS. Perfil do
atendimento à violência sexual no Brasil. Femina 2012; 40(6): 301-6.

\item[21] Brasil. Lei no 10.778, de 24 de novembro de 2003. Estabelece a
notificação compulsória, no território nacional, de caso de violência contra a
mulher que for atendida em serviços de saúde públicos ou privados. Disponível
em: http://www.planalto.gov.br/ccivil\_{}03/leis/2003/l10.778.htm. (Acessado em
28 de agosto de 2008).

\item[22] Brasil. Decreto no 5.099, de 3 de junho de 2004. Regulamenta a Lei
no. 10.778, de 24 de novembro de 2003, e institui os serviços de referência
sentinela. Disponível em:
http://www.planalto.gov.br/ccivil\_{}03/\_{}ato2004-2006/2004/decreto/d5099.htm.
(Acessado em 28 de agosto de 2008).

\item[23] Brasil. Portaria no 104, de 25 de janeiro de 2011. Define as
terminologias adotadas em legislação nacional, conforme o disposto no
Regulamento Sanitário Internacional 2005 (\textsc{rsi} 2005), a relação de doenças,
agravos e eventos em saúde pública de notificação compulsória em todo o
território nacional e estabelece fluxo, critérios, responsabilidades e
atribuições aos profissionais e serviços de saúde. Disponível em: http://bvsms.s
aude.gov.br/bvs/saudelegis/gm/2011/prt0104\_{}25\_{}01\_{}2011.html. (Acessado
em 8 de maio de 2014).

\item[24] Gomm R, Hammersley M, Foster P, eds. Case study methods. Thousand
Oaks: Sage Publications; 2000.

\item[25] Patton MQ. Qualitative evaluation and research methods. 2nd ed.
London: Sage; 1990.

\item[26] Kish L. Survey sampling. New York: John Wiley and Sons; 1965.

\item[27] Altman DG. Practical statistics for medical research. Boca Raton:
Champman \& Hall/\textsc{crc}; 1999.

\item[28] Minayo \textsc{mcs}. O desafio do conhecimento: Pesquisa qualitativa em
saúde. 5ª ed. São Paulo: Hucitec-Abrasco; 1998.

\item[29] Ziebland S, McPherson A. Making sense of qualitative data
analysis: an introduction with illustrations from \textsc{dipe}x (personal experiences of
health and illness). Med Educ 2006; 40(5): 405-14.

\item[30] Ministério da Saúde. Conselho Nacional de Saúde. Resolução nº
196/96 sobre pesquisa envolvendo seres humanos. Bioética 1996; 4: 15-25.

\item[31] Saliba O, Garbin \textsc{cas}, Garbin \textsc{aji}, Dossi AP. Responsabilidade do
profissional de saúde sobre a notificação de casos de violência doméstica. Rev
Saúde Pública 2007; 41(3): 472-7.

\item[32] Souza ER, Ribeiro AP, Penna \textsc{lhg}, Ferreira AL, Santos NC, Tavares
\textsc{cmm}. O tema violência intrafamiliar na concepção dos formadores dos
profissionais de saúde. Ciênc Saúde Coletiva 2009; 14(5): 1709-19.

\item[33] d'Oliveira \textsc{afpl}, Schraiber LB, Hanada H, Durand J. Atenção
integral à saúde de mulheres em situação de violência de gênero - uma
alternativa para a atenção primária em saúde. Cienc Saúde Coletiva 2009; 14(4):
1037-50.

\item[34] Vieira EM, Perdona \textsc{gcs}, Almeida AM, Nakano \textsc{mas}, Santos MA, Daltoso
D, et al. Conhecimento e atitude dos profissionais de saúde em relação à
violência de gênero. Rev Bras Epidemiol 2009; 12(4): 566-77.

\item[35] Gonçalves HS, Ferreira, AL. A notificação da violência
intrafamiliar contra crianças e adolescentes por profissionais de saúde. Cad
Saúde Pública 2002; 18(1): 315-9.

\item[36] Ministério da Saúde. Secretaria de Atenção à Saúde. Departamento
de Ações Programáticas Estratégicas. Área Técnica de Saúde da Mulher. Prevenção
e tratamento dos agravos resultantes da violência sexual contra mulheres e
adolescentes. Série A - Normas e Manuais Técnicos, Série Direitos Sexuais e
Direitos Reprodutivos, Caderno n° 6. 3ª ed. Brasília: Ministério da Saúde; 2011.

\item[37] Schraiber LB, d'Oliveira \textsc{afpl}. Violence against women and
Brazilian health care policies: a proposal for integrated care in primary care
services. Int J Gynecol Obstet 2002; 78(Suppl. 1): S21-5.

\item[38] Schraiber LB, d'Oliveira \textsc{afpl}, França-Júnior I, Pinho AA.
Violência contra a mulher: estudo em uma unidade de atenção primária à saúde.
Rev Saúde Pública 2002; 36(4): 470-7.

\item[39] Reis MJ, Lopes \textsc{mhbm}, Higa R, Turato ER, Chvatal \textsc{vls}, Bedone AJ.
Vivências de enfermeiros na assistência à mulher que sofre violência sexual. Rev
Saúde Pública 2010; 44(2): 325-31.

\end{itemize}

\end{document}
