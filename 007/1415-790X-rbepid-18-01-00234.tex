% Generated by jats2tex@0.11.1.0
\documentclass{article}
\usepackage{scielo}

\newcommand{\journalid}{Rev Bras Epidemiol}
\newcommand{\publisherid}{rbepid}
\newcommand{\journaltitle}{Revista Brasileira de Epidemiologia}
\newcommand{\abbrevjournaltitle}{Rev. bras. epidemiol.}
\newcommand{\issnppub}{1415-790X}
\newcommand{\issnepub}{1980-5497}
\newcommand{\publishername}{Associação Brasileira de Saúde Coletiva}
\newcommand\articledoi{\textsc{doi} 10.1590/1980-5497201500010018}
\def\subject{Artigos Originais}
\title{Prevalência de dor crônica e sua associação com a situação
sociodemográfica e atividade física no lazer em idosos de Florianópolis, Santa
Catarina: estudo de base populacional\titlegroup{}}
\author[{I}]{Santos, Franco Andrius Ache dos}
\author[{I}]{Souza, Juliana Barcellos de}
\author[{I}]{Antes, Danielle Ledur}
\author[{I}]{d'Orsi, Eleonora}
\affil[i]{Universidade Federal de Santa Catarina}
\def\authornotes{Autor correspondente: Franco Andrius Ache dos Santos. Centro de
Ciências da Saúde da Universidade Federal de Santa Catarina. Campus
Universitário Reitor João David Ferreira Lima, Trindade, \textsc{cep} 88040-970,
Florianópolis, SC, Brasil. E-mail: francoache@hotmail.com
Conflito de interesses: nada a delcarar}
\date{ 2015}
\def\volume{18}
\def\issue{1}
\def\fpage{234}
\def\lpage{247}
\newcommand{\kwdgroup}{Prevalência, Dor crônica, Fatores socioeconômicos,
Atividade motora, Idoso, Estudos transversais}
%%% Nota %%%%%%%%%%%%%%%%%%%%%%%%%%%%%%%%%%%%%%%%%%%%%%%%%%%%%%%%
\expandafter\newcommand\csname \endcsname{
Fonte de financiamento: Conselho Nacional de Desenvolvimento Científico e
Tecnológico (\textsc{cnp}q), Processo n. 569834/2008-2.}

\begin{document}
\selectlanguage{portuges}
\newcommand{\lingua}{Português}
\maketitle
\tableofcontents

This is an Open Access article distributed under the terms of the Creative
Commons Attribution Non-Commercial License, which permits unrestricted
non-commercial use, distribution, and reproduction in any medium, provided the
original work is properly cited.
\begingroup
\renewcommand{\section}[1]{\subsection*{#1}}

\begin{abstract}
\section{\textsc{objetivo}:}

Estimar a prevalência de dor crônica e sua associação com a situação
socioeconômica, demográfica e atividade física no lazer em idosos.

\section{\textsc{métodos}:}

Este estudo é parte do inquérito epidemiológico e transversal de base
populacional e domiciliar EpiFloripa Idoso 2009-2010 realizado com 1.705 idosos
(≥ 60 anos), residentes em Florianópolis, Santa Catarina. A partir da resposta
afirmativa de dor crônica, foram investigadas as associações com as variáveis
obtidas por meio de entrevista estruturada. Realizou-se a estatística
descritiva, incluindo cálculos de proporções e intervalos de confiança 95\%
(IC95\%). Na análise bruta e ajustada, empregou-se regressão de Poisson,
estimando-se as razões de prevalência, com intervalos de confiança de 95\% e
valores p ≤ 0,05.

\section{\textsc{resultados}:}

Dentre os idosos investigados, 29,3\% (IC95\% 26,5 - 32,2) relataram dor
crônica. Na análise ajustada, observou-se que as variáveis sexo feminino, menor
escolaridade e pior situação econômica ficaram associadas significativamente com
maior prevalência de dor crônica; ser fisicamente ativo no lazer ficou associado
significativamente com menor prevalência do desfecho.

\section{\textsc{conclusões}:}

Percebe-se que a dor crônica é um agravo que acomete considerável parcela de
idosos, havendo desigualdades sociais na sua frequência e sendo beneficamente
afetada pela atividade física no lazer. É necessário que políticas públicas de
saúde subsidiem programas multidisciplinares de controle da dor incluindo a
prática regular de atividade física, voltada especificamente à promoção da saúde
do idoso, evitando assim que a dor crônica comprometa a qualidade de vida desta
população.

\ifdef{\kwdgroup}{\iflanguage{portuges}{\medskip\noindent\textbf{Palavras-chave:} \kwdgroup}{}}{}
\ifdef{\kwdgroupen}{\iflanguage{english}{\medskip\noindent\textbf{Keywords:}
\kwdgroupen}{}}{}
\ifdef{\kwdgroupes}{\iflanguage{spanish}{\medskip\noindent\textbf{Palavras
claves:} \kwdgroupes}{}}{}
\ifdef{\kwdgroupfr}{\iflanguage{french}{\medskip\noindent\textbf{Mots clés:}
\kwdgroupfr}{}}{}
\end{abstract}
\endgroup
Conselho Nacional de Desenvolvimento Científico e
Tecnológico569834/2008-2\section{\textsc{introdução}}

A população mundial vem passando por um acelerado e gradual processo de
envelhecimento\textsuperscript{[}1\textsuperscript{]}. Acompanhando essa tendência, no Brasil, o envelhecimento populacional é uma
realidade. Segundo dados do último censo demográfico realizado pelo Instituto
Brasileiro de Geografia e Estatística (\textsc{ibge}) em 2010, o número de idosos (60
anos ou mais) representava 11,3\% da população
brasileira\textsuperscript{[}2\textsuperscript{]}.

Essa mudança na estrutura etária brasileira está diretamente relacionada à
transição epidemiológica, pois, à medida que a população envelhece, maior é a
prevalência de problemas crônicos de saúde. Entre as consequências que a
transição demográfica e a longevidade trazem para a sociedade, a dor é uma das
mais significativas; em muitos casos, a dor crônica é a principal queixa dos
indivíduos, interferindo consideravelmente na qualidade de vida dos
idosos\textsuperscript{[}3\textsuperscript{]}.

Em um contexto temporal, a dor pode ser classificada como aguda ou crônica. A
dor aguda está associada à lesão do organismo, é de curta duração e desaparece
com a cicatrização da lesão. A dor crônica, por sua vez, é persistente ou
recorrente e não está necessariamente associada a uma lesão no
organismo\textsuperscript{[}4\textsuperscript{]}. É considerada um evento complexo, de natureza biopsicossocial, que se
configura em problema de saúde coletiva e exige abordagem
multidisciplinar\textsuperscript{[}5\textsuperscript{]}.

A prevalência de dor crônica em estudos envolvendo idosos é bastante
diversificada, dependendo das características da população em estudo e da
metodologia utilizada. Em estudos internacionais, a prevalência de dor crônica
varia entre 28,9 e 59,3\%\textsuperscript{[}6\textsuperscript{]}\textsuperscript{-}\textsuperscript{[}8\textsuperscript{]}, enquanto no Brasil a prevalência varia entre 29,7 e
62,2\%\textsuperscript{[}3\textsuperscript{]}\textsuperscript{,}\textsuperscript{[}9\textsuperscript{]}\textsuperscript{-}\textsuperscript{[}10\textsuperscript{]}. Estudos transversais sugerem que o aumento da dor crônica está associado
principalmente com o sexo feminino, a idade avançada e o baixo nível
socioeconômico\textsuperscript{[}11\textsuperscript{]}. Menor prevalência de dor crônica tem sido associada a ter trabalho
remunerado\textsuperscript{[}3\textsuperscript{]}, níveis elevados de escolaridade e condição socioeconômica, bem como à prática
regular de atividade física\textsuperscript{[}12\textsuperscript{]}.

Apesar da considerável interferência negativa da dor crônica na qualidade de
vida dos idosos, configurando-se assim em um problema de saúde coletiva, em
nosso país, poucos estudos epidemiológicos de base populacional têm se dedicado
a esse assunto. Em Santa Catarina, este estudo é pioneiro sobre essa temática,
servindo de base para que outros estudos possam surgir, contribuindo com a
difusão desse conhecimento. Assim, o objetivo do presente estudo foi estimar a
prevalência de dor crônica e a sua associação com a situação socioeconômica e
demográfica e o nível de atividade física no lazer da população idosa de
Florianópolis, Santa Catarina.

\section{\textsc{métodos}}

Este é um estudo epidemiológico transversal, realizado com os dados do projeto
"Condições de Saúde da População Idosa do Município de Florianópolis, Santa
Catarina: Estudo de Base Populacional" (EpiFloripa Idoso 2009-2010).

O estudo foi desenvolvido na zona urbana do município de Florianópolis, capital
do estado de Santa Catarina. Florianópolis está localizada no centro-leste do
estado, grande parte do município (97,23\%) está situada na Ilha de Santa
Catarina. Segundo a Organização das Nações Unidas, o município apresentou um
índice de desenvolvimento humano municipal (\textsc{idh}-M) de 0,847 em 2010, colocando o
município na terceira posição dentre todos os municípios
brasileiros\textsuperscript{[}1\textsuperscript{]}.

De acordo com dados do último censo demográfico realizado pelo \textsc{ibge}, o município
apresenta esperança de vida ao nascer de 79,1 anos e taxa de fecundidade total
de 1,4 filhos por mulher\textsuperscript{[}2\textsuperscript{]}. A população estimada para Florianópolis em 2009 foi de 408.163 habitantes,
sendo 44.460 pertencentes à faixa etária com idade igual ou superior a 60 anos
(18.844 do sexo masculino e 25.616 do sexo feminino), representando, dessa
forma, 10,9\% da população total\textsuperscript{[}13\textsuperscript{]}. A população do estudo foi constituída por idosos de ambos os sexos, com 60
anos ou mais de idade, completos no ano da pesquisa, residentes na zona urbana
do município de Florianópolis, Santa Catarina. Para o cálculo do tamanho da
amostra, foram utilizados os seguintes parâmetros: a população foi igual a
44.460 idosos, prevalência para o desfecho desconhecido (50\%), erro amostral
igual a 4 pontos percentuais e intervalo de confiança de 95\% (IC95\%). O
tamanho da amostra obtido foi multiplicado por 2 em razão do efeito de
delineamento do estudo (\textit{deff}
), acrescido ainda de 20\% de perdas previstas e 15\% para estudos de
associação, totalizando 1.599 indivíduos.

Para a presente investigação, o cálculo da amostra foi realizado \textit{a
posteriori}, considerando-se uma prevalência de dor crônica em idosos de
51,4\%\textsuperscript{[}3\textsuperscript{]}, com 4 pontos percentuais de margem de erro, IC95\% efeito de delineamento de
2, acrescido de 20\% para eventuais perdas e 15\% para estudos de associação,
resultando em amostra mínima de 1.029 indivíduos. Como este estudo foi parte do
EpiFloripa Idoso, utilizou-se a maior amostra calculada. A seleção da amostra
foi realizada por conglomerados em dois estágios. No primeiro estágio, os 420
setores censitários urbanos do município foram colocados em ordem crescente
conforme a renda média mensal do chefe da família, sorteando-se sistematicamente
80 desses setores (8 setores em cada decil de renda).

As unidades de segundo estágio foram os domicílios. Uma etapa de atualização do
número de domicílios em cada setor (arrolamento) fez-se necessária uma vez que o
Censo mais recente havia sido realizado em 2000. Supervisores do estudo
percorreram os setores censitários sorteados e procederam à contagem de todos os
domicílios habitados, obedecendo a normas do \textsc{ibge}. O número de domicílios variou
de 61 a 725. A fim de diminuir o coeficiente de variação do número de domicílios
por setor, foi realizado o agrupamento de setores com menos de 150 domicílios e
divisão dos setores com mais de 500, respeitando o decil de renda
correspondente, originando 83 setores censitários. O coeficiente de variação
inicial era de 52,7\% (n = 80 setores) e o final foi de 35,5\% (n = 83 setores).

De acordo com dados do \textsc{ibge}\textsuperscript{[}14\textsuperscript{]}, estimou-se que deveriam ser visitados cerca de 60 domicílios por setor para se
encontrar 20 idosos. Os domicílios foram sorteados de forma sistemática e todos
os idosos residentes nestes foram convidados a participar da pesquisa. Em
virtude da disponibilidade de recursos financeiros, estimou-se realizar 23
entrevistas por setor censitário, permitindo maior variabilidade da amostra e
obtendo-se, dessa forma, 1.911 idosos elegíveis para o estudo. A taxa de não
resposta do estudo foi de 10,9\%, o que originou uma amostra final de 1.705
idosos efetivamente entrevistados. Idosos institucionalizados foram excluídos
deste estudo.

Consideraram-se perdas as entrevistas não realizadas após quatro tentativas
(sendo pelo menos uma no período noturno e uma no final de semana). As recusas
foram os sujeitos que negaram responder o questionário por opção pessoal, não
sendo admitidas substituições. A coleta de dados foi realizada por
entrevistadoras devidamente treinadas, por meio de questionário estruturado com
questões pré-codificadas aplicadas na forma de entrevista face a face,
utilizando-se o \textit{Personal Digital Assistants}
(\textsc{pda}), após realização de pré-teste e estudo-piloto em setores não amostrados
para a pesquisa. Houve verificação semanal da consistência dos dados e controle
de qualidade por meio de aplicação por telefone de um questionário reduzido em
10\% das entrevistas selecionadas aleatoriamente.

A variável desfecho ou dependente deste estudo foi a prevalência de dor crônica
na população idosa de Florianópolis, Santa Catarina. Para tal, foi utilizado o
questionário estruturado sobre dor crônica contendo cinco
questões\textsuperscript{[}15\textsuperscript{]}, considerando-se dor crônica aquela com duração igual ou superior a seis meses,
de caráter contínuo ou recorrente, conforme preconizado pela
\textit{International Association for the Study ofPain}
(\textsc{iasp})\textsuperscript{[}16\textsuperscript{]}. Os idosos foram indagados quanto a sentir dor na maioria dos dias (sim/não);
há quanto tempo (< 3 meses, entre 3 e 6 meses e > 6 meses); se no último mês ele
sentiu dores em várias partes do corpo, como, por exemplo, costas, pernas,
braços, pescoço ou cabeça (sim/não); se a dor durou mais de 15 dias (sim/não);
e, ainda, mediante uma escala de dor, na qual o valor zero representava ausência
de dor e cem a dor máxima suportável, como ele avaliava a dor na última semana.
Tendo em vista a subjetividade da questão sobre dor crônica, todas as
entrevistas respondidas por informante ou cuidador foram excluídas da análise.

As variáveis de controle ou independentes incluídas foram: sexo; faixa etária
(60-69; 70-79; 80 anos ou mais); situação conjugal (casado/companheiro,
solteiro/divorciado, viúvo); escolaridade (0 - 4; 5 - 8; 9 - 11; 12 anos ou
mais); renda familiar em quartis (1° quartil: ≤ R\$ 327,50, 2° quartil: R\$
327,50 a R\$ 700,00, 3° quartil: R\$ 700,00 a R\$ 1.500,00, 4° quartil: > R\$
1.500,00); se exercia trabalho remunerado (sim/não); e se a situação econômica
comparada aos 50 anos de idade piorou, permaneceu a mesma ou melhorou. O nível
de atividade física no lazer foi mensurado pela versão longa do
\textit{Questionário Internacional de Atividades Físicas}
(\textsc{ipaq}) adaptado e validado para idosos do
Brasil\textsuperscript{[}17\textsuperscript{]}. Idosos que praticavam pelo menos 150 minutos por semana ou mais de atividade
física no lazer, foram classificados como fisicamente ativos, e os que
praticaram menos de 150 minutos de atividade física no lazer, foram
classificados como insuficientemente ativos neste
domínio\textsuperscript{[}18\textsuperscript{]}.

Todas as variáveis do estudo foram analisadas de forma descritiva por meio de
frequência absoluta e relativa. A estatística descritiva incluiu cálculos de
proporções e IC95\% para variáveis categóricas. Para testar a associação entre o
desfecho (prevalência de dor crônica) e as variáveis de controle
(socioeconômicas e demográficas e atividade física no lazer), foram realizadas
análise bruta e ajustada por meio de regressão de Poisson, estimando-se as
razões de prevalência brutas e ajustadas, com IC95\% e valor p (obtido por meio
do teste de Wald)\textsuperscript{[}19\textsuperscript{]}.\textsuperscript{.}
Foram selecionadas para entrarem no modelo ajustado as variáveis que
apresentaram valores de p ≤ 0,05 na análise bruta, permanecendo no modelo se
atingissem valores de p ≤ 0,05 e/ou ajustassem o modelo. Utilizou-se o pacote
estatístico Stata 9.0 (Stata Corp., College Station, Estados Unidos),
considerando-se para todas as análises o efeito do desenho amostral por meio do
comando svy, projetado para análise de dados provenientes de amostras complexas.

O projeto foi aprovado no Comitê de Ética em Pesquisa da Universidade Federal de
Santa Catarina, sob protocolo de n° 352/2008 em 23 de dezembro de 2008. Os
sujeitos foram informados sobre os objetivos do estudo, e foi solicitada a
assinatura do termo de consentimento livre e esclarecido.

\section{\textsc{resultados}}

Da amostra original de 1.911 idosos elegíveis para o estudo, 1.705 indivíduos
(89,2\%) foram entrevistados, correspondendo a uma taxa de não resposta de
10,8\% (n = 206) variando entre 8,5\% no primeiro decil de renda e 22,0\% no
decil mais elevado. Os principais motivos de perdas (n = 206) foram: "não tinha
ninguém em casa", "o idoso estava viajando", "marcou com a entrevistadora e não
compareceu", "estava de férias", "estava muito doente", "tinha cachorro bravo no
terreno". Houve 3 perdas por motivo de hospitalização do idoso no momento da
entrevista, não afetando os resultados. Os principais motivos de recusas foram:
"não quis dar entrevista", "entrevista muito longa", "sem tempo para responder a
entrevista", "acha perda de tempo responder entrevistas", "não acredita em
pesquisas".

Dentre os 1.705 idosos investigados, foram excluídas da análise 49 entrevistas
respondidas por informante/cuidadores, totalizando, portanto, 1.656 idosos, dos
quais 29,3\% (IC95\% 26,5 - 32,2) apresentaram dor crônica. Na amostra do
estudo, houve predominância de mulheres (62,5\%), de idosos na faixa etária de
60 a 69 anos (51,7\%), casados ou vivendo com companheiro (58,9\%) e com baixa
escolaridade (40,0\%). A maioria dos idosos respondeu não trabalhar (86,5\%) e
48,6\% mencionaram que a situação econômica melhorou comparada aos 50 anos de
idade. Com relação ao nível de atividade física no lazer, 68,5\% dos idosos
foram classificados como insuficientemente ativos no lazer (Tabela 1).

Tabela 1.Descrição das variáveis socioeconômicas e demográficas, atividade
física no lazer e presença de dor crônica. Florianópolis, SC. EpiFloripa Idoso
2009 - 2010.\begin{table}
%\begin{adjustbox}{width=1.1\textwidth}
\small\centering
\begin{tabulary}{\linewidth}{ C C C C }
\hline
Variáveis & n & \% & IC95\%\\ \hline
\multicolumn{4}{l}{Sexo (n = 1.656)}
\\ \hline

Masculino
& 598
& 37,5
& 34,7 – 40,3
\\ \hline

Feminino
& 1.058
& 62,5
& 59,7 – 65,3
\\ \hline

\multicolumn{4}{l}{Faixa etária (anos) (n = 1.656)}
\\ \hline

60 a 69
& 846
& 51,7
& 48,8 – 54,7
\\ \hline

70 a 79
& 596
& 35,5
& 32,5 – 38,4
\\ \hline

≥ 80
& 214
& 12,8
& 10,3 – 15,3
\\ \hline

\multicolumn{4}{l}{Situação conjugal (n = 1.656)}
\\ \hline

Casado/com companheiro
& 974
& 58,9
& 55,5 – 62,4
\\ \hline

Solteiro/divorciado
& 225
& 13,7
& 11,7 – 15,7
\\ \hline

Viúvo
& 457
& 27,4
& 24,5 – 30,3
\\ \hline

\multicolumn{4}{l}{Renda em quartis (n = 1.656)}
\\ \hline

> R\$ 1.500,00
& 410
& 26,8
& 21,7 – 32,0
\\ \hline

R\$ 700,00 a R\$ 1.500,00
& 414
& 25,4
& 22,4 – 28,4
\\ \hline

R\$ 327,50 a R\$ 700,00
& 418
& 25,2
& 21,7 – 28,8
\\ \hline

≤ R\$ 327,50
& 414
& 22,6
& 18,6 – 26,4
\\ \hline

\multicolumn{4}{l}{Escolaridade (anos) (n = 1.648)}
\\ \hline

≥ 12
& 386
& 25,2
& 20,6 – 29,8
\\ \hline

9 a 11
& 231
& 16,2
& 12,6 – 19,8
\\ \hline

5 a 8
& 315
& 18,6
& 16,0 – 21,3
\\ \hline

0 a 4
& 716
& 40,0
& 33,9 – 46,0
\\ \hline

\multicolumn{4}{l}{Trabalho remunerado (n = 1.656)}
\\ \hline

Não
& 1.429
& 86,5
& 84,1 – 88,8
\\ \hline

Sim
& 227
& 13,5
& 11,2 – 15,8
\\ \hline

\multicolumn{4}{l}{Situação econômica comparada aos 50 anos (n = 1.654)}
\\ \hline

Melhor
& 775
& 48,6
& 44,3 – 53,0
\\ \hline

A mesma
& 460
& 25,9
& 22,5 – 29,1
\\ \hline

Pior
& 419
& 25,5
& 22,5 – 28,5
\\ \hline

\multicolumn{4}{l}{Atividade física no lazer (n = 1.656)}
\\ \hline

Insuficientemente ativo
& 1.165
& 68,5
& 63,5 – 73,6
\\ \hline

Fisicamente ativo
& 491
& 31,5
& 26,4 – 36,5
\\ \hline

\multicolumn{4}{l}{Dor crônica (n = 1.656)}
\\ \hline

Sim
& 497
& 29,3
& 26,5 – 32,2
\\ \hline

Não
& 1.159
& 70,7
& 67,8 – 73,5
\\ \hline

\end{tabulary}
%\end{adjustbox}
\caption*{\footnotesize }
\caption{}
\end{table}

Na análise bruta, associaram-se ao desfecho ser do sexo feminino, ter renda
entre 700 e 1.500 reais, menor escolaridade, exercer trabalho remunerado,
referir situação econômica pior comparada aos 50 anos de idade e ser fisicamente
ativos no lazer. Não esteve associada significativamente à dor crônica a
variável faixa-etária e situação conjugal (Tabela 2). A variável faixa-etária,
mesmo não estando associada significativamente ao desfecho na análise bruta, foi
mantida no modelo final de análise pelo fato de ser uma importante variável de
confundimento.

Tabela 2.Análise bruta da prevalência de dor crônica associada às variáveis
socioeconômicas e demográficas e atividade física no lazer. Florianópolis, SC.
EpiFloripa Idoso 2009 - 2010.\begin{table}
%\begin{adjustbox}{width=1.1\textwidth}
\small\centering
\begin{tabulary}{\linewidth}{ C C C C }
\hline
Variáveis & n (\%) & RP bruta (IC95\%) & Valor p\\ \hline
\multicolumn{3}{l}{Sexo (n = 1.656)}
& < 0,001*
\\ \hline

Masculino
& 106 (17,7)
& 1,00
&
\\ \hline

Feminino
& 391 (37,0)
& 1,99 (1,59 – 2,48)
&
\\ \hline

\multicolumn{3}{l}{Faixa etária (anos) (n = 1.656)}
& 0,212
\\ \hline

60 a 69
& 243 (28,7)
& 1,00
&
\\ \hline

70 a 79
& 188 (31,5)
& 1,19 (0,99 – 1,44)
&
\\ \hline

≥ 80
& 66 (30,8)
& 1,07 (0,82 – 1,40)
&
\\ \hline

\multicolumn{3}{l}{Situação conjugal (n = 1.656)}
& 0,816
\\ \hline

Casado/com companheiro
& 291 (29,9)
& 1,00
&
\\ \hline

Solteiro/divorciado
& 61 (27,1)
& 0,90 (0,71 – 1,15)
&
\\ \hline

Viúvo
& 145 (31,7)
& 1,04 (0,83 – 1,30)
&
\\ \hline

\multicolumn{3}{l}{Renda em quartis (n = 1.656)}
& 0,001*
\\ \hline

> R\$ 1.500,00
& 137 (33,1)
& 1,00
&
\\ \hline

R\$ 700,00 a R\$ 1.500,00
& 145 (34,7)
& 0,64 (0,48 – 0,85)
&
\\ \hline

R\$ 327,50 a R\$ 700,00
& 125 (30,2)
& 0,88 (0,70 – 1,11)
&
\\ \hline

≤ R\$ 327,50
& 90 (21,9)
& 1,13 (0,89 – 1,43)
&
\\ \hline

\multicolumn{3}{l}{Escolaridade (anos) (n = 1.648)}
& < 0,001*
\\ \hline

≥ 12
& 82 (21,2)
& 1,00
&
\\ \hline

9 a 11
& 58 (25,1)
& 1,16 (0,80 – 1,69)
&
\\ \hline

5 a 8
& 91 (28,9)
& 1,37 (1,01 – 1,86)
&
\\ \hline

0 a 4
& 264 (36,9)
& 1,83 (1,40 – 2,38)
&
\\ \hline

\multicolumn{3}{l}{Trabalho remunerado (n = 1.656)}
& 0,040*
\\ \hline

Não
& 444 (31,1)
& 1,00
&
\\ \hline

Sim
& 53 (23,3)
& 0,74 (0,56 – 0,99)
&
\\ \hline

\multicolumn{3}{l}{Situação econômica comparada aos 50 anos (n = 1.654)}
& 0,039*
\\ \hline

Melhor
& 227 (29,3)
& 1,00
&
\\ \hline

A mesma
& 125 (27,2)
& 0,92 (0,74 – 1,14)
&
\\ \hline

Pior
& 144 (34,4)
& 1,21 (1,03 – 1,42)
&
\\ \hline

\multicolumn{3}{l}{Atividade física no lazer (n = 1.656)}
& < 0,001*
\\ \hline

Insuficientemente ativo
& 386 (33,1)
& 1,00
&
\\ \hline

Fisicamente ativo
& 111 (22,6)
& 0,69 (0,55 – 0,85)
&
\\ \hline

\end{tabulary}
%\end{adjustbox}
\caption*{\footnotesize }
\caption{}
\end{table}
*
Valores estatisticamente significantes (p = 0,05); RP: Razão de prevalência.

Valores estatisticamente significantes (p = 0,05); RP: Razão de prevalência.

Na análise ajustada, apenas a variável sexo, escolaridade, situação econômica e
atividade física no lazer mantiveram-se associadas ao desfecho até o final da
análise.

As mulheres apresentaram prevalência 82\% maior de dor crônica em relação aos
homens (RP = 1,82; IC95\%1,45 - 2,29), e idosos com escolaridade entre 0 e 4
anos têm maior prevalência do desfecho em relação aos idosos com 12 ou mais anos
de estudo (RP = 1,43; IC95\% 1,10 - 1,85). Indivíduos com relato de pior
situação econômica comparada aos 50 anos de idade apresentaram prevalência 26\%
maior de dor em relação aos idosos que melhoraram a sua situação econômica (RP =
1,26; IC95\% 1,08 - 1,49). Ser fisicamente ativo no lazer apresentou menor
prevalência de dor crônica quando comparados aos idosos insuficientemente ativos
(RP = 0,80; IC95\% 0,65 - 0,99) (Tabela 3).

Tabela 3.Análise ajustada da prevalência de dor crônica associada às variáveis
socioeconômicas e demográficas e atividade física no lazer. Florianópolis, SC.
EpiFloripa Idoso 2009 - 2010.\begin{table}
%\begin{adjustbox}{width=1.1\textwidth}
\small\centering
\begin{tabulary}{\linewidth}{ C C C }
\hline
Variáveis & RP ajustada (IC95\%) & Valor p\\ \hline
\multicolumn{2}{l}{Sexo (n = 1.656)}
& < 0,001*
\\ \hline

Masculino
& 1,00
&
\\ \hline

Feminino
& 1,82 (1,45 – 2,29)
&
\\ \hline

\multicolumn{2}{l}{Faixa etária (anos) (n = 1.656)}
& 0,947
\\ \hline

60 a 69
& 1,00
&
\\ \hline

70 a 79
& 1,10 (0,90 – 1,34)
&
\\ \hline

≥ 80
& 0,95 (0,72 – 1,25)
&
\\ \hline

\multicolumn{2}{l}{Renda em quartis (n = 1.656)}
& 0,412
\\ \hline

> R\$ 1.500,00
& 1,00
&
\\ \hline

R\$ 700,00 a R\$ 1.500,00
& 0,90 (0,67 – 1,20)
&
\\ \hline

R\$ 327,50 a R\$ 700,00
& 1,08 (0,84 – 1,40)
&
\\ \hline

≤ R\$ 327,50
& 1,19 (0,93 – 1,52)
&
\\ \hline

\multicolumn{2}{l}{Escolaridade (anos) (n = 1.656)}
& 0,001*
\\ \hline

≥ 12
& 1,00
&
\\ \hline

9 a 11
& 0,95 (0,67 – 1,36)
&
\\ \hline

5 a 8
& 1,08 (0,80 – 1,45)
&
\\ \hline

0 a 4
& 1,43 (1,10 – 1,85)
&
\\ \hline

\multicolumn{2}{l}{Trabalho remunerado (n = 1.656)}
& 0,546
\\ \hline

Não
& 1,00
&
\\ \hline

Sim
& 0,92 (069 – 1,21)
&
\\ \hline

\multicolumn{2}{l}{Situação econômica comparada aos 50 anos (n = 1.654)}
& 0,012*
\\ \hline

Melhor
& 1,00
&
\\ \hline

A mesma
& 0,95 (0,77 – 1,18)
&
\\ \hline

Pior
& 1,26 (1,08 – 1,49)
&
\\ \hline

\multicolumn{2}{l}{Atividade física no lazer (n = 1.656) }
& 0,047*
\\ \hline

Insuficientemente ativo
& 1,00
&
\\ \hline

Fisicamente ativo
& 0,80 (0,65 – 0,99)
&
\\ \hline

\end{tabulary}
%\end{adjustbox}
\caption*{\footnotesize }
\caption{}
\end{table}
*
Valores estatisticamente significantes (p = 0,05); Teste de Wald: p < 0,001; RP:
Razão de prevalência.

Valores estatisticamente significantes (p = 0,05); Teste de Wald: p < 0,001; RP:
Razão de prevalência.

\section{\textsc{discussão}}

Os principais achados do presente estudo mostram importantes associações entre a
prevalência de dor crônica e a situação socioeconômica e demográfica e o nível
de atividade física no lazer da população idosa do município de Florianópolis. A
maior prevalência de dor crônica ficou associada significativamente com ser do
sexo feminino, ter menor escolaridade e pior situação econômica. Por outro lado,
atividade física no lazer associou-se com menor prevalência do desfecho.

Nesta investigação, a prevalência de dor crônica na população idosa do município
de Florianópolis foi de 29,3\%. Dellaroza et
al.\textsuperscript{[}3\textsuperscript{]}, que estudaram 529 idosos servidores municipais aposentados e em atividade de
Londrina, Paraná, os quais relataram dor há pelo menos seis meses, observaram
prevalência de dor crônica em 51,4\% da população estudada. O fato de esse
estudo ter sido realizado com uma população de conveniência, apenas servidores
municipais, pode ter contribuído para que a prevalência de dor crônica observada
tenha sido bem maior do que a encontrada no presente estudo.

Em estudo transversal realizado com 219 idosos na cidade de Taipei em Taiwan, Yu
et al.\textsuperscript{[}7\textsuperscript{]}
encontraram uma prevalência de dor crônica de 42,0\%. Em outra investigação
transversal de base populacional realizada na Colômbia na população em geral, a
prevalência de dor crônica encontrada em indivíduos acima de 65 anos foi de
43,8\%\textsuperscript{[}20\textsuperscript{]}. Dellaroza et al.\textsuperscript{[}9\textsuperscript{]}
entrevistaram 172 idosos residentes na área de abrangência de uma Unidade Básica
de Saúde localizada na zona norte da cidade de Londrina, Paraná, com dor há pelo
menos seis meses e com queixas frequentes de dor; a prevalência de dor crônica
encontrada no estudo foi de 62,2\%. Nesse estudo em específico, observa-se que a
amostra (n = 172) pode ser muito pequena e pouco representativa, uma vez que
esses idosos correspondem apenas aos residentes que utilizam uma unidade básica
de saúde de uma localidade específica do município de Londrina, não
representando assim o total da população do município. Ressalta-se ainda que os
idosos selecionados para o estudo já apresentavam queixas de dor, o que também
pode ter contribuído para o elevado percentual de dor crônica encontrada.

Dellaroza et al.\textsuperscript{[}10\textsuperscript{]}
realizaram um estudo transversal de base populacional com 1.271 idosos na cidade
de São Paulo, São Paulo, e observaram uma prevalência de dor crônica em 29,7\%
da população estudada, prevalência essa que se aproxima do valor de 29,3\%,
corroborando com o encontrado no presente estudo.

Nesta pesquisa, evidenciou-se que ser do sexo feminino ficou significativamente
associado a maior prevalência de dor crônica, consoante com o que foi encontrado
na literatura. A prevalência de dor crônica na população em geral tem sido maior
em mulheres, comparativamente aos homens\textsuperscript{[}21\textsuperscript{]}\textsuperscript{-}\textsuperscript{[}24\textsuperscript{]}. Em estudos internacionais, realizados na Espanha, França e
Colômbia\textsuperscript{[}8\textsuperscript{]}\textsuperscript{,}\textsuperscript{[}20\textsuperscript{]}\textsuperscript{,}\textsuperscript{[}25\textsuperscript{]}, a prevalência de dor crônica foi igualmente maior em mulheres. No estudo de
Dellaroza et al.\textsuperscript{[}3\textsuperscript{]}, das variáveis sociodemográficas analisadas, somente o sexo associou-se à
presença de dor crônica, mais frequente em mulheres. Em estudo realizado por
Leveille et al.\textsuperscript{[}26\textsuperscript{]}, os autores encontraram uma prevalência maior de dor musculoesquelética em
mulheres. Em geral, as mulheres idosas têm maior prevalência de dor crônica
comparativamente aos homens idosos\textsuperscript{[}7\textsuperscript{]}, o que vai ao encontro do observado no presente estudo.

Mulheres podem perceber o evento da dor com maior seriedade, pois as múltiplas
responsabilidades e papéis, resultantes de cuidados com parentes e administração
do lar, são razões para elas considerarem a dor ameaçadora. Além disso, o
significado da dor para homens e mulheres pode ser influenciado por normas
sociais e culturais que permitem à mulher a expressão ou manifestação de dor
enquanto encorajam os homens a desconsiderá-la. Esses fatores também devem ser
considerados como contribuintes para a maior queixa de dor entre o sexo
feminino\textsuperscript{[}21\textsuperscript{]}.

No presente estudo, idosos com menor nível de escolaridade apresentaram maiores
percentuais de dor crônica quando comparado aos demais níveis. Em estudo
realizado por Dellaroza et al.\textsuperscript{[}3\textsuperscript{]}, os autores encontraram que idosos que tinham entre dois e quatro anos de
estudo apresentaram maiores percentuais de dor crônica. Em outro estudo sobre a
prevalência de dor crônica na população de Salvador, indivíduos com nível de
escolaridade mais baixo apresentaram maiores percentuais de dor crônica quando
comparados aos níveis médio e alto\textsuperscript{[}22\textsuperscript{]}. Assim, baixa escolaridade sugere estar relacionada a elevados percentuais de
dor crônica entre os indivíduos.

Este resultado é expressivo, pois reflete as condições sociais do início do
século passado, demonstrando que o acesso à educação era restrito. A
possibilidade educacional há mais de meio século era muito baixa, e as pessoas
precisavam trabalhar para auxiliar no sustento da família. Tendo em vista que o
nível de escolaridade influencia sobremaneira no acesso à informação, este pode
ser determinante para a busca de tratamento, assim como é decisivo no
autocuidado, pois o idoso deve ser capaz de cuidar de si mesmo, e saber ler é
fator contributivo\textsuperscript{[}27\textsuperscript{]}.

Nesta investigação, idosos que relataram que a situação econômica piorou quando
comparada aos 50 anos de idade apresentaram maiores percentuais de dor crônica.
Embora algumas investigações\textsuperscript{[}9\textsuperscript{]}\textsuperscript{,}\textsuperscript{[}12\textsuperscript{]}
mencionem que o percentual de dor crônica é maior entre os indivíduos
pertencentes às classes sociais mais baixas, não foram encontrados outros
estudos que analisaram a relação entre dor crônica e situação econômica aos 50
anos comparada com a atual, o que evidencia a necessidade de maiores
investigações a respeito.

No presente estudo, constatou-se que ser fisicamente ativo no lazer, ficou
significativamente associado a menor prevalência de dor crônica. A prática de
atividades físicas pelos idosos, principalmente no lazer, proporciona
oportunidades para uma vida mais ativa, saudável e independente, contribuindo
para a manutenção da autonomia e melhora da qualidade de
vida\textsuperscript{[}28\textsuperscript{]}. Em estudo transversal com a população da Noruega, os autores encontraram entre
10 e 12\% menor prevalência de dor crônica entre indivíduos de 20 - 64 anos que
praticavam atividade física com intensidade moderada e frequência semanal de
três vezes; já entre os mais velhos, dependendo da intensidade do exercício,
houve redução de 21 - 38\% na prevalência de dor
crônica\textsuperscript{[}29\textsuperscript{]}.

Entre as estratégias empregadas por programas multidisciplinares destinados ao
tratamento da dor crônica, a atividade física é uma das mais
utilizadas\textsuperscript{[}30\textsuperscript{]}\textsuperscript{-}\textsuperscript{[}31\textsuperscript{]}. Uma das hipóteses mais aceitas sobre os benefícios da prática de atividade
física para a gestão da dor crônica se deve ao fato de sua influência estar
relacionada aos mecanismos endógenos de controle da
dor\textsuperscript{[}32\textsuperscript{]}.

Algumas limitações do presente estudo devem ser consideradas, sobretudo o
delineamento transversal, que não permite definir relações de causalidade entre
a prevalência de dor crônica e as demais variáveis investigadas e as medidas
autorreferidas das variáveis estudada. Entretanto, não há relevância em saber,
neste caso, se os idosos apresentaram menor prevalência para o desfecho por
serem fisicamente ativos ou se por serem fisicamente ativos apresentaram menor
prevalência para o desfecho, pois ser fisicamente ativo pode ter sido benéfico
tanto para a manutenção da saúde, evitando assim o surgimento da dor crônica,
como para que nos idosos que apresentaram dor crônica esta tivesse sua
intensidade e duração reduzida. Entre os pontos positivos, o estudo se destaca
pela relevância e originalidade do tema, servindo de base para outras
investigações, bem como pelo fato de a amostra ser ampla e representativa dos
idosos do município de Florianópolis. Ressalta-se ainda a elevada taxa de
resposta do estudo, que contribuiu para a validade interna dele, diminuindo a
chance de ocorrência de erros sistemáticos.

\section{\textsc{conclusão}}

A constatação de que as mulheres, os indivíduos com baixa escolaridade, com pior
situação econômica e insuficientemente ativos apresentam maior prevalência de
dor crônica entre os idosos representa um importante achado, que poderá
subsidiar políticas de saúde pública e focadas na atenção ao idoso.

Portanto, os resultados sugerem que campanhas de prevenção à dor crônica devam
visar prioritariamente mulheres, com baixa renda e insuficientemente ativas no
lazer. Faz-se necessário ainda o desenvolvimento de programas multidisciplinares
de gestão e controle da dor crônica, incluindo orientação aos profissionais de
saúde para atuarem na prevenção à dor crônica e programas de atividade física
voltados especificamente ao idoso, com o objetivo de evitar que a dor crônica
configure-se como fator responsável pelo comprometimento da qualidade de vida
dos idosos.

\section*{\textsc{referências}}
\begin{itemize}

\item[1] Programa das Nações Unidas - \textsc{pnud}. Atlas do Desenvolvimento Humano
no Brasil 2003. Disponível em http://www.pnud.org.br/atlas. (Acessado em maio de
2010).

\item[2] Instituto Brasileiro de Geografia e Estatística - \textsc{ibge}. Sinopse do
Censo Demográfico de 2010/2011. Disponível em http://www.ibge.gov.br/home/estati
stica/populacao/censo2010/default\_{}sinopse.shtm. (Acessado em maio de 2013).

\item[3] Dellaroza MS, Pimenta CA, Matsuo T. Prevalence and characterization
of chronic pain among the elderly living in the community. Cad Saúde Pública
2007; 23(5): 1151-60.

\item[4] Souza JB. Poderia a atividade física induzir analgesia em pacientes
com dor crônica? Rev Bras Med Esporte 2009; 15(2): 145-50.

\item[5] Rull M. Abordaje multidisciplinar del dolor de espalda. Rev Soc Esp
Dolor 2004; 11(3): 119-21.

\item[6] Jakobsson U, Klevsgård R, Westergren A, Hallberg IR. Old people in
pain: a comparative study. J Pain Symptom Manage 2003; 26(1): 625-36.

\item[7] Yu HY, Tang FI, Kuo BI, Yu S. Prevalence, interference, and risk
factors for chronic pain among Taiwanese community older people. Pain Manag Nurs
2006; 7(1): 2-11.

\item[8] Catàla E, Reig E, Artés M, Aliaga L, López J, Segú J. Prevalence of
pain in the Spanish population: telephone survey in 5000 homes. Eur J Pain 2002;
6(2): 133-40.

\item[9] Dellaroza MS, Furuya RK, Cabrera MA, Matsuo T, Trelha C, Yamada KN,
et al. Caracterização da dor crônica e métodos analgésicos utilizados por idosos
da comunidade. Rev Assoc Med Bras 2008; 54(1): 36-41.

\item[10] Dellaroza MS, Pimenta CA, Duarte YA, Lebrão ML. Dor crônica em
idosos residentes em São Paulo, Brasil: prevalência, características e
associação com capacidade funcional e mobilidade (Estudo \textsc{sabe}). Cad Saúde
Pública 2013; 29(2): 325-34.

\item[11] Blyth FM, March LM, Brnabic AJ, Jorm LR, Williamson M, Cousins MJ.
Chronic pain in Australia: a prevalence study. Pain 2001; 89(2): 127-34.

\item[12] Turner JA, Franklin G, Fulton-Kehoe D, Egan K, Wickizer TM, Lymp
JF, et al. Prediction of chronic disability in work-related musculoskeletal
disorders: a prospective, population-based study. \textsc{bmc} Musculoskel Disord 2004;
5(1): 14.

\item[13] Instituto Brasileiro de Geografia e Estatística - \textsc{ibge}.
Estimativas populacionais para o \textsc{tcu}. Estimativas da população para 1º de julho
de 2009 2009. Disponível em http://www.ibge.gov.br/home/estatistica/populacao/es
timativa2009/\textsc{pop}2009\_{}\textsc{dou}.pdf (Acessado em junho de 2010).

\item[14] Instituto Brasileiro de Geografia e Estatística - \textsc{ibge}. Perfil dos
Idosos responsáveis pelos domicílios no Brasil 2000. Disponível em
http://www.ibge.gov.br/home/estatistica/populacao/perfilidoso/default.shtm
(Acessado em junho de 2010).

\item[15] Perez C, Galvez R, Huelbes S, Insausti J, Bouhassira D, Diaz S, et
al. Validity and reliability of the Spanish version of the DN4 (Douleur
Neuropathique 4 questions) questionnaire for differential diagnosis of pain
syndromes associated to a neuropathic or somatic component. Health Qual Life
Outcomes 2007; 5: 66.

\item[16] Merskey H, Bogduk N (eds). Task Force on Taxonomy of the
International Association for the Study of Pain. Classification of chronic pain:
descriptions of chronic pain syndromes and definition of pain terms. Seattle:
\textsc{iasp}; 1994.

\item[17] Benedetti TB, Mazo GZ, Barros Md. Aplicação do Questionário
Internacional de Atividades Físicas para avaliação do nível de atividades
físicas de mulheres idosas: validade concorrente e reprodutibilidade
teste-reteste. Rev Bras Ciênc Mov 2004; 12(1): 25-34.

\item[18] Nelson ME, Rejeski JW, Blair SN, Duncan PW, Judge JO, King AC, et
al. Physical activity and public health in older adults: recommendation from the
American College of Sports Medicine and the American Heart Association.
Circulation 2007; 116(9): 1094-105.

\item[19] Barros AJ, Hirakata VN. Alternatives for logistic regression in
cross-sectional studies: an empirical comparison of models that directly
estimate the prevalence ratio. \textsc{bmc} Med Res Methodol 2003; 3(1): 21.

\item[20] Díaz CR, Marulanda MF, Sáenz X. Estudio epidemiológico del dolor
crónica en Caldas, Colombia (Estudio \textsc{dolca}). Acta Méd Colomb 2009; 34(3):
96-102.

\item[21] Kreling M, Cruz D, Pimenta CAdM. Prevalência de dor crônica em
adultos. Rev Bras Enferm 2006; 59(4): 509-13.

\item[22] Sá K, Baptista AF, Matos MA, LessaI I. Prevalência de dor crônica
e fatores associados na população de Salvador, Bahia. Rev Saude Publica 2009;
43(4): 622-30.

\item[23] Silva MC, Fassa AG, Valle NC. Dor lombar crônica em uma população
adulta do Sul do Brasil: prevalência e fatores associados. Cad Saúde Pública
2004; 20(2): 377-85.

\item[24] Vieira EB, Garcia JB, Silva AA, Araújo RL, Jansen RC, Bertrand AL.
Chronic pain, associated factors, and impact on daily life: are there
differences between the sexes? Cad Saúde Pública 2012; 28(8): 1459-67.

\item[25] Bouhassira D, Lantéri-Minet M, Attal N, Laurent B, Touboul C.
Prevalence of chronic pain with neuropathic characteristics in the general
population. Pain 2008; 136(3): 380-7.

\item[26] Leveille SG, Zhang Y, McMullen W, Kelly-Hayes M, Felson DT. Sex
differences in musculoskeletal pain in older adults. Pain 2005; 116(3): 332-8.

\item[27] Celich \textsc{kls}, Galon C. Dor crônica em idosos e sua influência nas
atividades da vida diária e convivência social. Rev Bras de Geriatr Gerontol
2009; 12(3): 345-59.

\item[28] Cress ME, Buchner DM, Prohaska T, Rimmer J, Brown M, Macera C, et
al. Best practices for physical activity programs and behavior counseling in
older adult populations. J Aging Phys Act 2005; 13(1): 61-74.

\item[29] Landmark T, Romundstad P, Borchgrevink PC, Kaasa S, Dale O.
Associations between recreational exercise and chronic pain in the general
population: evidence from the \textsc{hunt} 3 study. Pain 2011; 152(10): 2241-7.

\item[30] Bennett RM, Burckhardt C, Clark S, O'Reilly C, Wiens A, Campbell
S. Group treatment of fibromyalgia: a 6 month outpatient program. J Rheumatol
1996; 23(3): 521-8.

\item[31] Souza JB, Charest J, Marchand S. École interactionnelle de
fibromyalgie: description et évaluation. Douleur et Analgésie 2007; 20(4):
213-8.

\item[32] Koltyn KF. Analgesia following exercise: a review. Sports Med
2000; 29(2): 85-98.

\end{itemize}

   \section*{Metadados não aplicados}
    \begin{itemize}
    \ifdef{\lingua}{\item[\textbf{língua do artigo}] \lingua}{}
    \ifdef{\journalid}{\item[\textbf{journalid}] \journalid}{}
    \ifdef{\journaltitle}{\item[\textbf{journaltitle}] \journaltitle}{}
    \ifdef{\journalsubtitle}{\item[\textbf{journalsubtitle}] \journaltitle}{}
    \ifdef{\transjournaltitle}{\item[\textbf{journaltitle}] \journaltitle}{}
    \ifdef{\transjournalsubtitle}{\item[\textbf{journalsubtitle}] \journaltitle}{}
    \ifdef{\abbrevjournaltitle}{\item[\textbf{abbrevjournaltitle}] \abbrevjournaltitle}{}
    \ifdef{\issnppub}{\item[\textbf{issnppub}] \issnppub}{}
    \ifdef{\issnepub}{\item[\textbf{issnepub}] \issnepub}{}
    \ifdef{\alttitleauthor}{\item[\textbf{alttitle}] \alttitleauthor}{}
    \ifdef{\alttitle}{\item[\textbf{alttitleauthor}] \alttitle}{}
    \ifdef{\publishername}{\item[\textbf{publishername}] \publishername}{}
    \ifdef{\publisherid}{\item[\textbf{publisherid}] \publisherid}{}
    \ifdef{\subject}{\item[\textbf{subject}] \subject}{} 
    \ifdef{\transtitle}{\item[\textbf{transtitle}] \transtitle}{}
    \ifdef{\authornotes}{\item[\textbf{authornotes}] \authornotes}{}
    \ifdef{\articleid}{\item[\textbf{articleid}] \articleid}{}
    \ifdef{\articledoi}{\item[\textbf{articledoi}] \articledoi}{}
    \ifdef{\volume}{\item[\textbf{volume}] \volume}{}
    \ifdef{\issue}{\item[\textbf{issue}] \issue}{}
    \ifdef{\fpage}{\item[\textbf{fpage}] \fpage}{}
    \ifdef{\lpage}{\item[\textbf{lpage}] \lpage}{}
    \ifdef{\permissions}{\item[\textbf{permissions}] \permissions}{}
    \ifdef{\copyrightyear}{\item[\textbf{copyrightyear}] \copyrightyear}{}

    \end{itemize}
\end{document}
